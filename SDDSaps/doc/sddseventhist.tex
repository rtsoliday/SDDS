\begin{sddsprog}{sddseventhist}
  \item \textbf{description:}
  \verb|sddseventhist| analyzes labeled events in a dataset to provide histograms of the occurrences of each type of event. It can also histogram the overlap of all types of events with a single type of event.
  \item \textbf{examples:}
    \begin{verbatim}
sddseventhist events.sdds hist.sdds -dataColumn=Time -eventIdentifier=Event -bins=50
sddseventhist events.sdds hist.sdds -dataColumn=Time -eventIdentifier=Event -overlapEvent=Alarm
    \end{verbatim}
  \item \textbf{synopsis:}
    \begin{verbatim}
sddseventhist [-pipe=[input][,output]] [inputFile] [outputFile]
  -dataColumn=columnName -eventIdentifier=columnName [-overlapEvent=eventValue]
  [-bins=number | -sizeOfBins=value]
  [-lowerLimit=value] [-upperLimit=value]
  [-sides] [-normalize={sum | area | peak}]
    \end{verbatim}
  \item \textbf{files:}
  \emph{inputFile} is a file containing at least two columns of data. One column must contain string entries that serve as ``event identifiers''; for example, these might be the names of channels that issued an alarm. The other column must contain numerical data that will be histogrammed; for example, these might be the times at which alarms occurred. The \emph{outputFile} contains one histogram of this numerical data for each unique value of the event identifier; the histogram contains only the data that matches that identifier.
  \item \textbf{switches:}
    \begin{itemize}
      \item \verb|-pipe[=input][,output]| --- The standard SDDS Toolkit pipe option.
      \item \verb|-dataColumn=columnName| --- Specifies the name of the data column to be histogrammed.
      \item \verb|-eventIdentifier=columnName| --- Specifies the name of the string column that identifies events.
      \item \verb|-overlapEvent=eventValue| --- Requests computation of the overlap of the histograms of each event with the histogram of event \emph{eventValue}. Useful in determining which events always occur at the same time as event \emph{eventValue}.
      \item \verb|-bins=number| --- Specifies the number of bins to use. The default is 20.
      \item \verb|-sizeOfBins=value| --- Specifies the size of bins to use. The number of bins is computed from the range of the data.
      \item \verb|-lowerLimit=value| --- Specifies the lower limit of the histogram. By default, the lower limit is the minimum value in the data.
      \item \verb|-upperLimit=value| --- Specifies the upper limit of the histogram. By default, the upper limit is the maximum value in the data.
      \item \verb|-sides| --- Specifies that zero-height bins should be attached to the lower and upper ends of the event histogram. Many prefer the way this looks on a graph.
      \item \verb!-normalize={sum | area | peak}! --- Specifies that the histogram should be normalized, and how. The default is \verb|sum|. \verb|sum| normalization means that the sum of the heights will be 1. \verb|area| normalization means that the area under the histogram will be 1. \verb|peak| normalization means that the maximum height will be 1.
    \end{itemize}
  \item \textbf{see also:}
    \begin{itemize}
      \item \progref{sddscorrelate}
      \item \progref{sddshist}
      \item \progref{sddshist2d}
    \end{itemize}
  \item \textbf{author:} M. Borland, ANL/APS.
\end{sddsprog}
