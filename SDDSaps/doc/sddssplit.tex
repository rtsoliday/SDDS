\begin{sddsprog}{sddssplit}
  \item \textbf{description:} \verb|sddssplit| breaks up an SDDS file into one or more separate files, each containing a single page of data. This is useful when a program only processes the first page of an SDDS file.
  \item \textbf{examples:}
  \begin{verbatim}
sddssplit APS.twi
  \end{verbatim}
  \item \textbf{synopsis:}
  \begin{verbatim}
sddssplit {-pipe[=input] | inputFile} [-binary | -ascii]
  [-digits=number] [-rootname=string] [-extension=string]
  [-nameParameter=paramName]
  [-firstPage=number] [-lastPage=number] [-interval=number]
  \end{verbatim}
  \item \textbf{files:} \emph{inputFile} is an SDDS file to be split. By default, the output files are created by appending the page number to a ``rootname'' and adding an extension. That is, the output files have names \emph{rootname}\emph{Page}.\emph{extension}.
  The default rootname is the name of \emph{inputFile}, while the default extension is ``sdds''. By default, \emph{Page} is printed using ``\%03ld'' format.
  \item \textbf{switches:}
    \begin{itemize}
      \item {\tt -pipe[=input][,output]} --- The standard SDDS Toolkit pipe option.
      \item {\tt -binary}, {\tt -ascii} --- Specifies binary or ASCII output, with binary being the default.
      \item {\tt -digits=\emph{number}} --- Specifies the number of digits used in creating filenames. Leading zeros are included.
      \item {\tt -rootname=\emph{string}} --- Specifies the rootname used in creating filenames.
      \item {\tt -extension=\emph{string}} --- Specifies the extension used in creating filenames.
      \item {\tt -nameParameter=\emph{paramName}} --- Specifies that, instead of composing names for the output files, \verb|sddssplit| takes the names from the string parameter \emph{paramName} in the input file. This provides a limited capability to retrieve the original files from a file made with \verb|sddscombine|. Note that if the named parameter takes the same value on two pages, the file created for the first of the pages will be overwritten.
      \item {\tt -firstPage=\emph{number}} --- Specifies the first page of data to use.
      \item {\tt -lastPage=\emph{number}} --- Specifies the last page of data to use.
      \item {\tt -interval=\emph{number}} --- Specifies the interval between pages that are used.
    \end{itemize}
  \item \textbf{see also:}
    \begin{itemize}
      \item \progref{sddsbreak}
      \item \progref{sddscombine}
    \end{itemize}
  \item \textbf{author:} M. Borland, ANL/APS.
\end{sddsprog}
