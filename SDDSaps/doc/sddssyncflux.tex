\begin{sddsprog}{sddssyncflux}
  \item \textbf{description:}
    \verb|sddssyncflux| calculates synchrotron radiation photon flux of bend, wiggler, and undulator magnets. The calculation for undulators has not been implemented yet.
  \item \textbf{examples:}
    \begin{verbatim}
sddssyncflux bend.test -source=bend -mode=energy,linear,start=1,end=3,step=0.1
sddssyncflux wiggler.test -source=wiggler,period=5,numberOfPeriods=7,field=1 -mode=energy,linear,start=100,end=200,step=2
    \end{verbatim}
  \item \textbf{synopsis:}
    \begin{verbatim}
sddssyncflux <outputFile> -verbose [-pipe]
  [-fileValues=<filename>[,energy=<columnName or wavelength=columnName>]]
  [-mode=energy(wavelength),linear(logarithmic),start=<value>,end=<value>,step(factor)=<value>]
  [-source=bendMagnet[,field=xx[,radius=yy][,criticalEnergy=zz]]
  [-source=wiggler(undulator),period=xx[,field=yy][,K=zz],numberOfPeriods=<n>]
  [-eBeamEnergy=<value> [-eBeamCurrent=<value>] [-eBeamGamma=<value>]]
    \end{verbatim}
  \item \textbf{files:}
    \emph{outputFile} the results are written to an SDDS file.
  \item \textbf{switches:}
    \begin{itemize}
      \item \verb|-pipe| --- output result to the pipe.
      \item \verb|-fileValues=<filename>[,energy=<columnName or wavelength=columnName>]| --- get the energy or wavelength from \emph{filename} instead of by -mode option.
      \item \verb|-mode=energy,linear,start=<value>,end=<value>,step=<value>| --- Generate photon energy column in eV linearly, from start to end in steps.
      \item \verb|-mode=energy,logarithmic,start=<value>,end=<value>,factor=<value>| --- Generate photon energy column logarithmically, from start to end by multiplying factor from point to point.
      \item \verb|-mode=wavelength,linear,start=<value>,end=<value>,step=<value>| --- Generate photon wavelength column in nm linearly, from start to end in steps.
      \item \verb|-mode=wavelength,logarithmic,start=<value>,end=<value>,factor=<value>| --- Generate photon wavelength column in nm logarithmically, from start to end by multiplying factor from point to point.
      \item \verb|-source=bendMagnet[,field=xx][,radius=yy][,criticalEnergy=zz]| --- Bend magnet source, magnetic field \verb|xx| Tesla (default 0.6 Tesla), bend radius \verb|yy| meters, critical energy \verb|zz| eV. Only one of field, radius and critical energy is needed.
      \item \verb|-source=wiggler,period=xx[,field=yy][,K=zz],numberOfPeriods=n| --- Wiggler source, period \verb|xx| cm (default 5 cm), peak magnetic field \verb|yy| Tesla (default 1 Tesla), undulator parameter \verb|K=zz| (no default). Only two of period, field and K are needed. \verb|numberOfPeriods| must be provided.
      \item \verb|-source=undulator,period=xx[,field=yy][,K=zz],numberOfPeriods=n| --- Undulator source, period \verb|xx| cm (default 5 cm), peak magnetic field \verb|yy| Tesla (default 1 Tesla), undulator parameter \verb|K=zz| (no default). Only two of period, field and K are needed. \verb|numberOfPeriods| must be provided. \textbf{Note that only one source is accepted at one time.}
      \item \verb|-eBeamEnergy=<value>| --- Electron beam energy in GeV, default 7 GeV.
      \item \verb|-eBeamGamma=<value>| --- Electron beam gamma.
      \item \verb|-eBeamCurrent=<value>| --- Electron beam current in A.
      \item \verb|-g1ValueFile=<filename>| --- Give the file that contains the values of \emph{y} and \emph{yGy}, where \emph{yGy} is \emph{y} multiplied by the integral of the Bessel function \verb|K5/3| from \emph{y} to infinity.
    \end{itemize}
  \item \textbf{see also:}
    \begin{itemize}
      \item \progref{sddsplot}
    \end{itemize}
  \item \textbf{author:} H. Shang ANL/APS.
\end{sddsprog}

