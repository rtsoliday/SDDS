\begin{sddsprog}{sdds2dpfit}
  \item \textbf{description:} \verb|sdds2dpfit| does ordinary 2-dimensional polynomial fits to column data.
  \item \textbf{examples:}
    \begin{verbatim}
    sdds2dpfit data.sdds fit.sdds -independent=x,y -dependent=z -maximumOrder=2
    sdds2dpfit data.sdds fit.sdds -independent=x,y -dependent=z \
      -addOrders=1,1 -coefficients=coeff.sdds
    \end{verbatim}
  \item \textbf{synopsis:}
    \begin{verbatim}
    sdds2dpfit [inputfile] [outputfile] [-pipe=[input][,output]] \
      -independent=x1ColumnName,x2ColumnName -dependent=yColumnName[,sigma=columnName] \
      {-maximumOrder=value | [-addOrders=xOrder,yOrder [-addOrders=...]]} \
      [-coefficients=filename] \
      [-evaluate=locationsFilename,x1Name,x2Name,outputFilename] \
      [-copyParameters]
    \end{verbatim}
  \item \textbf{files:}
    \emph{inputFile} is an SDDS file containing columns of data to be fit. If it contains multiple pages, they are processed
    separately. \emph{outputFile} is an SDDS file containing one page for each page of \emph{inputFile}. It contains columns of
    the independent and dependent variable data, plus columns for the fit and residuals. The values of the fit
    and of the residuals are in columns named \emph{yName}{\tt Fit} and \emph{yName}{\tt Residual}. \emph{outputFile} also contains
    the following parameters:
    \begin{itemize}
      \item {\tt RmsResidual} --- Rms residual of the fit.
      \item {\tt ReducedChiSquared}: the reduced chi-squared of the fit:
        $$ \chi^2_\nu = \frac{\chi^2}{\nu} = \frac{1}{N-T}\sum_{i=0}^{N-1} \left(\frac{z_i - z(x_i, y_i)}{\sigma_i}\right)^2 $$,
        where $\nu = N-T$ is the number of degrees of freedom for a fit of N points with T terms.
      \item {\tt ConditionNumber} --- Condition number from SVD inversion of the matrix used in obtaining the fit.
      \item {\tt FitIsValid} --- If non-zero, the fit is valid.
      \item {\tt Terms} --- Number of terms in the fit.
      \item {\tt Coefficient_{\em mm}_{\em nn}} --- Coefficient $C_{mn}$ of the fit. E.g., if fitting $z$ as a function of $x$ and
        $y$, then
        \begin{equation}
          z = \sum_m \sum_n C_{mn} x^m y^n.
        \end{equation}
    \end{itemize}
  \item \textbf{switches:}
    \begin{itemize}
      \item {\tt -pipe[=input][,output]} --- The standard SDDS Toolkit pipe option.
      \item {\tt -independent={\em x1Name},{\em x2Name}} --- Gives the names of the columns containing values of the independent quantities.
      \item {\tt -dependent={\em yName}[,sigma={\em sigmaName}]} --- Gives the name of the column containing values of the dependent quantities, and optionally the name of the column containing error bars.
      \item {\tt -maximumOrder={\em p}} --- Request inclusion of all terms up to $x^m y^n$ such that $(n+m)\leq p$.
      \item {\tt -addOrders={\em m},{\em n}} --- Request inclusion of $x^m y^n$. May be repeated to request inclusion of additional terms.
      \item {\tt -coefficients={\em filename}} --- Request that fit coefficients be written to the named file.
      \item {\tt -evaluate={\em locationsFilename},{\em x1Name},{\em x2Name},{\em outputFilename}} --- Request evaluation of the fit for a set of $x$ and $y$ values given in \emph{locationsFilename}, with results written to \emph{outputFilename}.
      \item {\tt -copyParameters} --- If given, program copies all parameters from the input file into the main output file. By default, no parameters are copied.
    \end{itemize}
  \item \textbf{see also:}
    \begin{itemize}
      \item \hyperref[exampleData]{Data for Examples}
      \item \progref{sddspfit}
      \item \progref{sddsmpfit}
      \item \progref{sddsgfit}
      \item \progref{sddsplot}
    \end{itemize}
  \item \textbf{author:} M. Borland, ANL/APS.
\end{sddsprog}
