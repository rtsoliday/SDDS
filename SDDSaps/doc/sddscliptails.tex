%\begin{latexonly}
\newpage
%\end{latexonly}
\subsection{sddscliptails}
\label{sddscliptails}

\begin{itemize}
\item {\bf description:}
\verb|sddscliptails| removes the tails from functions, where a tail is a dubious feature
extending to the left or right of a peak.
\item {\bf examples:}
Remove tails from profiles of beam spots after baseline removal and prior to determining
rms spot properties.  This command clips the tails when the function falls to 1\% of
its peak value.
\begin{flushleft}{\tt
sddscliptails input.sdds output.sdds -columns=VideoLine* -fractional=0.01
}\end{flushleft}
\item {\bf synopsis:} 
\begin{flushleft}{\tt
sddscliptails [{\em input}] [{\em output}] [-pipe=[in][,out]]
[-columns={\em listOfNames}] [-fractional={\em value}] [-absolute={\em value}] [-fwhm={\em multiplier}]
[-afterzero[={\em bufferWidth}]]
}\end{flushleft}
\item {\bf files:}
{\em input} is an SDDS file containing one or more pages of data to be processed.
{\em output} is an SDDS file in which the result is placed.  The output file will
generally have fewer rows than the input file, corresponding the the number of
rows clipped.
\item {\bf switches:}
    \begin{itemize}
    \item {\tt -pipe=[input][,output]} --- The standard SDDS Toolkit pipe option.
    \item {\tt -columns={\em listOfNames}} --- Specifies an optionally-wildcarded list
        of names of columns from which to remove tails.
    \item {\tt -fractional={\em value}} --- Clip a tail if it falls below this fraction of the peak
        value of the column.
    \item {\tt -absolute={\em value}} --- Clip a tail if it falls below this absolute value.  This
        value might correspond, say, to a known noise level.
    \item {\tt -fwhm={\em multiplier}} --- Clip a tail if it is beyond {\em multiplier} times
        the full-width-at-half-maximum (FWHM) from the peak of the column.
    \item {\tt -afterzero[={\em bufferWidth}]} --- Clip a tail if it is separated from the 
        peak by values equal to zero.
        If {\em bufferWidth} is specified, then a region {\em bufferWidth} wide is kept
        on either side of the peak, if possible.
    \end{itemize}
\item {\bf see also:}
    \begin{itemize}
    \item \progref{sddsbaseline}
    \end{itemize}
\item {\bf author:} M. Borland, ANL/APS.
\end{itemize}
