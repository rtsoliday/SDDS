%\begin{latexonly}
\newpage
%\end{latexonly}
\subsection{sddsexpand}
\label{sddsexpand}

\begin{sddsprog}{sddsexpand}
  \item \textbf{description:} \verb|sddsexpand| reads data pages from an SDDS file and writes a new SDDS file containing a separate data page for every row in the input file. All column definitions from the input file are turned into parameter definitions in the output file. In addition all parameter definitions from the input file are copied to the output file. Each output data page contains the values of the columns from a single row of the input file, along with the values of the parameters from the same page. The output file contains no column or array definitions.

  \verb|sddsexpand| is essentially the inverse of \verb|sddscollapse| (except that the column data thrown out in collapsing a file is not recoverable).
  \item \textbf{examples:}
    \begin{verbatim}
sddsexpand input.sdds expanded.sdds
    \end{verbatim}
  \item \textbf{synopsis:}
    \begin{verbatim}
sddsexpand [inputFile] [outputFile] [-pipe[=input][,output]] [-noWarnings]
    \end{verbatim}
  \item \textbf{switches:}
    \begin{itemize}
      \item \verb|-pipe[=input][,output]| --- The standard SDDS Toolkit pipe option.
      \item \verb|-noWarnings| --- Suppresses warnings about name clashes between parameters and columns. If such a clash occurs, the parameter data is ignored.
    \end{itemize}
  \item \textbf{files:} \emph{inputFile} is the name of an SDDS data set to be expanded. \emph{outputFile} is the result. Note that \emph{outputFile} will not contain any information from any arrays that are in \emph{inputFile}.
  \item \textbf{see also:}
    \begin{itemize}
      \item \progref{sddscollapse}
      \item \progref{sddsbreak}
    \end{itemize}
  \item \textbf{author:} M. Borland, ANL/APS.
\end{sddsprog}
