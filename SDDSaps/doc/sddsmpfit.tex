\begin{sddsprog}{sddsmpfit}
  \item \textbf{description:}
  \verb|sddsmpfit| does ordinary and Chebyshev polynomial fits to column data,
  including error analysis. It will do fits with specified number of terms,
  with specific terms only, and with specific symmetry only. It will also
  eliminate spurious terms. The options for \verb|sddsmpfit| are very similar
  to those for \verb|sddspfit|.
  \item \textbf{examples:}
  Perform a second-order fit for column \verb|y| as a function of \verb|x| and
  create evaluation points for the fit.
  \begin{verbatim}
sddsmpfit data.sdds fit.sdds -independent=x -dependent=y \
  -terms=3 -evaluate=fitEval.sdds
  \end{verbatim}
  \item \textbf{synopsis:}
  \begin{verbatim}
sddsmpfit [-pipe=[input][,output]] [inputFile] [outputFile]
  -independent=xName [-sigmaIndependent=xSigmaName]
  -dependent=yName[,yName...] [-sigmaDependent=templateString]
  {-terms=number [-symmetry={none|odd|even}] | -orders=number[,number...]}
  [-reviseOrders[=threshold=chiValue][,verbose]] [-chebyshev[=convert]]
  [-xOffset=value] [-xFactor=value]
  [-sigmas=value,{absolute|fractional}]
  [-modifySigmas] [-generateSigmas[={keepLargest|keepSmallest}]]
  [-sparse=interval] [-range=lower,upper]
  [-normalize[=termNumber]] [-verbose]
  [-evaluate=filename[,begin=value][,end=value][,number=integer]]
  [-fitLabelFormat=sprintfString] [-infoFile=filename]
  \end{verbatim}
  \item \textbf{files:}
  \emph{inputFile} is an SDDS file containing columns of data to be fit.
  If it contains multiple pages, they are processed separately.
  \emph{outputFile} is an SDDS file containing one page for each page of
  \emph{inputFile}. It contains columns of the independent and dependent
  variable data, plus columns for error bars (``sigmas'') as appropriate.
  The values of the fit and of the residuals are in columns named
  \emph{yName}\verb|Fit| and \emph{yName}\verb|Residual|. In addition,
  various parameters having names beginning with \emph{yName} are created
  that give reduced chi-squared, slope, intercept, and so on.
  \item \textbf{switches:}
    \begin{itemize}
    \item \verb|-pipe[=input][,output]| --- The standard SDDS Toolkit pipe option.
    \item \verb|-evaluate=filename[,begin=value][,end=value][,number=integer]| ---
      Specifies creation of an SDDS file called \emph{filename} containing
      points from evaluation of the fit. The fit is normally evaluated over
      the range of the input data; this may be changed using the
      \verb|begin| and \verb|end| qualifiers. Normally, the number of points
      at which the fit is evaluated is the number of points in the input data;
      this may be changed using the \verb|number| qualifier.
    \item \verb|-infoFile=filename| --- Specifies creation of an SDDS file
      containing results of the fits in columns. A column called
      \emph{yName}\verb|Coefficient| is created for each column that is
      fitted.
    \item By default, an ordinary polynomial fit is done using a constant and
      linear term. Control of what fit terms are used is provided by the
      following switches:
      \begin{itemize}
      \item \verb|-terms=number| --- Specifies the number of terms to be used
        in fitting. 2 terms is a linear fit, 3 is quadratic, etc.
      \item \verb!-symmetry={none|odd|even}! --- When used with \verb|-terms|,
        allows specifying the symmetry of the N terms used. \verb|none| is the
        default. \verb|odd| implies using linear, cubic, etc., while
        \verb|even| implies using constant, quadratic, etc.
      \item \verb|-orders=number[,number...]| --- Specifies the polynomial
        orders to be used in fitting. The default is equivalent to
        \verb|-orders=0,1|.
      \item \verb|-reviseOrders[=threshold=value][,verbose]| --- Asks for
        adaptive fitting to be performed on the first data page to determine
        what orders to use. Any term that does not improve the reduced
        chi-squared by \emph{value} is discarded. Similar to but much less
        capable than the adaptive fitting feature of \verb|sddspfit|.
      \item \verb|-chebyshev[=convert]| --- Asks that Chebyshev T polynomials
        be used in fitting. If \verb|convert| is given, the output contains the
        coefficients for the equivalent ordinary polynomials.
      \end{itemize}
    \item \verb|-xOffset=value|, \verb|-xFactor=value| --- Specify offsetting
      and scaling of the independent data prior to fitting. The transformation
      is $x \rightarrow (x - \mathrm{Offset})/\mathrm{Factor}$. This feature
      can be used to make a fit about a point other than $x=0$, or to scale the
      data to make high-order fits more accurate.
    \item \verb|sddsmpfit| will compute error bars (``sigmas'') for fit
      coefficients if it has knowledge of the sigmas for the data points.
      These can be supplied using the \verb|-columns| switch, or generated
      internally in several ways:
      \begin{itemize}
      \item \verb!-sigmas=value{absolute|fractional}! --- Specifies that
        independent-variable errors be generated using a specified value for
        all points, or a specified fraction for all points.
      \item \verb|-modifySigmas| --- Specifies that independent-variable sigmas
        be modified to include the effect of uncertainty in the dependent
        variable values. If this option is not given, any x sigmas specified
        are ignored.
      \item \verb!-generateSigmas[={keepLargest|keepSmallest}]! --- Specifies
        that independent-variable errors be generated from the variance of an
        initial equal-weights fit. If errors are already given (via
        \verb|-column|), one may request that for every point \verb|sddsmpfit|
        retain the larger or smaller of the sigma in the data and the one given
        by the variance.
      \end{itemize}
    \item \verb|-sparse=interval| --- Specifies sparsing of the input data
      prior to fitting. This can greatly speed computations when the number of
      data points is large.
    \item \verb|-range=lower,upper| --- Specifies the range of independent
      variable over which to do fitting.
    \item \verb|-normalize[=termNumber]| --- Specifies that coefficients be
      normalized so that the coefficient for the indicated order is unity. By
      default, the 0-order term (i.e., the constant term) is normalized to
      unity.
    \item \verb|-verbose| --- Specifies that the results of the fit be printed
      to the standard error output.
    \item \verb|-fitLabelFormat=sprintfString| --- Specifies the format to use
      for printing numbers in the fit label. The default is ``\%g''.
    \end{itemize}
  \item \textbf{see also:}
    \begin{itemize}
    \item \hyperref[exampleData]{Data for Examples}
    \item \progref{sddspfit}
    \item \progref{sddsoutlier}
    \end{itemize}
  \item \textbf{author:} M. Borland, ANL/APS.
\end{sddsprog}

