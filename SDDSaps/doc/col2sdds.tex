\begin{sddsprog}{col2sdds}
  \item \textbf{description:}
  Converts a file in \verb|column| self-describing format to SDDS. This is of interest to
  APS users only, some of whom still have programs that generate \verb|column|-format files.
  \item \textbf{examples:}
    \begin{verbatim}
col2sdds input.col output.sdds
col2sdds data.col data.sdds -fixMplNames
    \end{verbatim}
  \item \textbf{synopsis:}
    \begin{verbatim}
col2sdds inputFile outputFile [-fixMplNames]
    \end{verbatim}
  \item \textbf{files:}
  \emph{inputFile} is a \verb|column|-format file, the SDDS equivalent of which is written to
  \emph{outputFile}. The ``auxiliary values'' of the \verb|column| file are converted into SDDS
  parameters. The \verb|column| table is converted into SDDS tabular data, all columns being
  double precision except the ``row label'', which becomes a string column.
  \item \textbf{switches:}
    \begin{itemize}
      \item \verb|-fixMplNames| --- Requests that any column or parameter names in the input file
      that contain \verb|mpl| character set escape sequences be ``fixed''. This results in simpler
      names. The escape sequences are always retained in definition of the symbol for each column
      or parameter, and hence will appear on graphs as expected.
    \end{itemize}
  \item \textbf{author:} M. Borland, ANL/APS.
\end{sddsprog}

