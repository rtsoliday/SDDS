% $Log: not supported by cvs2svn $
% Revision 1.1  2003/06/20 19:34:59  shang
% first version
%
% first version, Hairong Shang
%
%
% Template for making SDDS Toolkit manual entries.
%
%\begin{latexonly}
\newpage
%\end{latexonly}

%
% Substitute the program name for <programName>
%
\subsection{sddsmatrixop}
\label{sddsmatrixop}

\begin{itemize}
\item {\bf description:}
%
% Insert text of description (typicall a paragraph) here.
%
\verb+sddsmatrixop + performs general matrix operations. The matrices and 
operations are specified on the command line and the operations will
proceed in a rpn-like fashion.

String columns are ignored and not copied to the output file.

\item {\bf examples:} 
%
% Insert text of examples in this section.  Examples should be simple and
% should be preceeded by a brief description.  Wrap the commands for each
% example in the following construct:
% 
%
{\tt }C = A B  would be expressed as
\begin{flushleft}{\tt
\bf sddsmatrixop A.matrix C.matrix -push=B.matrix -multiply
}\end{flushleft}

Here A.matrix is the input matrix of the command line.
It is pushed on the "matrix" stack. In rpn, we always need
one quantity on the stack before doing any operations, so
the input file may as well be it. 
The command "push" pushes a second matrix on the stack.
The command -multiply does the multiplication of
A.matrix and B.matrix.
The matrix at the top of the stack will go in the output file
C.matrix.

{\tt }A more complicated command would be
\[{\rm Y = (1 + (A+B)C)^{-1}}\]

\begin{flushleft}{\tt
\bf sddsmatrixop A.matrix Y.matrix -push=B.matrix -add -push=C.matrix -mult -identity -add -invert
}\end{flushleft}
 where the -identity command pushes an identity matrix
with the same dimension as the top element on the stack.
{\tt }The above command will be executed as following:
\begin{flushleft}
\begin{verbatim}
command          execution                    stack (from top to bottom)   
A.matrix         push A into stack            A
                              
-push=B.matrix   push B into stack            B A

-add             pop matrix A,B from stack    temp1
                 execute: temp1=A+B
                 push temp1 into stack

-push=C.matrix   push C into stack            C temp1

-mult            pop C and temp1 from stack   temp2
                 execute: temp2=temp1*C
                 push temp2 into stack

-identity        pop temp2 from stack         I temp2
                 create unit matrix(I) that
                    has the same 
                    dimension as temp2
                 push temp2 into stack
                 push I into stack

-add             pop I and temp2 from stack   temp3
                 execute: temp3=temp2+I
                 push temp3 into stack

-invert         pop temp3 from stack          result
                execute: result = temp3^(-1)
                push result into stack.

at the end, the final result matrix is poped from the stack and writtend into output Y.matrix.

\end{verbatim}
\end{flushleft}

\item {\bf synopsis:} 
%
% Insert usage message here:
%
\begin{flushleft}{\tt
sddsmatrixop [inputmatrix] [outputmatrix] [-pipe=[in|out]] [-verbose]
   [-push=<matrix>] [-multiply]|[-add]|[-substract]|[-invert]...       
}\end{flushleft}

\item {\bf switches:}
%
% Describe the switches that are available
%
    \begin{itemize}
    \item {\tt  -pipe[=input][,output]} --- The standard SDDS Toolkit pipe option.
    \item {\tt  inputmatrix}    --- SDDS file which contains the input matrix -- the first element
                                     in the stack.
    \item {\tt  outputmatrix}   --- The result matrix is written into SDDS file named by outputmatrix.
    \item {\tt  -push=<matrix>}  --- The matrix that is going to be pushed into stack.
    \item {\tt  -verbose}        --- Write diagnostic messages to stderr.
    \item {\tt  -identity[=<number>]} --- push a unit matrix into stack. If {<number>} is provided, the unit matrix has the dimension provided by {<number>}, otherwise, the dimenstion of unit matrix is the same as
the top matrix in the stack.
    \end{itemize}
The available operations are as following:      
    \begin{itemize}
    \item {\tt  -add}       --- addition operator.
    \item {\tt  -substract}  --- substract operator.
    \item {\tt  -multiply[=hadamard]} --- matrix multiplication operator. if =hadamard is specified, the
                matrix multiplication is done element-by-element, similar to addition.
    \item {\tt  -divide=hadamard} --- element-by-element division.
    \item {\tt  -swap}   --- swap the top two elements in the stack.
    \item {\tt  -scalarmultiply=<value>}  --- multiply the matrix by a constant value.
    \item {\tt  -scalardivide=<value>}    --- divide the matrix by a constact value.
    \item {\tt  -transpose} --- matrix transpose operator.
    \item {\tt  -invert}    --- matrix inversion operator.
    \end{itemize}
the -push and operators can be repeated many times as needed.
%\item {\bf see also:}
%    \begin{itemize}
%
% Insert references to other programs by duplicating this line and 
% replacing {\em prog} with the program to be referenced:
%
%    \item \progref{<prog>}
%    \end{itemize}
%
% Insert your name and affiliation after the '}'
%
\item {\bf author: H. Shang } ANL
\end{itemize}



