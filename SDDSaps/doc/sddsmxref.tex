%\begin{latexonly} 
\newpage 
%\end{latexonly} 
\subsection{sddsmxref} 
\label{sddsmxref} 
 
\begin{itemize} 
\item {\bf description:} \hspace*{1mm}\\ 
{\tt sddsmxref} takes columns, parameters, and arrays from succesive pages from file {\em input2} and adds them to successive pages from {\em input1}. If {\em output} is given, the result is placed there; otherwise, {\em input1} is replaced. By default, all columns are taken from {\em input2}.
\item {\bf examples:} 
\begin{flushleft}
{\tt sddsmxref <input1> <input2> <output> -take=Values }
\end{flushleft} 
\item {\bf synopsis:}  
\begin{flushleft}
{\tt 
sddsmxref [{\em input1}] [{\em input2}] [{\em output}] [-pipe=[input][,output]] \\ \
[-ifis={column|parameter},{\em name}[,...]] \\ \
[-ifnot={parameter|column|array},{\em name}[,...]] \\ \
[-transfer={parameter|array},{\em name}[,...]] \\ \
[-take={\em column-name}[,...]]  \\ \
[-leave={\em column-name}[,...]] \\ \
[-fillIn] \\ \
[-match={\em column-name}[={\em column-name}][,...]] \\ \
[-equate={\em column-name}[={\em column-name}]]  \\ \
[-reuse[=[rows][,page]]] \\ \
[-rename={column|parameter|array},{\em oldname}={\em newname}[,...]] \\ \
[-editnames={column|parameter|array},{\em wildcard-string},{\em edit-string}] \\ \
[-nowarnings] \\ \
[-majorOrder=row|column]}
\end{flushleft} 
\item {\bf switches:} 
    \begin{itemize} 
    \item {\tt -pipe=[input][,output]} --- Standard SDDS pipe options for reading/writing files from stdin/stdout.
    \item {\tt -ifis} --- Specifies names of parameters, arrays, and columns that must exist in {\em input1} if the program is to run as asked.
    \item {\tt -ifnot} --- Specifies names of parameters, arrays, and columns that may not exist in {\em input1} if the program is to run as asked.
    \item {\tt -transfer} --- Specifies names of parameters or arrays to transfer from {\em input2}.
    \item {\tt -take} --- Specifies names of columns to take from {\em input2}.
    \item {\tt -leave} --- Specifies names of columns not to take from {\em input2}. Overrides -take if both name a given column. -leave=* results in no columns being taken.
    \item {\tt -fillIn} --- Specifies filling in NULL and 0 values in rows for which no match is found.  By default, such rows are omitted.
    \item {\tt -match} --- Specifies names of columns to match between {\em input1} and {\em input2} for selection and placement of data taken from {\em input2}.
    \item {\tt -equate} --- Specifies names of columns to equate between {\em input1} and {\em input2} for selection and placement of data taken from {\em input2}.
    \item {\tt -reuse} --- Specifies that rows of {\em input2} may be reused, i.e., matched with more than one row of {\em input1}.  Also, -reuse=page specifies that only the first page of {\em input2} is used.
    \item {\tt -rename} --- Specifies new names for entities in the output data set. The entities must still be referred to by their old names in the other commandline options.
    \item {\tt -editnames} --- Specifies creation of new names for entities of the specified type with names matching the specified wildcard string. Editing is performed using commands reminiscent of emacs keystrokes. if -editnames=<entity>{column|parameter|array},wildcard,ei/\%ld/ is specified, the entity names will be changed to <name>N, N is the position of input files in the command line.
    \item {\tt -nowarnings} --- Specifies that warning messages should be suppressed.
    \item {\tt -majorOrder=row|column} --- Specifies the binary SDDS layout.
\end{itemize} 

\item {\bf author:} C. Saunders, M. Borland, R. Soliday, H. Shang, ANL/APS. 
\end{itemize} 
