\begin{sddsprog}{sdds2mpl}
\item \textbf{description:}
  \verb|sdds2mpl| extracts data columns or parameters from an SDDS data set and creates \verb|mpl| data files. The
  program allows creation of \verb|mpl| labels from SDDS parameters. This tool is primarily of interest to APS
  users, some of whom still use the older {\tt mpl} Toolkit. It may be of interest to others who are interested in
  a simple format for use with programs that don't need the full power of SDDS protocol. Such applications can use
  {\tt sdds2mpl} and {\tt mpl2sdds} to mediate between themselves and SDDS-compliant programs.

\item \textbf{examples:}
  \begin{verbatim}
  sdds2mpl APS.twi -rootname=APS -output=column,z,betax -output=column,z,betay
  \end{verbatim}

\item \textbf{synopsis:}
  \begin{verbatim}
  usage: sdds2mpl [SDDSfile] [-pipe[=input]] [-rootName=string] [-separatePages]
  -output={column | parameter},xName,yName[{syName | sxName,syName}]
  [-announceOpenings] [-labelParameters=name[=format]][...]
  \end{verbatim}

\item \textbf{switches:}
  \begin{itemize}
  \item \verb|-pipe[=input]| --- The standard SDDS Toolkit pipe option.
  \item \verb|-announceOpenings| --- Requests that an informational message be printed whenever a new output file is opened.
  \item {\tt -rootName={\em string}} --- Gives the rootname for constructing output filenames.
  \item \verb|-separatePages| --- Requests that tabular-data column output from separate pages in the SDDS data set go to separate files.
  \item {\tt -labelParameters={\em name}[={\em format}][...]} --- Gives the names and optional \verb|printf| format specifications for parameters that will be printed on the title line of the \verb|mpl| files.
  \item {\tt -output\{column | parameter\},{\em xName},{\em yName}[,\{{\em syName} | {\em sxName},{\em syName}\}]} --- Requests that the named columns or parameters be put into a \verb|mpl| file or set of files. If \verb|-separate| is not given or if the data is for parameters, the name of the file is {\tt rootname\_{\em xName}\_{\em yName}.out}. For column output, if \verb|-separate| is given, the names of the files are {\tt rootname\_{\em N}\_{\em xName}\_{\em yName}.out}, where {\em N} is the page number. This option may be given any number of times.
  \end{itemize}

\item \textbf{files:}
  {\em SDDSfile} is the name of an SDDS file from which {\tt mpl}-format files will be made. Each {\tt mpl} file contains two to four columns of data.

\item \textbf{see also:}
  \begin{itemize}
  \item \hyperref[exampleData]{Data for Examples}
  \item \progref{mpl2sdds}
  \end{itemize}

\item \textbf{author:} M. Borland, ANL/APS.
\end{sddsprog}

