%\begin{latexonly}
\newpage
%\end{latexonly}
\subsection{sdds2tiff}
\label{sdds2tiff}

\begin{itemize}
\item {\bf description:} Converts an SDDS file to one or more TIFF images.  Each page of the input
file becomes a separate image named {\tt <output>.\#\#\#\#}.  Two styles of input are
supported: a single column file with {\tt Variable1Name} and {\tt Variable2Name}
parameters (plus dimension parameters), or a multi-column file containing
columns with a common prefix such as {\tt Line1}, {\tt Line2}, and so on.
\item {\bf example:}
\begin{flushleft}{\tt
sdds2tiff image.sdds image -fromPage=2 -toPage=5 -columnPrefix=Img -maxContrast
}\end{flushleft}
\item {\bf synopsis:}
\begin{flushleft}{\tt
sdds2tiff [{\em input}] [{\em output}] [-pipe[=in]] [-fromPage={\em number}]\\
  {}[-toPage={\em number}] [-columnPrefix={\em prefix}] [-maxContrast] [-16bit]
}\end{flushleft}
\item {\bf switches:}
  \begin{itemize}
  \item {\tt -pipe[=in]} --- The standard SDDS Toolkit pipe option.
  \item {\tt -fromPage={\em number}} --- Begin conversion with the specified page
    of the input file.
  \item {\tt -toPage={\em number}} --- Stop conversion after the specified page
    of the input file.
  \item {\tt -columnPrefix={\em prefix}} --- Prefix used to identify columns that
    contain image lines when the input has multiple columns.  The default is
    {\tt Line}.
  \item {\tt -maxContrast} --- Scale the output so the largest data value maps to
    the maximum gray level (255 or 65535).
  \item {\tt -16bit} --- Write 16-bit TIFF images rather than 8-bit images.
  \end{itemize}
\item {\bf author:} R. Soliday, ANL/APS.
\end{itemize}
