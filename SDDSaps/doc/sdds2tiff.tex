\begin{sddsprog}{sdds2tiff}
  \item \textbf{description:} Converts an SDDS file to one or more TIFF images. Each page of the input file becomes a separate image named \verb|<output>.####|. Two styles of input are supported: a single column file with parameters \verb|Variable1Name| and \verb|Variable2Name| (plus dimension parameters), or a multi-column file containing columns with a common prefix such as \verb|Line1|, \verb|Line2|, and so on.
  \item \textbf{examples:}
    \begin{verbatim}
    sdds2tiff image.sdds image -fromPage=2 -toPage=5 -columnPrefix=Img -maxContrast
    \end{verbatim}
  \item \textbf{synopsis:}
    \begin{verbatim}
    sdds2tiff [input] [output] [-pipe[=in]] [-fromPage=number]
      [-toPage=number] [-columnPrefix=prefix] [-maxContrast] [-16bit]
    \end{verbatim}
  \item \textbf{switches:}
    \begin{itemize}
      \item \verb|-pipe[=in]| --- The standard SDDS Toolkit pipe option.
      \item \verb|-fromPage=number| --- Begin conversion with the specified page of the input file.
      \item \verb|-toPage=number| --- Stop conversion after the specified page of the input file.
      \item \verb|-columnPrefix=prefix| --- Prefix used to identify columns that contain image lines when the input has multiple columns. The default is \verb|Line|.
      \item \verb|-maxContrast| --- Scale the output so the largest data value maps to the maximum gray level (255 or 65535).
      \item \verb|-16bit| --- Write 16-bit TIFF images rather than 8-bit images.
    \end{itemize}
  \item \textbf{see also:} \progref{tiff2sdds}.
  \item \textbf{author:} R. Soliday, ANL/APS.
\end{sddsprog}

