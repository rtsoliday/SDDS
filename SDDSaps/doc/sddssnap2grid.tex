\begin{sddsprog}{sddssnap2grid}
  \item \textbf{description:}
    \verb|sddssnap2grid| reads data pages from an SDDS file and writes a new SDDS file.
    The output data contains all of the input data, except that one or more columns may be
    modified to ``snap'' the values to a uniform grid.
  \item \textbf{examples:}
  \begin{verbatim}
sddssnap2grid fieldMap.sdds fieldMap1.sdds -column=x -column=y -column=z
sddssnap2grid fieldMap.sdds fieldMap1.sdds -column=x -column=y -column=z,deltaGuess=5e-4
  \end{verbatim}
  \item \textbf{synopsis:}
  \begin{verbatim}
sddssnap2grid [<options>] inputFile outputFile
  \end{verbatim}
  \item \textbf{switches:}
    \begin{itemize}
      \item \verb|-pipe[=input][,output]| --- The standard SDDS Toolkit pipe option.
      \item \verb!-column=name[{maximumBins=value | binFactor=value} | deltaGuess=value]! ---
        Specifies a column to be snapped to a grid. The column must contain numerical data.
        The algorithm uses histograms to group the data points into subsets; this grouping is
        considered valid if there are no adjacent bins in the histogram that have non-zero values.
        By default, the number of bins is 10 times the number of data points, which seems reliable
        if data is actually close to a grid. If the algorithm fails, the user can provide additional
        parameters to attempt to obtain a good result. The first thing to try is providing a guess of
        the grid spacing using \verb|deltaGuess|; in this case the initial number of bins is based on
        10 times the provided spacing. Next, one can try providing a value for the \verb|binFactor|
        parameter that is higher or lower than 10. Finally, the \verb|maximumBins| parameter can be set
        directly.
      \item \verb|-verbose| --- Prints grid parameters to standard output.
    \end{itemize}
  \item \textbf{files:}
    \emph{inputFile} is the name of an SDDS data set to be snapped. \emph{outputFile} is the result.
  \item \textbf{author:} M. Borland, ANL/APS.
\end{sddsprog}

