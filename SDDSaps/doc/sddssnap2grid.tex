%\begin{latexonly}
\newpage
%\end{latexonly}
\subsection{sddssnap2grid}
\label{sddssnap2grid}

\begin{itemize}
\item {\bf description:}
\verb|sddssnap2grid| reads data pages from an SDDS file and writes a new SDDS file.
The output data contains all of the input data, except that one or more columns may be
modified to ``snap'' the values to a uniform grid.
\item {\bf examples:} 

Snap x, y, and z values to a grid
\begin{flushleft}{\tt
sddssnap2grid fieldMap.sdds fieldMap1.sdds -column=x -column=y -column=z
}\end{flushleft}

Snap x, y, and z values to a grid, providing a hint for the z spacing.
\begin{flushleft}{\tt
sddssnap2grid fieldMap.sdds fieldMap1.sdds -column=x -column=y -column=z,deltaGuess=5e-4
}\end{flushleft}

\item {\bf files:}
{\em inputFile} is the name of an SDDS data set to be snapped.  {\em outputFile} is the result. 

\item {\bf switches:} 
\begin{itemize}
        \item {\tt -pipe[=input][,output]} --- The standard SDDS Toolkit pipe option.
        \item {\tt -column={\em name}[,\{maximumBins={\em value} | binFactor={\em value}\} | deltaGuess={\em value}]} 
          --- Specifies a column to be snapped to a grid. The column must contain 
          numerical data. 
          The algorithm uses histograms to group the data points into subsets; this grouping is considered valid
          if there are no adjacent bins in the histogram that have non-zero values.
          By default, the number of bins is 10 times the number of data points, which seems reliable if data
          is actually close to a grid.
          If the algorithm fails, the user can provide additional parameters to attempt to obtain a good result.
          The first thing to try is providing a guess of the grid spacing using {\tt deltaGuess}; in this
          case the initial number of bins is based on 10 times the provided spacing.
          Next, one can try providing a value for {\tt binFactor} parameter that is higher or lower than 10.
          Finally, the {\tt maximumBins} parameter can be set directly.
          \item {\tt -verbose} --- Prints grid parameters to standard output.
\end{itemize}

\item {\bf author:} M. Borland, ANL/APS.
\end{itemize}



