\begin{sddsprog}{sddsshift}
  \item \textbf{description:} \verb|sddsshift| shifts the given data columns by rows.
  \item \textbf{examples:}
\begin{verbatim}
sddsshift input.sdds output.sdds -columns=Values -shift=5 -zero
\end{verbatim}
  \item \textbf{synopsis:}
\begin{verbatim}
sddsshift [inputfile] [outputfile] [-pipe=[input][,output]]
          -columns=inputcol[,...] [-zero]
          [-shift=points | -match=matchcol]
          [-majorOrder=row|column]
\end{verbatim}
  \item \textbf{files:}
  The output file contains all the columns from the input file as well as new columns named \verb|Shifted<column>| and new parameters named \verb|<column>Shift|. Exposed end-points are set to zero if the \verb|-zero| option is provided, otherwise they are set to the value from the first or last row as appropriate.
  \item \textbf{switches:}
  \begin{itemize}
    \item \verb|-pipe=[input][,output]| --- Standard SDDS pipe options for reading/writing files from stdin/stdout.
    \item \verb|-columns=inputcol[,...]| --- The names of the columns to be shifted.
    \item \verb|-zero| --- Set exposed end-points to zero.
    \item \verb|-shift=points| --- Number of rows to shift columns. Positive and negative numbers are both allowed.
    \item \verb|-match=matchcol| --- The columns are shifted to minimize the least squares error relative to \emph{matchcol}.
    \item \verb+-majorOrder=row|column+ --- Specifies the binary SDDS layout.
  \end{itemize}
  \item \textbf{see also:}
  \begin{itemize}
    \item \progref{sddsshiftcor}
  \end{itemize}
  \item \textbf{author:} C. Saunders, M. Borland, R. Soliday, H. Shang, ANL/APS.
\end{sddsprog}
