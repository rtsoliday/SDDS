%
%\begin{latexonly}
\newpage
%\end{latexonly}
\subsection{hdf2sdds}
\label{hdf2sdds}

\begin{itemize}
\item {\bf description:} Converts Hierarchical Data Format (HDF) to SDDS.
\item {\bf example:} 
\begin{flushleft}{\tt
hdf2sdds data.hdf data.sdds -withIndex
}\end{flushleft}
\item {\bf synopsis:}
\begin{flushleft}{\tt
hdf2sdds [{\em HDFfile}] [{\em SDDSfile}] [-pipe[=out]]
[-query] [-ascii] [-binary] [-withIndex] [-reduceFactor={\em integer}[,keep={\em integer}]] [-3doutput]
}\end{flushleft}
\item {\bf files: }

{\em HDFfile} is the filename of the HDF file.

{\em SDDSfile} is the SDDS output that is created.

\item {\bf switches:}
\begin{itemize}
    \item {\tt -pipe[=out]} --- The standard SDDS Toolkit pipe option.
    \item {\tt -query} --- Print out the names of the groups and datasets in the HDF file.
    \item {\tt -ascii} --- Requests that the output be ASCII.
    \item {\tt -binary} --- Requests that the output be binary.
    \item {\tt -withIndex} --- An index column is added to the output file.
    \item {\tt -reduceFactor={\em integer}[,keep={\em integer}]} --- Write the first {\em keep}th value of every {\em reduceFactor} values in order to reduce the size of output file. The attributes are written to the output file by their original data types.
    \item {\tt -3doutput} --- Used to convert 3-dimensional HDF data.
\end{itemize}
\item {\bf author:} R. Soliday, ANL/APS.
\end{itemize}
