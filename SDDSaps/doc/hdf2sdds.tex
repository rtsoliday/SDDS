\begin{sddsprog}{hdf2sdds}
\item \textbf{description:} Converts Hierarchical Data Format (HDF) to SDDS.
\item \textbf{examples:}
\begin{verbatim}
hdf2sdds data.hdf data.sdds -withIndex
\end{verbatim}
\item \textbf{synopsis:}
\begin{verbatim}
hdf2sdds [HDFfile] [SDDSfile] [-pipe[=out]] [-query] [-ascii] [-binary]
[-withIndex] [-reduceFactor=integer[,keep=integer]] [-3doutput]
\end{verbatim}
\item \textbf{files:}
{\em HDFfile} is the filename of the HDF file.

{\em SDDSfile} is the SDDS output that is created.
\item \textbf{switches:}
\begin{itemize}
  \item \verb|-pipe[=out]| --- The standard SDDS Toolkit pipe option.
  \item \verb|-query| --- Print out the names of the groups and datasets in the HDF file.
  \item \verb|-ascii| --- Requests that the output be ASCII.
  \item \verb|-binary| --- Requests that the output be binary.
  \item \verb|-withIndex| --- An index column is added to the output file.
  \item {\tt -reduceFactor={\em integer}[,keep={\em integer}]} --- Write the first {\em keep}th value of every {\em reduceFactor} values in order to reduce the size of the output file. The attributes are written to the output file by their original data types.
  \item \verb|-3doutput| --- Used to convert 3-dimensional HDF data.
\end{itemize}
\item \textbf{see also:}
\begin{itemize}
  \item \progref{sddsconvert}
\end{itemize}
\item \textbf{author:} R. Soliday, ANL/APS.
\end{sddsprog}

