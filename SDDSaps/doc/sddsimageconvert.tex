%
%\begin{latexonly}
\newpage
%\end{latexonly}
\subsection{sddsimageconvert}
\label{sddsimageconvert}

\begin{sddsprog}{sddsimageconvert}
  \item \textbf{description:} \verb|sddsimageconvert| converts a single-column SDDS image file into a multi-column SDDS image file and vice versa.
  \item \textbf{examples:}
  \begin{verbatim}
sddsimageconvert image.input image.output
  \end{verbatim}
  \item \textbf{synopsis:}
  \begin{verbatim}
sddsimageconvert [Inputfile] [Outputfile] [-pipe[=in][,out]]
  [-ascii] [-binary]
  [-multicolumn=[indexName=name][,prefix=name]]
  [-singlecolumn=[imageColumn=name][,xVariableName=name][,yVariableName=name]]
  [-nowarnings]
  \end{verbatim}
  \item \textbf{files:}
  \emph{Inputfile} is the SDDS input image file. This file can be either a single-column image file or a multi-column image file.

  \emph{Outputfile} is a single-column SDDS image file if the \emph{Inputfile} is a multi-column SDDS image file, or it is a multi-column SDDS image file if the \emph{Inputfile} is a single-column SDDS image file.
  \item \textbf{switches:}
    \begin{itemize}
      \item {\tt -pipe[=in][,out]} --- The standard SDDS Toolkit pipe option.
      \item {\tt -ascii} --- The \emph{Outputfile} is written in ascii format.
      \item {\tt -binary} --- The \emph{Outputfile} is written in binary format.
      \item {\tt -multicolumn=[indexName=name][,prefix=name]} --- The multi-column SDDS image file will have or does have an index column with the name given by \verb|indexName|=\emph{name}. The default name is Index. It also will have or does have multiple columns with the prefix given by \verb|prefix|=\emph{name}. For an input file this defaults to the prefix of the first column found that is not the index column and that ends with a number. For an output file this defaults to the same name as the image column name in the single-column SDDS image file.
      \item {\tt -singlecolumn=[imageColumn=name][,xVariableName=name][,yVariableName=name]} --- The single-column SDDS image file will have or does have an image column with the name given by \verb|imageColumn|=\emph{name}. The default is the name of only column that exists in the single-column input file or the image prefix in the multi-column input file. If the output file is a single-column image file the \verb|xVariableName|=\emph{name} and \verb|yVariableName|=\emph{name} options will be used to define the x and y variable names. These default to x and y.
      \item {\tt -nowarnings} --- No warnings will be issued when the input file is overwritten.
    \end{itemize}
  \item \textbf{see also:}
    \begin{itemize}
      \item \progref{sddscongen}
      \item \progref{sddscontour}
      \item \progref{sddsimageprofiles}
      \item \progref{sddsspotanalysis}
    \end{itemize}
  \item \textbf{author:} R. Soliday, ANL/APS.
\end{sddsprog}

