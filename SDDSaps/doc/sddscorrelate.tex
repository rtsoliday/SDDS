\begin{sddsprog}{sddscorrelate}
\item \textbf{description:}
  {\tt sddscorrelate} computes correlation coefficients and correlation
  significance between column data. The correlation coefficient between
  columns i and j is defined as
  \[ {\rm C_{ij} = \frac{\langle (x_i-\langle x_i \rangle) (x_j-\langle x_j \rangle)\rangle}
  {\sqrt{\langle (x_i-\langle x_i \rangle)^2\rangle \langle (x_j - \langle x_j \rangle)^2 \rangle}}} \]
  If ${\rm C_{ij}=1}$, then the variables are perfectly correlated, whereas if ${\rm C_{ij}=-1}$,
  they are perfectly anticorrelated. The correlation significance is the probability that the
  observed correlation coefficient could happen by chance if the variables were in fact
  uncorrelated. Hence, a very small correlation significance means that the variables are
  probably correlated.
\item \textbf{examples:}
\begin{verbatim}
  sddscorrelate par.bpm par.cor -column='*x'
\end{verbatim}
\begin{verbatim}
  sddscorrelate par.bpm par.cor -column='*x' -withOnly=P1P1x
\end{verbatim}
\item \textbf{synopsis:}
\begin{verbatim}
  sddscorrelate [-pipe=[input][,output]] [inputFile] [outputFile]
    [-columns=columnNames] [-excludeColumns=columnNames]
    [-withOnly=columnName] [-rankOrder]
    [-stDevOutlier[=limit=factor][,passes=integer]]
\end{verbatim}
\item \textbf{files:}
  {\em inputFile} is an SDDS file containing two or more columns of data. For each page of
  the file, {\em outputFile} contains the correlation coefficients and significance for
  every possible pairing of variables requested. {\em outputFile} also contains three string
  columns: {\tt Correlate1Name}, {\tt Correlate2Name}, and {\tt CorrelatePair}. These are
  respectively the name first column in the analysis, the name of the second column in
  the analysis, and a string of the form {\em Name1}.{\em Name2}.
\item \textbf{switches:}
  \begin{itemize}
  \item {\tt -pipe=[input][,output]} --- The standard SDDS Toolkit pipe option.
  \item {\tt -columns={\em columnNames}} --- Specifies the names of columns to be included in the analysis.
    A comma-separated list of optionally wildcard-containing names may be given.
  \item {\tt -excludeColumns={\em columnNames}} --- Specifies the names of columns to be excluded from the
    analysis. A comma-separated list of optionally wildcard-containing names may be given.
  \item {\tt -withOnly={\em columnName}} --- Specifies that one of the variables for each correlation will be
    the named column.
  \item {\tt -rankOrder} --- Specifies computing rank-order correlations rather than standard correlations.
    This is considered more robust than standard correlations.
  \item {\tt -stDevOutlier[=limit={\em factor}][,passes={\em integer}]} --- Specifies standard-deviation-based
    outlier elimination on each pair of columns prior to computation of the correlation coefficient.
    Any pair of values is ignored if one or both values are outliers relative to the column from which they come.
    The {\tt limit} qualifier specifies the allowed deviation from the mean in standard deviations; the
    default is 1. The {\tt passes} qualifier specifies how many times the outlier elimination (including
    recomputation of the mean and standard deviation) is performed; the default is 1.
  \end{itemize}
\item \textbf{see also:}
  \begin{itemize}
  \item \hyperref[exampleData]{Data for Examples}
  \item \progref{sddsprocess}
  \end{itemize}
\item \textbf{author:} M. Borland, ANL/APS.
\end{sddsprog}

