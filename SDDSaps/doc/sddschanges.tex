%\begin{latexonly}
\newpage
%\end{latexonly}
\subsection{sddschanges}
\label{sddschanges}

\begin{sddsprog}{sddschanges}
  \item \textbf{description:}
    \verb|sddschanges| analyzes changes in column data from page to page in a file, relative to reference data in a baseline file or from the first page. It requires that every page in the file have the same number of rows. It produces a multipage output file containing the row-by-row difference between the reference data and the data on each page in the input file.

  \item \textbf{examples:}
  Compute the changes in the dispersion function for several APS lattices:
  \begin{verbatim}
sddschanges APS.twi APS.changes -copy=s -changesIn=betax,betay,etax
  \end{verbatim}
  The output file in this example would have one fewer pages than the input file. Each page would contain the column s from the first page, along with the differences from the first page for betax, betay, and etax. One could also compute the changes relative to the nominal lattice:
  \begin{verbatim}
sddschanges APS.twi -baseline=APS0.twi APS.changes -copy=s -changesIn=betax,betay,etax
  \end{verbatim}
  The output file would have one page for every page in the input.

  \item \textbf{synopsis:}
  \begin{verbatim}
sddschanges [-pipe[=input][,output]] [inputFile] [outputFile]
            [-copy=columnNames] [-changesIn=columnNames]
            [-baseline=referenceFileName [-parallelPages]]
  \end{verbatim}

  \item \textbf{switches:}
    \begin{itemize}
      \item \verb|-pipe=[input][,output]| --- The standard SDDS Toolkit pipe option.
      \item \verb|-copy=columnNames| --- Specifies that the named columns should be transferred to the output file without alteration. These data come from the baseline file or from the first page of the input file. A comma-separated list of optionally wildcard-containing strings may be given.
      \item \verb|-changesIn=columnNames| --- Specifies that the named columns should be transferred to the output file after subtracting the corresponding values from the baseline file or from the first page of the input file. A comma-separated list of optionally wildcard-containing strings may be given.
      \item \verb|-baseline=referenceFileName| --- Specifies the name of an SDDS file from which the reference data for changes should be taken.
      \item \verb|-parallelPages| --- Valid only with \verb|-baseline|. Specifies that the baseline data for each page of the input file shall be taken from the corresponding page of the baseline file. This results in page-by-page subtraction of the two files.
    \end{itemize}

  \item \textbf{files:}
    \emph{inputFile} is a multipage file containing the data for which changes are desired. \emph{outputFile} is a multipage file containing the changes. The column names in \emph{outputFile} for the changes are created from those in \emph{inputFile} by prepending the string \verb|ChangeIn|.

  \item \textbf{see also:}
    \begin{itemize}
      \item \hyperref[exampleData]{Data for Examples}
      \item \progref{sddsenvelope}
    \end{itemize}

  \item \textbf{author:} M. Borland, ANL/APS.
\end{sddsprog}
