%\begin{latexonly}
\newpage
%\end{latexonly}
\subsection{sddschanges}
\label{sddschanges}

\begin{itemize}
\item {\bf description:} {\tt sddschanges} analyzes changes in column data from page to page in a file,
relative to reference data in a baseline file or from the first page.  It requires that every page in the file have
the same number of rows.  It produces a multipage output file containing the row-by-row difference between the
reference data and the data each page in the input file.

\item {\bf examples:} 
Compute the changes in the dispersion function for several APS lattices:
\begin{flushleft}{\tt
sddschanges APS.twi APS.changes -copy=s -changesIn=betax,betay,etax
}\end{flushleft}
The output file in this example would have one fewer pages than the input
file.  Each page would contain the column s from the first page, along with
the differences from the first page for betax, betay, and etax.
One could also compute the changes relative to the nominal lattice:
\begin{flushleft}{\tt
sddschanges APS.twi -baseline=APS0.twi APS.changes -copy=s -changeIn=betax,betay,etax
}\end{flushleft}
The output file would have one page for every page in the input.
\item {\bf synopsis:}
\begin{flushleft}{\tt
sddschanges [-pipe[=input][,output]] [{\em inputFile}] [{\em outputFile}]
[-copy={\em columnNames}] [-changesIn={\em columnNames}]
[-baseline={\em referenceFileName} [-parallelPages]] 
}\end{flushleft}
\item {\bf files:}
      {\em inputFile} is a multipage file containing the data for which changes are
      desired.  {\em outputFile} is a multipage file containing the changes.  The
      column names in {\em outputFile} for the changes are created from those in
      {\em inputFile} by prepending the string ``ChangeIn''.
\item {\bf switches:}
    \begin{itemize}
    \item {\tt -pipe=[input][,output]} --- The standard SDDS Toolkit pipe option.
    \item {\tt -copy={\em columnNames}}--- Specifies that the named columns should be transferred
        to the output file without alteration.  These data come from the baseline
        file or from the first page of the input file.  A comma-separated list of optionally wildcard-containing
        strings may be given.
    \item {\tt -changesIn={\em columnNames}} --- Specifies that the named columns should be
        transferred to the output file after subtracting the corresponding values
        from the baseline file or from the first page of the input file.
        A comma-separated list of optionally wildcard-containing
        strings may be given.
    \item {\tt -baseline={\em referenceFileName}} --- Specifies the name of an SDDS file from which
        the reference data for changes should be taken.
    \item {\tt -parallelPages} --- Valid only with {\tt -baseline}.   Specifies that the ``baseline''
        data for each page of the input file shall be taken from the corresponding page of the
        baseline file.  This results in page-by-page subtraction of the two files.
    \end{itemize}
\item {\bf see also:}
    \begin{itemize}
    \item \hyperref[exampleData]{Data for Examples}
    \item \progref{sddsenvelope}
    \end{itemize}
\item {\bf author:} M. Borland, ANL/APS.
\end{itemize}

