\begin{sddsprog}{sddstimeconvert}
  \item \textbf{description:}
  \verb|sddstimeconvert| does conversions between calendar time in terms of (for example) day, month, and year, and ``time-since-epoch''. The latter is the time in seconds since a system-defined reference time (e.g., 0:00 on January 1, 1970). The hour data as used or created by \verb|sddstimeconvert| contains the floating-point time-of-day in hours. That is, the minutes and seconds are folded into the hour value. Year values must be the full four-digit year; e.g., year 99 is not 1999, but rather 99 AD.

  \item \textbf{examples:}
  \begin{verbatim}
  sddstimeconvert input.sdds output.sdds
    -epoch=column,Time,year=TheYear,month=TheMonth,day=DayOfMonth,hour=HourOfDay
  \end{verbatim}

  \item \textbf{synopsis:}
  \begin{verbatim}
  sddstimeconvert [inputFile] [outputFile] [-pipe[=input][,output]]
    -breakdown={column | parameter},timeName[,year=newName]
      [,julianDay=newName][,month=newName][,day=newName][,hour=newName][,text=newName]
    -epoch={column | parameter},newName,year=name
      [,julianDay=name | month=name,day=name],hour=name
  \end{verbatim}

  \item \textbf{switches:}
  \begin{itemize}
    \item \verb|-pipe[=input][,output]| --- The standard SDDS Toolkit pipe option.
    \item \verb!-breakdown={column | parameter},timeName[,year=newName][,julianDay=newName][,month=newName][,day=newName][,hour=newName][,text=newName]! --- Specifies conversion of the column or parameter data named \emph{timeName} to year, Julian day, month, day, hour, and/or a text string. \emph{timeName} contains the time expressed as seconds-since-epoch. Any number of these options may be given.
    \item \verb!-epoch={column | parameter},newName,year=name[,julianDay=name | month=name,day=name],hour=name! --- Specifies conversion of column or parameter data given as year, Julian day or month/day, and hour to seconds-since-epoch, with the result being placed in \emph{newName}.
  \end{itemize}

  \item \textbf{author:} M. Borland, ANL/APS.
\end{sddsprog}

