\begin{sddsprog}{sddscheck}
  \item \textbf{description:} \verb|sddscheck| is a simple tool to allow checking a file to see if it is a valid SDDS file or if it is corrupted. The primary use is in shell scripts that need to detect such conditions. \verb|sddscheck| issues one of four messages: \verb|ok|, \verb|nonexistent|, \verb|badHeader|, or \verb|corrupted|. (See \progref{sddsconvert} about recovering corrupted files.)
  \item \textbf{examples:}
    \begin{verbatim}
if (`sddscheck APS.twi` == "ok") plotTwissParameters APS.twi
    \end{verbatim}
    where \verb|plotTwissParameters| is a hypothetical plotting script.
  \item \textbf{synopsis:}
    \begin{verbatim}
sddscheck [-printErrors] filename
    \end{verbatim}
  \item \textbf{switches:}
    \begin{itemize}
      \item \verb|-printErrors| --- Causes the SDDS error traceback to be printed if the file is not \verb|ok|. This may be helpful in determining the problem with the file.
    \end{itemize}
  \item \textbf{files:} \emph{filename} is the name of a single file to be checked.
  \item \textbf{see also:} \progref{sddsconvert}
  \item \textbf{author:} M. Borland, ANL/APS.
\end{sddsprog}

