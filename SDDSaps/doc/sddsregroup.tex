% $:Log:$
%
% Template for making SDDS Toolkit manual entries.
%
%\begin{latexonly}
\newpage
%\end{latexonly}

%
% Substitute the program name for <programName>
%
\subsection{sddsregroup}
\label{sddsregroup}

\begin{itemize}
\item {\bf description:}
%
% Insert text of description (typicall a paragraph) here.
%
\verb|sddsregroup| swaps the row indexing and page indexing of data
in an SDDS file. That is, the ${\rm i {th}}$ row of all data pages in the input file are collected
and made into the ${\rm i {th}}$ data page of the output file.
\item {\bf examples:}
%
% Insert text of examples in this section.  Examples should be simple and
% should be preceeded by a brief description.  Wrap the commands for each
% example in the following construct:
% 
%
The file bpm.sdds contain the beam position monitor (bpm) readback as a function of time for a series
of consecutive bpms in a beamline. The defined columns are Time and x. The parameters
are bpmIndex. The file bpm.sdds is regrouped to produce data sets of x vs bpmIndex
for each time value. The output is suitable to plot as a movie with \verb|sddsplot|.
\begin{flushleft}{\tt
sddsregroup bpm.sdds bpm.movie -newParameters=Time -newColumns=bpmIndex
}\end{flushleft}
\item {\bf synopsis:} 
%
% Insert usage message here:
%
\begin{flushleft}{\tt
sddsregroup [-pipe=[input][,output]] {\em inputFile} {\em outputFile}
     [-newParameters={\em oldColumnName},...]
     [-newColumns={\em oldParameterName},...] [-warning] [-verbose]
}\end{flushleft}
\item {\bf files:}
% Describe the files that are used and produced
The input file contains the data sets to be regrouped. The output file
contains the regrouped data.  If only one file is specified,
then the input file is overwritten by the output.
\item {\bf switches:}
%
% Describe the switches that are available
%
    \begin{itemize}
    \item {\tt  -pipe[=input][,output]} --- The standard SDDS Toolkit pipe option.
        \item {\tt -newParameters} --- specifies which columns of the input file will become
                    parameters in the output file. By default no new parameters
                    are created, and all columns of the input file are transfered
                    to the output file.
        \item {\tt -newColumns} --- specifies which parameters of the input file will become
                    columns in the output file. The columns will necessarily be
                    duplicated in all pages. By default all parameters
                    values are lost.
    \end{itemize}
%\item {\bf see also:}
%    \begin{itemize}
%
% Insert references to other programs by duplicating this line and 
% replacing {\em prog} with the program to be referenced:
%
%    \item \progref{<prog>}
%    \end{itemize}
%
% Insert your name and affiliation after the '}'
%
\item {\bf author: L. Emery } ANL
\end{itemize}

