\begin{sddsprog}{sddspeakfind}
  \item \textbf{description:}
  \verb|sddspeakfind| finds the locations and values of peaks in a single column of an SDDS file.  It incorporates
  various features to help reject spurious peaks.  The column is considered a function of the row index for the
  purpose of finding peaks.  Hence, the data should be sorted if necessary using \verb|sddssort| prior to using this
  program.  I.e., if the data contains columns x and y, and one wants x values of peaks in y, then one should
  ensure that the rows are sorted into increasing or decreasing x order.

  It may also be helpful to smooth the data using \verb|sddssmooth| in order to eliminate spurious peaks due to
  noisy data.
  \item \textbf{examples:}
    Find peaks in a Fourier transform:
    \begin{verbatim}
sddspeakfind data.fft data.peaks -column=FFTamplitude
    \end{verbatim}
    Sort and smooth the data first:
    \begin{verbatim}
sddssort data.fft -column=f,increasing -pipe=out |
  sddssmooth -pipe -columns=FFTamplitude |
  sddspeakfind -pipe=in data.peaks -column=FFTamplitude
    \end{verbatim}
  \item \textbf{synopsis:}
    \begin{verbatim}
sddspeakfind [-pipe=[input][,output]] [inputFile] [outputFile]
  -column=columnName [-fivePoints] [-threshold=value]
  [-exclusionZone=fractionalInterval] [-changeThreshold=fractionalChange]
    \end{verbatim}
  \item \textbf{files:}
  \emph{inputFile} contains the data to be searched for peaks.  \emph{outputFile} contains all of the array and
  parameter data from \emph{inputFile}, plus data from all rows that contain a peak in the named column.  No new
  data elements are created.  If \emph{inputFile} contains multiple pages, each is treated separately and is
  delivered to a separate page of \emph{outputFile}.
  \item \textbf{switches:}
    \begin{itemize}
      \item \verb|-pipe[=input][,output]| --- The standard SDDS Toolkit pipe option.
      \item {\tt -column={\em columnName}} --- Specifies the name of the column to search for peaks.
      \item \verb|-fivePoints| --- Specifies peak analysis using five adjacent data points, rather than
        the default three.  For three-point mode, a peak is any point which is larger than both of
        its two nearest neighbors.  For five-point mode, the candidate point's nearest neighbors must in turn
        be higher than their nearest neighbors on the side away from the candidate point.
      \item {\tt -threshold={\em value}} --- Specifies a minimum value that a peak value must exceed in order
        to be included in the output.  By default, no threshold is applied.
      \item {\tt -exclusionZone={\em fractionalInterval}} --- Specifies elimination of smaller peaks within a given interval
        around a larger peak.  {\em fractionalInterval} is the width of the interval in units of the length of the data table.
      \item {\tt -changeThreshold={\em fractionalChange}} --- Specifies elimination of peaks for which the fractional
        change between the peak value and the nearest neighbor points is less than the given amount.  If
        \verb|-fivePoints| is given, the nearest neighbors in question are those 2 rows above and below the
        peak.
    \end{itemize}
  \item \textbf{see also:}
    \begin{itemize}
      \item \progref{sddsfft}
      \item \progref{sddssmooth}
    \end{itemize}
  \item \textbf{author:} M. Borland, ANL/APS.
\end{sddsprog}

