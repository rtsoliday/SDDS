\begin{sddsprog}{sddsbaseline}
  \item \textbf{description:}
    \verb|sddsbaseline| performs baseline removal for SDDS column data. Several methods of
    determining the baseline are provided.
  \item \textbf{examples:}
  \begin{verbatim}
sddsbaseline image.sdds image1.sdds -columns=VideoLine* -select=endpoints=10 -method=average
  \end{verbatim}
  \item \textbf{synopsis:}
  \begin{verbatim}
sddsbaseline [input] [output] [-pipe=[in][,out]]
  [-columns=listOfNames]
  -select={endPoints=number | -outsideFWHA=multiplier | -antiOutlier=passes}
  -method={fit | average}
  [-nonnegative [-despike=passes=number,widthlimit=value] [-repeats=count]]
  \end{verbatim}
  \item \textbf{switches:}
    \begin{itemize}
      \item \verb|-pipe=[input][,output]| --- The standard SDDS Toolkit pipe option.
      \item \verb|-columns=listOfNames| --- Specifies an optionally-wildcarded list
        of names of columns from which to remove baselines.
      \item \verb!-select={endPoints=number | -outsideFWHA=multiplier | -antiOutlier=passes}! --- Specifies how
        to select the points from which to determine the baseline. \verb|endPoints| specifies selecting \emph{number} values from the start and end of the column. \verb|outsideFWHA| selects all values that are outside \emph{multiplier} times the full-width-at-half-amplitude (FWHA) of the pixel count distribution. \verb|antiOutlier| selects all values that are \emph{not} deemed outliers in the 2-sigma sense in any of \emph{passes} inspections. These last two options implicitly assume that the statistical distribution of the pixel counts is baseline dominated.
      \item \verb!-method={fit | average}! --- Specifies how to compute the baseline from the selected
        points. \verb|fit| specifies fitting a line to the values (as a function of index). \verb|average| specifies taking a simple average of the values.
      \item \verb|-nonnegative [-despike=passes=number,widthlimit=value] [-repeats=count]| --- Specifies that the
        resulting function after baseline removal must be nonnegative. Any negative values are set to 0. In addition, despiking (as in \verb|sddssmooth|) may be applied after removal of negative values; this can result in the removal of positive noise spikes. Giving \verb|-repeats| allows applying the baseline removal procedure iteratively to the data.
    \end{itemize}
  \item \textbf{files:}
    \emph{input} is an SDDS file containing one or more pages of data to be processed.
    \emph{output} is an SDDS file in which the result is placed. Columns that are not processed are copied from \emph{input} to \emph{output} without change.
  \item \textbf{see also:}
    \begin{itemize}
      \item \progref{sddssmooth}
      \item \progref{sddscliptails}
    \end{itemize}
  \item \textbf{author:} M. Borland, ANL/APS.
\end{sddsprog}
