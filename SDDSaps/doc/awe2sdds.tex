%\begin{latexonly}
\newpage
%\end{latexonly}
\subsection{awe2sdds}
\label{awe2sdds}

\begin{itemize}
\item {\bf description:} 
Converts a file in \verb|awe| self-describing format to SDDS.  This is of interest to
only a few users at APS, as \verb|awe| format has been superseeded by SDDS and is rarely used.
\item {\bf example:} 
To convert {\tt awe} format Twiss parameter data from an old version of \verb|elegant|:
\begin{flushleft}{\tt 
awe2sdds APS.awe APS.sdds -labelColumnName=ElementName
}\end{flushleft}
\item {\bf synopsis:}
\begin{flushleft}{\tt 
awe2sdds {\em inputFile} {\em outputFile} [-labelColumnName={\em string}] [-asciiOutput]
}\end{flushleft}
\item {\bf files:} 
{\em inputFile} is an {\tt awe}-format file, the SDDS equivalent of which is written to {\em outputFile}.
The ``auxiliary values'' of the {\tt awe} file are converted into SDDS parameters.  The {\tt awe} tables
are converted into SDDS tabular data, all columns being double precision except the ``row label'', which
becomes a string column.
\item {\bf switches:}
    \begin{itemize}
    \item {\tt -labelColumnName={\em string}} --- Requests that the {\tt awe} row label be given the name {\em string}.
        By default, the row label is placed in a column named ``row-label''.
    \item \verb|-asciiOutput| --- Requests that output be in ASCII.  By default, the output is binary.
    \end{itemize}
\item {\bf author:} M. Borland, ANL/APS.
\end{itemize}

