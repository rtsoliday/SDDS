%\begin{latexonly}
\newpage
%\end{latexonly}
\subsection{awe2sdds}
\label{awe2sdds}

\begin{sddsprog}{awe2sdds}
  \item \textbf{description:} Converts a file in \verb|awe| self-describing format to SDDS. This is of interest to only a few users at APS, as \verb|awe| format has been superseded by SDDS and is rarely used.
  \item \textbf{examples:}
    \begin{verbatim}
awe2sdds APS.awe APS.sdds -labelColumnName=ElementName
    \end{verbatim}
  \item \textbf{synopsis:}
    \begin{verbatim}
awe2sdds inputFile outputFile [-labelColumnName=string] [-asciiOutput]
    \end{verbatim}
  \item \textbf{switches:}
    \begin{itemize}
      \item \verb|-labelColumnName=string| --- Requests that the \verb|awe| row label be given the name string. By default, the row label is placed in a column named ``row-label''.
      \item \verb|-asciiOutput| --- Requests that output be in ASCII. By default, output is binary.
    \end{itemize}
  \item \textbf{files:} \emph{inputFile} is an \verb|awe|-format file, the SDDS equivalent of which is written to \emph{outputFile}. The ``auxiliary values'' of the \verb|awe| file are converted into SDDS parameters. The \verb|awe| tables are converted into SDDS tabular data, all columns being double precision except the ``row label'', which becomes a string column.
  \item \textbf{author:} M. Borland, ANL/APS.
\end{sddsprog}

