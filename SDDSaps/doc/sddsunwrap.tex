\begin{sddsprog}{sddsunwrap}
  \item \textbf{description:}
    \verb|sddsunwrap| looks for discontinuities larger than a threshold in a set of data.
    After each discontinuity it adds the appropriate multiple of the modulo to the data set.
  \item \textbf{examples:}
    \begin{verbatim}
    sddsunwrap inputFile outputFile -columns=phase
    \end{verbatim}
  \item \textbf{synopsis:}
    \begin{verbatim}
    sddsunwrap [-pipe=[input][,output]] inputFile outputFile
      [-columns=columnName[,...]] [-threshold=value] [-modulo=value]
    \end{verbatim}
  \item \textbf{switches:}
    \begin{itemize}
      \item \verb|-pipe[=input][,output]| --- The standard SDDS Toolkit pipe option.
      \item \verb|-threshold=\emph{value}| --- Specifies the discontinuity threshold used to identify a wrap in the data, default is PI.
      \item \verb|-modulo=\emph{value}| --- Specifies the value used to unwrap the data, default is 2*PI.
      \item \verb|-columns=\emph{columnName}[,\emph{columnName}...]| --- Specifies the names of the columns to unwrap; wildcards are accepted. If not specified, all numerical columns in the input file are unwrapped. The output column is named as \verb|Unwrap<inputColumn>|.
    \end{itemize}
  \item \textbf{files:}
    The input file contains the column data to be unwrapped. The unwrapped data is saved into \verb|Unwrap<inputColumn>| column in the output file.
  \item \textbf{author:} H. Shang, ANL/APS.
\end{sddsprog}

