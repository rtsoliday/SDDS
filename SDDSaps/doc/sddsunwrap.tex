%\begin{latexonly} 
\newpage 
%\end{latexonly} 
\subsection{sddsunwrap} 
\label{sddsunwrap} 
 
\begin{itemize} 
\item {\bf description:} \hspace*{1mm}\\ 
{\tt sddsunwrap} looks for discontinuities of size $>$ threshold in a set of data. After each discontinuity it adds the appropriate multiple of the modulo to the data set.
\item {\bf examples:} 
\begin{flushleft}{\tt
   sddsunwrap <inputFile> <outputFile> -col=phase
}\end{flushleft} 
\item {\bf synopsis:}  
\begin{flushleft}{\tt 
sddsunwrap [-pipe=[input][,output]] {\em inputFile} {\em outputFile}
      [-column=list of columns] [-threshold=<value>] [-modulo<value>]
}\end{flushleft} 
\item {\bf files:} 
The input file contains the column data to be unwrapped. The unwrapped data is saved into Unwrap<inputColumn> column in the output file.
\item {\bf switches:} 
    \begin{itemize} 
    \item {\tt -pipe[=input]} --- The standard SDDS Toolkit pipe option. 
    \item {\tt -threshold={\em value} } --- Specifies the discontinuity threshold used to identify a wrap in the data, default is PI..
    \item {\tt -modulo={\em value}} --- Specifies the value used to unwrap the data, default is 2*PI.
    \item {\tt -columns={\em columnName}[,{\em columnName}...]} ---  
        Specifies the names of the columns to be unwrapped, separated by comma, and accept wild cards. If not specified, all numerical
        columns in the input file will be unwrapped. The output column is named as Unwrap<inputColumn>.
    \end{itemize} 

\item {\bf author:} H. Shang, ANL/APS. 
\end{itemize} 
 
