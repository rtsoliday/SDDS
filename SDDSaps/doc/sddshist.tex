\begin{sddsprog}{sddshist}
  \item \textbf{description:}
    \verb|sddshist| does weighted and unweighted one-dimensional histograms of column data from an SDDS file. It also does limited statistical analysis of data, and basic filtering of data.
  \item \textbf{examples:}
    Make a 20-bin histogram of a series of PAR x beam-position-monitor readouts:
    \begin{verbatim}
sddshist par.bpm par.bpmhis -data=P1P1x -bins=20
    \end{verbatim}
  \item \textbf{synopsis:}
    \begin{verbatim}
sddshist [-pipe=[input][,output]] [inputFile] [outputFile] \
  -dataColumn=columnName [-bins=number | -sizeOfBins=value] \
  [-lowerLimit=value] [-upperLimit=value] [-filter=columnName,lowerLimit,upperLimit] \
  [-weightColumn=columnName] [-sides] [-normalize={sum | area | peak}] \
  [-statistics] [-verbose]
    \end{verbatim}
  \item \textbf{files:}
    \emph{inputFile} is the name of an SDDS file containing data to be histogrammed, along with optional weight data. If \emph{inputFile} contains multiple data pages, each is treated separately. The histogram or histograms are placed in \emph{outputFile}, which has two columns. One column has the same name as the histogrammed variable, and consists of equispaced values giving the centers of the bins. The other column, named \verb|frequency|, contains the histogram frequencies. Its precise meaning is dependent on normalization modes and weighting. By default, it contains the number of data points in the corresponding bin.

    If requested, \emph{outputFile} will also contain parameters giving statistics for the data being histogrammed. See below for details.
  \item \textbf{switches:}
    \begin{itemize}
      \item \verb|-pipe[=input][,output]| --- The standard SDDS Toolkit pipe option.
      \item \verb|-dataColumn=columnName| --- Specifies the name of the data column to be histogrammed.
      \item \verb|-bins=number| --- Specifies the number of bins to use. The default is 20.
      \item \verb|-sizeOfBins=value| --- Specifies the size of bins to use. The number of bins is computed from the range of the data.
      \item \verb|-lowerLimit=value| --- Specifies the lower limit of the histogram. By default, the lower limit is the minimum value in the data.
      \item \verb|-upperLimit=value| --- Specifies the upper limit of the histogram. By default, the upper limit is the maximum value in the data.
      \item \verb|-filter=columnName,lowerLimit,upperLimit| --- Specifies the name of a column by which to filter the input rows. Rows for which the named data is outside the specified interval are discarded. Alternatively, one can use \progref{sddsprocess} to winnow data and pipe it into \verb|sddshist|.
      \item \verb|-weightColumn=columnName| --- Specifies the name of a column by which to weight the histogram. This means that data points with a higher corresponding weight value are counted proportionally more times in the histogram.
      \item \verb|-sides| --- Specifies that zero-height bins should be attached to the lower and upper ends of the histogram. Many prefer the way this looks on a graph.
      \item \verb!-normalize[={sum | area | peak}]! --- Specifies that the histogram should be normalized, and how. The default is \verb|sum|. \verb|sum| normalization means that the sum of the heights will be 1. \verb|area| normalization means that the area under the histogram will be 1. \verb|peak| normalization means that the maximum height will be 1.
      \item \verb|-statistics| --- Specifies that statistics should be computed for the data and placed in \emph{outputFile}. These presently include arithmetic mean, rms, and standard deviation. The parameters are named by appending the strings \verb|Mean|, \verb|RMS|, and \verb|StDev| to the name of the data column. If \verb|-weightColumn| is given, the statistics are weighted.
    \end{itemize}
  \item \textbf{see also:}
    \begin{itemize}
      \item \hyperref[exampleData]{Data for Examples}
      \item \progref{sddshist2d}
      \item \progref{sddsprocess}
    \end{itemize}
  \item \textbf{author:} M. Borland, ANL/APS.
\end{sddsprog}

