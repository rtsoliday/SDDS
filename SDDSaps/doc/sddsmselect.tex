%\begin{latexonly} 
\newpage 
%\end{latexonly} 
\subsection{sddsmselect} 
\label{sddsmselect} 
 
\begin{itemize} 
\item {\bf description:} \hspace*{1mm}\\ 
{\tt sddsmselect} selects data from {\em input1} for writing to {\em output} based on the presence or absence of matching data in {\em input2}. If {\em output} is not given, {\em input1} is replaced.
\item {\bf examples:} 
\begin{flushleft}
{\tt sddsmselect <input1> <input2> <output> -match=ControlName }
\end{flushleft} 
\item {\bf synopsis:}  
\begin{flushleft}
{\tt 
sddsmselect [{\em input1}] [{\em input2}] [{\em output}] [-pipe=[input][,output]] \\ \
[-match={\em column-name}[={\em column-name}][,...]] \\ \
[-equate={\em column-name}[={\em column-name}][,...]] \\ \
[-invert] \\ \
[-reuse[=[rows][,page]]] \\ \
[-nowarnings] \\ \
[-majorOrder=row|column]}
\end{flushleft} 
\item {\bf switches:} 
    \begin{itemize} 
    \item {\tt -pipe=[input][,output]} --- Standard SDDS pipe options for reading/writing files from stdin/stdout.
    \item {\tt -match} --- Specifies names of columns to match between {\em input1} and {\em input2} for selection and placement of data taken from {\em input1}.
    \item {\tt -equate} --- Specifies names of columns to equate between {\em input1} and {\em input2} for selection and placement of data taken from {\em input1}.
    \item {\tt -invert} --- Specifies that only non-matched rows are to be kept..
    \item {\tt -reuse} --- Specifies that rows of {\em input2} may be reused, i.e., matched with more than one row of {\em input1}.  Also, -reuse=page specifies that only the first page of {\em input2} is used.
    \item {\tt -nowarnings} --- Specifies that warning messages should be suppressed.
    \item {\tt -majorOrder=row|column} --- Specifies the binary SDDS layout.
\end{itemize} 

\item {\bf author:} C. Saunders, M. Borland, R. Soliday, H. Shang, ANL/APS. 
\end{itemize} 
