%\begin{latexonly}
\newpage
%\end{latexonly}
\subsection{sddsderiv}
\label{sddsderiv}

\begin{itemize}
\item {\bf description:}
{\tt sddsderiv} differentiates one or more columns of data as a function of a common
column.  The program will perform error propagation if error bars are provided in 
the data set.
\item {\bf examples:} 
Find the derivatives of columns {\tt J0} and {\tt J1} as a function of {\tt z}:
\begin{flushleft}{\tt
sddsderiv bessel.sdds bessel.deriv -differentiate=J0,J1 -versus=z
}\end{flushleft}
\item {\bf synopsis:} 
\begin{flushleft}{\tt
sddsderiv [-pipe=[input][,output]] [{\em input}] [{\em output}]
-differentiate={\em columnName}[,{\em sigmaName}] ...
-versus={\em columnName}[,{\em sigmaName}] [-interval={\em integer}]
[-SavitzkyGolay={\em left},{\em right},{\em fitOrder}[,{\em derivOrder}]]
[-mainTemplates={\em item}={\em string}[,...]] 
[-errorTemplates={\em item}={\em string}[,...]] 
}\end{flushleft}
\item {\bf files:}
{\em input} is an SDDS file containing columns of data to be
differentiated.  If it contains multiple data pages, each is treated
separately. The independent quantity along with the requested derivatives
are placed in columns in {\em output}.  By default, the
derivative column name is constructed by appending {\tt Deriv} to the
variable column name.  If applicable, the column name for the
derivative error is constructed by appending {\tt DerivSigma}.  The data with
respect to which the derivative is taken should be monotonically ordered.

\item {\bf switches:}
    \begin{itemize}
    \item {\tt -pipe[=input][,output]} --- The standard SDDS Toolkit pipe option.
    \item {\tt -differentiate={\em columnName}[,{\em sigmaName}} --- Specifies the name
        of a column to differentiate, and optionally the name of the column containing
        the error in the differentiated quantity.  May be given any number of times.
    \item {\tt -versus={\em columnName}[,{\em sigmaName}} --- Specifies the name
        of the independent variable column, and optionally the name of the column
        containing its error.
    \item {\tt -interval={\em integer}} --- Specifies the spacing of the data points
        used to approximate the derivative.  The default value of 2 specifies that
        the derivative for each point will be obtained from values 1 row above and
        1 row below the point.  In general (ignoring end points, which require
        special treatment):
        \[
        \frac{d y}{d x}[i] \approx \frac{y[i+Interval/2] - y[i-Interval/2]}{x[i+Interval/2] - x[i-Interval/2]}
        \]
    \item {\tt [-SavitzkyGolay={\em left},{\em right},{\em fitOrder}[,{\em derivOrder}]]} --- 
        Specifies using a Savitzky-Golay smoothing filter to perform the derivative,
        which involves fitting a polynomial of {\em fitOrder} through {\em left}+{\em right}+1
        points and then giving the derivative of the fitted curve.  {\em derivOrder} is 1
        by default and gives the order of derivative to take.
    \item {\tt -mainTemplates={\em item}={\em string}[,...]} --- Specifies template
        strings for  names and definition entries for the derivative columns
         in the output file.  {\em item} may be one of {\tt name}, {\tt description},
        {\tt symbol}.  The symbols ``\%x'' and ``\%y'' are used to represent
        the independent variable name and the name of the differentiated quantity, respectively.
    \item {\tt -errorTemplates={\em item}={\em string}[,...]} --- Specifies template
        strings for  names and definition entries for the derivative error columns
        in the output file.  {\em item} may be one of {\tt name}, {\tt description},
        the independent variable name and the name of the differentiated quantity, respectively.
    \end{itemize}
\item {\bf see also:}
    \begin{itemize}
    \item \progref{sddsinteg}
    \end{itemize}
\item {\bf author:} M. Borland, ANL/APS.
\end{itemize}

