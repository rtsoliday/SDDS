\begin{sddsprog}{sddsnaff}
  \item \textbf{description:}
    \verb|sddsnaff| is an implementation of Laskar's Numerical Analysis of Fundamental
    Frequencies (NAFF) algorithm. This algorithm determines the frequency components
    of a signal more accurately than Fast Fourier Transforms (FFT). FFTs are used as
    part of the analysis, so if an FFT is sufficient for an application, \progref{sddsfft}
    should be used as it will be much faster.
    The algorithm starts by removing the average value of the signal and applying a
    Hanning window. Next, the signal is FFT'd and the frequency at which the maximum
    FFT amplitude occurs is found. This is taken as the starting frequency for a
    numerical optimization of the ``overlap'' between the signal and $e^{i\omega t}$,
    which allows determining $\omega$ to resolution greater than the frequency spacing
    of the FFT. Once $\omega$ is determined, the overlap is subtracted from the
    original signal and the process is repeated, if desired.
  \item \textbf{examples:}
    Find the first fundamental frequency for each of the BPM signals in \verb|par.bpm|.
    \begin{verbatim}
sddsnaff par.bpm par.naff -column=Time,'P?P?x' -terminateSearch=frequencies=1
    \end{verbatim}
  \item \textbf{synopsis:}
    \begin{verbatim}
sddsnaff [inputfile] [outputfile]
  [-pipe=[input][,output]]
  [-columns=indep-variable[,depen-quantity[,...]]]
  [-pair=<column1>,<column2>]
  [-exclude=depen-quantity[,...]]
  [-terminateSearch={changeLimit=fraction[,maxFrequencies=number] | frequencies=number}]
  [-iterateFrequency=[cycleLimit=number][,accuracyLimit=fraction]]
  [-truncate] [-noWarnings]
    \end{verbatim}
  \item \textbf{files:}
    \emph{inputFile} contains the data to be NAFF'd. One column must be chosen as the
    independent variable. If \emph{inputFile} contains multiple pages, each is treated
    separately and delivered to a separate page of \emph{outputFile}.
    \emph{outputFile} contains two columns for each selected column in \emph{inputFile}.
    These columns have names like \emph{origColumn}\verb|Frequency| and
    \emph{origColumn}\verb|Amplitude|, giving the frequency and amplitude for
    \emph{origColumn}.
  \item \textbf{switches:}
    \begin{itemize}
      \item \verb|-pipe[=input][,output]| --- The standard SDDS Toolkit pipe option.
      \item \verb|-columns=indepVariable[,depenQuantityList]| --- Specifies the name of the
        independent variable column. Optionally, if no \verb|-pair| options are given, this
        specifies a comma-separated list (optionally with wildcards) of dependent quantities to be
        NAFF'd as a function of the independent variable. By default, all numerical columns except
        the independent column are NAFF'd.
      \item \verb|-pair=<column1>,<column2>| --- Specifies the names of conjugate pairs to give
        double the frequency range. \emph{column1} is used to obtain the basic frequency and
        \emph{column2} is used to obtain the phase at the frequency of the first column. The
        relative phase expands the resulting frequency from 0--Fn to 0--2\*Fn. Multiple
        \verb|-pair| options may be provided. The independent column is provided by \verb|-columns|,
        while dependent columns may be provided by either \verb|-columns| or \verb|-pair|.
      \item \verb|-exclude=depenQuantity,...| --- Specifies optionally wildcarded names of columns to
        exclude from analysis.
      \item \verb!-terminateSearch={changeLimit=fraction[,maxFrequencies=number] | frequencies=number}! ---
        Specifies when to stop searching for frequency components. If \verb|changeLimit| is given,
        the program stops when the RMS change in the signal is less than the specified \emph{fraction}
        of the original RMS value. The maximum number of frequencies returned in this mode is set with
        \verb|maxFrequencies| (default is 4). If \verb|frequencies| is given, the program finds the
        specified number of frequencies, if possible. By default, one frequency is found for each
        signal.
      \item \verb|-iterateFrequency=[cycleLimit=number][,accuracyLimit=fraction]| --- Controls the
        optimization procedure that searches for the best frequency. By default, the procedure
        executes 100 passes and attempts to determine the frequency to a precision of 0.00001 of the
        Nyquist frequency. \verb|cycleLimit| changes the number of passes, while \verb|accuracyLimit|
        specifies the desired precision.
      \item \verb|-truncate| --- Specifies that the data should be truncated so that the number of
        points is the largest product of primes from 2 to 19 not greater than the original number of
        points. In some cases, this results in significantly greater speed by making the FFTs faster.
      \item \verb|-noWarnings| --- Suppresses warning messages.
    \end{itemize}
  \item \textbf{see also:}
    \begin{itemize}
      \item \hyperref[exampleData]{Data for Examples}
      \item \progref{sddsfft}
    \end{itemize}
  \item \textbf{author:} M. Borland, ANL/APS.
\end{sddsprog}

