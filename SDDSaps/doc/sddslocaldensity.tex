%\begin{latexonly}
\newpage
%\end{latexonly}
\subsection{sddslocaldensity}
\label{sddslocaldensity}

\begin{sddsprog}{sddslocaldensity}
  \item \textbf{description:}
  \verb|sddslocaldensity| computes the local density of points in n-dimensional space for each point in the space.
  \item \textbf{examples:}
    \begin{verbatim}
sddslocaldensity <inputFile> <outputFile> -columns=none,x,y -kde=bins=40 -threads=20
    \end{verbatim}
  \item \textbf{synopsis:}
    \begin{verbatim}
sddslocaldensity [<inputfile>] [<outputfile>] [-pipe=[input][,output]] [-threads=<number>]
{-fraction=<value> | -spread=<value> |
 -kde=bins=<number>[,gridoutput=<filename>][,nsigma=<value>][,explimit=<value>]
      [,sample=<fraction>|use=<number>]}
-columns=<normalizationMode>,<name>[,...] [-output=<columnName>] [-verbose]
    \end{verbatim}
  \item \textbf{files:}
  The input file contains a collection of points in n-dimensional space, defined by named columns.
  The output file contains the same data, but with an additional column \verb|LocalDensity| that gives
  the density of points in the vicinity of each point.
  \item \textbf{switches:}
    \begin{itemize}
      \item \verb|-pipe[=input]| --- The standard SDDS Toolkit pipe option.
      \item \verb!-columns={none|range|rms},<column1Name>,<column2Name>...! ---
        Specifies the normalization mode and column names for analysis.
        Note that the normalization mode is irrelevant when \verb|-fraction|, \verb|-spread|, or \verb|-kde| options are used.
      \item \verb|-fraction|, \verb|-spread|, \verb|-kde| --- Specify the calculation mode, if different from the default.
        Note that all methods except the KDE method show $N^2$ growth in run time, where $N$ is the number of points.
        \begin{itemize}
          \item By default, the ``local density'' for each point is the inverse of the mean distance to all other points.
          \item For \verb|-fraction=f| mode, the ``local density'' for each point is the number of points inside a distance $d_i$
            in dimension $i$, where $d_i = f*(\max(x_i) - \min(x_i))$.
          \item For \verb|-spread=f| mode, the ``local density'' for each point is the sum over all other points
            of the product of unnormalized gaussian spread functions in each dimension, where the gaussian parameter
            is $\sigma = f*(\max(x_i) - \min(x_i))$. Since the gaussians are unnormalized, the result is roughly the
            number of nearby points.
          \item \verb|-kde=bins=<number>[,gridoutput=<filename>][,nsigma=<value>][,explimit=<value>][,sample=<fraction>|use=<number>]| ---
            Performs Kernel Density Estimation in n dimensions using the given number of bins in all dimensions, using
            Silverman's method to estimate the bandwidth in each dimension.
            The \verb|gridOutput| qualifier results in writing the density map to the named file.
            The \verb|nsigma| qualifier specifies truncation of contributions to the density outside the given number of
            bandwidths; it defaults to 5. The \verb|explimit| qualifier specifies a cutoff in the magnitude of $e^{-z^2/2}$
            for including contributions to the density; it defaults to $10^{-16}$. Setting \verb|nsigma| to a smaller value
            and \verb|explimit| to a larger value can significantly reduce run time with little impact on accuracy.
            The \verb|sample| and \verb|use| qualifiers allow reducing the number of input points used for computing the
            density map; this again can significantly reduce run time and may have little impact on results
            if the number of input points is large; regardless of this setting, the estimated local density is output for
            every input point.
        \end{itemize}
      \item \verb|-threads=<number>| --- Specify the number of threads to use for KDE-based computations. Defaults to 1.
        The speed-up is best when the number of KDE bins is relatively small or the number of dimensions is relatively small.
      \item \verb|-outputColumn=<name>| --- Gives the name of the output column containing the ``local density.'' Defaults to \verb|LocalDensity|.
      \item \verb|-verbose| --- If given, provides informational output during execution.
    \end{itemize}
  \item \textbf{author:} M. Borland, ANL/APS.
\end{sddsprog}
