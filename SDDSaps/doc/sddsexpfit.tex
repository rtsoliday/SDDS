\begin{sddsprog}{sddsexpfit}
  \item \textbf{description:}
    \verb|sddsexpfit| does exponential fits to a single column of an SDDS file as a function of another column, the independent variable. The fitting function is
    \[
      E(x) = C + F * e^{R*x},
    \]
    where $x$ is the independent variable, $C$ is the \emph{constant} term, $F$ is the \emph{factor}, and $R$ is the rate.
  \item \textbf{examples:}
  \begin{verbatim}
sddsexpfit vacDecay.sdds -columns=Time,Pressure vacDecay.fit
sddsexpfit vacDecay.sdds -columns=Time,Pressure \
  vacDecay.fit -clue=decays -tolerance=1e-12
  \end{verbatim}
  \item \textbf{synopsis:}
  \begin{verbatim}
sddsexpfit [-pipe=[input][,output]] [inputFile] [outputFile]
  [-columns=xName,yName] [-tolerance=value]
  [-clue={grows|decays}] [-guess=constant,factor,rate]
  [-verbosity=integer] [-fullOutput]
  \end{verbatim}
  \item \textbf{switches:}
    \begin{itemize}
      \item \verb|-pipe=[input][,output]| --- The standard SDDS Toolkit pipe option.
      \item \verb|-columns=xName,yName| --- Specifies the names of the independent and dependent columns of data.
      \item \verb|-tolerance=value| --- Specifies how close \verb|sddsexpfit| will attempt to come to the optimum fit in terms of the mean squared residual. The default is $10^{-8}$.
      \item \verb!-clue={grows|decays}! --- Helps \verb|sddsexpfit| decide whether the data is a decaying or growing exponential, i.e., whether $R$ is negative or positive. If \verb|sddsexpfit| is having trouble, this often helps.
      \item \verb|-guess=constant,factor,rate| --- Gives \verb|sddsexpfit| a starting point for each of the three fit parameters.
      \item \verb|-fullOutput| --- Specifies that \emph{outputFile} will contain the original dependent variable data and the fit residuals, in addition to the independent variable data and the fit values.
      \item \verb|-verbosity=integer| --- Requests informational printouts during fitting. A larger integer produces more output.
    \end{itemize}
  \item \textbf{files:}
    \emph{inputFile} contains the columns of data to be fit. If \emph{inputFile} contains multiple pages, each page is fit separately. \emph{outputFile} has columns containing the independent variable data and the corresponding values of the fit, named by appending the string \verb|Fit| to the name of the dependent variable. If \verb|-fullOutput| is given, \emph{outputFile} includes a column with the dependent values and the residual (dependent values minus fit values), named by appending \verb|Residual| to the dependent variable name. \emph{outputFile} contains four parameters: \verb|expfitConstant|, \verb|expfitFactor|, \verb|expfitRate|, and \verb|expfitRmsResidual|. The first three parameters are respectively $C$, $F$, and $R$ from the above equation. The last is the rms residual of the fit.
  \item \textbf{see also:}
    \begin{itemize}
      \item \hyperref[exampleData]{Data for Examples}
      \item \progref{sddspfit}
      \item \progref{sddsgfit}
      \item \progref{sddsoutlier}
    \end{itemize}
  \item \textbf{author:} M. Borland, ANL/APS.
\end{sddsprog}

