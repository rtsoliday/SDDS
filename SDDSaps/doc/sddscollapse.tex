%\begin{latexonly}
\newpage
%\end{latexonly}
\subsection{sddscollapse}
\label{sddscollapse}

\begin{itemize}
\item {\bf description:}
\verb|sddscollapse| reads data pages from an SDDS file and writes a new SDDS file containing a single data page.
This data page contains only the values of the parameters from the original file, with each parameter forming a 
column of the tabular data.
\item {\bf examples:} 
To create a new file containing the tunes and other parameters as columns:
\begin{flushleft}{\tt
sddscollapse APS.twi APS.parameters
}\end{flushleft}
To do a polynomial fit to nux as a function of nuy, and print the results out:
\begin{flushleft}{\tt
sddscollapse APS.twi -pipe=out | sddspfit -pipe=in fit.sdds -column=nux,nuy -verbose
}\end{flushleft}
\item {\bf synopsis:} 
\begin{flushleft}{\tt
sddscollapse [{\em inputFile}] [{\em outputFile}] [-pipe[=input][,output]]
}\end{flushleft}
\item {\bf files:}
{\em inputFile} is the name of an SDDS data set to be collapsed.  {\em
outputFile} is the result.  Note that {\em outputFile} will not
contain any information on the arrays or columns that are in {\em
inputFile}.

\item {\bf switches:} 
\begin{itemize}
        \item {\tt -pipe[=input][,output]} --- The standard SDDS Toolkit pipe option.
        \item {\tt -noWarnings} --- Suppresses warnings about file overwrites.
\end{itemize}

\item {\bf comment:} 
In spite of the simplicity of the commandline, this is an extremely useful program.  A typical use might
involve processing a multipage file using {\tt sddsprocess} to, for example, obtain statistical analyses of
columns for each page; the results of such analyses are placed in parameters.  Using {\tt sddscollapse} on
this file would produce columns of statistical analyses, with one row for each page.  One might then further
analyze the data using {\tt sddsprocess}.  One could also use {\tt sddscombine} to combine several collapsed,
processed data sets into a single file, which puts one formally back in the same position as when one started.
In this fashion, multi-level data analysis and collation is possible.  This is done with some magnetic
measurements at APS.
\item {\bf see also:}
    \begin{itemize}
    \item \hyperref[exampleData]{Data for Examples}
    \item \progref{sddsprocess}
    \item \progref{sddscombine}
    \item \progref{sddsexpand}
    \end{itemize}
\item {\bf author:} M. Borland, ANL/APS.
\end{itemize}



