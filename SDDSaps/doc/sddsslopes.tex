\begin{sddsprog}{sddsslopes}
  \item \textbf{description:} \verb|sddsslopes| makes straight line fits of column data of the input file with respect to a selected column used as the independent variable. The output file contains a one-row data set of slopes and intercepts for each data set of the input file. Errors on the slope and intercept may be calculated as an option.

  \item \textbf{examples:} The file \verb|corrector.sdds| contains beam position monitor (BPM) readbacks as a function of corrector setting. The defined columns are \verb|CorrectorSetpoint| and the series \verb|bpm1|, \verb|bpm2|, etc. The BPM response to the corrector setpoints is calculated with \verb|sddsslopes|:
  \begin{verbatim}
sddsslopes corrector.sdds corrector.slopes -independentVariable=CorrectorSetpoint \
  -columns='bpm*'
  \end{verbatim}

  \item \textbf{synopsis:}
  \begin{verbatim}
sddsslopes [-pipe=[input][,output]] inputFile outputFile
  -independentVariable=parameterName [-range=lower,upper]
  [-columns=listOfNames] [-excludeColumns=listOfNames]
  [-sigma[=generate]] [-residual=file] [-ascii] [-verbose]
  \end{verbatim}

  \item \textbf{files:} \emph{inputFile} contains the tabular data for fitting. Multiple data sets are processed one at a time. For optional error processing, additional columns of sigma values associated with the data to be fitted must be present. These sigma columns must be named \verb|<name>Sigma| or \verb|Sigma<name>|, the former being searched first.

  \emph{outputFile} contains a one-row data set for each data set in \emph{inputFile}. The columns defined have names such as \verb|nameSlope| and \verb|nameIntercept|, where \verb|name| is the name of the fitted data. A string column called \verb|IndependentVariable| is defined containing the name of the independent variable. If only one file is specified, then the input file is overwritten by the output.

  \item \textbf{switches:}
  \begin{itemize}
    \item \verb|pipe[=input][,output]| --- The standard SDDS Toolkit pipe option.
    \item \verb|-independentVariable=parameterName| --- Name of the independent variable (default is the first valid column).
    \item \verb|-range=lower,upper| --- The range of the independent variable over which the fit is calculated. By default, all data points are used.
    \item \verb|-columns=listOfNames| --- Columns to be individually paired with the independent variable for straight line fitting.
    \item \verb|-excludeColumns=listOfNames| --- Columns to exclude from fitting.
    \item \verb|-sigma[=generate]| --- Calculates errors by interpreting column names \verb|<name>Sigma| or \verb|Sigma<name>| as sigma of column \verb|<name>|. If these columns do not exist, the program generates a common sigma from the residual of a first fit and refits with these sigmas. If \verb|-sigma=generate| is given, sigmas are generated from the residual of a first fit for all columns irrespective of the presence of sigma columns.
    \item \verb|-residual=file| --- Specifies an output file into which the residual of the fits are written. The column names in the residual file are the same as they appear in the input file.
    \item \verb|-ascii| --- Make output file in ASCII mode (binary is the default).
    \item \verb|-verbose| --- Prints some output to stderr.
  \end{itemize}

  \item \textbf{see also:} \progref{sddspfit}

  \item \textbf{author:} L. Emery, ANL.
\end{sddsprog}

