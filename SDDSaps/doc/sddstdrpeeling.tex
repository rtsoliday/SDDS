\begin{sddsprog}{sddstdrpeeling}
  \item \textbf{description:} \verb|sddstdrpeeling| processes time-domain reflectometry (TDR) data using a recursive algorithm to determine the impedance of a nonuniform transmission line.
  \item \textbf{examples:}
    \begin{verbatim}
sddstdrpeeling <inputfile> <outputfile> -columns=Time,Values -inputVoltage=120 -z0=50
    \end{verbatim}
  \item \textbf{synopsis:}
    \begin{verbatim}
sddstdrpeeling [input] [output] [-pipe=[input][,output]]
  -col=time-col,data-column
  [-inputVoltage=value|@<parameter>]
  [-z0=value]
    \end{verbatim}
  \item \textbf{files:} The input file contains both the time and data columns. The output file contains the additional column \verb|PeeledImpedance|.
  \item \textbf{switches:}
    \begin{itemize}
      \item \verb|-pipe=[input][,output]| --- Standard SDDS pipe options for reading/writing files from stdin/stdout.
      \item \verb|-col=time-col,data-column| --- The names of the time and data columns.
      \item \verb|-inputVoltage=value| or \verb|@<parameter>| --- The input voltage in volts of TDR (default 0.2 V).
      \item \verb|-z0=value| --- The line impedance (default 50 ohms).
    \end{itemize}
  \item \textbf{author:} H. Shang, R. Soliday, ANL/APS.
\end{sddsprog}

