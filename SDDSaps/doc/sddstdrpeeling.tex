%\begin{latexonly} 
\newpage 
%\end{latexonly} 
\subsection{sddstdrpeeling} 
\label{sddstdrpeeling} 
 
\begin{itemize} 
\item {\bf description:} \hspace*{1mm}\\ 
{\tt sddstdrpeeling} process the time-domain reflectometry (TRD) data with recursive algorithm to find out the impedance of nonuniform transmission line.
\item {\bf examples:} 
\begin{flushleft}
{\tt sddstdrpeeling <inputfile> <outputfile> -columns=Time,Values -inputVoltage=120 -z0=50 }
\end{flushleft} 
\item {\bf synopsis:}  
\begin{flushleft}
{\tt 
sddstdrpeeling [{\em input}] [{\em output}] [-pipe=[input][,output]] \\ \
-col={\em time-col},{\em data-column} \\ \
{}[-inputVoltage={\em value}|{\em @<parameter>}] \\ \
{}[-z0={\em value}]}
\end{flushleft} 
\item {\bf files:} 
The input file contains both the time an data columns. The output file contains the additional column {\tt PeeledImpedance}.
\item {\bf switches:} 
    \begin{itemize} 
    \item {\tt -pipe=[input][,output]} --- Standard SDDS pipe options for reading/writing files from stdin/stdout.
    \item {\tt -col={\em time-col},{\em data-column}} --- The names of the time and data columns.
    \item {\tt -inputVoltage={\em value}|{\em @<parameter>}} --- The input voltage in volts of TDR (default .2V).
    \item {\tt -z0={\em value}} --- The line impedance (default 50 oms).
\end{itemize} 

\item {\bf author:} H. Shang, R. Soliday, ANL/APS. 
\end{itemize} 
 
