%\begin{latexonly}
\newpage
%\end{latexonly}
\subsection{sdds2dinterpolate}
\label{sdds2dinterpolate}

\begin{itemize}
\item {\bf description:}
{\tt sdds2dinterpolate} interpolates scalar data on a two--dimensional grid.  The program reads an SDDS file containing coordinates and one or more dependent quantities.  Data are interpolated using either natural--neighbour or cubic spline algorithms at a regular grid or at locations supplied in another file.

\item {\bf example:}
Interpolate column {\tt field} onto a 40~by~40 grid covering the range of the input data using natural neighbours:
\begin{flushleft}{\tt
sdds2dinterpolate input.sdds output.sdds -independentColumn=xcolumn=x,ycolumn=y \\
  -dependentColumn=field -outDimension=xdimension=40,ydimension=40 \\
  -algorithm=nn -npoints=10 -weight=1e-6
}\end{flushleft}

\item {\bf synopsis:}
\begin{flushleft}{\tt
sdds2dinterpolate [{\em input}] [{\em output}] [-pipe[=input][,output]] \\
  [-independentColumn=xcolumn={\em xName},ycolumn={\em yName}[,errorColumn={\em name}]] \\
  [-dependentColumn={\em zName}[,{\em zName}...]] \\
  [-scale=circle|square] [-outDimension=xdimension={\em nx},ydimension={\em ny}] \\
  [-range=xminimum={\em xmin},xmaximum={\em xmax},yminimum={\em ymin},ymaximum={\em ymax}] \\
  [-zoom={\em factor}] [-dimensionThin=xdimension={\em nx},ydimension={\em ny}] \\
  [-clusterThin={\em distance}] [-preprocess] \\
  [-algorithm=nn|csa[,linear|sibson|nonSibson][,average={\em nppc}][,sensitivity={\em value}]] \\
  [-weight={\em value}] [-vertex={\em id}] [-npoints={\em number}] [-verbose] [-merge] \\
  [-file={\em pointsFile}[,{\em xName},{\em yName}]] [-majorOrder=row|column]
}\end{flushleft}

\item {\bf switches:}
  \begin{itemize}
  \item {\tt -pipe[=input][,output]} --- Standard SDDS Toolkit pipe option.
  \item {\tt -independentColumn=xcolumn={\em name},ycolumn={\em name}[,errorColumn={\em name}]} --- Names of the columns containing the independent variables and, optionally, a column of errors.
  \item {\tt -dependentColumn={\em name}[,{\em name}...]} --- Dependent columns to be interpolated.
  \item {\tt -scale=circle|square} --- Scale the interpolation region to a unit circle or unit square.
  \item {\tt -outDimension=xdimension={\em nx},ydimension={\em ny}} --- Number of grid points in the x and y directions.
  \item {\tt -range=xminimum={\em xmin},xmaximum={\em xmax},yminimum={\em ymin},ymaximum={\em ymax}} --- Explicit limits of the interpolation grid.
  \item {\tt -zoom={\em factor}} --- Multiply the automatically determined grid range by the given factor.
  \item {\tt -dimensionThin=xdimension={\em nx},ydimension={\em ny}} --- Average data within the specified cell dimensions before interpolation.
  \item {\tt -clusterThin={\em distance}} --- Replace clusters of points closer than the given distance by a single point.
  \item {\tt -preprocess} --- Output thinned data without performing interpolation.
  \item {\tt -algorithm=nn|csa[,linear|sibson|nonSibson][,average={\em nppc}][,sensitivity={\em value}]} --- Select the interpolation algorithm and its parameters.  The {\tt nn} method uses natural neighbours with optional {\tt linear}, {\tt sibson} or {\tt nonSibson} weighting rules.  The {\tt csa} method performs cubic spline approximation and accepts the {\tt average} and {\tt sensitivity} qualifiers.
  \item {\tt -weight={\em value}} --- Minimum interpolation weight allowed.
  \item {\tt -vertex={\em id}} --- Output diagnostic information for the specified vertex.
  \item {\tt -npoints={\em number}} --- Number of points used by the interpolator.
  \item {\tt -verbose} --- Produce additional diagnostic output.
  \item {\tt -merge} --- Merge all input pages before processing.
  \item {\tt -file={\em pointsFile}[,{\em xName},{\em yName}]} --- Read interpolation locations from an external SDDS file.  Optional names specify the x and y columns.
  \item {\tt -majorOrder=row|column} --- Set the output data order.
  \end{itemize}
\item {\bf author:} H. Shang, R. Soliday, L. Emery, M. Borland, ANL/APS.
\end{itemize}
