%\begin{latexonly}
\newpage
%\end{latexonly}
\subsection{sdds2dinterpolate}
\label{sdds2dinterpolate}

\begin{sddsprog}{sdds2dinterpolate}
  \item \textbf{description:}
    \verb|sdds2dinterpolate| interpolates scalar data on a two--dimensional grid. The program reads an SDDS file containing coordinates and one or more dependent quantities. Data are interpolated using either natural--neighbour or cubic spline algorithms at a regular grid or at locations supplied in another file.

  \item \textbf{examples:}
  \begin{verbatim}
sdds2dinterpolate input.sdds output.sdds -independentColumn=xcolumn=x,ycolumn=y \
  -dependentColumn=field -outDimension=xdimension=40,ydimension=40 \
  -algorithm=nn -npoints=10 -weight=1e-6
  \end{verbatim}

  \item \textbf{synopsis:}
  \begin{verbatim}
sdds2dinterpolate [<input>] [<output>] [-pipe[=input][,output]] \
  [-independentColumn=xcolumn=<xName>,ycolumn=<yName>[,errorColumn=<name>]] \
  [-dependentColumn=<zName>[,<zName>...]] \
  [-scale=circle|square] [-outDimension=xdimension=<nx>,ydimension=<ny>] \
  [-range=xminimum=<xmin>,xmaximum=<xmax>,yminimum=<ymin>,ymaximum=<ymax>] \
  [-zoom=<factor>] [-dimensionThin=xdimension=<nx>,ydimension=<ny>] \
  [-clusterThin=<distance>] [-preprocess] \
  [-algorithm=nn|csa[,linear|sibson|nonSibson][,average=<nppc>][,sensitivity=<value>]] \
  [-weight=<value>] [-vertex=<id>] [-npoints=<number>] [-verbose] [-merge] \
  [-file=<pointsFile>[,<xName>,<yName>]] [-majorOrder=row|column]
  \end{verbatim}

  \item \textbf{switches:}
    \begin{itemize}
    \item \verb|-pipe[=input][,output]| --- Standard SDDS Toolkit pipe option.
    \item \verb|-independentColumn=xcolumn=<name>,ycolumn=<name>[,errorColumn=<name>]| --- Names of the columns containing the independent variables and, optionally, a column of errors.
    \item \verb|-dependentColumn=<name>[,<name>...]| --- Dependent columns to be interpolated.
    \item \verb+-scale=circle|square+ --- Scale the interpolation region to a unit circle or unit square.
    \item \verb|-outDimension=xdimension=<nx>,ydimension=<ny>| --- Number of grid points in the x and y directions.
    \item \verb|-range=xminimum=<xmin>,xmaximum=<xmax>,yminimum=<ymin>,ymaximum=<ymax>| --- Explicit limits of the interpolation grid.
    \item \verb|-zoom=<factor>| --- Multiply the automatically determined grid range by the given factor.
    \item \verb|-dimensionThin=xdimension=<nx>,ydimension=<ny>| --- Average data within the specified cell dimensions before interpolation.
    \item \verb|-clusterThin=<distance>| --- Replace clusters of points closer than the given distance by a single point.
    \item \verb|-preprocess| --- Output thinned data without performing interpolation.
    \item \verb+-algorithm=nn|csa[,linear|sibson|nonSibson][,average=<nppc>][,sensitivity=<value>]+ --- Select the interpolation algorithm and its parameters. The \verb|nn| method uses natural neighbours with optional \verb|linear|, \verb|sibson| or \verb|nonSibson| weighting rules. The \verb|csa| method performs cubic spline approximation and accepts the \verb|average| and \verb|sensitivity| qualifiers.
    \item \verb|-weight=<value>| --- Minimum interpolation weight allowed.
    \item \verb|-vertex=<id>| --- Output diagnostic information for the specified vertex.
    \item \verb|-npoints=<number>| --- Number of points used by the interpolator.
    \item \verb|-verbose| --- Produce additional diagnostic output.
    \item \verb|-merge| --- Merge all input pages before processing.
    \item \verb|-file=<pointsFile>[,<xName>,<yName>]| --- Read interpolation locations from an external SDDS file. Optional names specify the x and y columns.
    \item \verb+-majorOrder=row|column+ --- Set the output data order.
    \end{itemize}
  \item \textbf{author:} H. Shang, R. Soliday, L. Emery, M. Borland, ANL/APS.
\end{sddsprog}
