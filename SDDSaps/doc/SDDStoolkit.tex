\documentclass[11pt]{article}
\usepackage[latex2html]{hyperref}
\usepackage[dvips]{graphicx}
%%% PROGREF
% created by M. Borland
% inserted here 11/29/95 11:33:04: 
% latex2html perl script can't recognize this when this
% definition is in the style file.
%
%\newcommand{\progref}[1]{\hyperref{#1}{{\tt #1} (}{)}{#1}}
\newcommand{\progref}[1]{\hyperref[#1]{#1}}
%\newcommand{\progref}[1]{\hyperref[#1]{{#1}}}

\pagestyle{plain}
\tolerance=10000
\newenvironment{req}{\begin{equation} \rm}{\end{equation}}
\setlength{\topmargin}{0.15 in}
\setlength{\oddsidemargin}{0 in}
\setlength{\evensidemargin}{0 in} % not applicable anyway
\setlength{\textwidth}{6.5 in}
\setlength{\headheight}{-0.5 in} % for 11pt font size
%\setlength{\footheight}{0 in}
\setlength{\textheight}{9 in}
\begin{document}

\title{User's Guide for SDDS Toolkit Version 5.0}
\author{M. Borland, L. Emery, H. Shang, R. Soliday\\Advanced Photon Source\\ \date{\today}}
\maketitle

The Self Describing Data Sets (SDDS) file protocol is the basis for a
powerful and expanding toolkit of generic programs. These
programs are used for simulation postprocessing, graphics, data
preparation, program interfacing, and experimental data analysis.

This document describes Version 5.0 of the SDDS commandline toolkit.
Those wishing to write programs using SDDS should consult the {\em
Application Programmer's Guide for SDDS Version 1.5}\cite{SDDS_AP1.5}.
The first section of the present document is shared with this
reference.

This document does not describe SDDS-compliant EPICS applications, of
which they are many. Some of these will be covered in a separate
manual.

\section{Why Use Self-Describing Files?}

Before answering the question posed by the title of this section, it
is necessary to define what a self-describing file is.  As used here,
data in self-describing files has the following attributes:
\begin{itemize}
\item The data is accessed by name and by class.  For example, one
 might ask for ``the column of data called X'', or ``the array of data
 called Y''.  Self-describing data is {\em not} accessed by position
 in a file; e.g., one would not ask for ``the third column of data''.
\item Various attributes of the data that may be necessary to using it
 are available.  For example, one can ask ``what are the units of
 column X?'', ``what is the data-type of array Y?'', or ``how many
 dimensions does array Y have?'' .
\end{itemize}

The primary advantage of accessing data and its attributes by name
rather than the traditional position method is that one can then
construct generic tools to manipulate data. Self-describing data
contains the information that tools need to manipulate various types
of data correctly.  For example, one can plot data with a generic tool
that accepts the names of the quantities to plot; such a tool will be
able to plot data of different types (e.g., integer or
floating-point), and display relevant information (e.g., units) on the
plot.

Another advantage of self-describing data is that it makes the
interface between programs more robust and flexible.  Since programs
only look for data by name, insertion of additional data into a file
is irrelevant.  Multiple programs may interface to a single program
even in the face of differences in what data each places in its output
files.  E.g., program A may create data in single-precision, with
columns called X, Y, and Z.  Program B may create data in
double-precision, with columns called X, Y, and W.  If all programs
employ self-describing files, then a properly-written program C could
access X and Y from the output of either program A or B.  It could
also determine that the output of program B didn't contain data called
Z, and warn the user of this.

The SDDS file protocol incorporates these aspects of self-describing
data.  It has been found extremely valuable for storing data from
simulation, experiment, and accelerator operation at the Advanced
Photon Source (APS).  SDDS is made more valuable by the existence of a
growing ``toolkit'' of over 40 generic commandline programs that
perform many varied operations using SDDS files.  Indeed, while there
are more general self-describing protocols than SDDS, to the author's
knowledge only SDDS has a powerful, generic program toolkit built
around it.  In the author's opinion, this is possible because SDDS
protocol is general but not {\em too} general.  The SDDS Toolkit is
used to postprocess simulation output, to analyze experimental and
archival data, to prepare data for input to other programs, to provide
a bridge between separate simulation codes, to display data
graphically, to collate and section accelerator save/restore files,
and much more.

While it is very flexible, SDDS is also fairly simple.  Because SDDS
features interchangeable binary and ASCII formats, it is an easy
matter to create an SDDS data set ``by hand'', when necessary.  It is
also easy to modify existing programs to print in SDDS protocol, and
to create headers to convert existing text data to SDDS.  At the same
time, data archivers, large-scale simulations, and similar
applications can store data in binary for quick access and disk
economy.  These and other features contribute to the widespread use of
SDDS at APS.

\section{Definition of SDDS Protocol}

\subsection{Introduction}

An SDDS file is referred to as a ``data set''.  Each data set consists
of an ASCII header describing the data that is stored in the file,
followed by zero or more ``data pages'' or ``data tables'' (the former
term is preferred, though the latter is used in many places).  The
data may be in ASCII or unformatted (i.e., ``binary'').  Each data
page is an instance of the structure defined by the header.  That is,
while the specific data may vary from page to page, the structure of
the data may not.

Three types of entities may be present in each page: parameters,
arrays, and columns.  Each of these may contain data of a single data
type, with the choices being long and short integer, single and double
precision floating point, single character, and character string.  The
names, units, data types, and so forth of these entities are defined
in the header.

Parameters are scalar entities.  That is, each parameter defined in
the header has a single value for each page.  Each such value may be a
single number or a single character string, for example.

Arrays are multidimensional entities with potentially varying numbers
of elements.  While there is no restriction on the number of
dimensions an array may contain, this quantity is fixed throughout the
file for each array.  However, the size of the array may vary from
page to page.  Thus, a given two-dimensional array might be 2x2 in one
page, 3x5 in the next, etc.

Columns are vector entities.  All columns in a data set are organized
into a single table, called the ``tabular data section''.  Thus, all
columns must contain the same number of entries, that number being the
number of rows in the table.  There is no restriction on how many rows
the tabular data may contain, nor on the mixing of data types in the
tabular data.

It is possible to design more sophisticated data protocols than SDDS,
and this has in fact been done.  However, the more flexible a protocol
is, the more difficult it becomes to write generic programs that
operate on data.  Experience with SDDS has shown that there is very
little data that cannot be {\em conveniently} stored in one or more
SDDS files.  In fact, most applications need only the parameter and
tabular data facilities.  Frequently, complex data is separated into
several parallel files; the SDDS toolkit provides support for
multifile operations that make this convenient.

The following is an example of a very simple SDDS file.  Users who
would prefer not to read the detailed description of the protocol in
the next section may profit from using this example as a guide.
\begin{flushleft}
\begin{verbatim}
SDDS1
! This is a comment line.  The previous line is required and identifies 
! the file as SDDS.
! Define parameters: 
&parameter name=Description, type=string &end
&parameter name=xTune, type=double &end
&parameter name=yTune, type=double &end
! Define columns:
&column name=s, type=double, units=m, description="longitudinal distance" &end
&column name=betax, type=double, units=m, description="horizontal beta function" &end
&column name=betay, type=double, units=m, description="vertical beta function" &end
&column name=ElementName, type=string &end
! Declare ASCII data and end the header:
&data mode=ascii &end
! First come the parameter values for this page, in the order defined:
Twiss parameters for the APS
35.215
14.296
! Second comes the tabular data section for this page, which has
! 50 rows in this example:
50
   0.000000   14.461726    9.476181        _BEG_ 
   3.030000   15.096567   10.445020          L01 
   3.360000   15.242380   10.667547          L02 
   3.860000   17.308605    9.854735           Q1 
   3.975000   18.254680    9.419835          L11 
   4.190000   20.094943    8.640450          L12 
   4.520000   23.100813    7.529584          L13 
   5.320000   21.435972    7.949178           Q2 
   5.410000   20.278542    8.350441          L21 
   5.620000   17.705808    9.332877          L22 
   5.920000   14.341175   10.848446          L30 
   6.420000   10.719036   12.405601           Q3 
   7.120000    7.920453   12.969811          L41 
\end{verbatim}
\vdots
\begin{verbatim}
  27.600000   14.461726    9.476181          L01 
! The file may end at this point, or a new page may follow.
\end{verbatim}
\end{flushleft}

At this point, those who are new to SDDS may wish to skip to the \hyperref[ManualPagesOverview]{Manual Pages Overview} in order to get a feel for the capabilities of the Toolkit.  The details of SDDS protocol,
the subject of the next section, are less important than what can be done with data once it is in SDDS protocol.

\subsection{Structure of the SDDS Header}
\label{sect:header}

The first line of a data set must be of the form ``SDDS{\em n}'', where {\em n} is the integer SDDS version number.
This document describes version 1.

The SDDS header consists of a series of namelist-like constructs, called namelist commands.  These constructs
differ from FORTRAN namelists in that the SDDS routines scan each construct, determine which it is, and use the
data appropriately. There are six namelist commands recognized under Version 1.  Each is listed below along with
the data type and default values.

For each command, an example of usage is given.  Several styles of entering the namelist commands are exhibited.  I
suggest that the user choose a style that makes it easy to pick out the beginning of each command.  Note that while
each namelist command may occupy one or more lines, no two commands may occupy portions of the same line.

Any field value containing an ampersand must be enclosed in double quotes, as must string values containing
whitespace characters.

Another character with special meaning is the exclamation point, which introduces a comment.  An exclamation point
anywhere in a line indicates that the remainder of the line is a comment and should be ignored.  A literal
exclamation point is obtained with the sequence \verb|\!|, or by enclosing the exclamation point in double quotes.

The commands are briefly described in the following list, and described in detail in the following subsections:
\begin{itemize}
\item {\bf description} --- Specifies a data set description, consisting of informal and formal
        text descriptions of the data set.
\item {\bf column} --- Defines an additional column for the tabular-data section of the data pages.
\item {\bf parameter} -- Defines an additional parameter data element for the data pages.
\item {\bf array} --- Defines an additional array data element for the data pages.
\item {\bf include} --- Directs that header lines be read from a named file.  Rarely used.
\item {\bf data} --- Defines the data mode (ASCII or binary) along with layout parameters, and is
        always the last command in the header.
\end{itemize}

The {\tt column}, {\tt parameter}, and {\tt array} commands have a {\tt name} field that is used to identify the
data being defined.  Each type of data has a separate ``name-space'', so that one may, for example, use the same
name for a column and a parameter in the same file.  This is discouraged, however, because it may produce
unexpected results with some programs.  Names may contain any alphanumeric character, as well as any of the
following: {\tt @ : \# + - \% . \_ \$ \& / }.  The first letter of a name may not be a digit.


\subsubsection{Data Set Description}
\begin{verbatim}
&description 
    STRING text = NULL
    STRING contents = NULL
&end
\end{verbatim}

This optional command describes the data set in terms of two strings.  The first, {\tt text}, is an informal
description that is intended principly for human consumption.  The second, {\tt contents}, is intended to formally
specify the type of data stored in a data set.  Most frequently, the {\tt contents} field is used to record the
name of the program that created or most recently modified the file.

Example:
\begin{verbatim}
&description
        text = "Twiss parameters for APS lattice",
        contents = "Twiss parameters"
&end
\end{verbatim}

{\em Note:} In many cases it is best to use a string parameter for descriptive text instead of the {\tt description}
command.  The reason is that the Toolkit programs will allow manipulation of a string parameter.

\subsubsection{Tabular-Data Column Definition}
\begin{verbatim}
&column
    STRING name = NULL
    STRING symbol = NULL
    STRING units = NULL
    STRING description = NULL
    STRING format_string = NULL
    STRING type = NULL
    long field_length = 0
&end
\end{verbatim}

This optional command defines a column that will appear in the tabular data section of each data page.  The {\tt name}
field must be supplied, as must the {\tt type} field.  The type must be one of {\tt short}, {\tt long}, {\tt float},
{\tt double}, {\tt character}, or {\tt string}, indicating the corresponding C data types.  The {\tt string} type
refers to a NULL-terminated character string.

The optional {\tt symbol} field allows specification of a symbol to represent the column; it may contain escape
sequences, for example, to produce Greek or mathematical characters.  The optional {\tt units} field allows
specification of the units of the column.  The optional {\tt description} field provides for an informal description of
the column, that may be used as a plot label, for example.  The optional {\tt format\_string} field allows specification
of the {\tt printf} format string to be used to print the data (e.g., for ASCII in SDDS or other formats).

For ASCII data, the optional {\tt field\_length} field specifies the number of characters occupied by
the data for the column.  If zero, the data is assumed to be bounded by whitespace characters.  If negative,
the absolute value is taken as the field length, but leading and trailing whitespace characters will be
deleted from {\tt string} data.  This feature permits reading fixed-field-length FORTRAN output without
modification of the data to include separators.

The order in which successive {\tt column} commands appear is the order in which the columns are assumed to
come in each row of the tabular data.

Example (using {\tt sddsplot} conventions for Greek and subscript operations):
\begin{verbatim}
&column name=element, type=string, description="element name" &end
&column 
    name=z, symbol=z, units=m, type=double, 
    description="Longitudinal Position" &end
&column 
    name=alphax, symbol="$ga$r$bx$n", units=m, 
    type=double, description="Horizontal Alpha Function" &end
&column 
    name=betax, symbol="$gb$r$bx$n", units=m, 
    type=double, description="Horizontal Beta Function" &end
&column 
    name=etax, symbol="$gc$r$bx$n", units=m, 
    type=double, description="Horizontal Dispersion" &end
.
.
.
\end{verbatim}

\subsubsection{Parameter Definition}
\begin{verbatim}
&parameter
    STRING name = NULL
    STRING symbol = NULL
    STRING units = NULL
    STRING description = NULL
    STRING format_string = NULL
    STRING type = NULL
    STRING fixed_value = NULL
&end
\end{verbatim}

This optional command defines a parameter that will appear along with the tabular data section of each data page.  The
{\tt name} field must be supplied, as must the {\em type} field.  The type must be one of {\tt short}, {\tt long}, {\tt
float}, {\tt double}, {\tt character}, or {\tt string}, indicating the corresponding C data types.  The {\tt string}
type refers to a NULL-terminated character string.

The optional {\tt symbol} field allows specification of a symbol to represent the parameter; it may contain escape
sequences, for example, to produce Greek or mathematical characters.  The optional {\tt units} field allows
specification of the units of the parameter.  The optional {\tt description} field provides for an informal
description of the parameter.  The optional {\tt format} field allows specification of the {\tt printf} format
string to be used to print the data (e.g., for ASCII in SDDS or other formats).

The optional {\tt fixed\_value} field allows specification of a constant value for a given parameter.  This
value will not change from data page to data page, and is not specified along with non-fixed parameters or
tabular data.  This feature is for convenience  only; the parameter thus defined is treated like any other.

The order in which successive {\tt parameter} commands appear is the order in which the parameters are assumed to
come in the data.  For ASCII data, each parameter that does not have a {\tt fixed\_value} will occupy a separate
line in the input file ahead of the tabular data.

Example:
\begin{verbatim}
&parameter name=NUx, symbol="$gn$r$bx$n", 
  description="Horizontal Betatron Tune", type=double &end
&parameter name=NUy, symbol="$gn$r$by$n", 
  description="Vertical Betatron Tune", type=double &end
&parameter name=L, symbol=L, description="Ring Circumference", 
  type=double, fixed_value=30.6667 &end
.
.
.
\end{verbatim}

\subsubsection{Array Data Definition}

\begin{verbatim}
&array 
    STRING name = NULL
    STRING symbol = NULL
    STRING units = NULL
    STRING description = NULL
    STRING format_string = NULL
    STRING type = NULL
    STRING group_name = NULL
    long field_length = 0
    long dimensions = 1
&end
\end{verbatim}

This optional command defines an array that will appear along with the tabular data section of each data page.  The
{\tt name} field must be supplied, as must the {\tt type} field.  The type must be one of {\tt short}, {\tt long}, {\tt
float}, {\tt double}, {\tt character}, or {\tt string}, indicating the corresponding C data types.  The {\tt string}
type refers to a NULL-terminated character string.

The optional {\tt symbol} field allows specification of a symbol to represent the array; it may contain escape
sequences, for example, to produce Greek or mathematical characters.  The optional {\tt units} field allows
specification of the units of the array.  The optional {\tt description} field provides for an informal description
of the array.  The optional {\tt format\_string} field allows specification of the {\tt printf} format string to be used to
print the data (e.g., for ASCII in SDDS or other formats).  The optional {\tt group\_name} field allows
specification of a string giving the name of the array group to which the array belongs; such strings may be defined
by the user to indicate that different arrays are related (e.g., have the same dimensions, or parallel elements).
The optional {\tt dimensions} field gives the number of dimensions in the array.

The order in which successive {\tt array} commands appear is the order in which the arrays are assumed to come in
the data.  For ASCII data, each array will occupy at least one line in the input file ahead of the tabular data;
data for different arrays may not occupy portions of the same line.  This is discussed in more detail below.

Example:
\begin{verbatim}
&array name=Rx, units=R-standard-units, type=double, dimensions=2,
        description="Horizontal transport matrix in standard units",
        group_name="2x2 transport matrices" &end
&array name=R-standard-units, type=string, dimensions=2, 
        description="Standard units of 2x2 transport matrices",
        group_name="2x2 transport matrices" &end
&array name=P, units=P-standard-units, type=double, dimensions=1, 
        description="Particle coordinate vector in standard units" &end
&array name=P-standard-units, type=string, dimensions=1, 
        description="Standard units of particle coordinate vectors" &end
.
.
.
\end{verbatim}


\subsubsection{Header File Include Specification}
\begin{verbatim}
&include
    STRING filename = NULL
&end
\end{verbatim}

This optional command directs that SDDS header lines be read from the file named by the {\tt filename} field.  These
commands may be nested.

Example of a minimal header:
\begin{verbatim}
SDDS1
&include filename="SDDS.twiss-parameter-header" &end
! data follows:
.
.
.
\end{verbatim}

\subsubsection{Data Mode and Arrangement Defintion}
\begin{verbatim}
&data
    STRING mode = "binary"
    long lines_per_row = 1
    long no_row_counts = 0
    long additional_header_lines = 0
&end
\end{verbatim}

This command is optional unless {\tt parameter} commands without {\tt fixed\_value} fields, {\tt array} commands, or
{\tt column} commands have been given.

The {\tt mode} field is required, and may have one of the values ``ascii'' or ``binary''.  If binary mode is
specified, the other entries of the command are irrelevant and are ignored.  In ASCII mode, these entries are
optional.

In ASCII mode, each row of the tabular data occupies {\tt lines\_per\_row} rows in the file.  If {\tt lines\_per\_row}
is zero, however, the data is assumed to be in ``stream'' format, which means that line breaks are irrelevant.
Each line is processed until it is consumed, at which point the next line is read and processed.

Normally, each data page includes an integer specifying the number of rows in the tabular data section.  This allows
for preallocation of arrays for data storage, and obviates the need for an end-of-page indicator.  However, if {\tt
no\_row\_counts} is set to a non-zero value, the number of rows will be determined by looking for the occurence of an
empty line.  A comment line does {\em not} qualify as an empty line in this sense.

If {\tt additional\_header\_lines} is set to a non-zero value, it gives the number of non-SDDS data lines that
follow the {\tt data} command.  Such lines are treated as comments.

\subsection{Structure of SDDS ASCII Data Pages}

Since the user may wish to create SDDS data sets without using the SDDS function library, a more detailed
description of the structure of ASCII data pages is provided.  Comment lines (beginning with an exclamation point)
may be placed anywhere within a data page.  Since they essentially do not exist as far as the SDDS routines are
concern, I omit mention of them in what follows.

The first SDDS data page begins immediately following the {\tt data} command and the optional additional header
lines, the number of which is specified by the \verb|additional_header_lines| parameter of the {\tt data} command.

If parameters have been defined, then the next ${\rm N_p-N_{fp}}$ lines each contains the value of a single {\tt
parameter}, where ${\rm N_p}$ is the total number of parameters and ${\rm N_{fp}}$ is the number of parameters for
which the \verb|fixed_value| field was specified.  These will be assigned to the parameters in the order that the
\verb|parameter| commands occur in the header.  Multi-token string parameters need not be enclosed in quotation marks.

If arrays have been defined, then the data for these arrays comes next.  There must be at least one ASCII
line for each array.  This line must contain a list of whitespace-separated integer values giving the
size of the array in each dimension.  The number of values must be that given by the {\tt dimensions}
field of the {\tt array} definition.  If the number of elements in the array (given by the product of
these integers) is nonzero, then additional ASCII lines are read until the required number of elements
has been scanned.  It is an error for a blank line or end-of-file to appear before the required elements
have been scanned.

If tabular-data columns have been defined, the data for these elements follows. If the \verb|no_row_counts|
parameter of the {\tt data} command is zero, the first line of this section is expected to contain an integer giving
the number of rows in the upcoming data page.  If \verb|no_row_counts| is non-zero, no such line is expected.  The
remainder of the tabular data section has various forms depending on the parameters of the ${\rm data}$ command, as
discussed above.  The default format is that each line contains the whitespace-separated values for a single row of
the tabular data.  

For column and array data, string data containing whitespace characters must be enclosed in double-quotes.  For column,
array, and parameter data, nonprintable character data should be ``escaped'' using C-style octal sequences.

More than one data page may appear in the data set.  Subsequent data pages have the same structure as just described.
If \verb|no_row_counts=1| is given in the {\tt data} command, then a blank line is taken to end each data set.  An invalid
line (e.g., too few rows or invalid data) is treated as an error, and the rest of the file is ignored.

\subsection{Structure of SDDS Binary Data Pages}

Since the user may wish to read or write SDDS data sets without using the SDDS function library, a more detailed
description of the structure of the data pages is provided. 

The first SDDS data page begins immediately following the {\tt data}
command and the optional additional header lines, the number of which
is specified by the \verb|additional_header_lines| parameter of the
{\tt data} command.

All binary data is stored in the machine representation, except for
strings.  Strings are stored in a variable-record format that consists
of a long signed integers followed by a sequence of characters.  The number
of characters is equal to the value in the signed integer.  Note that
the SDDS library has features that allow recognition and interpretation
of big- and little-endian data representations, which are not described here.

The first element in the data page is the row count, which is a long
signed integer.  This exists even in files that do not contain any
columns.

If parameters have been defined, then their values follow in the order
that the {\tt parameter} definitions appear in the header.  Note that
if a parameter is define as ``fixed-value'' in the {\tt parameter}
definition, then its value will not appear.

If arrays have been defined, then they follow next, in the order that
the {\tt array} definitions appear in the header.  For each array, a
series of long signed integers is first given, one for each dimension
of the array.  For example, a two-dimensional array would have two
integers, specifying the size of the array in the first and second
dimension.  If the two integers are, say, {\tt n} and {\tt m} in that
order, then the declaration of the array in a C program would be, for
example, {\tt a[n][m]}.  Elements of the array are put in the file in
C storage order, which means that the outermost index varies fastest
as the data is accessed in storage order.

If tabular-data columns have been defined, then the table data follows.
Data is stored as rows, so that data for columns is intermixed.  The
order of the columns is the same as the order of the {\tt column} definitions
in the header.

\begin{thebibliography}{9}

\bibitem{SDDS_AP1.5}
    M. Borland and R. Soliday, ``Application Programmer's Guide for SDDS Version
1.5'', APS LS Note. Available \href{https://ops.aps.anl.gov/manuals/sdds/SDDS.html}{https://ops.aps.anl.gov/manuals/sdds/SDDS.html}

\bibitem{elegant}
        M. Borland, ``User's Manual for {\tt elegant}'', 
        APS Light Source Note, LS-287, September 2001.

\bibitem{SDDS_PAC95}
        M. Borland, ``A Self-Describing File Protocol for Simulation Integration and Shared Postprocessors'',
        to appear in {\em Proceedings of the 1995 Particle Accelerator Conference}, May 1995, Dallas.

\bibitem{thesis}
        M. Borland, ``A High-Brightness Thermionic Microwave Gun'', Stanford Ph.D. Thesis, 1991, Appendix A.

\bibitem{SDDSEPICS_PAC95}
        L. Emery, ``Commissioning Software Tools at the Advanced Photon Source'', 
        to appear in {\em Proceedings of the 1995 Particle Accelerator Conference}, May 1995, Dallas.

\bibitem{shower_PAC95}
        L. Emery, ``Beam Simulation and Radiation Dose Calculation at the Advanced Photon Source with
        {\tt shower}, an EGS4 Interface'',
         to appear in {\em Proceedings of the 1995 Particle Accelerator Conference}, May 1995, Dallas.

\bibitem{Abramowitz}
        M. Abramowitz and I. A. Stegun, eds., {\em Handbook of Mathematical Functions}, Dover Publications,
        New York, 1965.

\bibitem{GENESIS} S. Reiche, {\em NIM} A 429 (1999) 242.

\end{thebibliography}

\newpage
\section{Manual Pages Overview}

\label{ManualPagesOverview}

The intention is of this section is to provide a means by which the
reader can select programs that might suit a given need.  For each
program, a brief (and usually incomplete) description is given, along
with example applications.  The example applications provided for each
tool are drawn from experience at APS; it is hoped that most will make
sense to most readers.

This section is followed by manual pages that give detailed
descriptions of each program.  Many of the programs have a large
number of switches, most of which are optional.  In order to help the
new user, actual commandline examples are provided for simple use of
each program.  After understanding these, the user is in a good
position to explore the additional capabilities provided by the
options.

Note that many of the Toolkit programs process tabular data only
(i.e., columns).  To use these programs with parameter data, one can
use {\tt sddscollapse} to convert parameter data into tabular data.
Using pipes will make this more convenient.

Support for SDDS array elements is presently rather sparse in the
Toolkit.  This reflects the fact that almost all data can be
conveniently stored using parameter and column elements.  Hence, work
has concentrated on providing tools that manipulate such data.  Future
versions of the Toolkit will provide more array support.

Most of the Toolkit programs process data pages sequentially.  That
is, in many cases the requested processing is performed on each
successive page of the input file and delivered to successive pages of
the output file.

\subsection{SDDS Toolkit Programs by Category}

\subsubsection{ Mathematical Operations Tools}

\begin{itemize}

\item \progref{sddsbaseline} --- Remove baselines from column data.  Example
application: determining the noise level in a video signal and subtracting
it from the signal.

\item \progref{sddschanges} --- Analyzes changes in column data from
page to page in a file, relative to a reference file or the first
page.  Example application: finding changes in a waveform that is
acquired repeatedly, where successive waveforms are on successive
pages.

\item \progref{sddscliptails} --- Remove tails from column data, where a 
tail is dubious data on either side of a peak.  Example application: 
removing halo or noise tails from video images of beam spots.

\item \progref{sddsderiv} --- Does numerical differentiation of multiple data columns
versus a single column, with optional error propogation.

\item \progref{sddsinsideboundaries} --- Determines whether points in a two-dimensional space (x, y) are inside any of a
series of closed boundaries (or contours).

\item \progref{sddsinteg} --- Does numerical integration of multiple data columns versus a single column, with
optional error propogation.  Example application: finding the field integral an accelerator magnet from a
longitudinal field scan.

\item \progref{sddsinterp} --- Does interpolation of multiple data columns as a function of a single column.
Example application: finding the required current to obtain a desired excitation in a magnet, or interpolating a
curve at positions given in a second file.

\item \progref{sddslocaldensity} --- Computes the local density of points in n-dimensional space.

\item \progref{sddsnormalize} --- Normalizes data in multiple columns using various types of
normalization factors, determined from the data.  

\item \progref{sddspeakfind} --- Finds values of columns at locations of peaks in a single column.  Example
application: finding the position and height of peaks in a power spectrum obtained from a FFT.

\item \progref{sddsprocess} --- Probably the most-used toolkit program, excepting \verb|sddsplot|.  Allows
creating new parameters and columns with user-specified equations; filtering and matching operations; printing,
editing, scanning, and subprocess operations; statistical and waveform analysis of column data to produce new
parameters; and much more.

\item \progref{sddssmooth} --- Smooths columns of data using multipass nearest-neighbor averaging.  Example
application: reducing noise in a frequency spectrum prior to finding peaks.

\item \progref{sddszerofind} --- Finds values of columns at locations of interpolated zeroes in a single column.
Example application: finding zeros of a tabulated function that isn't known analytically.

\end{itemize}

\subsubsection{Statistics Tools}

\begin{itemize}

\item \progref{sddscorrelate} --- Computes correlation coefficients and
correlation significance between column data.  Example application: finding correlations among time series data
collected from process variables, and evaluating their signficance to find possible cause-and-effect relationships.

\item \progref{sddsdistest} --- Performs statistical tests on data to
determine whether the data is drawn from any of various distributions.
Example application: determining if a component failure rate matches a
Poisson distribution.

\item \progref{sddsenvelope} --- Analyzes column data across pages to find
minima, maxima, averages, standard-deviations, etc., on a row-by-row basis.   Example application: finding 
the envelope and average of a set of waveforms.

\item \progref{sddseventhist} --- Analyzes labeled events in a dataset
to provide histograms of the occurences of each type of event.  Can
also histogram the overlap off all types of events with a single type
of event.  Example application: correlating the occurence times of
alarm signals to determine which alarms usually occur together.

\item \progref{sddshist} --- Does histograms of column data.  Example application: finding the distribution of a
readback that is sampled many times, or of particle coordinates from an accelerator tracking simulation.

\item \progref{sddshist2d} --- Does two-dimensional histograms of column data.  Example applications: finding the
two-dimensional distribution of a pair of readbacks that are sampled many times, or of two particle coordinates
(e.g., x and y position) from an accelerator tracking simulation.

\item \progref{sddsmultihist} --- Does histograms of multiple columns
of data.  Example application: finding the distribution of a set of
similar readbacks that are sampled many times.

\item \progref{sddsoutlier} --- Eliminates statistical outliers from data.  Example application: eliminating bad
or nonrepresentative data points prior to searching for correlations with \verb|sddscorrelate|, or computing
statistics with \verb|sddsprocess|.

\item \progref{sddsprocess} --- Probably the most-used toolkit program, excepting \verb|sddsplot|.  Allows
creating new parameters and columns with user-specified equations; filtering and matching operations; printing,
editing, scanning, and subprocess operations; statistical and waveform analysis of column data to produce new
parameters; and much more.

\item \progref{sddsrowstats} --- Computes row-by-row statistics across multiple columns of data, creating
new columns to contain the statistics.  Example application: finding the mean value of a set of readout
values from time-series data collection, where each readout is in a separate column.

\item \progref{sddsrunstats} --- Computes running or blocked statistics of multiple columns.  Example
applications: smoothing noisy data; finding running averages and error bars for time-series data.

\item \progref{sddsshiftcor} --- Computes correlation coefficients
between column data as a function of shift position of a reference
column.  Example application: finding correlations among time series
data collected from process variables, including the possibility of
time-lags between the process variables due to physical or data
collection effects.

\end{itemize}

\subsubsection{Digital Signal Processing Tools}

\begin{itemize}

\item \progref{sddsconvolve} --- Does FFT convolution, deconvolution, and correlation. Example
application: computing the ideal impulse response of a system after you've measured the response to
a pulse.

\item \progref{sddsdigfilter} --- Performs time-domain digital filtering of column data.  Example applications:
low pass, high pass, band pass, or notch filtering of data to eliminate unwanted frequencies.

\item \progref{sddsfdfilter} --- Performs frequency-domain filtering of column data.  Example application:
applying a filter that is specified as a table of attenuation and phase as a function of frequency.

\item \progref{sddsfft} --- Does Fast Fourier Transforms of column data.  Example application: finding signficant
frequency components in time-varying data, or finding the integer tune of an accelerator from a difference orbit.

\item \progref{sddsnaff} --- Does Numerical Analysis of Fundamental Frequencies, a more accurate
method of determining principle frequencies in signals than the FFT.

\end{itemize}

\subsubsection{Data Fitting Tools}

\begin{itemize}

\item \progref{sddsexpfit} --- Does an exponential fit to column data.  Example
application: finding the exponential lifetime of a beam in a storage ring, or the half-life a radioactive
sample.

\item \progref{sddsgenericfit} --- Does generic fits to column data.  Example application:
fitting the sum of two gaussians.

\item \progref{sddsgfit} --- Does gaussian fits to column data.  Example application:
finding the width of a resonance, or the rms size of a beam profile.


\item \progref{sddsmpfit} --- Does polynomial fits to multiple columns
data, including error analysis.  Will do fits to specified orders,
fits of specified symmetry, and adaptive fitting.  Use {\tt sddspfit}
as a simpler and somewhat more capable alternative for fitting a
single column of data. Use {\tt sdds2dpfit} for two-dimensional fits.

\item \progref{sddspfit} --- Does polynomial fits to column data,
including error analysis.  Will do fits to specified orders, fits of
specified symmetry, and adaptive fitting.

\item \progref{sdds2dpfit} --- Does two-dimensional polynomial fits to column data.

\item \progref{sddssplinefit} --- Fits splines to column data. Useful for displaying a smooth curve in plots, or for feedforward based on experimental data. One can specify the order and the number of breakpoints on the command line, and create an evaluations file.

\end{itemize}

\subsubsection{ Data Manipulation Tools}
\begin{itemize}

\item \progref{sddsbreak} --- Breaks data pages into new, separate pages based on changes in column data and other
criterion.  Example applications: reorganizing a file to have a limited number of rows in each page, or to have a
new page started when a gap is seen in the data.

\item \progref{sddscast} --- change the data type of the elements in an SDDS file.

\item \progref{sddscollapse} --- Collapses a data set into a single data page by deleting the tabular data and
turning the parameters into columns.  Example application: abstraction of summary properties of data set following
analysis with \verb|sddsprocess|.

\item \progref{sddscollect} --- Reorganizes tabular data from the
input file to bring data from several groups of similarly named
columns together into a single column per group.  Example application:
collecting several statistical analyses of many columns into a single
column per analysis type.

\item \progref{sddscombine} --- Combines any number of data sets into a single data set by adding data from each
successive data set to a newly-created data set.  Example application: bringing together comparable but distinct
data for analysis with \verb|sddsprocess|.  Using \verb|sddsprocess|, \verb|sddscombine|, and \verb|sddscollapse|
in sequence repeatedly is a powerful way to analyze and collate large amounts of data.

\item \progref{sddsconvert} --- Allows conversion of a data set between binary and ASCII, with optional deletion
and renaming of columns, arrays, and parameters .  Example application: conversion to binary of an ASCII data set
created by a simple program, or by a text editor.  N.B.: it is {\em not} recommended to use \verb|sddsconvert| to
convert a binary SDDS file to ASCII, then strip the header off and read the ASCII file.  This completely
bypasses the self-describing aspects of the SDDS file and is not robust.  If the program that creates the SDDS
file is changed so that the columns are created in a different order, the program that reads the ASCII file
will produce unexpected results. Use \progref{sdds2plaindata}, \progref{sddsprintout}, or \progref{sdds2stream}
for conversion to non-self-describing files.  In this way, you can assure the order of the data is fixed.

\item \progref{sddsderef} --- Allows dereferencing (i.e., de-indexing) of array and column data.  Example
application: converting a column of integer indexes into a column of equivalent text messages, where the text
messages are stored once each in an array in the input file (for space-savings).
\item \progref{sddsdiff} --- Compares two SDDS files.

\item \progref{sddsendian} --- Converts from little-endian to
big-endian and vice-versa.  Example application: converting binary
data from the native format to a format used on another type of
computer prior to transferring the data to the other computer.

\item \progref{sddsexpand} --- Expands a data set into one page for each row, with column data promoted
to parameter data.  Essentially the inverse of \progref{sddscollapse}. 

\item \progref{sddsregroup} --- Swaps the row indexing and page
indexing of data in an SDDS file. That is, the ${\rm i {th}}$ row of
the ${\rm j {th}}$ data page in the input file becomes the ${\rm
j{th}}$ row of the ${\rm i {th}}$ data page of the output file.
Example application: viewing the long-term evolution of a
repeatedly-sampled waveform at each point in the waveform.

\item \progref{sddstranspose} --- Transposes the tabular data in the
input file, so that the output file contains one column for each row
in the input.  Example usage: tranpose an orbit response matrix as
part of preparing to use it for feedback.

\item \progref{sddsmakedataset} --- writes the input data into a file or 
pipe in SDDS format. It can be used to make add SDDS file consisting of
a small amount of data from the script. It is more convenient than {sdds save}.

\item \progref{sddsmatrixmult} --- Multiplies the tabular data in the
two input files to produce a file containing a matrix of the product.
Example usage: Multiply a vector of errors with a correction matrix to
obtain a vector of corrections to apply in a step-by-step feedback
system.

\item \progref{sddsmatrixop} --- performs general matrix operations. 
The matrices and operations are specified on the command line and 
the operations will proceed in a rpn-like fashion.

\item \progref{sddsselect} --- Copies rows from one file based on the
presence or absence of matching data in another file.  Example
application: finding all of the rows from one file that do not appear
in a second file.

\item \progref{sddssnap2grid} --- Snaps data that is quasi-uniformly spaced to a uniformly spaced grid.

\item \progref{sddssort} --- Sorts the tabular data section of a data set by the values in named columns.
Optionally eliminates duplicate rows.

\item \progref{sddssortcolumn} --- rearrange the columns of an SDDS data.

\item \progref{sddssplit} --- Places each page of a file in a separate, new file.  Example application: getting
selected pages of a file into separate, single-page files for use with a program that only recognizes the first
page.

\item \progref{sddsxref} --- Creates a new data set by adding selected
rows from one data set to another data set.  Example application:
cross-referencing the turn-by-turn coordinates of particles in a
tracking simulation with the initial coordinates using a particle ID
number.

\end{itemize}

\subsubsection{Graphics Tools}

\begin{itemize}

\item \progref{sddscontour} --- Makes contour and color-map plots from an SDDS data set column, or from a
\verb|rpn| expression of the values in the columns of a data set.  Supports FFT interpolation and filtering.
Example application: displaying data from a two-dimensional magnetic field scan.

\item \progref{sddsplot} --- A highly flexible, device-independent graphics program, equally capable of
``quick-and-dirty'' or publication quality graphics.  Example application: making an X-windows movie of several
columns of data that change from page to page in a file.

\end{itemize}

\subsubsection{Image Processing Tools}

\begin{itemize}

\item \progref{sddsimageconvert} --- Converts a single-column SDDS image file into a multi-column SDDS image file and vice versa.

\item \progref{sddsimageprofiles} --- Extracts the profile from a multi-column SDDS image file.

\item \progref{sddsspotanalysis} --- Used to locate and give details about spots in multi-column SDDS image files.

\end{itemize}

\subsubsection{Miscellaneous Tools}

\begin{itemize}

\item \progref{elegant2genesis} --- Processes particle output from the particle tracking code
        \verb|elegant|\cite{elegant}
        and makes a file suitable for use as the BEAMFILE with the FEL code GENESIS\cite{GENESIS}.

\item \progref{sddssampledist} --- Draws samples from one or more probability distributions.
        Suitable for making input particle distributions for tracking codes, for example, 
        using user-defined probability distributions.

\item \progref{sddssequence} --- Creates an SDDS data set of arithmetic sequences. 
Example application: generating values for an independent variable, whose values can be used by
{\tt sddsprocess} to produce a mathematical function.

\item \progref{sddscongen} --- Creates an SDDS data set by evaluating an \verb|rpn| expression over a defined 2
dimensional grid.  Example application: generating values of a function of two variables on a grid for plotting
with {\tt sddscontour}.

\item \progref{sddstimeconvert} --- Converts time data between seconds-since-epoch and calendar breakdown formats.
Example application: finding the year, month, and day corresponding to a system time value.

\end{itemize}

\subsubsection{File Protocol Conversion Tools}

\begin{itemize}
\item \progref{csv2sdds} --- Converts CSV (Comma-Separated-Values) data to SDDS.

\item \progref{citi2sdds} --- Converts Hewlett-Packard CITI files to SDDS.

\item \progref{hpif2sdds} --- Converts Hewlett-Packard HP54542 scope internal format to SDDS.

\item \progref{hpwf2sdds} --- Converts Hewlett-Packard HP54542 scope text format to SDDS.

\item \progref{hdf2sdds} --- Converts Hierarchical Data Format (HDF) to SDDS.

\item \progref{lba2sdds} --- Converts Spiricon Laser Beam Analyzer files to SDDS.

\item \progref{plaindata2sdds} --- Converts plain data file with simple formatting to SDDS.

\item \progref{sdds2math} --- Converts SDDS data to a format accepted by Mathematica.

\item \progref{sdds2mpl} --- Extracts data columns or parameters from an SDDS data set and creates \verb|mpl| data
files\cite{thesis}.

\item \progref{sdds2plaindata} --- Converts SDDS data to a plain data file with simple formatting.  This is
one of the recommended ways to convert SDDS data to plain ASCII or binary data for input to non-compliant
programs.  The advantage of using \verb|sdds2plaindata| is that you can guarantee that the data is 
emitted in a fixed order.

\item \progref{sdds2spreadsheet} --- Converts SDDS data to a format accepted by the Excel and Wingz spreadsheets.
Obsolete.  Use \progref{sddsprintout} instead.

\item \progref{TFS2sdds} --- Converts MAD/LEP TFS files to SDDS.

\end{itemize}


\subsubsection{Text-based Data-review Tools}

\begin{itemize}
\item \progref{sdds2stream} --- Takes column or parameter data from a list of SDDS data sets and delivers it to
the standard output as a stream of values.  Example application: getting data into a shell variable for use in a
script.  \verb|sdds2stream| may be used to convert SDDS data to plain ASCII text.

\item \progref{sddsprintout} --- Makes customized printouts from
column, parameter, and array data in an SDDS data set.  Also makes
spreadsheet-compatible data and plain ASCII text.  Example
application: making a nicely-formatted printout of data that needs to
be reviewed manually.

\item \progref{sddsquery} --- Prints a summary of the SDDS header for a data set.  Also prints bare lists of names
of defined entities, suitable to use with shell scripts that need to detect the existence of entities in the data
set.

\end{itemize}


\subsection{Toolkit Program Usage Conventions}

In order to make the multitude of Toolkit programs easier to use, the developers have attempted to use consistent
commandline argument styles.  The Toolkit programs all require at least one commandline argument.  Therefore, if a
program is executed without commandline arguments, it is assumed that the user is asking for help.  In this case, a
help message is printed that shows syntax and (usually) describes the meaning of the switches.  In general, program
usage is of the following form:\\
\hspace*{5mm}{\tt programName} {\em fileNames} {\em switches}.\\
Probably the simplest example would be \\ 
\hspace*{5mm}{\tt sddsquery } {\em fileName},\\
which would invoke {\tt sddsquery} to describe the contents of an SDDS file.
A slightly more complicated example would be\\
\hspace*{5mm}{\tt sddsquery } {\em fileName} {\tt -columnList},\\
which invokes {\tt sddsquery} to list just names of columns in a file.

Programs assume that any commandline argument beginning with a minus sign ('-') is an option; all others are
assumed to be filenames.  Note that case is ignored in commandline switches.  The specific meaning of a
filename is dictated by its order on the commandline.  For example, if two filenames are given, the first
would commonly be an input file while the second would commonly be an output file.  

In some cases, a command with a single filename implies replacement of the existing file.  For example,\\
\hspace*{5mm}{\tt sddsconvert} {\em fileName} {\tt -binary}\\
would replace the named file with a binary version of the same data.   This command is completely equivalent to\\
\hspace*{5mm}{\tt sddsconvert} {\tt -binary} {\em fileName} \\
That is, unlike many UNIX commands, the position of filenames relative to options is irrelevant.

One might also wish to make a new file, rather than replacing the existing file.  This could be done by\\
\hspace*{5mm}{\tt sddsconvert} {\tt -binary} {\em fileName} {\em fileName2} \\
Note that while the option may appear anywhere on the commandline, the order of the filenames is crucial to
telling the program what to do.

In following manual pages and in the program-generated help text, program usage is described using the following
conventions:
\begin{itemize}
\item The first token on the commandline  is the name of the program.
\item Items in square-brackets ({\tt []}) are optional.  Items not in square brackets are required.
\item Items in curly-brackets ({\tt \{\}}) represent a list of choices.  The choices are separated by
a \verb.|. character, as in\\
{\tt \{ {\em choice1} | {\em choice2} | {\em choice3} \}}
\item Items in italics are descriptions of arguments or data that must be supplied by the user.  These items are not typed 
literally as shown.
\item Items in normal print are typed as shown, with optional abbreviation.  These are
usually switch keywords or qualifiers.  Any unique abbreviation is acceptable.
\end{itemize}

In addition to using files, most toolkit programs also take input from pipes, which obviates the need for temporary
files in many cases.  For those programs that support pipes, one can employ the {\tt -pipe} option.  This option
provides a good example of what options look like.  For example, one could do the following to test binary-ascii
conversion:\\
\hspace*{5mm}{\tt sddsconvert -binary -pipe=out {\em fileName} \verb.|. sddsconvert -ascii -pipe=in {\em fileName1} }\\
The {\tt -pipe=out} option to {\tt sddsconvert} tells it to deliver its output to a pipe; it still
expects a filename for input.  Similarly, the {\tt -pipe=in} option to {\tt sddsquery} tells it to
accept input from a pipe.  

The {\tt -pipe} switch may be given in one of five forms: {\tt -pipe}, {\tt -pipe=input,output}, {\tt
-pipe=output,input}, {\tt -pipe=input}, {\tt -pipe=output} .  The first three forms are equivalent.  In a usage
message, these forms would be summarized as {\tt -pipe[=input][,output]}.  One could also use abbreviations like
{\tt -pipe=i}, {\tt -pipe=i,o}, etc.  For convenience in the manual, the data stream from or to a pipe will 
often be referred to by the name of the file for which it substitutes.  Note that you may not deliver more
than one file on the same pipe.

\subsection{Data for Examples}
\label{exampleData}

In order to make examples simpler to present, it helps to have hypothetical data files to refer to.  I will assume the
existence of several data files that I hope will be familiar to many readers.  An ASCII version of each file is
provided in the SDDS distribution package.  This gives new users some data to ``play with'' in getting familiar with
SDDS.  These files are also used in several demonstration scripts provided in the package.  

For each file, I've listed the names of the columns and parameters, and described each.  I've given the data types in
detail, even though only the distinction between numerical and nonnumerical data is relevant, just to emphasize that
data types can be freely mixed.  I've tried to include as little data as is necessary to make useful demonstrations,
without simplifying so much as to be trivial.

\subsubsection{Twiss Parameters}

The example of Twiss parameters for an accelerator is a familiar one.  Throughout these pages, it is assumed that two
files, {\t APS0.twi} and {\tt APS.twi}, exist containing the following data (a simplification of the Twiss output from the
accelerator simulation code \verb|elegant|):
\begin{itemize}
\item Parameters: 
        \begin{itemize} 
        \item {\tt nux}, {\tt nuy} -- Double-precision values of the x and y tunes.
        \item {\tt alphac} --- Double-precision values of the momentum compaction factor.
        \end{itemize}
\item Columns:
        \begin{itemize}
        \item {\tt s} -- A double precision column of element positions.  For simplicity, it is assumed to increase
                monotonically through the file.
        \item {\tt ElementName} -- A string column of element names.
        \item {\tt ElementType} -- A string column of element type identifiers.
        \item {\tt betax}, {\tt betay} --- Double-precision columns of the beta functions for the
        horizontal and vertical planes, respectively.
        \item {\tt psix}, {\tt psiy} --- Double-precision columns of the betatron phase advance.
        \item {\tt etax}, {\tt etay} --- Double-precision columns of the dispersion functions.
        \end{itemize}
\end{itemize}
To make it more interesting, {\tt APS0.twi} is a single-page file containing the APS design lattice, while {\tt
APS.twi} is a multi-page file with each page corresponding to a different configuration.

In passing, it is appropriate to mention the style of the names used.  It has been found helpful to use capitalization
at word boundaries to make long names more readable.  (In some cases, like {\tt betax}, a certain case is used because
it is significant.)  When doing so will not create confusion, we also tend to capitalize the first letter of a name,
which helps the name to stand out on the command line.  Abiding by these conventions tends to result in readable names
being created by Toolkit programs that have automatic name generation.  Underscores in names are avoided because they
increase the length of a name while adding less readability than capitalization.

\subsubsection{Data Logging Over Time}

One of the most common applications of SDDS for APS commissioning and
operation is logging of measured data values at intervals.  A set of
generic EPICS monitoring programs {\tt sddsmonitor}, {\tt sddsvmonitor}
(vector monitoring), and {\tt sddswmonitor} (waveform monitoring) are used
for this.  One example is the vacuum pressure in the APS ring, which is
logged continuously by {\tt sddsvmonitor}; this data consists of readings
from ion gauges around the ring.  Another example is logging of
beam-position-monitor readouts in the Positron Accumulator Ring (PAR) and
its input and output beam transport lines using the program {\tt
sddsmonitor}.

For use in examples, I'll assume the existence of two files called {\tt
SR.vac} and {\tt par.bpm}.  These are simplified from actual files
collected with the programs just mentioned.
 
{\tt SR.vac} is a file containing an arbitrary series of data pages, each
consisting of a snapshot of the vacuum gauge readings around the ring.
There are 40 such readings, one for each sector of the accelerator.
Typically, one set of readings is taken every 15 minutes.
\begin{itemize}
\item Parameters:
        \begin{itemize}
        \item {\tt TimeStamp} --- A string parameter containing the time at which the snapshot was taken.
        \item {\tt TimeOfDay} --- A double-precision parameter containing the time of day in hours since midnight.
        \end{itemize}
\item Columns:
        \begin{itemize}
        \item {\tt Index} --- A long-integer column containing the row index.
        \item {\tt SectorName} --- A string column containing the name of the sector each row corresponds to.
        \item {\tt Pressure} --- A double-precision column containing the pressure readout from the gauges at
                the time given by {\tt TimeStamp}.
        \end{itemize}
\end{itemize}

{\tt par.bpm} is a file containing a single page of data with any arbitrary
number of rows.  The PAR has 16 beam-position-monitors (BPMs), each
providing a horizontal (x) and vertical (y) readout.  In addition, the beam
transport line downstream of PAR (known as the PTB line), contains five
BPMs for x and five for y.  The data included in the distribution contains
only the x values, since these are more interesting:
\begin{itemize}
\item Parameters: 
        \begin{itemize}
        \item {\tt TimeStamp} --- A string parameter giving the starting time of the data collection.
        \end{itemize}
\item Columns:
        \begin{itemize}
        \item \verb|Time| --- A double-precision column giving the elapsed number of seconds since monitoring
        begain. The values are approximately equispaced.
        \item \verb|TimeOfDay| --- A double-precision column giving the time of day in hours since midnight.
        \item \verb|P|{\em quadrant}\verb|P|{\em number}\verb|x| --- 16 single-precision 
        readouts of the horizontal beam orbit just prior
        to beam extraction.  {\em quadrant} ranges from 1 to 4, as does {\em number}.
        \item \verb|P|{\em quadrant}\verb|P|{\em number}\verb|y| --- 16 single-precision 
        readouts of the vertical beam orbit just prior
        to beam extraction.  {\em quadrant} ranges from 1 to 4, as does {\em number}.
        \item \verb|PTB:PH|{\em number}\verb|x| --- four single-precision readouts of the
        horizontal beam trajectory as the beam passes
        through the PTB transfer line.  {\em number} ranges from 2 to 5.
        \end{itemize}
\end{itemize}

\newpage
\section{Manual Pages}
\begin{center}
{\em Manual pages are written by the program author unless otherwise noted.}
\end{center}

\label{ManualPages}

\begin{sddsprog}{csv2sdds}
  \item \textbf{description:} Converts Comma-Separated-Values (CSV) data and similar data to SDDS. CSV data is commonly used by spreadsheet programs.
  \item \textbf{examples:}
    \begin{verbatim}
    csv2sdds data.csv -columnData=name=x,type=float,units=m -columnData=name=Name,type=string data.sdds
    \end{verbatim}
  \item \textbf{synopsis:}
    \begin{verbatim}
    csv2sdds [CSVfile] [SDDSfile] [-pipe[=in][,out]]
      [-asciiOutput] [-spanLines] [-maxRows=integer]
      [-schFile=SCHfilename] [-skiplines=integer]
      [-delimiters=start=character,end=character] [-separator=character]
      [-columnData=name=string,type=string[,units=string] ...]
      [-uselabels[=units]] [-majorOrder=<row|column>]
    \end{verbatim}
  \item \textbf{files:}
    {\em CSVfile} is a comma-separated-values file. Such a file consists of M rows each containing
    N items of data, forming N columns. The items on each row are separated by commas (or by a
    specified separator). The items may also be delimited by double quotation marks (or by specified
    delimiters).

    {\em SDDSfile} is the SDDS output that is created.

    The optional {\em SCHfilename} is a way of specifying the column headers. The file is
    expected to contain a series of lines of the form {\em tag}={\em valueList}, where {\em
    valueList} is a comma-separated list of one or more items. Lines not matching this format are
    ignored. The {\em tag} may be one of the following:
    \begin{itemize}
      \item \verb|Filetype|: optional. If given, must have {\em valueList} of \verb|Delimited|.
      \item \verb|Delimiter|: optional. If given, the first character of {\em valueList} is used for
        the start and end delimiters.
      \item \verb|Separator|: optional. If given, the first character of {\em valueList} is used for
        the separator.
      \item \verb|CharSet|: optional. If given, must have {\em valueList} of \verb|ascii|.
      \item \verb|FieldN|, where {\em N} is an integer: one or more required. The integers {\em N}
        must be consecutive. The first item in {\em valueList} is taken as the column name.
        The second item is interpreted as the data type. At present, only the \verb|Float| data type is actually
        interpreted as anything other than character data. All others are
        treated as character string types. If needed, \verb|sddsprocess| may be used to process the
        resulting string columns to produce other data types.
    \end{itemize}
  \item \textbf{switches:}
    \begin{itemize}
      \item \verb|-pipe[=in][,out]| --- The standard SDDS Toolkit pipe option.
      \item \verb|-asciiOutput| --- Specifies ASCII output.
      \item \verb|-spanLines| --- Specifies that the program should ignore line breaks in parsing the
        input data.
      \item \verb|-maxRows=integer| --- The maximum number of rows expected. This allows
        optimization of the program, but isn't essential.
      \item \verb|-schFile=filename| --- Specifies the name of a SCH file specifying the
        format of the CSV file. I don't know what SCH stands for, but apparently some PC
        programs generate such files.
      \item \verb|-skiplines=integer| --- Skip the specified number of lines at the beginning
        of the input file.
      \item \verb|-delimiters=start=character,end=character| --- Specifies start and end
        delimiters for data. The default is to use a double-quotation mark for both.
      \item \verb|-separator=character| --- Specifies separator to use. The default is a comma.
      \item \verb|-columnData=name=string,type=string[,units=string]| ---
        Specifies the name and data
        type of a column of data in the CSV file. One of these options should be given for each
        column in the input file, in the same order as the columns appear in that file.
      \item \verb|-uselabels[=units]| --- The column names and optionally the units are defined in the
        file prior to the data.
      \item \verb!-majorOrder=<row|column>! --- Writes output file in row or column major order.
    \end{itemize}
  \item \textbf{see also:}
    \begin{itemize}
      \item \progref{sdds2stream}
      \item \progref{sddsprocess}
      \item \progref{plaindata2sdds}
    \end{itemize}
  \item \textbf{author:} M. Borland, ANL/APS.
\end{sddsprog}


%\begin{latexonly}
\newpage
\begin{center}{\Large\verb|elegant2genesis|}\end{center}
%\end{latexonly}
\subsection{elegant2genesis}
\label{elegant2genesis}

\begin{itemize}
\item {\bf description:}
\verb|elegant2genesis| analyzes particle output data from \verb|elegant| and prepares
a ``beamfile'' for input to GENESIS\cite{GENESIS}, a 3-D time-dependent FEL code by 
S. Reiche.  The beamfile contains slice analysis of the particle data, and may be
useful in other applications as well.

\item {\bf synopsis:}
\begin{flushleft}{\tt
elegant2genesis {\em inputfile} {\em outputfile} 
[-pipe=[in][,out]] [-textOutput]
[-totalCharge={\em coulombs} | -chargeParameter={\em name}]
[-wavelength={\em meters} | -slices={\em integer}]
[-steer] [-removePTails=deltaLimit={\em value}[,fit][,beamOutput={\em filename}]]
[-reverseOrder] [-localFit]
}\end{flushleft}

\item {\bf files:}
\begin{itemize}
\item {\em inputfile} --- A particle output file from \verb|elegant| or any other program that
uses the same column names and units.  
\item {\em outputfile} --- Contains the slice analysis, suitable for use with SDDS-compliant GENESIS.  The columns are
as follows:
\begin{itemize}
\item \verb|s|, \verb|t| --- Location of slice in the bunch, in meters or seconds, respectively. \verb|s| is defined
  so that the head of the bunch is a {\em larger} values, contrary to {\tt elegant}'s convention.
\item \verb|gamma| --- The average value of $\gamma$ for the slice.
\item \verb|dgamma| --- The standard deviation of $\gamma$ for the slice.
\item \verb|xemit|, \verb|yemit| --- The normalized slice emittances in the horizontal and vertical plane, respectively.
\item \verb|xrms|, \verb|yrms| --- The slice rms beam sizes.
\item \verb|xavg|, \verb|yavg| --- The slice centroid positions.
\item \verb|pxavg|, \verb|pyavg| --- The slice centroids for x and y slopes.
\item \verb|alphax|, \verb|alphay| --- The slice values of the Twiss parameter $\alpha$.
\item \verb|current| --- The slice current.
\item \verb|wakez| --- Defined for convenience to be 0.  See GENESIS manual for the meaning.
\item \verb|N| --- The number of particles in the slice.
\end{itemize}
\end{itemize}

\item {\bf switches:}
\begin{itemize}
\item \verb|-pipe[in][,out]| --- The standard SDDS toolkit pipe option.
\item \verb|-textOutput| --- Requests text output instead of SDDS output, which may be
        useful for input to non-SDDS-complaint versions of GENESIS.
\item \verb|-totalCharge=|{\em coulombs} --- Gives the total charge of the beam in Coulombs.
\item \verb|-chargeParameter=|{\em name} --- Gives the name of a parameter in {\em inputfile} where
        the total charge in the beam is given.
\item \verb|-wavelength=|{\em meters} --- This option is misnamed.  It is actually the slice length
        in meters.
\item \verb|-slices=|{\em integer} --- The number of analysis slices to use.
\item \verb|-steer| --- If given, then the transverse centroids for the bulk beam are all set to
        zero.  The relative centroid offsets of the slices are, of course, unchanged.
\item \verb|-removePTails=deltaLimit=|{\em value}\verb|[,fit][,beamOutput={\em filename}\verb|]| ---
        Remvoes the momentum tails from the beam.  \verb|deltaLimit| is the maximum absolute value
        of $(p-\langle p \rangle)/\langle p \rangle$ that will be accepted.  If \verb|fit| is given,
        then a linear fit to $p$ as a function of $t$ is performed, and removal is based on the
        residuals from that fit.  If \verb|beamOutput| is given, then the filtered beam data is
        written to the named file for review.
\item \verb|-reverseOrder| --- By default, the data for the head of the beam comes first.  This
        option causes elegant to put the data for the tail of the beam first.
\item \verb|-localFit| --- If given, then for each slice a local
 linear fit is used to remove any momentum chirp prior to compute the
 momentum spread.  This produces a momentum spread that is more
 independent of the number of slices.  Should not be used if the FEL
 cooperation length is greater than the slice length.
\end{itemize}

\item {\bf author:} R. Soliday, M. Borland, ANL/APS.
\end{itemize}




\begin{sddsprog}{hdf2sdds}
\item {\bf description:} Converts Hierarchical Data Format (HDF) to SDDS.
\item {\bf examples:}
\begin{verbatim}
hdf2sdds data.hdf data.sdds -withIndex
\end{verbatim}
\item {\bf synopsis:}
\begin{verbatim}
hdf2sdds [HDFfile] [SDDSfile] [-pipe[=out]] [-query] [-ascii] [-binary]
[-withIndex] [-reduceFactor=integer[,keep=integer]] [-3doutput]
\end{verbatim}
\item {\bf files:}
{\em HDFfile} is the filename of the HDF file.

{\em SDDSfile} is the SDDS output that is created.
\item {\bf switches:}
\begin{itemize}
  \item \verb|-pipe[=out]| --- The standard SDDS Toolkit pipe option.
  \item \verb|-query| --- Print out the names of the groups and datasets in the HDF file.
  \item \verb|-ascii| --- Requests that the output be ASCII.
  \item \verb|-binary| --- Requests that the output be binary.
  \item \verb|-withIndex| --- An index column is added to the output file.
  \item {\tt -reduceFactor={\em integer}[,keep={\em integer}]} --- Write the first {\em keep}th value of every {\em reduceFactor} values in order to reduce the size of the output file. The attributes are written to the output file by their original data types.
  \item \verb|-3doutput| --- Used to convert 3-dimensional HDF data.
\end{itemize}
\item {\bf see also:}
\begin{itemize}
  \item \progref{sddsconvert}
\end{itemize}
\item {\bf author:} R. Soliday, ANL/APS.
\end{sddsprog}


\begin{sddsprog}{image2sdds}
  \item \textbf{description:} \verb|image2sdds| converts raw binary image files to SDDS format. It can
    produce single column or multi-column image files and supports both
    binary and ASCII output. Options are provided to transpose the image,
    write data as a 2D array, and generate headers compatible with
    \verb|sddscontour|.
  \item \textbf{examples:}
    \begin{verbatim}
    image2sdds camera.raw image.sdds -2d -xdim 640 -ydim 480 -xmax 10 -ymax 10
    image2sdds camera.raw image.sdds -multicolumnmode -transpose -ascii
    \end{verbatim}
  \item \textbf{synopsis:}
    \begin{verbatim}
    image2sdds <IMAGE infile> <SDDS outfile> [-2d] [-ascii] [-contour]
        [-multicolumnmode] [-transpose] [-xdim value] [-ydim value]
        [-xmin value] [-xmax value] [-ymin value]
        [-ymax value] [-debug] [-help]
    \end{verbatim}
  \item \textbf{switches:}
    \begin{itemize}
    \item {\tt -2d} --- Write the image as a single SDDS 2D array. May not be used
      with \verb|-multicolumnmode|.
    \item {\tt -ascii} --- Produce ASCII SDDS output rather than binary.
    \item {\tt -contour} --- Add parameters expected by \progref{sddscontour}.
    \item {\tt -multicolumnmode} --- Write image data in multiple columns. Incompatible
      with \verb|-2d|.
    \item {\tt -transpose} --- Transpose the image about its diagonal prior to output.
    \item {\tt -xdim {\em value}} --- Number of pixels in the horizontal dimension
      (default {\tt 482}).
    \item {\tt -ydim {\em value}} --- Number of pixels in the vertical dimension
      (default {\tt 512}).
    \item {\tt -xmin {\em value}} --- Minimum x value for the image data (default 0).
    \item {\tt -xmax {\em value}} --- Maximum x value for the image data. If not given
      the spacing is determined from {\tt -xdim}.
    \item {\tt -ymin {\em value}} --- Minimum y value for the image data (default 0).
    \item {\tt -ymax {\em value}} --- Maximum y value for the image data. If not given
      the spacing is determined from {\tt -ydim}.
    \item {\tt -debug {\em level}} --- Enable debugging output at the specified level.
    \item {\tt -help} or {\tt -?} --- Print command usage information.
    \end{itemize}
  \item \textbf{see also:}
    \begin{itemize}
    \item \progref{sddscontour}
    \item \progref{sddsimageconvert}
    \item \progref{sddsimageprofiles}
    \end{itemize}
  \item \textbf{author:} J. Stein, J. Anderson, R. Soliday, ANL/APS.
\end{sddsprog}




% $Log: not supported by cvs2svn $
%
%\begin{latexonly}
\newpage
%\end{latexonly}
\subsection{plaindata2sdds}
\label{plaindata2sdds}

\begin{itemize}
\item {\bf description:} Converts plain data files with a simple format to SDDS.
\item {\bf example:} 
\begin{flushleft}{\tt
plaindata2sdds data.input data.output -inputMode=ascii "-separator= " -parameter=time,long -column=x,double -column=y,double
}\end{flushleft}
\item {\bf synopsis:}
\begin{flushleft}{\tt
plaindata2sdds [{\em Inputfile}] [{\em Outputfile}] [-pipe[=in][,out]] 
[-inputMode=<ascii|binary>]
[-outputMode=<ascii|binary>]
[-separator={\em character}]
[-commentCharacters={\em characters}]
[-noRowCount]
[-order=<rowMajor|columnMajor>]
[-parameter={\em name},{\em type}[,units={\em string}][,desc={\em string}][,symbol={\em string}] ...]
[-column={\em name},{\em type}[,units={\em string}][,desc={\em string}][,symbol={\em string}] ...]
[-skipcolumn={\em type}]
[-nowarnings]
[-majorOrder=<row|column>]
}\end{flushleft}
\item {\bf files: }
{\em Inputfile} is a file that is similar to SDDS files in that it contains parameter and column data. However this file does not contain SDDS header information. The column data does not need to be preceded by a row count but it is recommended. Also the column data can be separated by a user supplied character. White space on either side of the separator is allowed. Binary plaindata files are also allowed.

{\em Outputfile} is the SDDS output that is created.

\item {\bf switches:}
    \begin{itemize}
    \item {\tt -pipe[=in][,out]} --- The standard SDDS Toolkit pipe option.
    \item {\tt -inputMode=<ascii|binary>} --- The plain data file can be read in ascii or binary format.
    \item {\tt -outputMode=<ascii|binary>} --- The SDDS data file can be written in ascii or binary format.
    \item {\tt -separator={\em character}} --- In ascii mode the columns of the plain data file are separated 
	by the given character.
    \item {\tt -commentCharacters={\em characters}} --- In ascii mode the rows beginning with any comment characters are ignored.
    \item {\tt -noRowCount} --- The row count is not included prior to the beginning of the column data. 
	If the plain data file is a binary file then the row count must be included.
    \item {\tt -order=<rowMajor|columnMajor>} --- Row major order is the default. Here each row of the 
	plain data file consists of one element from each column. In column major order each column 
	is located entirely on one row.
    \item {\tt -parameter={\em name},{\em type}[,units={\em string}][,description={\em string}][,symbol={\em string}]} --- 
	Add this option for each parameter in the plain data file.
    \item {\tt -column={\em name},{\em type}[,units={\em string}][,description={\em string}][,symbol={\em string}]} --- 
	Add this option for each column in the plain data file.
    \item {\tt -skipcolumn={\em type}} --- Skip a column in the plain data file. It may be used multiple times.
    \item {\tt -nowarnings} --- Do not print warning messages.
    \item {\tt -majorOrder=<row|column>} --- Writes output file in row or column major order.
    \end{itemize}
\item {\bf see also:}
    \begin{itemize}
    \item \progref{sdds2plaindata}
    \item \progref{csv2sdds}
    \end{itemize}
\item {\bf author:} R. Soliday, ANL/APS.
\end{itemize}


\begin{sddsprog}{raw2sdds}
  \item \textbf{description:}
  \verb|raw2sdds| converts a binary data stream into the SDDS format.
  \item \textbf{examples:}
\begin{verbatim}
raw2sdds inputfile outputfile -definition=Screen1,type=character -size=484,512
\end{verbatim}
  \item \textbf{synopsis:}
\begin{verbatim}
raw2sdds inputfile outputfile -definition=name,definition-entries
  [-size=horiz-pixels,vert-pixels] [-majorOrder=row|column]
\end{verbatim}
  \item \textbf{files:}
  The input file contains the binary data for horizontal and vertical pixels of a screen.
  Only single byte data, per pixel, is currently allowed.
  \item \textbf{switches:}
    \begin{itemize}
      \item \verb|-definition=name,definition-entries| --- The name of the output column.
      \item \verb|-size=horiz-pixels,vert-pixels| --- Specifies dimensions of the screen.
      \item \verb!-majorOrder=row|column! --- Specifies the binary SDDS layout.
    \end{itemize}
  \item \textbf{see also:} \progref{image2sdds}.
  \item \textbf{author:} C. Saunders, R. Soliday, ANL/APS.
\end{sddsprog}
 

%\begin{latexonly}
\newpage
%\end{latexonly}
\subsection{sdds2dfft}
\label{sdds2dfft}

\begin{itemize}
\item {\bf description:}
{\tt sdds2dfft} performs two-dimensional fast Fourier transforms of data stored in SDDS tables. One column is taken as the independent variable while the remaining numerical columns are transformed. Options allow complex input, power spectral density output, normalization and more.

\item {\bf example:}
\begin{flushleft}{\tt
sdds2dfft image.sdds fft.sdds -columns=t,Image*
}\end{flushleft}

\item {\bf synopsis:}
\begin{flushleft}{\tt
sdds2dfft [-pipe=[input][,output]] [{\em inputFile}] [{\em outputFile}]\\
  -columns={\em indepVariable}[,{\em depenQuantity}[,...]]\\
  {}[-complexInput[=unfolded|folded]]\\
  {}[-exclude={\em depenQuantity}[,...]]\\
  {}[-sampleInterval={\em number}]\\
  {}[-normalize]\\
  {}[-fullOutput[=unfolded|folded][,unwrapLimit={\em value}]]\\
  {}[-psdOutput[=plain][,{integrated|rintegrated[={\em cutoff}]}]]\\
  {}[-inverse]\\
  {}[-padwithzeroes[={\em exponent}]]\\
  {}[-truncate]\\
  {}[-suppressaverage]\\
  {}[-noWarnings]\\
  {}[-majorOrder=row|column]
}\end{flushleft}

\item {\bf files:}
{\em inputFile} supplies the data to be transformed. The independent variable column must exist and the dependent columns may include wildcards. Each page of {\em inputFile} is processed separately. The resulting {\em outputFile} contains a column {\tt f} giving frequency plus FFT results for each selected quantity. Additional columns may appear for phase, real and imaginary parts, and PSD depending on the options.

\item {\bf switches:}
    \begin{itemize}
    \item {\tt -pipe[=input][,output]} --- Standard SDDS Toolkit pipe option.
    \item {\tt -columns={\em indepVariable}[,{\em depenQuantity}[,...]]} --- Specifies the name of the independent variable and a list of dependent columns to transform.
    \item {\tt -complexInput[=unfolded|folded]} --- Indicates that input columns contain complex data as real and imaginary pairs. The qualifiers control whether the input frequency space is unfolded or folded.
    \item {\tt -exclude={\em depenQuantity}[,...]} --- Excludes columns from transformation using wildcard patterns.
    \item {\tt -sampleInterval={\em number}} --- Specifies a sampling interval for selecting input rows.
    \item {\tt -normalize} --- Normalizes the FFT results to a peak magnitude of~1.
    \item {\tt -fullOutput[=unfolded|folded][,unwrapLimit={\em value}]} --- Requests output of magnitude, real, imaginary and phase components. The frequency spectrum may be unfolded or folded. When {\tt unwrapLimit} is supplied the phase is unwrapped when the relative magnitude exceeds the limit.
    \item {\tt -psdOutput[=plain][,{integrated|rintegrated[={\em cutoff}]}]} --- Requests power spectral density output. Qualifiers allow plain PSD, integrated PSD, or reverse-integrated PSD with an optional cutoff frequency.
    \item {\tt -inverse} --- Performs the inverse Fourier transform. Effective only with complex input.
    \item {\tt -padwithzeroes[={\em exponent}]} --- Pads data with zeros so the number of points is a power of two or a multiple of small primes. The exponent determines extra powers of two to use.
    \item {\tt -truncate} --- Truncates the data to the nearest product of small primes when padding is not desired.
    \item {\tt -suppressaverage} --- Subtracts the average value before transforming.
    \item {\tt -noWarnings} --- Suppresses warning messages.
    \item {\tt -majorOrder=row|column} --- Selects row-major or column-major ordering for the output file.
    \end{itemize}
\item {\bf author:} H. Shang and R. Soliday, ANL/APS.
\end{itemize}

%\begin{latexonly}
\newpage
%\end{latexonly}
\subsection{sdds2dinterpolate}
\label{sdds2dinterpolate}

\begin{itemize}
\item {\bf description:}
{\tt sdds2dinterpolate} interpolates scalar data on a two--dimensional grid.  The program reads an SDDS file containing coordinates and one or more dependent quantities.  Data are interpolated using either natural--neighbour or cubic spline algorithms at a regular grid or at locations supplied in another file.

\item {\bf example:}
Interpolate column {\tt field} onto a 40~by~40 grid covering the range of the input data using natural neighbours:
\begin{flushleft}{\tt
sdds2dinterpolate input.sdds output.sdds -independentColumn=xcolumn=x,ycolumn=y \\
  -dependentColumn=field -outDimension=xdimension=40,ydimension=40 \\
  -algorithm=nn -npoints=10 -weight=1e-6
}\end{flushleft}

\item {\bf synopsis:}
\begin{flushleft}{\tt
sdds2dinterpolate [{\em input}] [{\em output}] [-pipe[=input][,output]] \\
  {}[-independentColumn=xcolumn={\em xName},ycolumn={\em yName}[,errorColumn={\em name}]] \\
  {}[-dependentColumn={\em zName}[,{\em zName}...]] \\
  {}[-scale=circle|square] [-outDimension=xdimension={\em nx},ydimension={\em ny}] \\
  {}[-range=xminimum={\em xmin},xmaximum={\em xmax},yminimum={\em ymin},ymaximum={\em ymax}] \\
  {}[-zoom={\em factor}] [-dimensionThin=xdimension={\em nx},ydimension={\em ny}] \\
  {}[-clusterThin={\em distance}] [-preprocess] \\
  {}[-algorithm=nn|csa[,linear|sibson|nonSibson][,average={\em nppc}][,sensitivity={\em value}]] \\
  {}[-weight={\em value}] [-vertex={\em id}] [-npoints={\em number}] [-verbose] [-merge] \\
  {}[-file={\em pointsFile}[,{\em xName},{\em yName}]] [-majorOrder=row|column]
}\end{flushleft}

\item {\bf switches:}
  \begin{itemize}
  \item {\tt -pipe[=input][,output]} --- Standard SDDS Toolkit pipe option.
  \item {\tt -independentColumn=xcolumn={\em name},ycolumn={\em name}[,errorColumn={\em name}]} --- Names of the columns containing the independent variables and, optionally, a column of errors.
  \item {\tt -dependentColumn={\em name}[,{\em name}...]} --- Dependent columns to be interpolated.
  \item {\tt -scale=circle|square} --- Scale the interpolation region to a unit circle or unit square.
  \item {\tt -outDimension=xdimension={\em nx},ydimension={\em ny}} --- Number of grid points in the x and y directions.
  \item {\tt -range=xminimum={\em xmin},xmaximum={\em xmax},yminimum={\em ymin},ymaximum={\em ymax}} --- Explicit limits of the interpolation grid.
  \item {\tt -zoom={\em factor}} --- Multiply the automatically determined grid range by the given factor.
  \item {\tt -dimensionThin=xdimension={\em nx},ydimension={\em ny}} --- Average data within the specified cell dimensions before interpolation.
  \item {\tt -clusterThin={\em distance}} --- Replace clusters of points closer than the given distance by a single point.
  \item {\tt -preprocess} --- Output thinned data without performing interpolation.
  \item {\tt -algorithm=nn|csa[,linear|sibson|nonSibson][,average={\em nppc}][,sensitivity={\em value}]} --- Select the interpolation algorithm and its parameters.  The {\tt nn} method uses natural neighbours with optional {\tt linear}, {\tt sibson} or {\tt nonSibson} weighting rules.  The {\tt csa} method performs cubic spline approximation and accepts the {\tt average} and {\tt sensitivity} qualifiers.
  \item {\tt -weight={\em value}} --- Minimum interpolation weight allowed.
  \item {\tt -vertex={\em id}} --- Output diagnostic information for the specified vertex.
  \item {\tt -npoints={\em number}} --- Number of points used by the interpolator.
  \item {\tt -verbose} --- Produce additional diagnostic output.
  \item {\tt -merge} --- Merge all input pages before processing.
  \item {\tt -file={\em pointsFile}[,{\em xName},{\em yName}]} --- Read interpolation locations from an external SDDS file.  Optional names specify the x and y columns.
  \item {\tt -majorOrder=row|column} --- Set the output data order.
  \end{itemize}
\item {\bf author:} H. Shang, R. Soliday, L. Emery, M. Borland, ANL/APS.
\end{itemize}

\begin{sddsprog}{sdds2headlessdata}
  \item \textbf{description:} Converts selected columns from an SDDS file into a raw binary stream with no header. Data may be written in row-major or column-major order.
  \item \textbf{examples:}
    \begin{verbatim}
sdds2headlessdata data.sdds output.bin -column=Col1,Col2 -order=rowMajor
    \end{verbatim}
  \item \textbf{synopsis:}
    \begin{verbatim}
sdds2headlessdata [input] [output] [-pipe[=in][,out]] -column=name[,name...] [-order=rowMajor|columnMajor]
    \end{verbatim}
  \item \textbf{files:}
    \begin{itemize}
    \item \verb|input| --- SDDS file providing column data.
    \item \verb|output| --- Raw binary file receiving selected columns.
    \end{itemize}
  \item \textbf{switches:}
    \begin{itemize}
    \item {\tt -pipe[=in][,out]} --- The standard SDDS Toolkit pipe option.
    \item {\tt -column=\emph{name}[,\emph{name}...]} --- Specifies one or more columns to write to the output file. This option is required.
    \item {\tt -order=\emph{rowMajor|columnMajor}} --- Determines the order in which column data are written. Row-major order (default) writes one element from each column per row; column-major order writes each column sequentially.
    \end{itemize}
  \item \textbf{see also:}
    \begin{itemize}
    \item \progref{sdds2plaindata}
    \end{itemize}
  \item \textbf{author:} H. Shang, ANL/APS.
\end{sddsprog}

%\begin{latexonly}
\newpage
%\end{latexonly}
\subsection{sdds2tiff}
\label{sdds2tiff}

\begin{itemize}
\item {\bf description:} Converts an SDDS file to one or more TIFF images.  Each page of the input
file becomes a separate image named {\tt <output>.\#\#\#\#}.  Two styles of input are
supported: a single column file with {\tt Variable1Name} and {\tt Variable2Name}
parameters (plus dimension parameters), or a multi-column file containing
columns with a common prefix such as {\tt Line1}, {\tt Line2}, and so on.
\item {\bf example:}
\begin{flushleft}{\tt
sdds2tiff image.sdds image -fromPage=2 -toPage=5 -columnPrefix=Img -maxContrast
}\end{flushleft}
\item {\bf synopsis:}
\begin{flushleft}{\tt
sdds2tiff [{\em input}] [{\em output}] [-pipe[=in]] [-fromPage={\em number}]\\
  {}[-toPage={\em number}] [-columnPrefix={\em prefix}] [-maxContrast] [-16bit]
}\end{flushleft}
\item {\bf switches:}
  \begin{itemize}
  \item {\tt -pipe[=in]} --- The standard SDDS Toolkit pipe option.
  \item {\tt -fromPage={\em number}} --- Begin conversion with the specified page
    of the input file.
  \item {\tt -toPage={\em number}} --- Stop conversion after the specified page
    of the input file.
  \item {\tt -columnPrefix={\em prefix}} --- Prefix used to identify columns that
    contain image lines when the input has multiple columns.  The default is
    {\tt Line}.
  \item {\tt -maxContrast} --- Scale the output so the largest data value maps to
    the maximum gray level (255 or 65535).
  \item {\tt -16bit} --- Write 16-bit TIFF images rather than 8-bit images.
  \end{itemize}
\item {\bf author:} R. Soliday, ANL/APS.
\end{itemize}

%\begin{latexonly}
\newpage
%\end{latexonly}
\subsection{sdds2dpfit}
\label{sdds2dpfit}

\begin{itemize}
\item {\bf description:} 
{\tt sdds2dpfit} does ordinary 2-dimensional polynomial fits to column data.
\item {\bf synopsis:} 
\begin{flushleft}{\tt
sdds2dpfit [{\em inputfile}] [{\em outputfile}] [-pipe=[input][,output]]
  -independent={\em x1ColumnName},{\em x2ColumnName} -dependent={\em yColumnName}[,sigma={\em columnName}]
  {-maximumOrder={\em value} |  [-addOrders={\em xOrder},{\em yOrder} [-addOrders=...]]}
  [-coefficients={\em filename}] [-evaluate={\em locationsFilename},{\em x1Name},{\em x2Name},{\em outputFilename}]
  [-copyParameters]
}\end{flushleft}
\item {\bf files:}
{\em inputFile} is an SDDS file containing columns of data to be fit.  If it contains multiple pages, they are processed
separately.  {\em outputFile} is an SDDS file containing one page for each page of {\em inputFile}.  It contains columns of
the independent and dependent variable data, plus columns for the fit and residuals.   The values of the fit
and of the residuals are in a columns named {\em yName}{\tt Fit} and {\em yName}{\tt Residual}.  {\em outputFile} also contains the
following parameters:
\begin{itemize}
\item {\tt RmsResidual} --- Rms residual of the fit.
\item {\tt ReducedChiSquared}: the reduced chi-squared of the fit:
  $$ \chi^2_\nu = \frac{\chi^2}{\nu} = \frac{1}{N-T}\sum_{i=0}^{N-1} \left(\frac{z_i - z(x_i, y_i)}{\sigma_i}\right)^2 $$,
  where $\nu = N-T$ is the number of degrees of freedom for a fit of N points with T terms.
\item {\tt ConditionNumber} --- Condition number from SVD inversion of the matrix used in obtaining the fit.
\item {\tt FitIsValid } --- If non-zero, the fit is valid.
\item {\tt Terms} --- Number of terms in the fit.
\item {\tt Coefficient\_{\em mm}\_{\em nn}} --- Coefficient $C_{mn}$ of the fit. E.g., if fitting $z$ as a function of $x$ and
  $y$, then
  \begin{equation}
    z = \sum_m \sum_n C_{mn} x^m y^n.
  \end{equation}
\end{itemize}

\item {\bf switches:}
    \begin{itemize}
    \item {\tt -pipe[=input][,output]} --- The standard SDDS Toolkit pipe option.
    \item {\tt -independent={\em x1Name},{\em x2Name}} --- Gives the names of the columns containing values of
      the independent quantities.
    \item {\tt -dependent={\em yName}[,sigma={\em sigmaName}]} --- Gives the name of the column containing value sof
      the dependent quantities, and optionally the name of the column containing error bars.
    \item {\tt -maximumOrder={\em p}} --- Request inclusion of all terms up to $x^m y^n$ such that $(n+m)\leq p$.
    \item {\tt -addOrders={\em m},{\em n}} --- Request inclusion of $x^m y^n$. May be repeated to request inclusion of
      additional terms.
    \item {\tt -coefficients={\em filename}} --- Request that fit coefficients be written to the named file.
    \item {\tt -evaluate={\em locationsFilename},{\em x1Name},{\em x2Name},{\em outputFilename}} --- Request evaluation
      of the fit for a set of $x$ and $y$ values given in {\em locationsFilename}, with results written to 
      {\em outputFilename}.
    \item {\tt -copyParameters} --- If given, program copies all parameters from the input file
      into the main output file.  By default, no parameters are copied.
    \end{itemize}
\item {\bf see also:}
    \begin{itemize}
    \item \hyperref[exampleData]{Data for Examples}
    \item \progref{sddspfit}
    \item \progref{sddsmpfit}
    \item \progref{sddsgfit}
    \item \progref{sddsplot}
    \end{itemize}
\item {\bf author:} M. Borland, ANL/APS.


\end{itemize}

%\begin{latexonly} 
\newpage 
%\end{latexonly} 
\subsection{sdds2math} 
\label{sdds2math} 
 
\begin{itemize} 
\item {\bf description:} 
\verb|sdds2math| converts an SDDS file to a file that can be read into Mathematica. 
The file contains a single Mathematica variable of the form: 
 
\begin{verbatim} 
sdds={description,coldef,pardef,arraydef,associates,tables} 
  description={text,contents} 
  coldef={coldef-1, coldef-2, ...} 
    coldef-n={name,units,symbol,format,type,fieldlength,description} 
  pardef={pardef-1, pardef-2, ...} 
    pardef-n={name,fixed_value,units,symbol,type,description} 
  arraydef={arraydef-1, arraydef-2, ...} 
    arraydef-n={name,units,symbol,format,type,fieldlength,group,description} 
  associates={associate-1, associate-2,...} 
    associate-n={sdds,filename,path,contents,description} 
  tables={table-1, table-2, ...} 
    table-n={parameters,data} 
      parameters={parameter-1, parameter-2, ...} 
      data={row-1, row-2, ...} 
        row-n={val-1, val-2, ...} 
\end{verbatim} 
 
A number of Mathematica programs to extract information from this 
variable are available in the file SDDS.m.  To include these routines 
in your Mathematica program, put this file in your working directory 
and use the following line in your Mathematica program: 
\begin{verbatim} 
Needs["SDDS`"]; 
\end{verbatim} 
 
The programs are: 
\begin{itemize} 
\item \verb|SDDSRead[filename_String]|---returns an SDDS structure from a file. 
\item \verb|SDDSWrite[sdds_,filename_String]|---writes an SDDS structure to a file. 
\item \verb|SDDSGetColumnDefinitions[sdds_]|---returns the list of column definitions. 
\item \verb|SDDSGetParameterDefinitions[sdds_]|---returns the list of parameter definitions. 
\item \verb|SDDSGetArrayDefinitions[sdds_]|---returns the list of array definitions. 
\item \verb|SDDSGetAssociates[sdds_]|---returns the list of associates. 
\item \verb|SDDSGetTable[sdds_,n_:1]|---returns the nth table {parameters,data}. 
\item \verb|SDDSGetParameters[sdds_,n_:1]|---returns the parameters from the nth table. 
\item \verb|SDDSGetParameter[sdds_,p_String,n_:1]|---returns the value of parameter p from the nth table. 
\item \verb|SDDSGetData[sdds_,n_:1]|---returns the data matrix from the nth table. 
\item \verb|SDDSGetColumn[sdds_,c_String,n_:1]|---returns the column named c from the nth table. 
\item \verb|SDDSGetColumn[sdds_,m_,n_:1]|---returns the mth column from the nth table. 
\item \verb|SDDSGetRow[sdds_,m_,n_:1]|---returns the mth row from the nth table. 
\item \verb|SDDSGetNColumns[sdds_]|---returns the number of columns. 
\item \verb|SDDSGetNParameters[sdds_]|---returns the number of parameters. 
\item \verb|SDDSGetNArrays[sdds_]|---returns the number of arrays. 
\item \verb|SDDSGetNAssociates[sdds_]|---returns the number of associates. 
\item \verb|SDDSGetNTables[sdds_]|---returns the number of tables. 
\item \verb|SDDSGetNRows[sdds_,n_:1]|---returns the number of rows in the nth table. 
\item \verb|SDDSGetColumnNames[sdds_]|---returns the list of column names. 
\item \verb|SDDSGetParameterNames[sdds_]|---returns the list of parameter names. 
\item \verb|SDDSGetArrayNames[sdds_]|---returns the list of array names 
\item \verb|SDDSGetAssociateNames[sdds_]|---returns the list  
of associate names. 
\end{itemize} 
 
\item {\bf examples:}  
Convert a snapshot to a Mathematica file. 
\begin{flushleft}{\tt 
sdds2math par.050695.snap par.050695.m 
}\end{flushleft} 
 
\item {\bf synopsis:}  
\begin{flushleft}{\tt 
sdds2math [{\em SDDSfilename}] [{\em outputname}]  
   [-pipe[=input][,output]] [-comments] [-verbose] 
   [-format={\em printfString}] 
}\end{flushleft} 
 
\item {\bf switches:} 
    \begin{itemize} 
    \item \verb|pipe| --- The standard SDDS Toolkit pipe option. 
    \item \verb|comments| --- Put helpful Mathematica comments in the file. 
    \item \verb|verbose| --- Write header information to the terminal like sddsquery. 
    \item \verb|format| --- Format for doubles (Default: \%g) 
    \end{itemize} 
\item {\bf author:} K. Evans, Jr., ANL/APS. 
\end{itemize} 
 

% $Log: not supported by cvs2svn $
%
%\begin{latexonly}
\newpage
%\end{latexonly}
\subsection{sdds2plaindata}
\label{sdds2plaindata}

\begin{itemize}
\item {\bf description:} Converts SDDS data to a plain data file with simple formatting.
\item {\bf example:} 
\begin{flushleft}{\tt
sdds2plaindata data.input data.output -outputMode=binary "-separator= " -parameter=time -column=x -column=y
}\end{flushleft}
\item {\bf synopsis:}
\begin{flushleft}{\tt
sdds2plaindata [{\em Inputfile}] [{\em Outputfile}] [-pipe[=in][,out]] 
[-outputMode=<ascii|binary>]
[-separator={\em string}]
[-noRowCount]
[-order=<rowMajor|columnMajor>]
[-parameter={\em name},[,format={\em string}] ...]
[-column={\em name},[,format={\em string}] ...]
[-nowarnings]
}\end{flushleft}
\item {\bf files: }
{\em Inputfile} is the SDDS input file.

{\em Outputfile} is a file that is similar to SDDS files in that it contains parameter and column data. However this file does not contain SDDS header information. The column data does not need to be preceded by a row count but it is recommended. Also the column data can be separated by a user supplied string. Binary plaindata files are also allowed.

\item {\bf switches:}
    \begin{itemize}
    \item {\tt -pipe[=in][,out]} --- The standard SDDS Toolkit pipe option.
    \item {\tt -outputMode=<ascii|binary>} --- The plain data file can be written in ascii or binary format.
    \item {\tt -separator={\em string}} --- In ascii mode the columns can be separated by the given string.
    \item {\tt -noRowCount} --- The number of rows will not be included in the plain data file. 
	If binary mode is used the number of rows will always be written to the file.
    \item {\tt -order=<rowMajor|columnMajor>} --- Row major order is the default. 
	Here each row consists of one element from each column. 
	In column major order each column is written entirely on one row.
    \item {\tt -parameter={\em name},[,format={\em string}]} --- 
	Add this option for each parameter to add to the plain data file.
    \item {\tt -column={\em name},[,format={\em string}]} --- 
	Add this option for each column to add to the plain data file.
    \end{itemize}
\item {\bf see also:}
    \begin{itemize}
    \item \progref{plaindata2sdds}
    \end{itemize}
\item {\bf author:} R. Soliday, ANL/APS.
\end{itemize}


\begin{sddsprog}{sdds2spreadsheet}
\item \textbf{description:} \verb|sdds2spreadsheet| converts an SDDS file to a file that can be read into most spreadsheet programs. You need to consult your particular spreadsheet program to see how it reads ASCII files. For Wingz, the conversion is automatic. Excel 5.0 will bring up its Text Import Wizard.

  Notes:
  \begin{enumerate}
    \item Excel lines must be shorter than 255 characters. The Wingz delimiter can only be \verb|\t|.
    \item The program \verb|sddsprintout| with the \verb|-spreadSheet| option is intended to replace the function of \verb|sdds2spreadsheet|. It allows greater control of what data is output and how it is formatted.
  \end{enumerate}

\item \textbf{examples:}
  \begin{verbatim}
sdds2spreadsheet par.050695.snap par.050695.wkz
sdds2spreadsheet par.050695.snap p050695.txt
  \end{verbatim}

\item \textbf{synopsis:}
  \begin{verbatim}
sdds2spreadsheet [SDDSfilename] [outputname] [-pipe[=input][,output]] [-delimiter=string] [-all] [-verbose]
  \end{verbatim}

\item \textbf{switches:}
  \begin{itemize}
    \item \verb|pipe| --- The standard SDDS Toolkit pipe option.
    \item \verb|delimiter| --- Delimiter string (Default is \verb|\t|).
    \item \verb|all| --- Write parameter, column, and array information (Default is data and parameters only).
    \item \verb|verbose| --- Write header information to the terminal like sddsquery.
  \end{itemize}

\item \textbf{see also:} \progref{sddsprintout}

\item \textbf{author:} K. Evans, Jr., ANL/APS.
\end{sddsprog}


\begin{sddsprog}{sdds2stream}
  \item \textbf{description:}
    {\tt sdds2stream} provides stream output to the standard output of data values from a group of columns or parameters.
    Each line of the output contains a different row of the tabular data or a different parameter.
    Values from different columns are separated by the delimiter string.
    If \verb|-page| is not employed, all data pages are output sequentially.
    If multiple filenames are given, the files are processed sequentially in the order given.
  \item \textbf{examples:}
    \begin{verbatim}
sdds2stream APS.twi -parameters=nux,nuy -delimiter=" "
sdds2stream APS.twi -column=ElementName,betax -page=1
    \end{verbatim}
  \item \textbf{synopsis:}
    \begin{verbatim}
sdds2stream {inputFileList | -pipe[=input]} [-page=pageNumber] [-delimiter=delimitingString]
            { -columns=columnName[,columnName...] |
              -parameters=parameterName[,parameterName...] |
              -arrays=arrayName[,arrayName...] }
            [-filenames] [-rows[=bare]] [-npages[=bare]] [-noquotes]
            [-ignoreFormats] [-description]
    \end{verbatim}
  \item \textbf{files:}
    {\tt inputFileList} is a space-separated list of SDDS filenames.
  \item \textbf{switches:}
    \begin{itemize}
      \item {\tt -pipe[=input]} --- The standard SDDS Toolkit pipe option.
      \item {\tt -page=pageNumber} --- Specifies the number of the data page for which output is desired. Recall that pages are numbered sequentially beginning with 1. More complete control of which pages are output may be obtained using {\tt sddsconvert} or {\tt sddsprocess} as a filter.
      \item {\tt -delimiter=delimitingString} --- Specifies the delimiting string to be printed to separate row entries or parameters. The delimiter is printed with \verb|printf|, so that any of the usual escape sequences may be employed.
      \item {\tt -columns=columnName[,columnName...]} --- Specifies the names of the columns for which output is desired. For each row of each data page, the specified columns are printed on a single line, separated by the delimiting string. The default delimiting string is a single space.
      \item {\tt -parameters=parameterName[,parameterName...]} --- Specifies the names of the parameters for which output is desired. For each row of each data page, the specified parameters are printed on a single line, separated by the delimiting string. However, since the default delimiting string is a newline, the parameters end up on separate lines.
      \item {\tt -arrays=arrayName[,arrayName...]} --- Specifies the names of the arrays for which output is desired.
      \item {\tt -filenames} --- Specifies that the filename will be printed out as each file is processed.
      \item \verb|-rows[=bare]| --- Specifies that the number of rows per page for the tabular data section will be printed out. If the \verb|bare| qualifier is given, only the numerical values are printed, without the word ``rows.''
      \item \verb|-npages[=bare]| --- Specifies that the number of pages will be printed out. If the \verb|bare| qualifier is given, only the numerical values are printed, without the word ``pages.''
      \item \verb|-noquotes| --- Specifies that whitespace-containing string data will be printed without the default double-quotes.
      \item \verb|-ignoreFormats| --- Specifies that the format data supplied in the file is to be ignored. Guarantees printing of floating point data to full precision.
      \item \verb|-description| --- Specifies printing of the description data for the data set.
    \end{itemize}
  \item \textbf{see also:}
    \begin{itemize}
      \item \hyperref[exampleData]{Data for Examples}
      \item \progref{sddsprintout}
      \item \progref{sddsconvert}
      \item \progref{sddsprocess}
    \end{itemize}
  \item \textbf{author:} M. Borland, ANL/APS.
\end{sddsprog}


%\begin{latexonly}
\newpage
%\end{latexonly}
\subsection{sddsbaseline}
\label{sddsbaseline}

\begin{itemize}
\item {\bf description:}
\verb|sddsbaseline| performs baseline removal for SDDS column data.  Several methods of
determining the baseline are provided.
\item {\bf examples:}
Remove baselines from a video image organized with each scan line in a separate column.
The baseline is determined by looking at 10 points at either end of each line and averaging
the pixel count for these points.
\begin{flushleft}{\tt
sddsbaseline image.sdds image1.sdds -columns=VideoLine* -select=endpoints=10 -method=average 
}\end{flushleft}
\item {\bf synopsis:} 
\begin{flushleft}{\tt
sddsbaseline [{\em input}] [{\em output}] [-pipe=[in][,out]]
[-columns={\em listOfNames}]
-select=\{endPoints={\em number} | -outsideFWHA={\em multiplier} | -antiOutlier={\em passes}\}
-method=\{fit | average\}
[-nonnegative [-despike=passes={\em number},widthlimit={\em value}] [-repeats={\em count}]]
}\end{flushleft}
\item {\bf files:}
{\em input} is an SDDS file containing one or more pages of data to be processed.
{\em output} is an SDDS file in which the result is placed.  Columns that are not
processed are copied from {\em input} to {\em output} without change.
\item {\bf switches:}
    \begin{itemize}
    \item {\tt -pipe=[input][,output]} --- The standard SDDS Toolkit pipe option.
    \item {\tt -columns={\em listOfNames}} --- Specifies an optionally-wildcarded list
        of names of columns from which to remove baselines.
    \item {\tt -select=\{endPoints={\em number} | -outsideFWHA={\em multiplier} | -antiOutlier={\em passes\}}} --- Specifies how to select the points from which to determine the baseline.  \verb|endPoints| specifies selecting {\em number} values from the start and end of the column.  \verb|outsideFWHA| specifies selecting all values that are outside {\em multiplier} times the full-width-at-half-amplitude (FWHA) 
of the pixel count distribution.  \verb|antiOutlier| specifies selecting all values that are {\em not} deemed outliers in the 2-sigma sense in any of {\em passes} inspections.  These last two options implicitly assume that the statistical distribution of the pixel counts is baseline dominated.
    \item {\tt -method=\{fit | average\}} --- Specifies how to compute the baseline from the selected
    points.  \verb|fit| specifies fitting a line to the values (as a function of index).  \verb|average|
    specifies taking a simple average of the values.
    \item {\tt -nonnegative [-despike=passes={\em number},widthlimit={\em value}] [-repeats={\em count}]} --- Specifies that the resulting function after baseline removal must be nonnegative.  Any 
    negative values are set to 0.  In addition, despiking (as in \verb|sddssmooth|) may be applied
    after removal of negative values; this can result in the removal of positive noise spikes.
    Giving \verb|-repeats| allows applying the baseline removal procedure iteratively to the
    data.
    \end{itemize}
\item {\bf see also:}
    \begin{itemize}
    \item \progref{sddssmooth}
    \item \progref{sddscliptails}
    \end{itemize}
\item {\bf author:} M. Borland, ANL/APS.
\end{itemize}

\begin{sddsprog}{sddsbreak}
  \item \textbf{description:} \verb|sddsbreak| reads pages from an SDDS file and writes a new SDDS file containing the same data, but with each of the input pages potentially separated into several output pages. The separation involves breaking each input page at one or more locations as determined by one of several user-defined criteria.
  \item \textbf{examples:}
  \begin{verbatim}
sddsbreak par.bpm par.bpm1 -rowlimit=500
sddsbreak par.bpm par.bpm1 -gapin=Time,amount=15
  \end{verbatim}
  \item \textbf{synopsis:}
  \begin{verbatim}
sddsbreak [-pipe=[input][,output]] [inputFile] [outputFile]
  { -gapIn=columnName[, {amount=value | factor=value}] |
    -increaseOf=columnName | -decreaseOf=columnName |
    -changeOf=columnName[,amount=value[,base=value]] |
    -rowLimit=integer }
  \end{verbatim}
  \item \textbf{files:} \emph{inputFile} is an SDDS file containing one or more pages of data to be broken up. \emph{outputFile} is an SDDS file in which the result is placed. Each page of \emph{outputFile} contains the parameter and array values from the page of \emph{inputFile} that is its source.
  \item \textbf{switches:}
    \begin{itemize}
      \item {\tt -pipe=[input][,output]} --- The standard SDDS Toolkit pipe option.
      \item {\tt -gapIn=\emph{columnName}[,\{amount=\emph{value} | factor=\emph{value}\}]} --- Breaks the page when the value in the named column has a gap. If the \verb|amount| qualifier is given, then a gap is defined as any occurrence of successive values different by more than \emph{value}. If this qualifier is not given, then the \emph{value} is computed as follows: the mean absolute difference (MAD) between successive values for the first page which has more than 1 row is computed; if the {\tt factor} qualifier is given, then the gap amount is the MAD times the given value; otherwise, it is the MAD times two.
      \item {\tt -increaseOf=\emph{columnName}}, {\tt -decreaseOf=\emph{columnName}} --- These options cause a page break whenever the value in the named column increases or decreases, respectively.
      \item {\tt -changeOf=\emph{columnName}[,amount=\emph{value}[,base=\emph{value}]]} --- Breaks the page when the value in the named column changes. If the {\tt amount} qualifier is not given, then any change is sufficient to break the page. Otherwise, the page is broken whenever the quantity ${\rm \lfloor (V - B)/A \rfloor}$ changes, where V is the value in the column, A is the value given for {\tt amount}, and B is the value given for {\tt base}. If {\tt base} is not given, then the value in first row for the column is used.
      \item {\tt -rowLimit=\emph{integer}} --- Breaks the page after the specified number of rows.
    \end{itemize}
  \item \textbf{see also:}
    \begin{itemize}
      \item \hyperref[exampleData]{Data for Examples}
      \item \progref{sddscombine}
    \end{itemize}
  \item \textbf{author:} M. Borland, ANL/APS.
\end{sddsprog}

\begin{sddsprog}{sddscast}
  \item \textbf{description:} \verb|sddscast| converts numeric columns, parameters, or arrays from one data type to another within an SDDS file.
  \item \textbf{examples:}
\begin{verbatim}
sddscast APS.twi -cast=col,*,double,float
sddscast APS.twi -cast=col,'(col1,col2,col3)',double,float
sddscast APS.twi -cast=col,'(col1,col4,col5)','(double,float,float)',long
sddscast APS.twi '-cast=col,(col1,col4,col5),(double,float,float),long'
\end{verbatim}
  \item \textbf{synopsis:}
\begin{verbatim}
sddscast [inputFile] [outputFile]
         [-pipe=[input][,output]]
         [-noWarnings]
         [-majorOrder={row|column}]
         -cast={column|parameter|array},<names>,<typeNames>,<newType>
\end{verbatim}
  \item \textbf{files:}
  \begin{itemize}
    \item \emph{inputFile} \textendash{} SDDS file containing data to be cast.
    \item \emph{outputFile} \textendash{} SDDS file to receive the result; if omitted and no output pipe is given, the input is replaced.
  \end{itemize}
  \item \textbf{switches:}
  \begin{itemize}
    \item \verb+-cast={column|parameter|array},<names>,<typeNames>,<newType>+ \textendash{} Cast data types of specified entities.
    \item \verb|-pipe[=input][,output]| \textendash{} Use pipes for input and/or output.
    \item \verb|-noWarnings| \textendash{} Suppress warning messages.
    \item \verb+-majorOrder=row|column+ \textendash{} Set column data ordering.
  \end{itemize}
  \item \textbf{see also:}
  \begin{itemize}
    \item \progref{sddsprocess}
    \item \progref{sddsconvert}
  \end{itemize}
  \item \textbf{author:} H. Shang, ANL/APS.
\end{sddsprog}

%\begin{latexonly}
\newpage
%\end{latexonly}
\subsection{sddschanges}
\label{sddschanges}

\begin{itemize}
\item {\bf description:} {\tt sddschanges} analyzes changes in column data from page to page in a file,
relative to reference data in a baseline file or from the first page.  It requires that every page in the file have
the same number of rows.  It produces a multipage output file containing the row-by-row difference between the
reference data and the data each page in the input file.

\item {\bf examples:} 
Compute the changes in the dispersion function for several APS lattices:
\begin{flushleft}{\tt
sddschanges APS.twi APS.changes -copy=s -changesIn=betax,betay,etax
}\end{flushleft}
The output file in this example would have one fewer pages than the input
file.  Each page would contain the column s from the first page, along with
the differences from the first page for betax, betay, and etax.
One could also compute the changes relative to the nominal lattice:
\begin{flushleft}{\tt
sddschanges APS.twi -baseline=APS0.twi APS.changes -copy=s -changeIn=betax,betay,etax
}\end{flushleft}
The output file would have one page for every page in the input.
\item {\bf synopsis:}
\begin{flushleft}{\tt
sddschanges [-pipe[=input][,output]] [{\em inputFile}] [{\em outputFile}]
[-copy={\em columnNames}] [-changesIn={\em columnNames}]
[-baseline={\em referenceFileName} [-parallelPages]] 
}\end{flushleft}
\item {\bf files:}
      {\em inputFile} is a multipage file containing the data for which changes are
      desired.  {\em outputFile} is a multipage file containing the changes.  The
      column names in {\em outputFile} for the changes are created from those in
      {\em inputFile} by prepending the string ``ChangeIn''.
\item {\bf switches:}
    \begin{itemize}
    \item {\tt -pipe=[input][,output]} --- The standard SDDS Toolkit pipe option.
    \item {\tt -copy={\em columnNames}}--- Specifies that the named columns should be transferred
        to the output file without alteration.  These data come from the baseline
        file or from the first page of the input file.  A comma-separated list of optionally wildcard-containing
        strings may be given.
    \item {\tt -changesIn={\em columnNames}} --- Specifies that the named columns should be
        transferred to the output file after subtracting the corresponding values
        from the baseline file or from the first page of the input file.
        A comma-separated list of optionally wildcard-containing
        strings may be given.
    \item {\tt -baseline={\em referenceFileName}} --- Specifies the name of an SDDS file from which
        the reference data for changes should be taken.
    \item {\tt -parallelPages} --- Valid only with {\tt -baseline}.   Specifies that the ``baseline''
        data for each page of the input file shall be taken from the corresponding page of the
        baseline file.  This results in page-by-page subtraction of the two files.
    \end{itemize}
\item {\bf see also:}
    \begin{itemize}
    \item \hyperref[exampleData]{Data for Examples}
    \item \progref{sddsenvelope}
    \end{itemize}
\item {\bf author:} M. Borland, ANL/APS.
\end{itemize}


\begin{sddsprog}{sddscheck}
  \item \textbf{description:} \verb|sddscheck| is a simple tool to allow checking a file to see if it is a valid SDDS file or if it is corrupted. The primary use is in shell scripts that need to detect such conditions. \verb|sddscheck| issues one of four messages: \verb|ok|, \verb|nonexistent|, \verb|badHeader|, or \verb|corrupted|. (See \progref{sddsconvert} about recovering corrupted files.)
  \item \textbf{examples:}
    \begin{verbatim}
if (`sddscheck APS.twi` == "ok") plotTwissParameters APS.twi
    \end{verbatim}
    where \verb|plotTwissParameters| is a hypothetical plotting script.
  \item \textbf{synopsis:}
    \begin{verbatim}
sddscheck [-printErrors] filename
    \end{verbatim}
  \item \textbf{switches:}
    \begin{itemize}
      \item \verb|-printErrors| --- Causes the SDDS error traceback to be printed if the file is not \verb|ok|. This may be helpful in determining the problem with the file.
    \end{itemize}
  \item \textbf{files:} \emph{filename} is the name of a single file to be checked.
  \item \textbf{see also:} \progref{sddsconvert}
  \item \textbf{author:} M. Borland, ANL/APS.
\end{sddsprog}


\begin{sddsprog}{sddscliptails}
  \item \textbf{description:}
  \verb|sddscliptails| removes the tails from functions, where a tail is a dubious feature
  extending to the left or right of a peak.
  \item \textbf{examples:}
  Remove tails from profiles of beam spots after baseline removal and prior to determining
  rms spot properties. This command clips the tails when the function falls to 1\% of
  its peak value.
  \begin{verbatim}
sddscliptails input.sdds output.sdds -columns=VideoLine* -fractional=0.01
  \end{verbatim}
  \item \textbf{synopsis:}
  \begin{verbatim}
sddscliptails [input] [output] [-pipe=[in][,out]]
  [-columns=listOfNames] [-fractional=value] [-absolute=value] [-fwhm=multiplier]
  [-afterzero[=bufferWidth]]
  \end{verbatim}
  \item \textbf{files:}
  {\em input} is an SDDS file containing one or more pages of data to be processed.
  {\em output} is an SDDS file in which the result is placed. The output file will
  generally have fewer rows than the input file, corresponding to the number of
  rows clipped.
  \item \textbf{switches:}
    \begin{itemize}
    \item {\tt -pipe=[input][,output]} --- The standard SDDS Toolkit pipe option.
    \item {\tt -columns={\em listOfNames}} --- Specifies an optionally-wildcarded list
      of names of columns from which to remove tails.
    \item {\tt -fractional={\em value}} --- Clip a tail if it falls below this fraction of the peak
      value of the column.
    \item {\tt -absolute={\em value}} --- Clip a tail if it falls below this absolute value. This
      value might correspond, say, to a known noise level.
    \item {\tt -fwhm={\em multiplier}} --- Clip a tail if it is beyond {\em multiplier} times
      the full-width-at-half-maximum (FWHM) from the peak of the column.
    \item {\tt -afterzero[={\em bufferWidth}]} --- Clip a tail if it is separated from the
      peak by values equal to zero.
      If {\em bufferWidth} is specified, then a region {\em bufferWidth} wide is kept
      on either side of the peak, if possible.
    \end{itemize}
  \item \textbf{see also:}
    \begin{itemize}
    \item \progref{sddsbaseline}
    \end{itemize}
  \item \textbf{author:} M. Borland, ANL/APS.
\end{sddsprog}


\begin{sddsprog}{sddscollapse}
  \item {\bf description:} \verb|sddscollapse| reads data pages from an SDDS file and writes a new SDDS file containing a single data page. This data page contains only the values of the parameters from the original file, with each parameter forming a column of the tabular data. In spite of the simplicity of the command line, this is an extremely useful program. A typical use might involve processing a multipage file using \verb|sddsprocess| to, for example, obtain statistical analyses of columns for each page; the results of such analyses are placed in parameters. Using \verb|sddscollapse| on this file would produce columns of statistical analyses, with one row for each page. One might then further analyze the data using \verb|sddsprocess|. One could also use \verb|sddscombine| to combine several collapsed, processed data sets into a single file, which puts one formally back in the same position as when one started. In this fashion, multi-level data analysis and collation is possible. This is done with some magnetic measurements at APS.

  \item {\bf examples:}
  \begin{verbatim}
  sddscollapse APS.twi APS.parameters
  sddscollapse APS.twi -pipe=out | sddspfit -pipe=in fit.sdds -column=nux,nuy -verbose
  \end{verbatim}

  \item {\bf synopsis:}
  \begin{verbatim}
  sddscollapse [inputFile] [outputFile] [-pipe[=input][,output]]
  \end{verbatim}

  \item {\bf files:} \verb|inputFile| is the name of an SDDS data set to be collapsed.  \verb|outputFile| is the result.  Note that \verb|outputFile| will not contain any information on the arrays or columns that are in \verb|inputFile|.

  \item {\bf switches:}
  \begin{itemize}
    \item \verb|-pipe[=input][,output]| --- The standard SDDS Toolkit pipe option.
    \item \verb|-noWarnings| --- Suppresses warnings about file overwrites.
  \end{itemize}

  \item {\bf see also:}
  \begin{itemize}
    \item \hyperref[exampleData]{Data for Examples}
    \item \progref{sddsprocess}
    \item \progref{sddscombine}
    \item \progref{sddsexpand}
  \end{itemize}

  \item {\bf author:} M. Borland, ANL/APS.
\end{sddsprog}


\begin{sddsprog}{sddscollect}
  \item \textbf{description:}
    \verb|sddscollect| reorganizes tabular data from the input file to bring data from several groups of
    similarly named columns together into a single column per group. In doing so it creates one page
    of output for every row on the input file and creates parameters to hold data from columns that
    are not included in any group.
  \item \textbf{examples:}
    Take a data set with several columns of PAR BPM x and y readings (one set per row) and create a
    new file with one column for x readings and one column for y readings, with each page containing
    one set of readings.
    \begin{verbatim}
sddscollect par.bpm par.orbits -collect=suffix=x -collect=suffix=y
    \end{verbatim}
    The output file has three columns, \verb|x|, \verb|y|, and \verb|Rootname|, the last containing the
    original column names without the suffix.
    Do statistics on PAR BPM x and y readbacks, then collect the statistics into columns, one column
    for each type of statistic:
    \begin{verbatim}
sddsprocess par.bpm -pipe=out -process=P?P?[xy],spread,%sSpread \
  -process=P?P?[xy],ave,%sMean -process=P?P?[xy],stand,%sStDev | sddscollapse | \
sddscollect -pipe=in parbpm.stat \
  -collect=suffix=xSpread -collect=suffix=xMean -collect=suffix=xStDev \
  -collect=suffix=ySpread -collect=suffix=yMean -collect=suffix=yStDev
    \end{verbatim}
    The output file has columns \verb|xSpread|, \verb|ySpread|, \verb|xMean|, \verb|yMean|,
    \verb|xStDev|, and \verb|yStDev|, plus \verb|Rootname| containing the remainder of each original
    column name after the suffix is removed. All collections must produce matching sets of name
    remainders.
  \item \textbf{synopsis:}
    \begin{verbatim}
sddscollect [input] [output] [-pipe[=input][,output]] \
  -collect={suffix|prefix|match}=match[,column=newName][,editCommand=<string>]
    \end{verbatim}
  \item \textbf{switches:}
    \begin{itemize}
      \item \verb|-pipe[=input][,output]| --- The standard SDDS Toolkit pipe option.
      \item \verb!-collect={suffix|prefix|match}=match[,column=newName][,editCommand=<string>]! --- Specifies
        a string, \emph{match}, to look for at the end (suffix mode), beginning (prefix mode), or
        anywhere (match mode) in column names. Data from all matching columns is transferred into a
        single column in the output file. For prefix and suffix modes, the default name of this
        column is the suffix or prefix string, while for match mode it is created by applying the
        edit command to the matching column names. This may be changed with the \verb|column|
        qualifier. All collections must produce the same number of matches and the set of name
        remainders (i.e., the original column name less the prefix or suffix, or following editing)
        must be the same for each collection.
    \end{itemize}
  \item \textbf{files:}
    Input and output are SDDS files. Each output page corresponds to one input row.
  \item \textbf{see also:}
    \begin{itemize}
      \item \progref{sddsregroup}
      \item \progref{sddstranspose}
      \item \progref{sddsEditing}
    \end{itemize}
  \item \textbf{author:} M. Borland, ANL/APS.
\end{sddsprog}


\begin{sddsprog}{sddscombine}
  \item \textbf{description:} \verb|sddscombine| combines data from a series of SDDS files into a single SDDS file with one page for each page in each file. Data is added from files in the order that they are listed on the command line. All of the data files must contain the columns and parameters contained by the first; the program ignores any columns or parameters in a subsequent data file that are not in the first data file.

  \item \textbf{examples:}
    \begin{verbatim}
    sddscombine APS1.twi APS2.twi APS3.twi APSall.twi
    \end{verbatim}

  \item \textbf{synopsis:}
    \begin{verbatim}
    sddscombine [inputFileList] [outputFile] [-pipe[=input][,output]]
               [-merge=parameterName] [-overWrite] [-sparse=integer]
               [-collapse]
               [-delete={columns | parameters | arrays},matchingString[,...]]
               [-retain={columns | parameters | arrays},matchingString[,...]]
               [-xzLevel=integer]
    \end{verbatim}

  \item \textbf{files:} \verb|inputFileList| is a list of space-separated filenames to be combined. \verb|outputFile| is a filename into which the combined data is placed. If no \verb|-pipe| options are given, the \emph{outputFile} is taken as the last filename on the command line. To specify an output file with input from a pipe, one uses \verb|sddscombine| \verb|-pipe=input| \emph{outputFile}. Similarly, to specify output to a pipe with many input files, use \verb|sddscombine| \verb|-pipe=output| \emph{inputFileList}. Since accidentally leaving off the \verb|-pipe=output| option for the last command might result in replacement of an intended input file, the program refuses to overwrite an existing file unless the \verb|-overWrite| option is given. A string parameter (\verb|Filename|) is included in \emph{outputFile} to show the source of each page.

  \item \textbf{switches:}
    \begin{itemize}
      \item \verb|-pipe[=input][,output]| --- The standard SDDS Toolkit pipe option.
      \item \verb|-merge=parameterName| --- Specifies that all pages of all files are to be merged into a single page of the output file. If a \verb|parameterName| is given, successive pages are merged only if the value of the named parameter is the same.
      \item \verb|-overWrite| --- Forces \verb|sddscombine| to overwrite \emph{outputFile} if it exists.
      \item \verb|-sparse=integer| --- Specifies sparsing the tabular data in the input to retain only every \emph{integer}-th row.
      \item \verb|-collapse| --- Specifies collapsing the data, as done by \verb|sddscollapse|.
      \item \verb|-xzLevel=integer| --- Sets the LZMA compression level when writing \verb|.xz| files.
      \item {\tt -delete=\{columns | parameters | arrays\},matchingString[,...]},\\
            {\tt -retain=\{columns | parameters | arrays\},matchingString[,...]} --- These options specify wildcard strings to select entities (i.e., columns, parameters, or arrays) that will respectively be deleted or retained. The selection is performed by determining which input entities have names matching any of the strings. If \verb|retain| is given but \verb|delete| is not, only those entities matching one of the \verb|retain| strings are retained. If both \verb|delete| and \verb|retain| are given, then all entities are retained except those that match a \verb|delete| string without matching any of the \verb|retain| strings.
    \end{itemize}

  \item \textbf{see also:}
    \begin{itemize}
      \item \hyperref[exampleData]{Data for Examples}
      \item \progref{sddscollapse}
    \end{itemize}

  \item \textbf{author:} M. Borland, ANL/APS.
\end{sddsprog}


\begin{sddsprog}{sddscongen}
  \item \textbf{description:}
    Creates an SDDS data set by evaluating an \verb|rpn| expression over a defined 2 dimensional grid. This data set may be plotted using \verb|sddscontour|.
  \item \textbf{examples:}
\begin{verbatim}
sddscongen example.sdds -xRange=-1,1,101 -yRange=-1,1,101
-zEquation="x x * y y * + 4 * pi * sin"
sddscontour example.sdds -shade example.sdds -equalAspect
\end{verbatim}
  \item \textbf{synopsis:}
\begin{verbatim}
sddscongen outputfile -xRange=lower,upper,nPoints -yRange=lower,upper,nPoints
  -zEquation=rpnExpression [-rpnCommand=rpnExpression]
  [-rpnDefinitions=rpn-defnsFile]
\end{verbatim}
  \item \textbf{switches:}
    \begin{itemize}
      \item {\tt xRange=lower,upper,nPoints}, {\tt yRange=lower,upper,nPoints} --- Specifies the 2 dimensional grid over which data is generated. x is the horizontal variable and y the vertical.
      \item {\tt -zEquation=rpnExpression} --- Specifies the \verb|rpn| expression that is evaluated at each point of the grid.
      \item {\tt -rpnCommand=rpnExpression} --- Specifies the name of a file containing \verb|rpn| input. The named file is read before any other operations are performed.
      \item {\tt -rpnDefinitions=rpn-defnsFile} --- Specifies a string to submit to \verb|rpn| prior to beginning evaluation of the equation on the grid.
    \end{itemize}
  \item \textbf{see also:}
    \begin{itemize}
      \item \progref{sddscontour}
      \item \progref{rpn}
    \end{itemize}
  \item \textbf{author:} M. Borland, ANL/APS.
\end{sddsprog}


\begin{sddsprog}{sddscontour}
  \item {\bf description:}
  \verb|sddscontour| makes contour and color-map plots from an SDDS data set column, or from a \verb|rpn| expression
  in terms of the values in the columns of a data set. It supports FFT interpolation and filtering. If the
  data set contains more than one data page, data from successive pages is plotted on separate pages.

  \item {\bf examples:}
  This will generate a two-dimensional color-shaded map of the function ${\rm sin(4\pi (x^2 + y^2))}$ on
  the region x:[-1, 1] and y:[-1, 1]:
  \begin{verbatim}
sddscongen example.sdds -xRange=-1,1,101 -yRange=-1,1,101 \\
  -zEquation="x x * y y * + 4 * pi * sin"
sddscontour example.sdds -shade example.sdds -equalAspect
  \end{verbatim}
  \item {\bf synopsis:}
  \begin{verbatim}
sddscontour SDDSfilename switches
  \end{verbatim}
  \item {\bf switches:}
    \begin{itemize}
    \item Choice of what to plot:
\begin{flushleft}{\tt
[{-quantity={\em columnName} | -equation={\em rpnExpression} | 
 -columnMatch={\em indepColumnName},{\em matchingExpression}} |
 -waterfall={\em parameter=<parameter>,independentColumn=<xColumn>, 
             colorColumn=<colorColumn>[,scroll=vertical|horizontal]}]
 -xyz={\em xColumnName},{\em yColumnName},{\em zColumnName} 
}\end{flushleft}
\begin{itemize}
        \item \verb|quantity| --- Specifies the name of the column to make a contour or color map of.
        \item \verb|equation| --- Specifies a \verb|rpn| expression to make a contour or color map of.
        The expression may refer to the values in the columns by the appropriate column name, and may
        also refer to the variable values by name.
        \item \verb|columnMatch| --- Specifies plotting of all columns matching {\em matchingExpression}
        as a function of the column {\em indepColumnName}.  Each matching column is displayed as a horizontal 
        color bar. 
        \item \verb|waterfall| --- Specifies plotting of {\em colorColumn} in all pages as a function of the
        {\em independentColumn}. The {\em parameter} in each page is displayed as horizontal or vertical (provided
        by the {\em scroll}) bar, the default is horizontal. The {\em independentColumn} should be the same in
        each page.
        \item \verb|xyz| --- Specifies the names of two independent columns {\em xColumnName} and {\em yColumnName}, along with 
          a third dependent column {\em zColumnName} that is plotted as a function of the others.
          The x and y values must form a grid.
        \end{itemize}

        In the case of the first two choices, the file must contain
tabular data with at least one numeric column, which will be organized
into a 2d array with R rows and C columns.  By default, the values are
assumed to come in row-major order (i.e., the file should contain a
series of R sequences each containing the C values of a single row).
The parameters of the 2d grid over which the plot is to be made are
communicated to the program in one of two ways:

\begin{enumerate}

\item The string parameters \verb|Variable1Name| and \verb|Variable2Name| contain the names of the 
x and y axis variables, which I'll represent as {\em x} and {\em y} respectively.  The program expects to find
six more parameters, with names {\em x}\verb|Minimum|, {\em x}\verb|Interval|, and {\em x}\verb|Dimension|,
and similarly for {\em y}.  These parameters must be numeric, and contain the minimum value, the interval
between grid points, and the number of points, respectively, for the dimension in question.
The data must be arranged so that {\em y} varies fastest as the row in the file increases.  Put another
way, variable 1 is the row index and variable 2 is the column index.
\item The numeric parameters \verb|NumberOfRows| and \verb|NumberOfColumns| contain the values of R and
C, respectively.
\end{enumerate}

    \item \verb|rpn| control:
\begin{flushleft}{\tt
[-rpnDefinitionsFiles={\em filename}[,{\em filename}...]]
[-rpnExpressions={\em setupExpression}[,{\em setupExpression}...]]
}\end{flushleft}
        \begin{itemize}
        \item \verb|rpnDefinitionsFiles| --- Specifies the names of files containing \verb|rpn| expressions
        to be executed before any other processing takes place.
        \item \verb|rpnExpressions| --- Specifies \verb|rpn| expressions to be executed before any other processing
         takes place, immediately after any definitions files.
        \end{itemize}
    \item Shade and contour control:
\begin{flushleft}{\tt
\{-shade={\em number}[,{\em min},{\em max},gray] | -contours={\em number}[,{\em min},{\em max}]\}
[-labelContours={\em interval}[,{\em offset}]]
}\end{flushleft}
        \begin{itemize}
        \item \verb|shade| --- Specifies that a color (or grey-scale) map should be produced, with the
        indicated {\em number} of shades mapped onto the range from {\em min} to {\em max}.  If {\em min}
        and {\em max} are not given, they are taken to be equal to the minimum and maximum data values.
        \item \verb|contours| --- Specifies that contour lines should be drawn, with the 
        indicated {\em number} of lines for  the range from {\em min} to {\em max}.  If {\em min}
        and {\em max} are not given, they are taken to be equal to the minimum and maximum data values.
        \item \verb|labelContours| --- Specifies that every {\em interval}$ {th}$ contour line, starting with
        the {\em offset}$ {th}$ line, should be labeled with the contour value.
        \end{itemize}
    \item Image processing:
\begin{flushleft}{\tt
[-interpolate={\em nx},{\em ny}[,{floor | ceiling | antiripple}]] [-filter={\em xcutoff},{\em ycutoff}]
}\end{flushleft}
        \begin{itemize}     
        \item \verb|interpolate| --- Specifies that the 2d map should be interpolated to have {\em nx} times
        more rows (or x grid points) and {\em ny} times more columns (or y grid points).  Since FFTs are used to
        do the interpolation, the original number of grid points must be a power of 2, as must the factor.  Giving
        a factor of 1 disables interpolation for the dimension in question.  \verb|floor|, \verb|ceiling|,
        and \verb|antiripple| specify image processing of the interpolated map.  \verb|floor| and \verb|ceiling|
        respectively force values below (above) the minimum (maximum) value of the data to be set equal to that
        value.  \verb|antiripple| causes the map to be altered so that non-zero values in the new map between
        zero values on the original map are set to zero; this suppresses ripples that sometimes occur in regions
        where the data was originally all zero.
        \item \verb|filter| --- Applies low-pass filters to the data with the specified normalized cutoff 
        frequencies.  The integer cutoff values give the number of frequencies starting at the Nyquist frequency
        that are to be eliminated.  
        \end{itemize}
    \item Plot labeling:
\begin{flushleft}{\tt
[-xLabel={\em string|@<parameter-name>}] [-yLabel={\em string|@<parameter-name>}] 
[-title={\em string|@<parameter-name>|file[,edit=<string>]}] [-topline={\em string|@<parameter-name>|file[,edit=<string>]}] 
[-topTitle] [-noLabels] [-noScales] [-dateStamp] 
}\end{flushleft}
        \begin{itemize}
        \item \verb|xLabel|, \verb|yLabel|, \verb|title|, \verb|topline| --- These specify strings to be placed
                in the various label locations on the plot. If @<parameter-name> is provided, the value of given parameter will be printed; If {\em-topline=file[,edit=<string>]} or {\em-title=file[,edit=<string>]} option is provided, then the input file name or edited file name (if edit command is also provided) will be printed to the topline or title.
        \item \verb|topTitle| --- Requests that the title label be placed at the top of the plot, rather than
                at the bottom.
        \item \verb|noLabels| --- Requests that no labels be placed on the plot.
        \item \verb|noScales| --- Requests omission of the numeric scales.
        \item \verb|noBorder| --- Requests omission of the border around the data.  Implies \verb|-no_scales|.
        \item \verb|dateStamp| --- Requests that the date and time be placed on the pot.
        \end{itemize}
    \item Plot tick labeling: (only valid for -columnMatch plot)
\begin{flushleft}{\tt
[-xrange=mimum={\em value}|@{\em parameterName},maximum={\em value}|@{\em parameterName}] 
[-yrange=mimum={\em value}|@{\em parameterName},maximum={\em value}|@{\em parameterName}]
[-xaxis=scaleValue=<value>|scaleParameter=<name>
   [,offsetValue=<number>|offsetParameter=<name>] 
[-yaxis=scaleValue=<value>|scaleParameter=<name>
   [,offsetValue=<number>|offsetParameter=<name>]
}\end{flushleft}
        \begin{itemize} 
        \item \verb|xrange| --- specifies the minimum and maximum value of x axis, the value can be provided or obtained from parameters. If -xrange is provided, the indepentColumn will be ignored.
        \item \verb|yrange| --- specifies the minimum and maximum value of y axis, the value can be provided or obtained from parameters. If -yrange is provided, the y tick labels will be numberically labeled with provided range.
        \item \verb|yaxis| --- specifies the scale and offset value of y axis, the value can be provided or obtained from parameters. Only one of the -yrange and -yaxis can be provided. If -yaxis is provided, the y tick labels will be numberically labeled with provided scale and offset.
        \end{itemize}
        {\bf For example}, {\em origin1}, {\em delta1}, {\em max\_ext1}, {\em origin2}, {\em delta2} and {\em max\_ext2} are the parameters in \href{https://ops.aps.anl.gov/manuals/example_files/sddscontour.input1}{sddscontour.input1} file, {\em origin1}, {\em delta1} and {\em max\_ext1} represent the minimum, delta and maximum values of x coordinate, {\em origin2}, {\em delta2}, and {\em max\_ext2} represents the minimum, delta and maximum values of y coordinate. The {\em Index} column represents the index of x coordinate, i.e. value of {\em x=Index * delta1 + origin1}; The Ex\_{\em n} column represents the Ex field at {\em n}th y value, where {\em y=(n-1)*delta2 + origin2}. If no -xrange and -yrange provided as in following command, the actual value of x and y will not be shown in the plot. (click the show\_plot button will show you the corresponding plot.) 
       \begin{flushleft}{\tt \bf
            sddscontour sddscontour.input1 -columnMatch=Index,Ex* -ystring=sparse=10 -ylabel=y -shade
         \href{https://ops.aps.anl.gov/manuals/example_files/sddscontour1_img.html}{show\_plot}
        }\end{flushleft}

        We can use -ystring to remove the string part of y label as following:
        \begin{flushleft}{\tt \bf
            sddscontour sddscontour.input1 -columnMatch=Index,Ex* -ystring=sparse=10,edit=\%/Ex\_// -ylabel=y -shade    \href{https://ops.aps.anl.gov/manuals/example_files/sddscontour2_img.html}{show\_plot}
        }\end{flushleft}

        The above y tick label still shows the index of y coordinate, not y values. Following command allows us to see the y values:
        \begin{flushleft}{\tt \bf
            sddscontour sddscontour.input1 -columnMatch=Index,Ex* -yrange=minimum=@origin2,maximum=@max\_ext2 -ylabel=y -shade          \href{https://ops.aps.anl.gov/manuals/example_files/sddscontour3_img.html}{show\_plot}
        }\end{flushleft}

        Now, for the x tick labels, the above plot shows the Index value. Following command will show the values of x coordinate:
    
        \begin{flushleft}{\tt \bf
        sddscontour sddscontour.input1 -column=Index,Ex* -yrange=min=@origin2,max=@max\_ext2 -xrange=min=@origin1,max=@max\_ext1 -xlabel=x -ylabel=y  -shade   \href{https://ops.aps.anl.gov/manuals/example_files/sddscontour4_img.html}{show\_plot}
        }\end{flushleft}  
        
        The independent column - Index in the above command is useless. Therefore, -xrange provides a way for plotting a set of columns with contour without indepent column. If use sddsprocess to create x column through {\em x=Index * delta1 + origin1}, the above plot can be created using following command, note that the titles in two plots are different because the independent column names are different since the title is automatically generated from input column names if it is not provided.

        \begin{flushleft}{\tt \bf
        sddscontour sddscontour.input1 -column=x,Ex* -yrange=min=@origin2,max=@max\_ext2 -ylabel=y -shade   \href{https://ops.aps.anl.gov/manuals/example_files/sddscontour5_img.html}{show\_plot}
        }\end{flushleft}  


       Here shows the examples of providing xrange and yrange from parameters, however, they can be provided by fixed values from commandline also.

    \item Data scaling:
\begin{flushleft}{\tt
[-deltas[=\{fractional | normalize\}]] [-logscale[={\em floor}]]
}\end{flushleft}
        \begin{itemize}
        \item \verb|deltas| --- For use with \verb|-columnMatch| and \verb|-waterfall| option.  Specifies plotting 
        only differential values (relative to the mean of each column).  If the \verb|fractional| 
        qualifier is given, then the differential values normalized to the individual
        means are plotted.  If the \verb|normalize| qualifier is given, then all differential values
        are normalized to the range [-1, 1] before plotting.
        \item \verb|logscale| --- Specifies plotting the base-10 logarithm of the values.  If a
        {\em floor} value is given, it is added to each value prior to taking the logarithm; this
        can help prevent taking the log of zero, for example.
        \end{itemize}
    \item Miscellaneous plot control:
\begin{flushleft}{\tt
[-scales={\em xl},{\em xh},{\em yl},{\em yh}] 
[-swapxy] [-equalAspect[={-1,1}]]
[-noBorder] [-layout={\em nx},{\em ny}]
[-ticksettings={xy}time] [-nocolorbar] 
[-drawLine=\{x0value={\em value} | p0value={\em value} | x0parameter={\em name} | p0parameter={\em name}\},
            \{x1value={\em value} | p1value={\em value} | x1parameter={\em name} | p1parameter={\em name}\},
            \{y0value={\em value} | q0value={\em value} | y0parameter={\em name} | q0parameter={\em name}\},
            \{y1value={\em value} | q1value={\em value} | y1parameter={\em name} | q1parameter={\em name}\}]
}\end{flushleft}
        \begin{itemize}
        \item \verb|scales| --- Specifies the extent of the plot region.
        \item \verb|swapxy| --- Requests that the horizontal and vertical coordinates be interchanged.
        \item \verb|equalAspect| --- Requests plotting with an aspect ratio of 1.  If the '1' qualifier
        is given, then the aspect ratio is achieved by changing the size of the plot region within the window;
        this is the default.
        If the '-1' qualifier is given, then the aspect ratio is achieved by changing the size of the plot region
        in user's coordinates.  
        \item \verb|noBorder| --- Specifies that no border will be placed around the graph.
        \item \verb|layout| --- Specifies that each page of the plot should have a {\em nx} by {\em ny} grid of contour plots.
        \item \verb|tickSettings| --- Specify use of time mode for tick settings.
        \item \verb|nocolorbar| --- Specify suppression of the color bar in \verb|-shade| mode.
        \item \verb|xaxis|, \verb|yaxis| --- Modifies the labels on the x or y axis, through scaling and offseting.
          The scale/offset values may be given literally or drawn from parameters in the data file.
        \item \verb|drawLine| --- Requests drawing of lines on the plot, using any combination of real coordinate values
          or plot-space values, either specified as literal values or drawn from parameters in the data file.
          Suitable for multi-page files.
        \end{itemize}
    \item Miscellaneous:
\begin{flushleft}{\tt
[-device={\em name}[,{\em deviceArguments}]] 
[-output={\em filename}] [-verbosity[=level]]
}\end{flushleft}
        \begin{itemize}
        \item \verb|device| --- Specifies the device name and optional device-specific arguments. Qt device
        arguments include \verb|-dashes <0|1>|, \verb|-linetype <filename>|, \verb|-movie 1| [\verb|-interval <sec>|],
        \verb|-keep <number>|, \verb|-share <name>|, \verb|-timeoutHours <hours>|, and \verb|-spectrum|. png devices
        take rootname and template identifiers. {\tt rootname={\em string}} specifies a rootname
        for automatic filename generation; the resulting filenames are of the form {\em rootname}.DDD, where DDD
        is a three-digit integer. {\tt template={\em string}} provides a more general facility; one uses it to
        specify an sprintf-style format string to use in creating filenames. For example, the behavior obtained
        using {\tt rootname={\em name}} may be obtained  using {\tt template={\em name}.\%03ld}.
        \item \verb|output| --- Requests SDDS output of a new file containing the data with any modifications
                resulting in the processing requested.
        \item \verb|verbosity| --- Sets the verbosity level of informational printouts. Higher integer values
                of the \verb|level| parameter result in more output.
        \end{itemize}
    \end{itemize}
  \item {\bf see also:}
    \begin{itemize}
      \item \progref{sddscongen}
      \item \progref{sddshist2d}
      \item \progref{sddsimageconvert}
      \item \progref{sddsimageprofiles}
      \item \progref{sddsplot}
      \item \progref{sddsspotanalysis}
      \item \progref{rpn}
    \end{itemize}
  \item {\bf author:} M. Borland, ANL/APS.
\end{sddsprog}


\begin{sddsprog}{sddsconvert}
  \item \textbf{description:} \verb|sddsconvert| converts SDDS files between ASCII and binary and allows wildcard-based filtering out of unwanted columns and/or rows, as well as renaming of columns. N.B.: it is \emph{not} recommended to use \verb|sddsconvert| to convert a binary SDDS file to ASCII, then strip the header off and read the ASCII file. This completely bypasses the self-describing aspects of the SDDS file and is not robust. If the program that creates the SDDS file is changed so that the columns are created in a different order, the program that reads the ASCII file will produce unexpected results. Use \progref{sdds2plaindata}, \progref{sddsprintout}, or \progref{sdds2stream} for conversion to non-self-describing files. In this way, you can assure the order of the data is fixed.

  \item \textbf{examples:}
  Convert \verb|APS.twi| to binary:
  \begin{verbatim}
  sddsconvert -binary APS.twi
  \end{verbatim}
  Convert \verb|APS.twi| to binary and delete the \verb|alphax| and \verb|alphay| columns:
  \begin{verbatim}
  sddsconvert -binary APS.twi -delete=column,'alpha?'
  \end{verbatim}

  \item \textbf{synopsis:}
  \begin{verbatim}
  sddsconvert [inputFile] [outputFile] [-pipe[=input][,output]] \
    [-binary | -ascii] [-fromPage=number] [-toPage=number] \
    [-delete={columns | parameters | arrays},matchingString[,matchingString...]] \
    [-retain={columns | parameters | arrays},matchingString[,matchingString...]] \
    [-rename={columns | parameters | arrays},oldname=newname[,oldname=newname...]] \
    [-editNames={columns | parameters | arrays},matchingString,editString] \
    [-description=text,contents] \
    [-recover[=clip]] [-linesPerRow=number] [-nowarnings] [-acceptAllNames] \
    [-xzLevel=integer]
  \end{verbatim}

  \item \textbf{files:}
  \emph{inputFile} is an SDDS file containing data to be processed. The \emph{outputFile} argument is optional. If it is not given, and if an output pipe is not selected, then the input file will be replaced.

  \item \textbf{switches:}
    \begin{itemize}
    \item {\tt \{-binary | -ascii\}} --- Requests that the output be binary or ASCII.
    \item {\tt fromPage=\emph{number}} --- Specifies the first data page of the file that will appear in the output. By default, the output starts with data page~1.
    \item {\tt toPage=\emph{number}} --- Specifies the last page of the file that will appear in the output. By default, the output ends with the last data page in the file.
    \item {\tt -delete=\{columns | parameters | arrays\},\emph{matchingString}[,\emph{matchingString}...]}, {\tt -retain=\{columns | parameters | arrays\},\emph{matchingString}[,\emph{matchingString}...]} --- These options specify wildcard strings to be used to select entities (i.e., columns, parameters, or arrays) that will respectively be deleted or retained (i.e., that will not or will appear in the output). The selection is performed by determining which input entities have names matching any of the strings. If \verb|retain| is given but \verb|delete| is not, only those entities matching one of the strings given with \verb|retain| are retained. If both \verb|delete| and \verb|retain| are given, then all entities are retained except those that match a \verb|delete| string without matching any of the \verb|retain| strings.
    \item {\tt -rename=\{columns | parameters | arrays\},\emph{oldname}=\emph{newname}[,\emph{oldname}=\emph{newname}...]} --- Specifies new names for entities in the output data set. The entities must still be referred to by their old names in the other commandline options.
    \item {\tt -editNames=\{columns | parameters | arrays\},\emph{matchingString},\emph{editString}} --- Specifies creation of new names for entities of the specified type with names matching the specified wildcard string. Editing is performed using commands reminiscent of \verb|emacs| keystrokes. For details on editing commands, see \progref{SDDS editing}.
    \item {\tt -description=\emph{text},\emph{contents}} --- Sets the description fields for the output.
    \item {\tt -recover[=clip]} --- Asks for attempted recovery of corrupted data. If the qualifier is given, then all data from a corrupted page is ignored. Otherwise, \verb|sddsconvert| tries to save as much data from the corrupted page as it can; typically, it is able to save part of the tabular data and all of the array and parameter data.
    \item {\tt -linesPerRow=\rm number} --- Sets the number of lines of text output per row of the tabular data, for ASCII output only.
    \item {\tt -noWarnings} --- Suppresses warning messages, such as file replacement warnings.
    \item {\tt -xzLevel} --- Sets the LZMA compression level when writing .xz files.
    \item {\tt -acceptAllNames} --- Forces acceptance of any name for an SDDS data element (e.g., a column). This can be used with the \verb|rename| or \verb|editNames| options to fix invalid names in SDDS files. This option is provided for backward compatibility to the original version of SDDS, which allowed arbitrary characters in element names.
    \end{itemize}

  \item \textbf{see also:}
    \begin{itemize}
    \item \hyperref[exampleData]{Data for Examples}
    \item \progref{sddsprocess}
    \item \progref{SDDS editing}
    \end{itemize}

  \item \textbf{author:} M. Borland, ANL/APS.
\end{sddsprog}

%\begin{latexonly}
\newpage
%\end{latexonly}
\subsection{sddsconvolve}
\label{sddsconvolve}

\begin{itemize}
\item {\bf description:} 
{\tt sddsconvolve} performs discrete Fourier convolution/deconvolution/correlation of
signals in two files.  It assumes that spacing of points is the same in both input files.
\item {\bf example:}
Compute the result of a signal applied to a system with a known impulse response.
\begin{flushleft}{\tt
sddsconvolve signal.sdds impulseResponse.sdds signalResponse.sdds
-signalColumns=t,VSignal -responseColumns=t,VImpulse -outputColumns=t,VOutput
}\end{flushleft}

\item {\bf synopsis:}
\begin{flushleft}{\tt
sddsconvolve  {\em signal-file} {\em response-file} {\em output} [-pipe[=in][,out]]
 -signalColumns={\em indepColumn},{\em dataName}
 -responseColumns={\em indepColumn},{\em dataName} [-reuse]
 -outputColumns={\em indepColumn},{\em dataName}  
[{-deconvolve [-noiseFraction={\em value}] | -correlate}]
}\end{flushleft}
\item {\bf files:}
The meaning of the files depends on whether the {\tt -deconvolve} or {\tt -correlate}
options are given.
If neither option is given, then
{\em signal-file} is the file containing the signal that is imposed on the system,
{\em response-file} is the impulse response of the system, and 
{\em output} is the computed response of the system to the signal.If {\tt -deconvolve} is given, then {\em signal-file} is the response of the
system to the signal, {\em response-file} is the impulse response of the system, and
{\em output} is the computed signal imposed on the system.
If {\tt -correlate} is given, then {\em signal-file} and {\em response-file} contain
two equivalent signals, while {\em output} contains the computed Fourier correlation;
physically, this tells over what time scale the two functions have correlated values.
\item {\bf switches:}
    \begin{itemize}
    \item {\tt -pipe=[input][,output] } --- The standard SDDS Toolkit pipe option.
        \item {\tt -signalColumns={\em indepColumn},{\em dataName}} --- Specifies the
        names of the data columns from {\em signal-file} (the first data file).
        \item {\tt -responseColumns={\em indepColumn},{\em dataName}} --- Specifies the
        names of the data columns from {\em response-file} (the second data file).
        \item {\tt -reuse} --- Specifies that the first page of the response file will
          be used with all pages of the signal file. Particularly useful if the
          response file has one page but the signal file has many.
        \item {\tt -outputColumns={\em indepColumn},{\em dataName}} --- Specifies the
        desired names of the result in the file {\em output}.
        \item {\tt -deconvolve} --- Specifies deconvolution instead of convolution.
        \item {\tt -noiseFraction={\em value}} --- Specifies the amount of noise to
        allow in the deconvolution to prevent division by zero, as a fraction of the
        maximum power in the impulse response function.
        \item {\tt -correlate} --- Specifies correlation instead of convolution.
    \end{itemize}
\item {\bf author:} M. Borland, ANL/APS.
\end{itemize}



\begin{sddsprog}{sddscorrelate}
\item \textbf{description:}
  {\tt sddscorrelate} computes correlation coefficients and correlation
  significance between column data. The correlation coefficient between
  columns i and j is defined as
  \[ {\rm C_{ij} = \frac{\langle (x_i-\langle x_i \rangle) (x_j-\langle x_j \rangle)\rangle}
  {\sqrt{\langle (x_i-\langle x_i \rangle)^2\rangle \langle (x_j - \langle x_j \rangle)^2 \rangle}}} \]
  If ${\rm C_{ij}=1}$, then the variables are perfectly correlated, whereas if ${\rm C_{ij}=-1}$,
  they are perfectly anticorrelated. The correlation significance is the probability that the
  observed correlation coefficient could happen by chance if the variables were in fact
  uncorrelated. Hence, a very small correlation significance means that the variables are
  probably correlated.
\item \textbf{examples:}
\begin{verbatim}
  sddscorrelate par.bpm par.cor -column='*x'
\end{verbatim}
\begin{verbatim}
  sddscorrelate par.bpm par.cor -column='*x' -withOnly=P1P1x
\end{verbatim}
\item \textbf{synopsis:}
\begin{verbatim}
  sddscorrelate [-pipe=[input][,output]] [inputFile] [outputFile]
    [-columns=columnNames] [-excludeColumns=columnNames]
    [-withOnly=columnName] [-rankOrder]
    [-stDevOutlier[=limit=factor][,passes=integer]]
\end{verbatim}
\item \textbf{files:}
  {\em inputFile} is an SDDS file containing two or more columns of data. For each page of
  the file, {\em outputFile} contains the correlation coefficients and significance for
  every possible pairing of variables requested. {\em outputFile} also contains three string
  columns: {\tt Correlate1Name}, {\tt Correlate2Name}, and {\tt CorrelatePair}. These are
  respectively the name first column in the analysis, the name of the second column in
  the analysis, and a string of the form {\em Name1}.{\em Name2}.
\item \textbf{switches:}
  \begin{itemize}
  \item {\tt -pipe=[input][,output]} --- The standard SDDS Toolkit pipe option.
  \item {\tt -columns={\em columnNames}} --- Specifies the names of columns to be included in the analysis.
    A comma-separated list of optionally wildcard-containing names may be given.
  \item {\tt -excludeColumns={\em columnNames}} --- Specifies the names of columns to be excluded from the
    analysis. A comma-separated list of optionally wildcard-containing names may be given.
  \item {\tt -withOnly={\em columnName}} --- Specifies that one of the variables for each correlation will be
    the named column.
  \item {\tt -rankOrder} --- Specifies computing rank-order correlations rather than standard correlations.
    This is considered more robust than standard correlations.
  \item {\tt -stDevOutlier[=limit={\em factor}][,passes={\em integer}]} --- Specifies standard-deviation-based
    outlier elimination on each pair of columns prior to computation of the correlation coefficient.
    Any pair of values is ignored if one or both values are outliers relative to the column from which they come.
    The {\tt limit} qualifier specifies the allowed deviation from the mean in standard deviations; the
    default is 1. The {\tt passes} qualifier specifies how many times the outlier elimination (including
    recomputation of the mean and standard deviation) is performed; the default is 1.
  \end{itemize}
\item \textbf{see also:}
  \begin{itemize}
  \item \hyperref[exampleData]{Data for Examples}
  \item \progref{sddsprocess}
  \end{itemize}
\item \textbf{author:} M. Borland, ANL/APS.
\end{sddsprog}


\begin{sddsprog}{sddsderiv}
  \item \textbf{description:} \verb|sddsderiv| differentiates one or more columns of data as a function of a common column. The program will perform error propagation if error bars are provided in the data set.
  \item \textbf{examples:}
    \begin{verbatim}
    sddsderiv bessel.sdds bessel.deriv -differentiate=J0,J1 -versus=z
    \end{verbatim}
  \item \textbf{synopsis:}
    \begin{verbatim}
    sddsderiv [-pipe=[input][,output]] [input] [output]
      -differentiate=columnName[,sigmaName] ...
      -versus=columnName[,sigmaName] [-interval=integer]
      [-SavitzkyGolay=left,right,fitOrder[,derivOrder]]
      [-mainTemplates=item=string[,...]]
      [-errorTemplates=item=string[,...]]
    \end{verbatim}
  \item \textbf{files:} {\em input} is an SDDS file containing columns of data to be differentiated. If it contains multiple data pages, each is treated separately. The independent quantity along with the requested derivatives are placed in columns in {\em output}. By default, the derivative column name is constructed by appending \verb|Deriv| to the variable column name. If applicable, the column name for the derivative error is constructed by appending \verb|DerivSigma|. The data with respect to which the derivative is taken should be monotonically ordered.
  \item \textbf{switches:}
    \begin{itemize}
      \item \verb|-pipe[=input][,output]| --- The standard SDDS Toolkit pipe option.
      \item \verb|-differentiate=\emph{columnName}[,\emph{sigmaName}]| --- Specifies the name of a column to differentiate, and optionally the name of the column containing the error in the differentiated quantity. May be given any number of times.
      \item \verb|-versus=\emph{columnName}[,\emph{sigmaName}]| --- Specifies the name of the independent variable column, and optionally the name of the column containing its error.
      \item \verb|-interval=\emph{integer}| --- Specifies the spacing of the data points used to approximate the derivative. The default value of 2 specifies that the derivative for each point will be obtained from values 1 row above and 1 row below the point. In general (ignoring end points, which require special treatment):
        \[
        \frac{d y}{d x}[i] \approx \frac{y[i+\textit{Interval}/2] - y[i-\textit{Interval}/2]}{x[i+\textit{Interval}/2] - x[i-\textit{Interval}/2]}
        \]
      \item \verb|-SavitzkyGolay=\emph{left},\emph{right},\emph{fitOrder}[,\emph{derivOrder}]| --- Specifies using a Savitzky-Golay smoothing filter to perform the derivative, which involves fitting a polynomial of \emph{fitOrder} through \emph{left}+\emph{right}+1 points and then giving the derivative of the fitted curve. \emph{derivOrder} is 1 by default and gives the order of derivative to take.
      \item \verb|-mainTemplates=\emph{item}=\emph{string}[,...]| --- Specifies template strings for names and definition entries for the derivative columns in the output file. \emph{item} may be one of \verb|name|, \verb|description|, \verb|symbol|. The symbols ``\%x'' and ``\%y'' are used to represent the independent variable name and the name of the differentiated quantity, respectively.
      \item \verb|-errorTemplates=\emph{item}=\emph{string}[,...]| --- Specifies template strings for names and definition entries for the derivative error columns in the output file. \emph{item} may be one of \verb|name| or \verb|description|. The symbols ``\%x'' and ``\%y'' are used to represent the independent variable name and the name of the differentiated quantity, respectively.
    \end{itemize}
  \item \textbf{see also:} \progref{sddsinteg}
  \item \textbf{author:} M. Borland, ANL/APS.
\end{sddsprog}

%\begin{latexonly} 
\newpage 
%\end{latexonly} 
 
\subsection{sddsderef} 
\label{sddsderef} 
 
\begin{itemize} 
\item {\bf description:} 
\verb|sddsderef| allows array and column dereferencing based on constants 
or on data in parameters and columns. 
 
\item {\bf examples:}  
Let \verb|arrayData| be a file containing a string array named  
\verb|MessageText| 
and a column named \verb|MessageIndex| containing integers. 
The integers are indices into the string array.  To create a new column 
called \verb|Message| giving the message text for each index, the 
following command would be used: 
\begin{flushleft}{\tt  
sddsderef arrayData -column=Message,arraySource=MessageText,MessageIndex 
} 
\end{flushleft} 
\item {\bf synopsis:}  
\begin{flushleft}{\tt 
sddsderef [{\em input}] [{\em output}] [-pipe[=input][,output]] 
\\ -column={\em newName},\{columnSource | arraySource\}={\em name},{\em indexName}[,{\em indexName1}...] 
\\ -parameter={\em newName},\{columnSource | arraySource\}={\em name},
\hspace*{10mm}{\em indexName}[,{\em indexName1}...] 
\\ -constant={\em newName},\{columnSource | arraySource\}={\em name},
\hspace*{10mm} {\em indexValue}[,{\em indexValue}...] 
\\ \hspace*{0.1mm} [-outOfBounds=\{exit | delete\}]
}\end{flushleft} 
 
\item {\bf switches:} 
    \begin{itemize} 
   \item {\tt -pipe[=input][,output]} --- The standard SDDS Toolkit pipe option. 
    \item {\tt -column={\em newName},\{columnSource | arraySource\}={\em name},{\em indexName}} \\
        {\tt [,{\em indexName1}...]}
    --- Creates a new column named {\em newName} containing values 
    found by dereferencing the array (or column) {\em name} 
    using the index or indices in the named columns. 
    \item {\tt -parameter={\em newName},\{columnSource | arraySource\}={\em name},{\em indexName}}\\
         {\tt [,{\em indexName1}...]}
    --- Creates a new parameter named {\em newName} containing values 
    found by dereferencing the array (or column) {\em name} 
    using the index or indices in the named parameters.  A new value 
    is thus generated for the parameter for each page. 
    \item {\tt -constant={\em newName},\{columnSource | arraySource\}={\em name},{\em indexValue}}\\
         {\tt [,{\em indexValue}...]}  
    --- Creates a new parameter named {\em newName} containing values 
    found by dereferencing the array (or column) {\em name} using 
    the index or indices given (as explicit values).  Unless the array (or 
    column) varies from page to page, the new parameter will have the 
    same value on each page. 
    \item {\tt -outOfBounds=\{exit | delete\}} --- Specifies behavior in the event
        that an index value is out of bounds (i.e., less than 0, or greater than
        or equal to the number of array elements or rows).  If {\tt exit} is given,
        the program aborts with an error; this is the default behavior.
        If {\tt delete} is given, then the source row or page for the index
        is omitted from the output.
    \end{itemize} 
\item {\bf author:} M. Borland, ANL/APS. 
\end{itemize} 

%\begin{latexonly}
\newpage
%\end{latexonly}

\subsection{sddsdigfilter}
\label{sddsdigfilter}

\begin{itemize}
\item {\bf description:}

{\tt sddsdigfilter} performs time-domain digital filtering of columns of data. Filters can
be combined in series and/or cascade to produce complex filter characteristics. In
addition to allowing simple 1-pole lowpass and highpass filters, filter charateristics can
be defined using either digital 'Z' or analog 'S' domain transfer functions.

A digital filter has a Z transform given by
\[
\frac{b_0 + b_1 z^{-1} + \ldots + b_n z^{-n}}{a_0 + a_1 z^{-1} + \ldots + a_n z^{-n}},
\]
while an analog filter has a Laplace transform given by
\[
\frac{d_0 + d_1 s^{1} + \ldots + d_n s^{n}}{c_0 + c_1 s^{1} + \ldots + c_n s^{n}},
\]

\item {\bf examples:} 
These examples assume the existence of a file {\tt data.wf} containing a waveform
stored as a column {\tt value} that is a function of a column {\tt time} that has
units of seconds.

 Pass data through lowpass filter with a -3dB cutoff of 0.01 Hz:

  \begin{flushleft}{\tt
  sddsdigfilter data.wf -col=time,value result.wf -low=1,0.01.
  }\end{flushleft}

  Bandstop filter between 10 Hz and 100 Hz:

  \begin{flushleft}{\tt
  sddsdigfilter data.wf -col=time,value result.wf -low=1,10 -high=1,100
  }\end{flushleft}

  Bandpass filter between 10 Hz and 100 Hz:

  \begin{flushleft}{\tt
  sddsdigfilter data.wf -col=time,value result.wf -low=1,100 -cascade -high=1,10
  }\end{flushleft}

  Analog transfer function:

  \begin{flushleft}{\tt
  sddsdigfilter data.wf -col=time,value result.wf -analog=D,1.0,0.01,C,0.1,0.3,1.6
  }\end{flushleft}

  Five-sample digital delay:

  \begin{flushleft}{\tt
  sddsdigfilter data.wf -col=time,value result.wf -digital=B,0,0,0,0,0,1
  }\end{flushleft}
 
\item {\bf synopsis:} 
\begin{flushleft}{\tt
sddsdigfilter [{\em inputFile}] [{\em outputFile}] [-pipe=[input][,output]]
  -columns={\em xName},{\em yName}
 [-proportional={\em gain}]
 [-lowpass={\em gain},{\em cutoffFrequency}]
 [-highpass={\em gain},{\em cutoffFrequency}]
 [-digitalfilter={\em sddsfile},{\em aCoeffName},{\em bCoeffName}
 [-digitalfilter=[A,{\em a0},{\em a1},..,{\em am}][,B,{\em b0},{\em b1},..,{\em bn}]
 [-analogfilter={\em sddsfile},{\em cCoeffName},{\em dCoeffName}
 [-analogfilter=[C,{\em c0},{\em c1},..,{\em cm}][,D,{\em d0},{\em d1},..,{\em dn}]
 [-cascade]
 [-verbose]

}\end{flushleft}
\item {\bf files:}
 Two file names are required: the name of the existing input file,
 and the name of the output file to be produced. The input file must
 contain at least two columns: one containing to data to be filtered
 ({\em yName}) and the other giving time information ({\em xName}). A linear time
 scale is assumed for {\em xName}.
 The output file is a copy of the input file with an additional column
 called {\tt DigFiltered{\em yName}} where {\em yName} would be the name of the
 original y-column.

\item {\bf switches:}
%
% Describe the switches that are available
%
    \begin{itemize}

   \item {\tt -pipe[=input][,output]} --- The standard SDDS Toolkit pipe switch.
   \item {\tt -columns={\em xName},{\em yName}} --- The names of the input file data columns.
   \item {\tt -proportional={\em gain}} --- Defines a gain stage, where {\em gain} is the multiplier applied to the data.

   \item {\tt -lowpass={\em gain},{\em cutoffFrequency}} --- Defines a lowpass filter stage, where
{\em gain} is the mutiplier applied to the data and {\em cutoffFrequency} is the -3dB point of the
filter in units appropriate to the supplied {\em xName}.

   \item {\tt -highpass={\em gain},{\em cutoffFrequency}} --- Defines a highpass filter stage,
where {\em gain} is the multiplier applied to the data and {\em cutoffFrequency} is the -3dB point
of the filter in units appropriate to the supplied {\em xName}.

   \item {\tt -digitalfilter={\em sddsfile},{\em aCoeffName},{\em bCoeffName}} --- Defines a digital
filter with coefficients in the supplied SDDS coefficient file. This file must cointain two
columns containing the A and B coefficients of a digital 'Z' transfer function. Note that
control theory convention assumes that the A0 coefficient is always 1.0. To ensure
consistency with the SDDS file, the a0 coefficient is the first row in the A-column and
must be implicitly supplied. Although there is little benefit to setting a0
to anything other than 1.0, it is allowed.

 \item {\tt -digitalfilter=[A,{\em a0},{\em a1},...,{\em am}][,B,{\em b0},{\em b1},...,{\em bn}]} --- Defines a
digital filter with the A and B coefficients of the digital 'Z' transfer function supplied
on the command line. Either A or B or both coefficients can be supplied. If no A
coefficients are supplied, a0 is set to 1.0. Equally, if no B coefficients are supplied,
b0 is set to 1.0. If different numbers of A and B coefficients are suppied, the filter
order is determined from the largest order.

   \item {\tt -analogfilter={\em sddsfile},{\em cCoeffName},{\em dCoeffName}} --- Defines an analog
filter with coefficients in the supplied sdds cefficient file. This file must cointain two
columns containing the C and D coefficients of an analog 's' transfer function. Conversion
to the digital domain is done using a bilinear transform. Note that the user must ensure
adequate data sampled, since the general format does not allow frequency warping based on
the filter cutoff frequency.

 \item {\tt -analogfilter=[A,{\em a0},{\em a1},...,{\em am}][,B,{\em b0},{\em b1},...,{\em bn}]} --- Defines an
analog filter with the C and D coefficients of the analog 'S' transfer function supplied
on the command line. Either C or D or both coefficients can be supplied. If no C
coefficients are supplied, then c0 is set to 1.0. Equally, if no D coefficients are
supplied, then d0 is set to 1.0. Conversion to the digital domain is done using a bilinear
transform. Note that the user must ensure adequate data sampled, since the general format
does not allow frequency warping based on the filter cutoff frequency.

   \item {\tt -cascade} --- Defines the start of a new filter stage. Any number of filter
stages can be supplied for a single data set. If more than one filter is defined, then the
outputs are summed unless the {\tt -cascade} switch is supplied between the filter definitions
in which case the output of the first filter stage is fed into the input of the subsequent
filter stage.

   \item {\tt -verbose} --- Prints the filter coefficients for each filter stage.
   \end{itemize}
\item {\bf references} --- 
	The digital filtering routines were adapted from Stearns and David, {\em Signal Processing Algorithms in Fortran and C}, Prentice Hall, 1993

\item {\bf author}: John Carwardine, Argonne National Laboratory
\end{itemize}


% $Log: not supported by cvs2svn $
% 
% Template for making SDDS Toolkit manual entries.
%
%\begin{latexonly}
\newpage
%\end{latexonly}

%
% Substitute the program name for <programName>
%
\subsection{sddsdiff}
\label{sddsdiff}

\begin{itemize}
\item {\bf description:}
%
% Insert text of description (typicall a paragraph) here.
%
\verb+sddsdiff + compares two SDDS files and print out message of comparison results.
\item {\bf examples:} 
%
% Insert text of examples in this section.  Examples should be simple and
% should be preceeded by a brief description.  Wrap the commands for each
% example in the following construct:
% 
%
{\tt }
\begin{flushleft}{\tt
\bf sddsdiff file1 file2
}\end{flushleft}
{\tt}If two files are the same, message {``file1 and file2 are identical''} will be printed to the standard output:
file1 and file2 are identical.
If two files are different, the differences will be printed to the standard error output.

\item {\bf synopsis:} 
%
% Insert usage message here:
%
\begin{flushleft}{\tt
sddsdiff <file1> <file2>        
}\end{flushleft}

\item {\bf author: H. Shang } ANL
\end{itemize}




%\begin{latexonly}
\newpage
%\end{latexonly}
\subsection{sddsdistest}
\label{sddsdistest}

\begin{itemize}
\item {\bf description:} 
{\tt sddsdistest} performs the Kolmogorov-Smirnov (K-S) test on a
set of numbers to determine how likely those numbers are to have been drawn from
a specified statistical distribution (e.g., gaussian, poisson).

\item {\bf example:}
Try the K-S test on random numbers generated by {\tt sddsprocess}
\begin{flushleft}{\tt
sddssequence -pipe=out -define=i,type=long -sequence=begin=0,end=9999,delta=1 
| sddsprocess -pipe -define=column,gaussRN,grnd -define=column,uniformRN,rnd 
| sddsdistest -pipe -test=ks -gaussian -column=gaussRN -column=uniformRN 
| sddsprintout -pipe -column=ColumnName -column=distestSigLevel
}\end{flushleft}
The result is 
\begin{flushleft}{\tt
    ColumnName      distestSigLevel 
-------------------------------------
     gaussRN         4.019061e-01 
    uniformRN        1.598565e-32 
}\end{flushleft}
which shows that the K-S test accurately distinguishes between numbers
drawn from the two distributions.  The probability that the numbers in
column {\tt uniformRN} are from a gaussian distribution is very small, whereas
the probability that the numbers in column {\tt gaussRN} are from a
gaussian distribution is 40\%.
\item {\bf synopsis:}
\begin{flushleft}{\tt
sddsdistest [{\em input}] [{\em output}] [-pipe=[in][,out]] 
-column={\em name}[,sigma={\em name}] ... -exclude={\em name}[,{\em name}...] ...
{-gaussian | -poisson | -student | -chisquared }
[-degreesOfFreedom={{\em value} | @{\em parameterName}}]
}\end{flushleft}
\item {\bf switches:}
    \begin{itemize}
    \item {\tt -pipe=[input][,output] } --- The standard SDDS Toolkit pipe option.
        \item {\tt -column={\em name}[,sigma={\em name}]} --- Specifies the name of
        a column to test, and optionally the name of the column with the measurement
        error for the each test value.  {\em name} may contain wildcards.  The sigma
        name may contain ``\%s'', for which each column name is substituted to obtain
        the corresponding sigma name.  Multiple {\tt column} options may be given.
        \item {\tt -exclude={\em name}[,{\em name}...]} --- Specifies the names of
        columns to exclude from testing.
        \item {\tt -gaussian | -poisson | -student | -chisquared } --- Specifies the
        model distribution against which to test the data.
        \item {\tt -degreesOfFreedom={{\em value} | @{\em parameterName}}} ---
        Specifies the number of degrees of freedoms to assume for the model distribution
        in the case of student and chi-squared distribution.  The first form specifies a
        fixed {\em value}, whereas the second specifies taking the value for each page
        from the named parameter.
    \end{itemize}
\item {\bf author:} M. Borland, ANL/APS.
\end{itemize}



\begin{sddsprog}{sddsendian}
  \item \textbf{description:} \verb|sddsendian| converts numerical data in an SDDS file between big-endian and little-endian formats. Use this tool before transferring binary data between computers with different endianness.
  \item \textbf{examples:}
    \begin{verbatim}
    sddsendian input.sdds output.sdds
    sddsendian input.sdds -pipe=output > output.sdds
    \end{verbatim}
  \item \textbf{synopsis:}
    \begin{verbatim}
    sddsendian [inputFile] [outputFile] [-pipe[=input][,output]]
    \end{verbatim}
  \item \textbf{switches:}
    \begin{itemize}
      \item \verb|-pipe[=input][,output]| --- The standard SDDS Toolkit pipe option.
    \end{itemize}
  \item \textbf{files:} \emph{inputFile} is the source SDDS data set and \emph{outputFile} is the converted data set.
  \item \textbf{see also:}
    \begin{itemize}
      \item \progref{sddsconvert}
    \end{itemize}
  \item \textbf{author:} M. Borland, ANL/APS.
\end{sddsprog}


\begin{sddsprog}{sddsenvelope}
  \item \textbf{description:} {\tt sddsenvelope} analyzes column data across pages to find minima, maxima, averages, standard-deviations, etc., on a row-by-row basis. It produces a single-page output file containing one column for each analysis requested. It will also copy through data from the first page into the output file. It requires that each page of the input file have the same number of rows.
  \item \textbf{examples:} Find the minimum and maximum beta functions for a set of APS lattices:
  \begin{verbatim}
sddsenvelope APS.twi APS.twi.env -copy=s -minimum=beta? -maximum=beta?
  \end{verbatim}
  \item \textbf{synopsis:}
  \begin{verbatim}
sddsenvelope [-pipe=[input][,output]] [input] [output] [-copy=columnNames]
             [-maximum=columnNames] [-minimum=columnNames]
             [-mean=columnNames] [-sum=power,columnNames]
             [-standardDeviation=columnNames] [-rms=columnNames]
             [-slope=independentVariableName,columnNames]
             [-intercept=independentVariableName,columnNames]
             [-median=columnNames] [-decileRange=columnNames]
  \end{verbatim}
  \item \textbf{switches:}
  \begin{itemize}
    \item \verb|-pipe=[input][,output]| --- The standard SDDS Toolkit pipe option.
    \item \verb|-copy=\emph{columnNames}| --- Specifies that the named columns should be transferred to the output file without alteration. These data come from the first page of the input file. A comma-separated list of optionally wildcard-containing strings may be given.
    \item \verb|-maximum=\emph{columnNames}|, \verb|-minimum=\emph{columnNames}|, \verb|-mean=\emph{columnNames}|, \verb|-rms=\emph{columnNames}|, \verb|-median=\emph{columnNames}|, \verb|-decileRange=\emph{columnNames}| --- Specifies that the named columns should be analysed in the indicated fashion. A comma-separated list of optionally wildcard-containing strings may be given. Decile range is the spread between the 90\% and 10\% points on the distribution.
    \item \verb|-sum=\emph{power},\emph{columnNames}| --- Specifies that the named columns should be analysed in the indicated fashion, i.e., that each output row should be the sum of the values to the indicated power. A comma-separated list of optionally wildcard-containing strings may be given.
    \item \verb|-slope=\emph{independentVariableName},\emph{columnNames}|, \verb|-intercept=\emph{independentVariableName},\emph{columnNames}| --- Specifies that the named columns should be analysed to get the slope or intercept with respect to the parameter \emph{independentVariableName}. A comma-separated list of optionally wildcard-containing strings may be given for the \emph{columnNames}.
  \end{itemize}
  \item \textbf{files:} \emph{inputFile} is a multipage file containing the data for which row-by-row statistics are desired. \emph{outputFile} is a single-page file containing the statistics. The column names in \emph{outputFile} are created from those in the input file by appending the appropriate suffix from the following list: {\tt Max}, {\tt Min}, {\tt Mean}, {\tt StDev}, {\tt RMS}, {\tt Sum}, {\tt Slope}, or {\tt Intercept}.
  \item \textbf{see also:}
  \begin{itemize}
    \item \hyperref[exampleData]{Data for Examples}
    \item \progref{sddschanges}
  \end{itemize}
  \item \textbf{author:} M. Borland, ANL/APS.
\end{sddsprog}


%\begin{latexonly}
\newpage
%\end{latexonly}
\subsection{sddseventhist}
\label{sddseventhist}

\begin{itemize}
\item {\bf description:} 
{\tt sddseventhist} analyzes labeled events in a dataset to provide
histograms of the occurences of each type of event.  It can also
histogram the overlap off all types of events with a single type of
event.
\item {\bf synopsis:} 
\begin{flushleft}{\tt
sddseventhist [-pipe=[input][,output]] [{\em inputFile}] [{\em outputFile}]
-dataColumn={\em columnName} -eventIdentifier={\em columnName} [-overlapEvent={\em eventValue}]
[{-bins={\em number} | -sizeOfBins={\em value}}] 
[-lowerLimit={\em value}] [-upperLimit={\em value}] 
[-sides] [-normalize[=\{sum | area | peak\}]] 
}\end{flushleft}
\item {\bf files:}
{\em inputFile} is a file containing at least two columns of data.
One column must contain string entries that serve as ``event
identifiers''; for example, these might be the names of channels that
issued an alarm.  The other column must contain numerical data that
will be histogrammed; for example, these might be the times at which
alarms occured.  The {\em outputFile} contains one histogram of this
numerical data for each unique value in of the event identifier; the
histogram contains only the data that matches that identifier.
\item {\bf switches:}
    \begin{itemize}
    \item \verb|-pipe[=input][,output]| --- The standard SDDS Toolkit pipe option.
    \item {\tt -dataColumn={\em columnName}} --- Specifies the name of the data column to be histogrammed.
    \item {\tt -eventIdentifier={\em columnName}} --- Specifies the name of the string column that
        identifies events.
    \item {\tt -overlapEvent={\em eventValue}} --- Requests computation of the overlap of the
        histograms of each event with the histogram of event {\em eventValue}.  Useful in determining
        which events always occur at the same time as event {\em eventValue}.
    \item {\tt -bins={\em number}} --- Specifies the number of bins to use.  The default is 20.
    \item {\tt -sizeOfBins={\em value}} --- Specifies the size of bins to use.  The number of bins is
        computed from the range of the data.
    \item {\tt -lowerLimit={\em value}} --- Specifies the lower limit of the histogram.  By default,
        the lower limit is the minimum value in the data.
    \item {\tt -upperLimit={\em value}} --- Specifies the upper limit of the histogram.  By default,
        the upper limit is the maximum value in the data.
    \item {\tt -sides} --- Specifies that zero-height bins should be attached to the lower
        and upper ends of the eventhistogram.  Many prefer the way this looks on a graph.
    \item {\tt -normalize[=\{sum | area | peak\}]} --- Specifies that the histogram should be normalized, and how.
        The default is {\tt sum}.  {\tt sum} normalization means that the sum of the heights will be 1.
        {\tt area} normalization means that the area under the histogram will be 1.
        {\tt peak} normalization means that the maximum height will be 1.
    \end{itemize}
\item {\bf see also:}
    \begin{itemize}
    \item \progref{sddscorrelate}
    \item \progref{sddshist}
    \item \progref{sddshist2d}
    \end{itemize}
\item {\bf author:} M. Borland, ANL/APS.
\end{itemize}


%\begin{latexonly}
\newpage
%\end{latexonly}
\subsection{sddsexpand}
\label{sddsexpand}

\begin{itemize}
\item {\bf description:}

\verb|sddsexpand| reads data pages from an SDDS file and writes a new
SDDS file containing a separate data page for every row in the input
file.  All column definitions from the input file are turned into
parameter definitions in the output file.  In addition all parameter
definitions from the input file are copied to the output file.  Each
output data page contains the values of the columns from a single row
of the input file, along with the values of the parameters from the
same page.  The output file contains no column or array definitions.

\verb|sddsexpand| is essentially the inverse of \verb|sddscollapse|
(except that the column data thrown out in collapsing a file is not
recoverable).

\item {\bf synopsis:} 
\begin{flushleft}{\tt
sddsexpand [{\em inputFile}] [{\em outputFile}] [-pipe[=input][,output]]
}\end{flushleft}

\item {\bf files:} {\em inputFile} is the name of an SDDS data set to
be expanded.  {\em outputFile} is the result.  Note that {\em
outputFile} will not contain any information from any arrays that are
in {\em inputFile}.

\item {\bf switches:} 
\begin{itemize}
        \item {\tt -pipe[=input][,output]} --- The standard SDDS Toolkit pipe option.
        \item {\tt -noWarnings} --- Suppresses warnings about name clashes between
        parameters and columns.  If such a clash occurs, the parameter data is ignored.
\end{itemize}

\item {\bf see also:}
    \begin{itemize}
    \item \progref{sddscollapse}
    \item \progref{sddsbreak}
    \end{itemize}
\item {\bf author:} M. Borland, ANL/APS.
\end{itemize}




%\begin{latexonly}
\newpage
%\end{latexonly}
\subsection{sddsexpfit}
\label{sddsexpfit}

\begin{itemize}
\item {\bf description:}
{\tt sddsexpfit} does exponential fits to a single column of an SDDS file as a function of another column (the independent 
variable).  The fitting function is
\[ E(x) = C + F * e {R*x}, \]
where x is the independent variable, C is the {\em constant} term, F is the {\em factor}, and R is the {\rm rate}.
\item {\bf examples:} 
Fit an  exponential decay to vacuum pressure versus time during a pumpdown:
\begin{flushleft}{\tt
sddsexpfit vacDecay.sdds -columns=Time,Pressure vacDecay.fit
}\end{flushleft}
Same, but give the program a hint and force it to get a better fit
\begin{flushleft}{\tt
sddsexpfit vacDecay.sdds -columns=Time,Pressure vacDecay.fit -clue=decays -tolerance=1e-12
}\end{flushleft}
\item {\bf synopsis:} 
\begin{flushleft}{\tt
sddsexpfit [-pipe=[input][,output]] [{\em inputFile}] [{\em outputFile}]
[-columns={\em xName},{\em yName}] [-tolerance={\em value}] 
[-clue=\{grows | decays\}] [-guess={\em constant},{\em factor},{\em rate}] 
[-verbosity={\em integer}] [-fullOutput] 
}\end{flushleft}
\item {\bf files:}
{\em inputFile} contains the columns of data to be fit.  If {\em inputFile} contains multiple
pages, each page of data is fit separately.  {\em outputFile} has columns containing the
independent variable data and the corresponding values of the fit.  The name of the latter
column is constructed by appending the string {\tt Fit} to the name of the dependent variable.
In addition, if {\tt -fullOutput} is given, {\em outputFile} includes a column with the dependent values and
the residual (dependent values minus fit values).  The name of the residual column is
constructed by appending the string {\tt Residual} to the name of the dependent variable.  {\em
outputFile} contains four parameters: {\tt expfitConstant}, {\tt expfitFactor}, {\tt expfitRate},
and {\tt expfitRmsResidual}.  The first three parameters are respectively C, F, and R from the above
equation.  The last is the rms residual of the fit.
\item {\bf switches:}
    \begin{itemize}
    \item {\tt -pipe=[input][,output]} --- The standard SDDS Toolkit pipe option.
    \item {\tt -columns={\em xName},{\em yName}} --- Specifies the names of the independent and dependent columns of data.
    \item {\tt -tolerance={\em value}} --- Specifies how close {\tt sddsexpfit} will attempt to come to the optimum fit,
        in terms of the mean squared residual.  The default is ${\rm 10 {-8}}$.
    \item {\tt -clue=\{grows | decays\}} --- Helps {\tt sddsexpfit} decide whether the data is a decaying or growing exponential,
        i.e., whether R is negative or positive, respectively.  If {\tt sddsexpfit} is having trouble, this
        will often help.
    \item {\tt -guess={\em constant},{\em factor},{\em rate}} --- Gives {\tt sddsexpfit} a stating point for each of the three fit parameters.
    \item {\tt -fullOutput} --- Specifies that {\em outputFile} will contain the original dependent variable
        data and the fit residuals, in addition to the independent variable data and the fit values.
    \item {\tt -verbosity={\em integer}} --- Specifies that informational printouts are desired during fitting.  A larger
        integer produces more output.
    \end{itemize}
\item {\bf see also:}
    \begin{itemize}
    \item \hyperref[exampleData]{Data for Examples}
    \item \progref{sddspfit}
    \item \progref{sddsgfit}
    \item \progref{sddsoutlier}
    \end{itemize}
\item {\bf author:} M. Borland, ANL/APS.
\end{itemize}



\begin{sddsprog}{sddsfdfilter}
  \item \textbf{description:}
    \verb|sddsfdfilter| performs frequency-domain filtering of data columns using
    Fourier transforms.  It supports thresholding, highpass, lowpass, notch,
    bandpass, and user-defined filters.  Multiple filters may be cascaded or
    applied in parallel, and the program can create new or difference columns in
    the output.
  \item \textbf{examples:}
    \begin{verbatim}
    sddsfdfilter data.sdds filtered.sdds -columns=t,value -lowpass=start=1,end=10
    sddsfdfilter data.sdds filtered.sdds -columns=t,value \
      -notch=center=60,flatWidth=2,fullWidth=5
    \end{verbatim}
  \item \textbf{synopsis:}
    \begin{verbatim}
    sddsfdfilter [-pipe[=input][,output]] [inputfile] [outputfile]
                 [-columns=indep-variable[,depen-quantity[,depen-quantity...]]]
                 [-exclude=depen-quantity[,depen-quantity]]
                 [-clipFrequencies=[high=number][,low=number]]
                 [-threshold=level=value[,fractional][,start=freq][,end=freq]]
                 [-highpass=start=freq,end=freq]
                 [-lowpass=start=freq,end=freq]
                 [-notch=center=center,flatWidth=width1,fullWidth=width2]
                 [-bandpass=center=center,flatWidth=width1,fullWidth=width2]
                 [-filterFile=filename=filename,frequency=columnName,
                             {real=columnName,imaginary=columnName |
                              magnitude=columnName}]
                 [-cascade] [-newColumns] [-differenceColumns]
    \end{verbatim}
  \item \textbf{files:}
    The optional \verb|inputfile| is an existing SDDS file to filter.  If omitted,
    the program reads from standard input when \verb|-pipe=input| is given.  The
    \verb|outputfile| receives the filtered data, or standard output is used when
    \verb|-pipe=output| is specified.
  \item \textbf{switches:}
    \begin{itemize}
    \item {\tt -pipe[=input][,output]} --- The standard SDDS Toolkit pipe option.
    \item {\tt -columns={\em indepVariable}[,{\em depenQuantity}][,{\em depenQuantity}...]} ---
      Gives the name of the independent variable (typically time).  If no
      {\em depenQuantity} qualifiers are given, all numerical columns are filtered;
      otherwise, only the named columns are processed.
    \item {\tt -exclude={\em depenQuantity}[,{\em depenQuantity}]} --- Specifies columns to
      exclude from filtering, modifying selections from {\tt -columns}.
    \item {\tt -clipFrequencies[=low={\em frequency}][,high={\em frequency}]} --- Clips
      frequencies below the {\tt low} value and/or above the {\tt high} value; clipped
      components are set to zero.
    \item {\tt -threshold=level={\em value}[,fractional][,start={\em freq}][,end={\em freq}]} ---
      Suppress components below a threshold.  With {\tt fractional}, the level is a
      fraction of the largest component.  {\tt start} and {\tt end} restrict the
      frequency range.
    \item {\tt -highpass=start={\em freq},end={\em freq}} --- Apply a highpass filter.
    \item {\tt -lowpass=start={\em freq},end={\em freq}} --- Apply a lowpass filter.
    \item {\tt -notch=center={\em center},flatWidth={\em width1},fullWidth={\em width2}} ---
      Apply a notch filter centered on a frequency.
    \item {\tt -bandpass=center={\em center},flatWidth={\em width1},fullWidth={\em width2}} ---
      Apply a bandpass filter centered on a frequency.
    \item {\tt -filterFile=filename={\em filename},frequency={\em columnName},}
      {\tt \{real={\em columnName},imaginary={\em columnName} | magnitude={\em columnName}\}} ---
      Specifies a filter via an SDDS file of attenuation values.  The column named
      with the {\tt frequency} qualifier gives the frequency values.  Components
      outside the provided range are unaffected.
    \item {\tt -cascade} --- Cascade multiple filter definitions.
    \item {\tt -newColumns} --- Create new columns containing filtered results.
    \item {\tt -differenceColumns} --- Create columns containing differences between
      original and filtered data.
    \end{itemize}
  \item \textbf{see also:}
    \begin{itemize}
    \item \progref{sddsdigfilter}
    \item \progref{sddssmooth}
    \end{itemize}
  \item \textbf{author:} M. Borland, ANL/APS.
\end{sddsprog}


%\begin{latexonly}
\newpage
%\end{latexonly}
\subsection{sddsfft}
\label{sddsfft}

\begin{itemize}
\item {\bf description:}
{\tt sddsfft} takes Fast Fourier Transforms of real data in columns.  It will transform any number of columns
simultaneously as a function of a single independent variable.  
Strictly speaking, the independent variable values should be equispaced; if they are not, {\tt sddsfft} uses
the average spacing.  The number of data points need not be a power of two.  Output of the magnitude only is the 
default, but phase and complex values are available.
\item {\bf examples:} 
Take the FFT of time series samples of PAR x beam-position-monitor readouts:
\begin{flushleft}{\tt
sddsfft par.bpm par.fft -column=Time,'P?P?x'
}\end{flushleft}
\item {\bf synopsis:} 
\begin{flushleft}{\tt
sddsfft [-pipe=[input][,output]] [{\em inputFile}] [{\em outputFile}]
-columns={\em indepVariable}[,{\em depenQuantityList}]
[-padWithZeroes | -truncate] [-sparse={\em integer}] 
[-window[=\{hanning | welch | parzen\}]] [-complexInput[=folded|unfolded]]
[-normalize] [-suppressAverage] [-fullOutput[=folded|unfolded]] [-psdOutput] [-inverse]
}\end{flushleft}
\item {\bf files:}
{\em inputFile} contains the data to be FFT'd.  One column from this file must be chosen as the independent
variable.   By default, all other columns are taken as dependent variables.  If {\em inputFile} contains multiple
pages, each is treated separately and is delivered to a separate page of {\em outputFile}.

{\em outputFile} contains a column {\tt f} for the frequency, along with one or more columns for each independent
variable.  By default, {\em outputFile} has one column named {\tt FFT}{\em indepName} containing the magnitude of the
FFT for each independent variable. If {\tt -fullOutput} is specified, {\em outputFile} contains additional
columns for, respectively, the phase (or argument), real part, and imaginary part of the FFT: {\tt Arg}{\em indepName},
{\tt Real}{\em indepName}, and {\tt Imag}{\em indepName}.  If power-spectral-density output is requested, then
a column {\tt PSD}{\em indepName} is also created.  

{\em outputFile} also contains two parameters, {\tt fftFrequencies} and {\tt fftFrequencySpacing}, giving
the number of frequencies and the frequency spacing, respectively.

\item {\bf switches:}
    \begin{itemize}
    \item \verb|pipe[=input][,output]| --- The standard SDDS Toolkit pipe option.
    \item {\tt -columns={\em indepVariable}[,{\em depenQuantityList}]} --- Specifies the name of the
        independent variable column.  Optionally, specifies a list of comma-separated, optionally
        wildcard-containing names of dependent quantities to be FFT'd as a function of the independent variable. 
        By default, all numerical columns except the independent column are FFT'd.
    \item {\tt -exclude={\em depenQuantity},...} --- Specifies optionally wildcarded names of columns
        to exclude from analysis.
    \item \verb|-padWithZeros| --- Specifies that the independent data should be padded with zeros to
        make the number of points equal to the nearest power of two.  In some cases, this will result in
        significantly greater speed.
    \item \verb|-truncate| --- Specifies that the data should be truncated so that the number of points is 
        the largest product of primes from 2 to 19 not greater than the original number of points.  
        In some cases, this will result in significantly greater speed.
    \item {\tt sparse={\em integer}} --- Specifies that the data should be uniformly sampled at the
        given integer interval.  While this reduces frequency span of the FFT, it may result in greater
        speed.
    \item {\tt window[=\{hanning | welch | parzen\}} --- Specifies that data windowing should be performed
        prior to taking FFT's, and optionally specifies the type of window.  The default is {\tt hanning}.
        Usually used to improve visibility of small features or accuracy of amplitudes for data that is
        not periodic in the total sampling time or a submultiple thereof.
    \item \verb|normalize| --- Specifies that FFT's will be normalized to give a maximum magnitude of 1.
    \item \verb|suppressAverage| --- Specifies that the average value of the data will be subtracted from
        every point prior to taking the FFT.  This may improve accuracy and visibility of small components.
    \item \verb|complexInput| --- Specifies that the names of the input columns are of the form Real<rootname>
        and Imag<rootname>, giving the real and imaginary part of a function to be analyzed. In this case, 
        the <depen-quantity> entries in the -columns option give the rootname, not the full quantity name.
        It has options folded and unfolded, unfolded means the input frequency space input is unfolded and
        it must have negative frequency. default is "folded", If no option is given, and if the input file
        has "SpectrumFolded" parameter, then it will be defined by this parameter.
    \item \verb|fullOutput| --- Specifies that in addition to the magnitude, the phase, real part, and imaginary
        part of each FFT will be included in the output. It also has folded and unfolded options, while the unfold
        option outputs the unfolded frequency-space (full FFT spectrum), but the folded option outputs the folded
        spectrum (half FFT).
    \item \verb|inverse| --- produce inverse fourier transform. when it is given, the output is always unfolded spectrum and
        it only works with complexInput that has imaginary data.
    \item \verb|psdOutput| --- Specifies that in in addition to ordinary FFT data, the power-spectral-densities
        will also be included in the output.  The units of the PSD are of the form ${\rm x^2/t}$, where
        x (t) represents the units of the independent (dependent) variable.  These units are conventional with
        PSDs, which are normalized to the frequency spacing so that integrating the PSD gives the signal power.
    \end{itemize}
\item {\bf How to use the folded and unfolded options} There are three cases as listed in the following:
    \begin{itemize}
    \item real input without inverse option
        \begin{table}[hbt]
        \caption{real input without inverse option}
        \begin{tabular}{|c|c|c|c|}  
        Input & \multicolumn{3}{|c|}{Output} \\ 
        Real Number & Real Number & Imaginary Number & output option \\ \hline
        N & N/2 & N/2  &folded \\ \hline
        N & N   & N  &unfolded \\ \hline
        \end{tabular}
        \label{table1}
        \end{table}
    \item complexInput without inverse option
        \begin{table}[hbt]
        \caption{complexInput without inverse option}
        \begin{tabular}{|c|c|c|c|c|c|c|}  
        \multicolumn{4}{|c|}{ComplexInput}  &\multicolumn{3}{|c|}{Output} \\ 
        Real Number & Imaginary Number & Condition & Input option & Real Number & Imaginary Number & output option \\ \hline
        N & N & & folded N & N & N & folded \\ \hline
        N & N & last imag=0 & folded & 2*(N-1) & 2*(N-1) &unfolded \\ \hline
        N & N & last imag!=0 & folded & 2*(N-1)+1) & 2*(N-1)+1 & unfolded \\ \hline
        N & N & & unfolded & N/2 & N/2 & folded \\ \hline
        N & N & & unfolded & N & N & unfolded \\ \hline
        \end{tabular}
        \label{table2}
        \end{table}
     \item with inverse option, since inverse only works with complexInptut, and inverse spectrum is always unfolded, so -fullOutput=folded will be changed to -fullOutput=unfolded with -inverse option.
        \begin{table}[hbt]
        \caption{complexInput with inverse option}
        \begin{tabular}{|c|c|c|c|c|c|} 
        \multicolumn{4}{|c|}{ComplexInput}  &\multicolumn{2}{|c|}{Inverse Output} \\
        Real Number & Imaginary Number & Condition & Input option & Real Number & Imaginary Number  \\ \hline
        N & N & last imag=0 & folded & 2*(N-1) & 2*(N-1) \\ \hline
        N & N & last imag!=0 & folded & 2*(N-1)+1) & 2*(N-1)+1\\ \hline
        N & N & & unfolded & N & N\\ \hline
        \end{tabular}
        \label{table3}
        \end{table}
     \end{itemize}
\item {\bf Simple FFT and inverse FFT pairs} If you want to take the FFT of real data, manipulate the result (e.g.,
  to apply a custom filter), then invert the FFT, use the following options
  \begin{itemize}
  \item For forward FFT: \verb|-fullOutput=folded|.
  \item For inverse FFT: \verb|-complexInput=folded -fullOutput|. The quantity of interest is the real part of the inverse
    FFT, presumably.
  \end{itemize}
\item {\bf see also:}
    \begin{itemize}
    \item \hyperref[exampleData]{Data for Examples}
    \item \progref{sddsdigfilter}
    \item \progref{sddsnaff}
    \end{itemize}
\item {\bf author:} M. Borland, ANL/APS.
\end{itemize}


%\begin{latexonly}
\newpage
%\end{latexonly}
\subsection{sddsgenericfit}
\label{sddsgenericfit}

\begin{sddsprog}{sddsgenericfit}
  \item \textbf{description:} \verb|sddsgenericfit| does fits to an arbitrary functional form specified by the user.
  \item \textbf{examples:}
    Fit a gaussian to a beam profile to get the rms beam size. In this example, the
    file \verb|beamProfile.sdds| is assumed to contain data to be fit in columns
    \verb|x| and \verb|Intensity|.
    \begin{verbatim}
    sddsgenericfit beamProfile.sdds beamProfile.gfit -column=x,Intensity \
      "-equation=height x center - sqr 2 / sigma sqr / exp / baseline + Intensity - sqr" \
      -variable=name=height,start=1,lower=0,upper=10,step=0.1 \
      -variable=name=center,start=0,lower=-10,upper=10,step=0.1 \
      -variable=name=sigma,start=1,lower=0.1,upper=10,step=0.1 \
      -variable=name=baseline,start=0,lower=-1,upper=1,step=0.1
    \end{verbatim}
  \item \textbf{synopsis:}
    \begin{verbatim}
    sddsgenericfit [-pipe=[input][,output]] [inputfile] [outputfile]
      -columns=x-name,y-name[,ySigma=sy-name]
      -equation=string[,algebraic] [-target=value] [-tolerance=value]
      [-simplex=[restarts=integer][,cycles=integer][,evaluations=integer]]
      -variable=name=name,lowerLimit=value,upperLimit=value,stepsize=value,
        startingValue=value[,units=string][,heat=value]
      [-variable=...] [-verbosity=integer] [-startFromPrevious]
    \end{verbatim}
  \item \textbf{switches:}
    \begin{itemize}
      \item \verb|-pipe=[input][,output]| --- The standard SDDS Toolkit pipe option.
      \item \verb|-columns=x-name,y-name[,ySigma=name]| --- Specifies the names of the independent and dependent columns of data, and
        optionally the name of the column containing the errors in the dependent column.
      \item \verb|-equation=string[,algebraic]| --- Specifies the functional form of the fit, $y(x, \{p_1, p_2, ...\})$,
        where $\{p_1, p_2, ...\}$ represents the parameters of the function and $x$ is the independent variable value.
        The fitting process attempts to minimize the penalty function $\sum_{i=0}^{N-1} (y(x_i, \{p_1, p_2, ...\}) - y_i)^2$
        (or $\sum_{i=0}^{N-1} (y(x_i, \{p_1, p_2, ...\}) - y_i)^2/\sigma_{y,i}^2$ if y sigma values are given)
        by changing the values of the parameters $\{p_1, p_2, ...\}$.
      \item \verb|-tolerance=value| --- Specifies the minimum change in the (weighted) rms residual that is considered
        significant enough to justify continuing optimization.
      \item \verb|-simplex=[restarts=nRestarts][,cycles=nCycles][,evaluations=nEvals]| --- Specifies parameters of the simplex optimization used to perform the fit.
        Each start or restart allows \emph{nCycles} cycles with up to \emph{nEvals} evaluations of the function.
        Defaults are 10 restarts, 10 cycles, and 5000 evaluations.
      \item \verb|-variablename=name,lowerLimit=value,upperLimit=value,stepsize=value,startingValue=value[,units=string][,heat=value]| ---
        Specifies a parameter of the fitting function. If the \verb|heat| qualifier is given, then prior to each restart the code ``heats'' the values by adding random
        numbers to the result of the last iteration. This can help avoid getting stuck
        in a local minimum. You must give one \verb|variable| option for every parameter of the fit.
      \item \verb|-startFromPrevious| --- Meaningful for multipage input files only. If given,
        then the optimization for each page starts from the parameter values from the fit
        to the previous page. By default, fitting for each page starts from the values
        specified on the commandline.
      \item \verb|-verbosity=integer| --- Specifies that informational printouts are desired during fitting. A larger integer produces more output.
    \end{itemize}
  \item \textbf{files:} \emph{inputFile} contains the columns of data to be fit. If \emph{inputFile} contains multiple pages, each page of data is
    fit separately. \emph{outputFile} has columns containing the independent variable data and the corresponding values of the fit.
    The name of the column is constructed by appending the string \verb|Fit| to the name of the dependent variable. The name of the residual
    column is constructed by appending the string \verb|Residual| to the name of the dependent variable. \emph{outputFile} also contains
    parameters giving the values of the fit parameters.
  \item \textbf{see also:}
    \begin{itemize}
      \item \progref{sddsexpfit}
      \item \progref{sddsgfit}
      \item \progref{sddsoutlier}
      \item \progref{sddspfit}
    \end{itemize}
  \item \textbf{author:} M. Borland, ANL/APS.
\end{sddsprog}


%\begin{latexonly}
\newpage
%\end{latexonly}
\subsection{sddsgfit}
\label{sddsgfit}

\begin{sddsprog}{sddsgfit}
  \item \textbf{description:}
  \verb|sddsgfit| does gaussian fits to a single column of an SDDS file as a function of another column
  (the independent variable). The fitting function is
  \[
    G(x) = B + H e^{-(x-\mu)^2/(2\sigma^2)}
  \]
  where $x$ is the independent variable, $B$ is the baseline, $H$ is the height, $\mu$ is the mean, and
  $\sigma$ is the width.

  \item \textbf{examples:}
  Fit a gaussian to a beam profile to get the rms beam size:
  \begin{verbatim}
  sddsgfit beamProfile.sdds beamProfile.gfit -column=x,Intensity
  \end{verbatim}

  \item \textbf{synopsis:}
  \begin{verbatim}
  sddsgfit [-pipe=[input][,output]] [inputFile] [outputFile]
    -columns=x-name,y-name[,sy-name]
    [-fitRange=lower,upper] [-fullOutput]
    [-guesses=[baseline=value][,mean=value][,height=value][,sigma=value]]
    [-fixValue=[baseline=value][,mean=value][,height=value][,sigma=value]]
    [-stepSize=factor] [-tolerance=value]
    [-limits=[evaluations=number][,passes=number]]
    [-verbosity=integer]
  \end{verbatim}

  \item \textbf{switches:}
  \begin{itemize}
    \item \verb|-pipe=[input][,output]| --- The standard SDDS Toolkit pipe option.
    \item \verb|-columns=x-name,y-name| --- Specifies the names of the independent and dependent columns of data.
    \item \verb|-fitRange=lower,upper| --- Specifies the range of independent variable values to use in the fit.
    \item \verb|-guesses=[baseline=value][,mean=value][,height=value][,sigma=value]| --- Gives \verb|sddsgfit| a starting
    point for one or more parameters.
    \item \verb|-fixValue=[baseline=value][,mean=value][,height=value][,sigma=value]| --- Gives \verb|sddsgfit| a fixed
    value for one or more parameters. If given, then \verb|sddsgfit| will not attempt to fit the parameters in question.
    \item \verb|-stepSize=factor| --- Specifies the starting stepsize for optimization as a fraction of the starting
    values. The default is 0.01.
    \item \verb|-tolerance=value| --- Specifies how close \verb|sddsgfit| will attempt to come to the optimum fit, in
    terms of the mean squared residual. The default is $10^{-8}$.
    \item \verb|-limits=[evaluations=number][,passes=number]| --- Specifies limits on how many fit function
    evaluations and how many minimization passes will be done in the fitting. The defaults are 5000 and 100,
    respectively. If the fit is not converging, try increasing one or both of these. If the number of evaluations is
    too small, you may get warning messages about optimization failures.
    \item \verb|-fullOutput| --- Specifies that \verb|outputFile| will contain the original dependent variable data and
    the fit residuals, in addition to the independent variable data and the fit values.
    \item \verb|-verbosity=integer| --- Specifies that informational printouts are desired during fitting. A larger
    integer produces more output.
  \end{itemize}

  \item \textbf{files:}
  \emph{inputFile} contains the columns of data to be fit. If \emph{inputFile} contains multiple pages, each page of
  data is fit separately. \emph{outputFile} has columns containing the independent variable data and the corresponding
  values of the fit. The name of the latter column is constructed by appending the string \verb|Fit| to the name of the
  dependent variable. In addition, if \verb|-fullOutput| is given, it includes a column with the dependent values and
  the residual (dependent values minus fit values). The name of the residual column is constructed by appending the
  string \verb|Residual| to the name of the dependent variable. \emph{outputFile} contains five parameters:
  \verb|gfitBaseline|, \verb|gfitHeight|, \verb|gfitMean|, \verb|gfitSigma|, and \verb|gfitRmsResidual|. The first four
  parameters are respectively $B$, $H$, $\mu$, and $\sigma$ from the equation above. The last is the rms residual of the
  fit.

  \item \textbf{see also:}
  \begin{itemize}
    \item \progref{sddspfit}
    \item \progref{sddsexpfit}
    \item \progref{sddsoutlier}
  \end{itemize}

  \item \textbf{author:} M. Borland, ANL/APS.
\end{sddsprog}


%\begin{latexonly}
\newpage
%\end{latexonly}
\subsection{sddshist}
\label{sddshist}

\begin{itemize}
\item {\bf description:} 
{\tt sddshist} does weighted and unweighted one-dimensional histograms of column data from an SDDS file.
It also does limited statistical analysis of data, and basic filtering of data.
\item {\bf examples:} 
Make a 20-bin histogram of a series of PAR x beam-position-monitor readouts:
\begin{flushleft}{\tt
sddshist par.bpm par.bpmhis -data=P1P1x -bins=20
}\end{flushleft}
\item {\bf synopsis:} 
\begin{flushleft}{\tt
sddshist [-pipe=[input][,output]] [{\em inputFile}] [{\em outputFile}]
-dataColumn={\em columnName} [{-bins={\em number} | -sizeOfBins={\em value}}] 
[-lowerLimit={\em value}] [-upperLimit={\em value}] [-filter={\em columnName},{\em lowerLimit},{\em upperLimit}] 
[-weightColumn={\em columnName}] [-sides] [-normalize[=\{sum | area | peak\}]] 
[-statistics] [-verbose]
}\end{flushleft}
\item {\bf files:}
{\em inputFile} is the name of an SDDS file containing data to be
histogrammed, along with optional weight data.  If {\em inputFile}
contains multiple data pages, each is treated separately.  The
histogram or histograms are placed in {\em outputFile}, which has two
columns.  One column has the same name as the histogrammed variable,
and consists of equispaced values giving the centers of the bins.  The
other column, named {\tt frequency}, contains the histogram
frequencies. Its precise meaning is dependent on normalization modes
and weighting.  By default, it contains the number of data points in
the corresponding bin.  

If requested, {\em outputFile} will also contain parameters giving
statistics for the data being histogrammed.  See below for details.

\item {\bf switches:}
    \begin{itemize}
    \item \verb|-pipe[=input][,output]| --- The standard SDDS Toolkit pipe option.
    \item {\tt -dataColumn={\em columnName}} --- Specifies the name of the data column to be histogrammed.
    \item {\tt -bins={\em number}} --- Specifies the number of bins to use.  The default is 20.
    \item {\tt -sizeOfBins={\em value}} --- Specifies the size of bins to use.  The number of bins is
        computed from the range of the data.
    \item {\tt -lowerLimit={\em value}} --- Specifies the lower limit of the histogram.  By default,
        the lower limit is the minimum value in the data.
    \item {\tt -upperLimit={\em value}} --- Specifies the upper limit of the histogram.  By default,
        the upper limit is the maximum value in the data.
    \item {\tt -filter={\em columnName},{\em lowerLimit},{\em upperLimit}} --- Specifies the name of a column by which to filter the input rows.
        Rows for which the named data is outside the specified interval are discarded.
        Alternatively, one can use \progref{sddsprocess}
        to winnow data and pipe it into {\tt sddshist}.
    \item {\tt -weightColumn={\em columnName}} --- Specifies the name of a column by which to weight the
        histogram.  This means that data points with a higher corresponding weight value
        are counted proportionally more times in the histogram.  
    \item {\tt -sides} --- Specifies that zero-height bins should be attached to the lower
        and upper ends of the histogram.  Many prefer the way this looks on a graph.
    \item {\tt -normalize[=\{sum | area | peak\}]} --- Specifies that the histogram should be normalized, and how.
        The default is {\tt sum}.  {\tt sum} normalization means that the sum of the heights will be 1.
        {\tt area} normalization means that the area under the histogram will be 1.
        {\tt peak} normalization means that the maximum height will be 1.
    \item {\tt -statistics} --- Specifies that statistics should be computed for the data and
        placed in {\em outputFile}.  These presently include arithmetic mean, rms, and standard deviation.
        The parameters are named by appending the strings {\tt Mean}, {\tt RMS}, and {\tt StDev} to the
        name of the data column.  If {\tt -weigthColumn} is given, the statistics are weighted.  
    \end{itemize}
\item {\bf see also:}
    \begin{itemize}
    \item \hyperref[exampleData]{Data for Examples}
    \item \progref{sddshist2d}
    \item \progref{sddsprocess}
    \end{itemize}
\item {\bf author:} M. Borland, ANL/APS.
\end{itemize}


%\begin{latexonly}
\newpage
%\end{latexonly}

\subsection{sddshist2d}
\label{sddshist2d}

\begin{itemize}
\item {\bf description:}

{\tt sddshist2d} makes two-dimensional histograms of data, producing
output that is suitable for plotting with {\tt sddscontour}.
The two-dimensional histogram may include data from two columns, or may
show the histograms of a single column versus page number.

\item {\bf examples:} 
Make a two-dimensional histogram of two PAR bpm readouts, then plot the result:
\begin{flushleft}{\tt 
sddshist2d par.bpm par.bpm.h2d -column=P1P1x,P1P2x -xparam=50 -yparam=50
sddscontour -shade=32 par.bpm.h2d -quantity=frequency
}
\end{flushleft}

\item {\bf synopsis:} 
\begin{flushleft}{\tt
sddshist2d [-pipe[=input][,output]] [{\em inputfile}] [{\em outputfile}]
-columns=\{{\em xName},{\em yName} | {\em yName}\}
[-weights={\em columnName}[,average]]
[-xParameters={\em bins}[,{\em lower},{\em upper}]] [-yParameters={\em bins}[,{\em lower},{\em upper}]]
[-outputName={\em string}] 
[-sameScale] [-combine] [-normalize[=sum]] [-smooth[={\em passes}]] 
[-verbose] 
}\end{flushleft}

\item {\bf switches:}
    \begin{itemize}
    \item {\tt -pipe[=input][,output]} --- The standard SDDS Toolkit pipe option.
    \item {\tt -columns=\{{\em xName},{\em yName} | {\em yName}\}} --- Specifies the data from
        the input to histogram.  If both {\em xName} and {\em yName} are given, then {\tt sddshist2d}
        does a two-dimensional histogram of the values in the named columns.  If only {\tt yName}
        is given, then {\tt sddshist2d} does a series of one-dimensional histograms of the named
        column, one for each data pages; these histograms are then assembled as a two-dimensional
        histogram with one axis being the page number.
    \item {\tt -weights={\em columnName}[,average]} --- Specifies the name of a column of data with
        which to proportionally weight the count value of points in the histogram.  If the {\tt average}
        qualifier is given, then each bin value is normalized to contain the average value of the weight for
        all points in the bin.
    \item {\tt -xParameters={\em bins}[,{\em lower},{\em upper}]} --- Specifies the number of bins and
        optionally the histogrammed region for the x values.  Ignored if only {\em yName} is given.
        By default, 21 bins are used encompassing all of the data points.
    \item {\tt -yParameters={\em bins}[,{\em lower},{\em upper}]} --- Specifies the number of bins and
        optionally the histogrammed region for the y values.  By default, 21 bins are used encompassing
        all of the data points.
    \item {\tt -outputName={\em string}} --- Specifies the name of the histogram data.  The default
        is {\tt frequency}.
    \item {\tt -sameScale} --- Specifies that for multipage input files, the histogram region should be
        the same for all pages.  The region is set to encompass all data points from all pages.
    \item {\tt -combine} --- Specifies that for multipage input files, the data from all pages should be
        placed in a single histogram.
    \item {\tt -normalize[=sum]} --- Specifies normalization of the histogram.  If the {\tt sum} qualifier
        is not given,  the histogram is normalized to unit amplitude; otherwise, it is normalized so that
        the sum of all frequencies is unity.
    \item {\tt -smooth[={\em passes}]} --- Specifies smoothing by nearest-neighbor-averaging.  If {\em passes}
        is omitted, only one pass is performed.
    \item {\tt -verbose} --- Requests informational output during processing.
    \end{itemize}
\item {\bf see also:}
    \begin{itemize}
    \item \progref{sddshist}
    \item \progref{sddscontour}
    \item \progref{sddscongen}
    \end{itemize}
\item {\bf author:} M. Borland, ANL/APS.
\end{itemize}

%
%\begin{latexonly}
\newpage
%\end{latexonly}
\subsection{sddsimageconvert}
\label{sddsimageconvert}

\begin{sddsprog}{sddsimageconvert}
  \item \textbf{description:} \verb|sddsimageconvert| converts a single-column SDDS image file into a multi-column SDDS image file and vice versa.
  \item \textbf{examples:}
  \begin{verbatim}
sddsimageconvert image.input image.output
  \end{verbatim}
  \item \textbf{synopsis:}
  \begin{verbatim}
sddsimageconvert [Inputfile] [Outputfile] [-pipe[=in][,out]]
  [-ascii] [-binary]
  [-multicolumn=[indexName=name][,prefix=name]]
  [-singlecolumn=[imageColumn=name][,xVariableName=name][,yVariableName=name]]
  [-nowarnings]
  \end{verbatim}
  \item \textbf{files:}
  \emph{Inputfile} is the SDDS input image file. This file can be either a single-column image file or a multi-column image file.

  \emph{Outputfile} is a single-column SDDS image file if the \emph{Inputfile} is a multi-column SDDS image file, or it is a multi-column SDDS image file if the \emph{Inputfile} is a single-column SDDS image file.
  \item \textbf{switches:}
    \begin{itemize}
      \item {\tt -pipe[=in][,out]} --- The standard SDDS Toolkit pipe option.
      \item {\tt -ascii} --- The \emph{Outputfile} is written in ascii format.
      \item {\tt -binary} --- The \emph{Outputfile} is written in binary format.
      \item {\tt -multicolumn=[indexName=name][,prefix=name]} --- The multi-column SDDS image file will have or does have an index column with the name given by \verb|indexName|=\emph{name}. The default name is Index. It also will have or does have multiple columns with the prefix given by \verb|prefix|=\emph{name}. For an input file this defaults to the prefix of the first column found that is not the index column and that ends with a number. For an output file this defaults to the same name as the image column name in the single-column SDDS image file.
      \item {\tt -singlecolumn=[imageColumn=name][,xVariableName=name][,yVariableName=name]} --- The single-column SDDS image file will have or does have an image column with the name given by \verb|imageColumn|=\emph{name}. The default is the name of only column that exists in the single-column input file or the image prefix in the multi-column input file. If the output file is a single-column image file the \verb|xVariableName|=\emph{name} and \verb|yVariableName|=\emph{name} options will be used to define the x and y variable names. These default to x and y.
      \item {\tt -nowarnings} --- No warnings will be issued when the input file is overwritten.
    \end{itemize}
  \item \textbf{see also:}
    \begin{itemize}
      \item \progref{sddscongen}
      \item \progref{sddscontour}
      \item \progref{sddsimageprofiles}
      \item \progref{sddsspotanalysis}
    \end{itemize}
  \item \textbf{author:} R. Soliday, ANL/APS.
\end{sddsprog}


%
%\begin{latexonly}
\newpage
%\end{latexonly}
\subsection{sddsimageprofiles}
\label{sddsimageprofiles}

\begin{itemize}
\item {\bf description:} Extracts the profile from a multi-column SDDS image file.
\item {\bf example:} 
\begin{flushleft}{\tt
sddsimageprofiles image.input image.output -profileType=x -method=peak
}\end{flushleft}
\item {\bf synopsis:}
\begin{flushleft}{\tt
sddsimageprofiles [{\em Inputfile}] [{\em Outputfile}] [-pipe[=in][,out]] 
[-columnPrefix={\em prefix}]
[-profileType=<x|y>]
[-method=<centerLine|integrated|averaged|peak>]
[-background={\em filename}]
[-aVector={\em ax},{\em ay}]
[-bVector={\em bx},{\em by}]
[-offset={\em x},{\em y}]
}\end{flushleft}
\item {\bf files: }

{\em Inputfile} is the multi-column SDDS input image file.

{\em Outputfile} is a simple SDDS file containing x and y columns which can be plotted with sddsplot to show the image profile.

\item {\bf switches:}
    \begin{itemize}
    \item {\tt -pipe[=in][,out]} --- The standard SDDS Toolkit pipe option.
    \item {\tt -columnPrefix={\em prefix}} --- The prefix for the image columns in the multi-column SDDS image file.
    \item {\tt -profileType=<x|y>} --- Used to select the profile along the X or Y axis.
    \item {\tt -method=<centerLine|integrated|averaged|peak>} --- If this option is not specified it is a real profile. If centerLine is specified it will find the row with the greatest integrated profile and display that line only. If integrated is specified it will sum all the profiles together. If averaged is specified it will divide the sum of all the profiles by the number of profiles. If peak is specified it will find the peak point and display the profile for that row.
    \item {\tt -background={\em filename}} --- Used to specify a background image file which will be subtracted from the input image file.
    \end{itemize}
\item {\bf see also:}
    \begin{itemize}
    \item \progref{sddscongen}
    \item \progref{sddscontour}
    \item \progref{sddsimageconvert}
    \item \progref{sddsspotanalysis}
    \end{itemize}
\item {\bf author:} R. Soliday, ANL/APS.
\end{itemize}


%\begin{latexonly} 
\newpage 
%\end{latexonly} 
\subsection{sddsinsideboundaries} 
\label{sddsinsideboundaries} 
 
\begin{itemize} 
\item {\bf description:} \hspace*{1mm}\\ 
{\tt sddsinsideboundaries} determines whether points in a two-dimensional space (x, y) are inside any of a
series of closed boundaries (or contours).
\item {\bf examples:} 
\begin{flushleft}{\tt
   sddsinsideboundaries <inputFile> <outputFile> -columns=Z,X -boundary=exclusionRegions.sdds,Z,X -keep=outside
}\end{flushleft} 
\item {\bf synopsis:}  
\begin{flushleft}{\tt 
sddsinsideboundaries [<inputfile>] [<outputfile>] [-pipe=[input][,output]]
-columns=<x-name>,<y-name>
-boundary=<filename>,<x-name>,<y-name>
[-insideColumn=<columnName>]
[-keep={inside|outside}] [-threads=<number>]
}\end{flushleft} 
\item {\bf files:} 
The input file contains the columns defining the test points. By default, the output file will contain the same columns,
but also an additional column indicating if each test point is inside any of the boundaries.
The boundary file provides a series of closed boundaries, each on a separate page.
\item {\bf switches:} 
  \begin{itemize} 
  \item {\tt -pipe[=input]} --- The standard SDDS Toolkit pipe option. 
  \item {\tt -columns={\em xName},{\em yName}} ---
    Specifies the columns containing the x and y coordinates of the probe points.
  \item {\tt -boundary={\em filename},{\em xName},{\em yName}} --- Specifies the name of the file containing the
    x and y coordinates of the boundaries. Each boundary should form a closed curve (first and last point the same).
    Each boundary is on a separate page.
  \item {\tt -insideColumn={\em columnName}} --- By default, the output file contains all the input data, plus a new
    column called {\tt InsideSum}, which gives the number of boundaries that enclose the point in question.
    This option allows changing the name of the column giving this sum.
  \item {\tt -keep={inside|outside}} --- By default, all input rows appear in the output file. If this option is given,
    the user may elect to keep only those rows that are inside at least one boundary, or only those rows that are outside
    all boundaries.
  \item {\tt -threads={\em number}} --- Specify the number of threads to use for computations. Defaults to 1.
    Using more threads tends to help more when there are many complex boundary contours and when the number of
    output points is a small fraction of the number of input points.
  \end{itemize} 

\item {\bf author:} M. Borland, ANL/APS. 
\end{itemize} 
 

\begin{sddsprog}{sddsinteg}
  \item \textbf{description:} \verb|sddsinteg| integrates one or more columns of data as a function of a common column. The program will perform error propagation if error bars are provided in the data set.
  \item \textbf{examples:}
    Find the integral ${\rm \int \eta_x ds}$ for APS lattices
    \begin{verbatim}
    sddsinteg APS.twi APS.integ -integrate=etax -versus=s
    \end{verbatim}
  \item \textbf{synopsis:}
    \begin{verbatim}
    sddsinteg [-pipe=[input][,output]] [input] [output]
      -integrate=columnName[,sigmaName] ...
      -versus=columnName[,sigmaName] [-mainTemplates=item=string[,...]]
      [-errorTemplates=item=string[,...]]
      [-method=methodName] [-printFinal[=bare][,stdout]]
    \end{verbatim}
  \item \textbf{files:}
    {\em input} is an SDDS file containing columns of data to be integrated. If it contains multiple data pages, each is treated separately. The independent quantity along with the requested integrals is placed in columns in {\em output}. By default, the integral column name is constructed by appending ``Integ'' to the variable column name. If applicable, the column name for the integral error is constructed by appending ``IntegSigma''.
  \item \textbf{switches:}
    \begin{itemize}
      \item \verb|-pipe[=input][,output]| --- The standard SDDS Toolkit pipe option.
      \item \verb|-integrate=columnName[,sigmaName]| --- Specifies the name of a column to integrate, and optionally the name of the column containing the error in the integrand. May be given any number of times.
      \item \verb|-versus=columnName[,sigmaName]| --- Specifies the name of the independent variable column, and optionally the name of the column containing its error.
      \item \verb|-mainTemplates=item=string[,...]| --- Specifies template strings for names and definition entries for the integral columns in the output file. {\em item} may be one of \verb|name|, \verb|description|, \verb|symbol|. The symbols \verb|\%x| and \verb|\%y| are used to represent the independent variable name and the name of the integrand, respectively.
      \item \verb|-errorTemplates=item=string[,...]| --- Specifies template strings for names and definition entries for the integral error columns in the output file. {\em item} may be one of \verb|name|, \verb|description|, \verb|symbol|. The symbols \verb|\%x| and \verb|\%y| are used to represent the independent variable name and the name of the integrand, respectively.
      \item \verb|-method=methodName| --- Specifies the integration method. The default method is ``trapazoid,'' for trapizoid rule integration. Another method is ``GillMiller,'' which is based on cubic interpolation and is much more accurate than trapazoid rule; unlike most higher-order formulae, it is not restricted to equispaced points. (See P.E. Gill and G. F. Miller, The Computer Journal, Vol. 15, N. 1, 80-83, 1972.) Error propagation is available for trapazoid rule integration only. If higher-order integration is needed, one can first interpolate the data with reduced spacing of the independent variable using \verb|sddsinterp| with the \verb|-equispaced| option and \verb|-order=2| or higher, then integrate using \verb|sddsinteg|. This is mathematically equivalent to using a higher-order formula, but is more general as it is not restricted to initially equispaced data. However, using Gill-Miller is probably the best approach.
      \item \verb|-printFinal[=bare][,stdout]| --- Specifies that the final value of each integral should be printed out. By default, the printout goes to stderr and includes the name of the integral. If \verb|bare| is given, the names are omitted. If \verb|stdout| is given, the printout goes to stdout.
    \end{itemize}
  \item \textbf{see also:}
    \begin{itemize}
      \item \progref{sddsderiv}
      \item \progref{sddsinterp}
    \end{itemize}
  \item \textbf{author:} M. Borland, ANL/APS.
\end{sddsprog}


\begin{sddsprog}{sddsinterp}
  \item \textbf{description:}
  \verb|sddsinterp| does polynomial interpolation of one or more columns of data as a function of a common independent variable. Interpolation may be done at specified points, at a sequence of points, or at points given in another SDDS file.
  \item \textbf{examples:}
\begin{verbatim}
sddsinterp APS.twi APS.interp -column=s,betax,betay -order=2 -sequence=250
\end{verbatim}
  \item \textbf{synopsis:}
\begin{verbatim}
sddsinterp [-pipe[=input][,output]] [inputFile] [outputFile]
  [-columns=independentQuantity,name[,name...]]
  { -atValues=valuesList |
    -fillIn |
    -sequence=points[,start,end] |
    -fileValues=valuesFile[,column=columnName][,parallelPages] }
  [-order=number] [-printOut[=bare][,stdout]]
  [-belowRange={value=value | skip | saturate | extrapolate | wrap}[,{abort | warn}]
  [-aboveRange={value=value | skip | saturate | extrapolate | wrap}[,{abort | warn}]
\end{verbatim}
  \item \textbf{files:}
  \emph{inputFile} is an SDDS file containing columns of data to be interpolated. One column is selected as the independent variable. Any number of others may be specified as dependent variables. If \emph{inputFile} contains multiple data pages, each is treated separately. \emph{outputFile} contains the independent variable values at which interpolation was performed, in a column with the same name as the independent variable in \emph{inputFile}. Similarly, the interpolated values are placed in \emph{outputFile} under the same names as the independent columns from \emph{inputFile}.
  \item \textbf{switches:}
    \begin{itemize}
      \item \verb|-pipe[=input][,output]| --- The standard SDDS Toolkit pipe option.
      \item {\tt -columns={\em independentQuantity},{\em name}[,{\em name}...]} --- Specifies the names of the independent and dependent variable columns.
      \item {\tt -atValues={\em valuesList}} --- Specifies a comma-separated list of values at which interpolation is done.
      \item {\tt -sequence={\em points}[,{\em start},{\em end}]} --- Specifies a sequence of equispaced points at which interpolation is done. If {\em start} and {\em end} are given, they specify the range of these points. If they are not given, the range is the range of the independent data.
      \item \verb|-fillIn| --- Somewhat like {\tt -sequence={\em points}}, except the number of points is chosen so that the spacing of the interpolation points is equal to the minimum point spacing in the file. In other words, if you have a data file with non-equidistant points, this option interpolates to give you equidistant points with the same minimum spacing as the original data. This tends to fill in the space between widely-spaced points, hence the name.
      \item {\tt -fileValues={\em valuesFile}[,column={\em columnName}][,parallelPages]} --- Specifies a set of values at which interpolation is to be done. In this case, the values are extracted from a column ({\em columnName}) of an SDDS file ({\em valuesFile}). If {\tt parallelPages} is given, then successive pages of {\em inputFile} are interpolated at points given by successive pages of {\em valuesFile}. Otherwise, each page of {\em inputFile} is interpolated at the values in all pages of {\em valuesFile}; this can take quite some time if both files have many pages with many rows.
      \item {\tt -order={\em number}} --- The order of the polynomials to use for interpolation. The default is 1, indicating linear interpolation.
      \item {\tt -printOut[=bare][,stdout]} --- Specifies that interpolated values should be printed to stderr. By default, the printout contains text identifying the quantities; this may be suppressed by specifying {\tt bare}. Output may be directed to the standard output by specifying {\tt stdout}.
      \item \begin{raggedright}{\tt -belowRange=\{value={\em value} | skip | saturate | extrapolate | wrap\}[,\{ abort | warn\}], -aboveRange=\{value={\em value} | skip | saturate | extrapolate | wrap\}[,\{ abort | warn\}] }\end{raggedright} --- These options specify the behavior in the event that an interpolation point is, respectively, below or above the range of the independent data. If such an out-of-range point occurs, the default behavior is to assign the value at the nearest endpoint of the data; this is identical to specifying {\tt saturate}. One may specify use of a specific value with {\tt value={\em value}}. {\tt skip} specifies that offending points should be discarded. {\tt extrapolate} specifies extrapolation beyond the limits of the data. {\tt wrap} specifies that the data should be treated as periodic. {\tt abort} specifies that the program should terminate. {\tt warn} requests warnings for out-of-bounds points.
    \end{itemize}
  \item \textbf{see also:} \hyperref[exampleData]{Data for Examples}, \progref{sddspfit}.
  \item \textbf{author:} M. Borland, H. Shang, R. Soliday ANL/APS.
\end{sddsprog}

%\begin{latexonly} 
\newpage 
%\end{latexonly} 
\subsection{sddsinterpset} 
\label{sddsinterpset} 
 
\begin{itemize} 
\item {\bf description:} \hspace*{1mm}\\ 
{\tt sddsinterpset} is used to perform multiple interpolations.

\item {\bf synopsis:}  
\begin{flushleft}
{\tt 
sddsinterpset [{\em input}] [{\em output}] [-pipe=[input],[output]] \\ \
{}[-order={\em number}] \\ \
{}[-data=fileColumn={\em colName},interpolate={\em colName}, \\ \
 functionOf={\em colName},[column={\em colName}|atValue={\em value}]] \\ \
{}[-belowRange={value={\em value}|skip|saturate|extrapolate|wrap}[,{abort|warn}]] \\ \
{}[-aboveRange={value={\em value}|skip|saturate|extrapolate|wrap}[,{abort|warn}]] \\ \
{}[-verbose] \\ \
{}[-majorOrder=row|column]}
\end{flushleft} 

\item {\bf files:}
    \begin{itemize} 
    \item {\em input} Each row contains the name of an SDDS file used as the source of an interpolation table.
    \end{itemize} 

\item {\bf switches:} 
    \begin{itemize} 
    \item {\tt -pipe=[input][,output]} --- Standard SDDS pipe options for reading/writing files from stdin/stdout.
    \item {\tt -order} --- The order of the polynomials to use for interpolation. The default is 1, indicating linear interpolation.
    \item {\tt -data} --- 
      \begin{itemize} 
        \item {\tt fileColumn} --- Gives the name of a column in {\em input} that contains the names of files with tables of data.
        \item {\tt interpolate} --- Gives the name of a column that must exist in all the files named in fileColumn.  This column will exist in the output file.
        \item {\tt functionOf} --- Gives the name of a column that must exist in all the files named in fileColumn.  The 'interpolate' column is viewed as a function of this column:  I(F).
        \item {\tt column} --- Gives the name of a column in {\em input}.  The primary output is I(C).
        \item {\tt atValue} --- Gives a number at which to perform the interpolations.
      \end{itemize} 
    \item {\tt -belowRange}
    \item {\tt -aboveRange} --- They have the same options, which specify the behavior in the event that an interpolation point is, respectively, below or above the range of the independent data. If such an out-of-range point occurs, the default behavior is to assign the value at the nearest endpoint of the data; this is identical to specifying saturate. One may specify use of a specific value with value=value. skip specifies that offending points should be discarded. extrapolate specifies extrapolation beyond the limits of the data. wrap specifies that the data should be treated as periodic. abort specifies that the program should terminate. warn requests warnings for out-of-bounds points.
    \item {\tt -verbose} --- Errors are printed to stdout.
    \item {\tt -majorOrder=row|column} --- Specifies the binary SDDS layout.
    \end{itemize} 

\item {\bf author:} H. Shang, R. Soliday, X. Jiao, ANL/APS. 
\end{itemize} 

%\begin{latexonly} 
\newpage 
%\end{latexonly} 
\subsection{sddslocaldensity} 
\label{sddslocaldensity} 
 
\begin{itemize} 
\item {\bf description:} \hspace*{1mm}\\ 
{\tt sddslocaldensity} the local density of points in n-dimensional space for each point in the space.
\item {\bf examples:} 
\begin{flushleft}{\tt
   sddslocaldensity <inputFile> <outputFile> -columns=none,x,y -kde=bins=40 -threads=20
}\end{flushleft} 
\item {\bf synopsis:}  
\begin{flushleft}
\begin{verbatim}
sddslocaldensity [<inputfile>] [<outputfile>] [-pipe=[input][,output]] [-threads=<number>]
\{-fraction=<value> | -spread=<value> | 
 -kde=bins=<number>[,gridoutput=<filename>][,nsigma=<value>][,explimit=<value>]
      [,sample=<fraction>|use=<number>]\}
-columns=<normalizationMode>,<name>[,...] [-output=<columnName>] [-verbose]
\end{verbatim}
\end{flushleft} 
\item {\bf files:} 
The input file contains a collection of points in n-dimensional space, defined by named columns.
The output file contains the same data, but with an additional column {\tt LocalDensity} that gives 
the density of points in the vicinity of each point.
\item {\bf switches:} 
  \begin{itemize} 
  \item {\tt -pipe[=input]} --- The standard SDDS Toolkit pipe option. 
  \item {\tt -columns={none|range|rms},{\em column1Name},{\em column2Name ...}} --- 
    Specifies the normalization mode and column names for analysis.
    Note that the normalization mode is irrelevant when fraction, spread, or kde options are used.
  \item {-fraction | -spread | -kde} --- Specify the calculation mode, if different from the default.
    Note that all methods except the KDE method show $N^2$ growth in run time, where $N$ is the number of points.
    \begin{itemize}
      \item By default, the ``local density'' for each point is the inverse of the mean distance to all other points.
      \item For {\tt -fraction=f} mode, the ``local density'' for each point is the number of points inside a distance $d_i$
        in dimension $i$, where $d_i = f*(\max(x_i) - \min(x_i))$.
      \item For {\tt -spread=f} mode, the ``local density'' for each point is the sum over all other points
        of the product of unnormalized gaussian pread functions in each dimensional, where the gaussian parameter
        is $\sigma = f*(\max(x_i) - \min(x_i))$. Since the gaussians are unnormalized, the result is roughly the
        number of nearby points.
      \item {\tt -kde=bins=<number>[,gridoutput=<filename>][,nsigma=<value>]}
        {\tt [,explimit=<value>][,sample=<fraction>|use=<number>]} ---
        Performs Kernel Density Estimation in n dimensions using the given number of bins in all dimensions, using
        Silverman's method to estimate the bandwidth in each dimension.
        The {\tt gridOutput} qualifier results in writing the density map to the named file.
        The {\tt nsigma} qualifier specifies truncation of contributions to the density outside the given number of
        bandwidths; it defaults to 5. The {\tt explimit} qualifier specifies a cutoff in the magnitude of $e^{-z^2/2}$ 
        for including contributions to the density; it defaults to $10^{-16}$. Setting {\tt nsigma} to a smaller value
        and {\tt explimit} to a larger value can significantly reduce run time with little impact on accuracy.
        The {\tt sample} and {\tt use} qualifiers allow reducing the number of input points used for computing the
        density map; this again can significantly reduce run time and may have little impact on results 
        if the number of input points is large; regardless of this setting, the estimated local density is output for
        every input point.
    \end{itemize}
  \item {\tt -threads={\em number}} --- Specify the number of threads to use for KDE-based computations. Defaults to 1.
    The speed-up is best when the number of KDE bins is relatively small or the number of dimensions is relatively small.
  \item {\tt -outputColumn={\em name}} --- Gives the name of the output column containing the ``local density.''
    Defaults to {\tt LocalDensity}.
  \item {\tt -verbose} --- If given, provides informational output during execution.
  \end{itemize} 

\item {\bf author:} M. Borland, ANL/APS. 
\end{itemize} 
 

\begin{sddsprog}{sddsmakedataset}
\item \textbf{description:} \verb|sddsmakedataset| writes input data into an SDDS file or pipe. It can be used to create a small SDDS file from a script, and is more convenient than \verb|sddssave|.
\item \textbf{examples:}
\begin{verbatim}
sddsmakedataset mydata.sdds \
  -parameter=pi,type=double -data=3.1415926 \
  -parameter=UserName,type=string -data=somebody \
  -column=index,type=short -data=1,2,3,4,5,6,7,8,9,10 \
  -column=primeNumbers,type=long -data=1,2,3,5,7,11,13,17,19,23 \
  -column=lettersOfAlphabet,type=character -data=a,b,c,d,e,f,g,h,i,j \
  -ascii
\end{verbatim}
An ASCII file \verb|mydata.sdds| is created by this command. The printout using \progref{sddsprintout} is:
\begin{verbatim}
Printout for SDDS file mydata.sdds

pi =           3.141593e+00  UserName =         somebody

 index  primeNumbers  lettersOfAlphabet
----------------------------------------
     1             1                  a
     2             2                  b
     3             3                  c
     4             5                  d
     5             7                  e
     6            11                  f
     7            13                  g
     8            17                  h
     9            19                  i
    10            23                  j
\end{verbatim}
\item \textbf{synopsis:}
\begin{verbatim}
sddsmakedataset [<outputFile> | -pipe=out] \
  [-defaultType={double|float|long|short|string|character}] \
  [-parameter=<name>[,type=<string>][,units=<string>][,symbol=<string>][,description=<string>]] \
  [-data=<value>] -parameter=... -data=... \
  [-column=<name>[,type=<string>][,units=<string>][,symbol=<string>][,description=<string>]] \
  [-data=<listOfCommaSeparatedValue>] -column=... -data=... \
  [-noWarnings] [-description=<string>] [-contents=<string>] [-mode=<string>]
\end{verbatim}
\item \textbf{switches:}
\begin{itemize}
  \item \verb|outputFile| --- SDDS output file for writing the data.
  \item \verb|-pipe=out| --- Output the data in SDDS format to the pipe instead of to a file.
  \item \verb|-defaultType| --- Specify the default data type for parameters and columns if not specified in the parameter or column definition.
  \item \verb|-parameter| --- Specify the parameter name, data type, units, symbol and description.
  \item \verb|-column| --- Specify the column name, data type, units, symbol and description.
  \item \verb|-noWarnings| --- Do not print warning messages.
  \item \verb|-ascii| --- Output the file in ASCII mode; default is binary.
  \item \verb|-description| --- Description of output file.
  \item \verb|-contents| --- Contents of the description.
\end{itemize}
\item \textbf{see also:}
\begin{itemize}
  \item \progref{sddsprintout}
\end{itemize}
\item \textbf{author:} H. Shang, ANL/APS.
\end{sddsprog}

%
%\begin{latexonly}
\newpage
%\end{latexonly}

%
% Substitute the program name for <programName>
%
\subsection{sddsmatrixmult}
\label{sddsmatrixmult}

\begin{itemize}
\item {\bf description:}
%
% Insert text of description (typicall a paragraph) here.
%
\verb+sddsmatrixmult+ multiplies the matrices represented in the
two input files and puts the results in the output file.

String columns are ignored and not copied to the output file.

\item {\bf examples:} 
%
% Insert text of examples in this section.  Examples should be simple and
% should be preceeded by a brief description.  Wrap the commands for each
% example in the following construct:
% 
%
In an accelerator beamline a linear relationship exists between the
corrector dipole setpoints and the beam position monitor (BPM) readbacks.
The matrix data in file response is multiplied with the columns of 
file corrector to produce a new file containing values of expected bpm change:
\begin{flushleft}{\tt
sddsmatrixmult response correctorChange bpmExpectedChange
}\end{flushleft}
\item {\bf synopsis:} 
%
% Insert usage message here:
%
\begin{flushleft}{\tt
sddsmatrixmult [-pipe=[input][,output]] [{\em file1}] {\em file2} [{\em output}]
          [-commute] [-reuse] [-verbose] [-ascii]
}\end{flushleft}
\item {\bf files:}
% Describe the files that are used and produced
The first file ({\em file1}) is the SDDS file for left-hand matrix of product.
The second file ({\em file2}) is the SDDS file for right-hand matrix of product.
The third file contains the product matrix data.

\item {\bf switches:}
%
% Describe the switches that are available
%
    \begin{itemize}
    \item {\tt  -pipe[=input][,output]} --- The standard SDDS Toolkit pipe option.
    \item {\tt  -commute} --- Use {\em file1} for right-hand matrix and 
      {\em file2} for 
      left-hand matrix.  Useful with -pipe option
    \item {\tt  -reuse} --- If one file runs out of data pages, 
         then reuse the last one.
    \item {\tt  -ascii}  --- Produces an output in ascii mode. Default is binary.
    \item {\tt  -verbose} --- Write diagnostic messages to stderr.
    \end{itemize}
%\item {\bf see also:}
%    \begin{itemize}
%
% Insert references to other programs by duplicating this line and 
% replacing {\em prog} with the program to be referenced:
%
%    \item \progref{<prog>}
%    \end{itemize}
%
% Insert your name and affiliation after the '}'
%
\item {\bf author: L. Emery } ANL
\end{itemize}




\begin{sddsprog}{sddsmatrixop}
  \item \textbf{description:}
  \verb|sddsmatrixop| performs general matrix operations. The matrices and operations are
  specified on the command line and are executed in an RPN-like fashion. String columns are
  ignored and not copied to the output file.
  \item \textbf{examples:}
  Multiply two matrices:
    \begin{verbatim}
sddsmatrixop A.matrix C.matrix -push=B.matrix -multiply
    \end{verbatim}
  Compute $Y = (1 + (A+B)C)^{-1}$:
    \begin{verbatim}
sddsmatrixop A.matrix Y.matrix -push=B.matrix -add -push=C.matrix -mult -identity -add -invert
    \end{verbatim}
  \item \textbf{synopsis:}
    \begin{verbatim}
sddsmatrixop [inputmatrix] [outputmatrix] [-pipe=[in|out]] [-verbose]
  [-push=<matrix>] [-multiply] [-add] [-substract] [-invert] ...
    \end{verbatim}
  \item \textbf{files:}
  \emph{inputmatrix} is the SDDS file containing the initial matrix on the stack.
  \emph{outputmatrix} is the SDDS file to receive the resulting matrix.
  \item \textbf{switches:}
    \begin{itemize}
      \item \verb|-pipe[=input][,output]| --- The standard SDDS Toolkit pipe option.
      \item \verb|-push=<matrix>| --- Push a matrix onto the stack.
      \item \verb|-verbose| --- Write diagnostic messages to \verb|stderr|.
      \item \verb|-identity[=<number>]| --- Push an identity matrix. If a number is given, it sets the
        dimension; otherwise uses the dimension of the top matrix.
      \item \verb|-add| --- Addition operator.
      \item \verb|-substract| --- Subtraction operator.
      \item \verb|-multiply[=hadamard]| --- Matrix multiplication. If \verb|=hadamard| is given,
        performs element-by-element multiplication.
      \item \verb|-divide=hadamard| --- Element-by-element division.
      \item \verb|-swap| --- Swap the top two elements of the stack.
      \item \verb|-scalarmultiply=<value>| --- Multiply the matrix by a constant.
      \item \verb|-scalardivide=<value>| --- Divide the matrix by a constant.
      \item \verb|-transpose| --- Matrix transpose.
      \item \verb|-invert| --- Matrix inversion.
    \end{itemize}
  \item \textbf{see also:}
    \begin{itemize}
      \item \progref{sddsmatrixmult}
      \item \progref{sddsmatrix2column}
    \end{itemize}
  \item \textbf{author:} H. Shang, ANL.
\end{sddsprog}


%\begin{latexonly}
\newpage
%\end{latexonly}
\subsection{sddsmatrix2column}
\label{sddsmatrix2column}

\begin{sddsprog}{sddsmatrix2column}
  \item \textbf{description:} \verb|sddsmatrix2column| transfers a matrix into a single-column SDDS file.
  \item \textbf{examples:}
    \begin{verbatim}
    sddsmatrix2column matrix.sdds column.sdds
    sddsmatrix2column matrix.sdds column.sdds -rowNameColumn=Label -dataColumnName=Value -rootnameColumnName=ID -majorOrder=row
    \end{verbatim}
  \item \textbf{synopsis:}
    \begin{verbatim}
    sddsmatrix2column [inputfile] [outputfile] [-pipe=[input][,output]]
      [-rowNameColumn=string] [-dataColumnName=string]
      [-rootnameColumnName=string] [-majorOrder=row|column]
    \end{verbatim}
  \item \textbf{files:}
    \begin{itemize}
      \item {\em inputfile} Contains an optional string column and multiple numerical columns. If the string column or \verb|-rowNameColumn| is not provided, Row$<$row\_index$>$ will be used as row names in the output file.
      \item {\em outputfile} Contains two columns: a string column and a data column. The string column combines the input string column (or Row$<$row\_index$>$) and the input data column names.
    \end{itemize}
  \item \textbf{switches:}
    \begin{itemize}
      \item \verb|-pipe=[input][,output]| --- Standard SDDS pipe option for reading/writing from stdin/stdout.
      \item \verb|-rowNameColumn=<string>| --- Column name for row names in the input file. If not provided, Row$<$row\_index$>$ will be used.
      \item \verb|-dataColumnName=<string>| --- Column name of the data column in the output file. Defaults to \verb|Rootname|.
      \item \verb|-rootnameColumnName=<string>| --- Column name of the string column in the output file.
      \item \verb+-majorOrder=row|column+ --- Specifies the binary SDDS layout.
    \end{itemize}
  \item \textbf{see also:} \progref{sddsarray2column}
  \item \textbf{author:} H. Shang, R. Soliday, ANL/APS.
\end{sddsprog}


%\begin{latexonly} 
\newpage 
%\end{latexonly} 
\subsection{sddsminterp} 
\label{sddsminterp} 
 
\begin{itemize} 
\item {\bf description:} \hspace*{1mm}\\ 
{\tt sddsminterp} performs multiplicative renormalized model interpolation of a data set using another data set as a model function.
\item {\bf synopsis:}  
\begin{flushleft}
{\tt 
sddsminterp [{\em input-file}] [{\em output-file}] [-pipe=[input],[output]] \\ \
-columns={\em independent-quantity},{\em name} \\ \
{}[-interpOrder={\em order}] \\ \
-model={\em modelFile},abscissa={\em column},ordinate={\em column}[,interp={\em order}] \\ \
-fileValues={\em valuesFile}[,abscissa={\em column}] \\ \
{}[-majorOrder=row|column] \\ \
-verbose \\ \
-ascii}
\end{flushleft} 
\item {\bf switches:} 
    \begin{itemize} 
    \item {\tt -pipe=[input][,output]} --- Standard SDDS pipe options for reading/writing files from stdin/stdout.
    \item {\tt -columns} --- Specifies the data in the input file to be interpolated.
    \item {\tt -interpOrder} --- Interpolation order of the multiplicative factor.
    \item {\tt -model} --- Data representing the model function.
    \item {\tt -fileValues} --- Specifies abscissa at which interpolated values are calculated. If not present, then the abscissa values of the model file are used.
    \item {\tt -majorOrder=row|column} --- Specifies the binary SDDS layout.
    \item {\tt -verbose} --- Print verbose messages to stdout.
    \item {\tt -ascii} --- Save output in ASCII SDDS format.
\end{itemize} 

\item {\bf author:} L. Emery, C. Saunders, M. Borland, R. Soliday, H. Shang, ANL/APS. 
\end{itemize} 

\begin{sddsprog}{sddsmpfit}
  \item \textbf{description:}
  \verb|sddsmpfit| does ordinary and Chebyshev polynomial fits to column data,
  including error analysis. It will do fits with specified number of terms,
  with specific terms only, and with specific symmetry only. It will also
  eliminate spurious terms. The options for \verb|sddsmpfit| are very similar
  to those for \verb|sddspfit|.
  \item \textbf{examples:}
  Perform a second-order fit for column \verb|y| as a function of \verb|x| and
  create evaluation points for the fit.
  \begin{verbatim}
sddsmpfit data.sdds fit.sdds -independent=x -dependent=y \
  -terms=3 -evaluate=fitEval.sdds
  \end{verbatim}
  \item \textbf{synopsis:}
  \begin{verbatim}
sddsmpfit [-pipe=[input][,output]] [inputFile] [outputFile]
  -independent=xName [-sigmaIndependent=xSigmaName]
  -dependent=yName[,yName...] [-sigmaDependent=templateString]
  {-terms=number [-symmetry={none|odd|even}] | -orders=number[,number...]}
  [-reviseOrders[=threshold=chiValue][,verbose]] [-chebyshev[=convert]]
  [-xOffset=value] [-xFactor=value]
  [-sigmas=value,{absolute|fractional}]
  [-modifySigmas] [-generateSigmas[={keepLargest|keepSmallest}]]
  [-sparse=interval] [-range=lower,upper]
  [-normalize[=termNumber]] [-verbose]
  [-evaluate=filename[,begin=value][,end=value][,number=integer]]
  [-fitLabelFormat=sprintfString] [-infoFile=filename]
  \end{verbatim}
  \item \textbf{files:}
  \emph{inputFile} is an SDDS file containing columns of data to be fit.
  If it contains multiple pages, they are processed separately.
  \emph{outputFile} is an SDDS file containing one page for each page of
  \emph{inputFile}. It contains columns of the independent and dependent
  variable data, plus columns for error bars (``sigmas'') as appropriate.
  The values of the fit and of the residuals are in columns named
  \emph{yName}\verb|Fit| and \emph{yName}\verb|Residual|. In addition,
  various parameters having names beginning with \emph{yName} are created
  that give reduced chi-squared, slope, intercept, and so on.
  \item \textbf{switches:}
    \begin{itemize}
    \item \verb|-pipe[=input][,output]| --- The standard SDDS Toolkit pipe option.
    \item \verb|-evaluate=filename[,begin=value][,end=value][,number=integer]| ---
      Specifies creation of an SDDS file called \emph{filename} containing
      points from evaluation of the fit. The fit is normally evaluated over
      the range of the input data; this may be changed using the
      \verb|begin| and \verb|end| qualifiers. Normally, the number of points
      at which the fit is evaluated is the number of points in the input data;
      this may be changed using the \verb|number| qualifier.
    \item \verb|-infoFile=filename| --- Specifies creation of an SDDS file
      containing results of the fits in columns. A column called
      \emph{yName}\verb|Coefficient| is created for each column that is
      fitted.
    \item By default, an ordinary polynomial fit is done using a constant and
      linear term. Control of what fit terms are used is provided by the
      following switches:
      \begin{itemize}
      \item \verb|-terms=number| --- Specifies the number of terms to be used
        in fitting. 2 terms is a linear fit, 3 is quadratic, etc.
      \item \verb!-symmetry={none|odd|even}! --- When used with \verb|-terms|,
        allows specifying the symmetry of the N terms used. \verb|none| is the
        default. \verb|odd| implies using linear, cubic, etc., while
        \verb|even| implies using constant, quadratic, etc.
      \item \verb|-orders=number[,number...]| --- Specifies the polynomial
        orders to be used in fitting. The default is equivalent to
        \verb|-orders=0,1|.
      \item \verb|-reviseOrders[=threshold=value][,verbose]| --- Asks for
        adaptive fitting to be performed on the first data page to determine
        what orders to use. Any term that does not improve the reduced
        chi-squared by \emph{value} is discarded. Similar to but much less
        capable than the adaptive fitting feature of \verb|sddspfit|.
      \item \verb|-chebyshev[=convert]| --- Asks that Chebyshev T polynomials
        be used in fitting. If \verb|convert| is given, the output contains the
        coefficients for the equivalent ordinary polynomials.
      \end{itemize}
    \item \verb|-xOffset=value|, \verb|-xFactor=value| --- Specify offsetting
      and scaling of the independent data prior to fitting. The transformation
      is $x \rightarrow (x - \mathrm{Offset})/\mathrm{Factor}$. This feature
      can be used to make a fit about a point other than $x=0$, or to scale the
      data to make high-order fits more accurate.
    \item \verb|sddsmpfit| will compute error bars (``sigmas'') for fit
      coefficients if it has knowledge of the sigmas for the data points.
      These can be supplied using the \verb|-columns| switch, or generated
      internally in several ways:
      \begin{itemize}
      \item \verb!-sigmas=value{absolute|fractional}! --- Specifies that
        independent-variable errors be generated using a specified value for
        all points, or a specified fraction for all points.
      \item \verb|-modifySigmas| --- Specifies that independent-variable sigmas
        be modified to include the effect of uncertainty in the dependent
        variable values. If this option is not given, any x sigmas specified
        are ignored.
      \item \verb!-generateSigmas[={keepLargest|keepSmallest}]! --- Specifies
        that independent-variable errors be generated from the variance of an
        initial equal-weights fit. If errors are already given (via
        \verb|-column|), one may request that for every point \verb|sddsmpfit|
        retain the larger or smaller of the sigma in the data and the one given
        by the variance.
      \end{itemize}
    \item \verb|-sparse=interval| --- Specifies sparsing of the input data
      prior to fitting. This can greatly speed computations when the number of
      data points is large.
    \item \verb|-range=lower,upper| --- Specifies the range of independent
      variable over which to do fitting.
    \item \verb|-normalize[=termNumber]| --- Specifies that coefficients be
      normalized so that the coefficient for the indicated order is unity. By
      default, the 0-order term (i.e., the constant term) is normalized to
      unity.
    \item \verb|-verbose| --- Specifies that the results of the fit be printed
      to the standard error output.
    \item \verb|-fitLabelFormat=sprintfString| --- Specifies the format to use
      for printing numbers in the fit label. The default is ``\%g''.
    \end{itemize}
  \item \textbf{see also:}
    \begin{itemize}
    \item \hyperref[exampleData]{Data for Examples}
    \item \progref{sddspfit}
    \item \progref{sddsoutlier}
    \end{itemize}
  \item \textbf{author:} M. Borland, ANL/APS.
\end{sddsprog}


%\begin{latexonly} 
\newpage 
%\end{latexonly} 
\subsection{sddsmselect} 
\label{sddsmselect} 
 
\begin{itemize} 
\item {\bf description:} \hspace*{1mm}\\ 
{\tt sddsmselect} selects data from {\em input1} for writing to {\em output} based on the presence or absence of matching data in {\em input2}. If {\em output} is not given, {\em input1} is replaced.
\item {\bf examples:} 
\begin{flushleft}
{\tt sddsmselect <input1> <input2> <output> -match=ControlName }
\end{flushleft} 
\item {\bf synopsis:}  
\begin{flushleft}
{\tt 
sddsmselect [{\em input1}] [{\em input2}] [{\em output}] [-pipe=[input][,output]] \\ \
{}[-match={\em column-name}[={\em column-name}][,...]] \\ \
{}[-equate={\em column-name}[={\em column-name}][,...]] \\ \
{}[-invert] \\ \
{}[-reuse[=[rows][,page]]] \\ \
{}[-nowarnings] \\ \
{}[-majorOrder=row|column]}
\end{flushleft} 
\item {\bf switches:} 
    \begin{itemize} 
    \item {\tt -pipe=[input][,output]} --- Standard SDDS pipe options for reading/writing files from stdin/stdout.
    \item {\tt -match} --- Specifies names of columns to match between {\em input1} and {\em input2} for selection and placement of data taken from {\em input1}.
    \item {\tt -equate} --- Specifies names of columns to equate between {\em input1} and {\em input2} for selection and placement of data taken from {\em input1}.
    \item {\tt -invert} --- Specifies that only non-matched rows are to be kept..
    \item {\tt -reuse} --- Specifies that rows of {\em input2} may be reused, i.e., matched with more than one row of {\em input1}.  Also, -reuse=page specifies that only the first page of {\em input2} is used.
    \item {\tt -nowarnings} --- Specifies that warning messages should be suppressed.
    \item {\tt -majorOrder=row|column} --- Specifies the binary SDDS layout.
\end{itemize} 

\item {\bf author:} C. Saunders, M. Borland, R. Soliday, H. Shang, ANL/APS. 
\end{itemize} 

\begin{sddsprog}{sddsmultihist}
  \item \textbf{description:} \verb|sddsmultihist| does one-dimensional histograms of multiple columns of data from an SDDS file.
    All columns are histogrammed on the same interval and with the same number of bins. It is similar to \verb|sddshist|,
    except that the latter program only histograms a single column at a time. Unlike \verb|sddshist|, \verb|sddsmultihist|
    does not presently do statistical analyses or filtering.
  \item \textbf{examples:}
    \begin{verbatim}
sddsmultihist par.bpm par.bpmhis -columns=P?P?x -bins=20 -abscissa=xReadout
    \end{verbatim}
  \item \textbf{synopsis:}
    \begin{verbatim}
sddsmultihist [-pipe=[input][,output]] [inputFile] [outputFile]
  -columns=columnName[,columnName...] -abscissa=newName
  [-separate]
  [-exclude=columnName[,columnName...]]
  [{-bins=integer | -sizeOfBins=value | -autobins=target=number[,minimum=integer][,maximum=integer]}]
  [-lowerLimit=value] [-upperLimit=value]
  [-sides]
    \end{verbatim}
  \item \textbf{files:} \emph{inputFile} is the name of an SDDS file containing data to be histogrammed. If \emph{inputFile}
    contains multiple data pages, each is treated separately. The histograms are placed in \emph{outputFile}, which has one
    column of histogram frequencies for each histogrammed input column, plus a column giving the abscissa values for the frequency
    distributions. The former columns have names of the form \emph{columnName}\verb|Frequency|, containing the number of points
    in each bin. The latter column has a name given by the user.
  \item \textbf{switches:}
    \begin{itemize}
      \item \verb|-pipe[=input][,output]| --- The standard SDDS Toolkit pipe option.
      \item \verb|-columns=columnName[,columnName...]| --- Specifies the names of the data columns to be histogrammed. The
        \emph{columnName} items may contain wildcards.
      \item \verb|-separate| --- Specifies that a separate abscissa shall be created for each histogrammed column. If
        \verb|-abscissa| is not given, then the abscissa names are the names of the columns being histogrammed.
      \item \verb|-abscissa=newName[,newName...]| --- Specifies the name or names of the abscissa columns for the histogram
        output. If \verb|-separate| is not given, then only one name is permitted. The units are taken from the units of the
        columns being histogrammed.
      \item \verb|-exclude=columnName[,columnName...]| --- Specifies the names of data columns to exclude from histogramming. The
        \emph{columnName} items may contain wildcards.
      \item \verb|-bins=number| --- Specifies the number of bins to use. The default is 20.
      \item \verb|-sizeOfBins=value| --- Specifies the size of bins to use. The number of bins is computed from the range of the
        data.
      \item \verb|-autoBins=target=number[,minimum=integer][,maximum=integer]| --- Specifies that the number of bins should be
        chosen to attempt to give a target number of samples per bin on average. If \verb|minimum| is given, then no fewer than
        the specified number of bins will be used (default: 5). If \verb|maximum| is given, then no more than the specified number
        of bins will be used (default: number of samples).
      \item \verb|-lowerLimit=value| --- Specifies the lower limit of the histogram. By default, the lower limit is the minimum
        value in the data.
      \item \verb|-upperLimit=value| --- Specifies the upper limit of the histogram. By default, the upper limit is the maximum
        value in the data.
      \item \verb|-sides| --- Specifies that zero-height bins should be attached to the lower and upper ends of the histogram.
        Many prefer the way this looks on a graph.
    \end{itemize}
  \item \textbf{see also:}
    \begin{itemize}
      \item \hyperref[exampleData]{Data for Examples}
      \item \progref{sddshist}
      \item \progref{sddshist2d}
    \end{itemize}
  \item \textbf{author:} M. Borland, ANL/APS.
\end{sddsprog}


%\begin{latexonly} 
\newpage 
%\end{latexonly} 
\subsection{sddsmxref} 
\label{sddsmxref} 
 
\begin{itemize} 
\item {\bf description:} \hspace*{1mm}\\ 
{\tt sddsmxref} takes columns, parameters, and arrays from succesive pages from file {\em input2} and adds them to successive pages from {\em input1}. If {\em output} is given, the result is placed there; otherwise, {\em input1} is replaced. By default, all columns are taken from {\em input2}.
\item {\bf examples:} 
\begin{flushleft}
{\tt sddsmxref <input1> <input2> <output> -take=Values }
\end{flushleft} 
\item {\bf synopsis:}  
\begin{flushleft}
{\tt 
sddsmxref [{\em input1}] [{\em input2}] [{\em output}] [-pipe=[input][,output]] \\ \
[-ifis={column|parameter},{\em name}[,...]] \\ \
[-ifnot={parameter|column|array},{\em name}[,...]] \\ \
[-transfer={parameter|array},{\em name}[,...]] \\ \
[-take={\em column-name}[,...]]  \\ \
[-leave={\em column-name}[,...]] \\ \
[-fillIn] \\ \
[-match={\em column-name}[={\em column-name}][,...]] \\ \
[-equate={\em column-name}[={\em column-name}]]  \\ \
[-reuse[=[rows][,page]]] \\ \
[-rename={column|parameter|array},{\em oldname}={\em newname}[,...]] \\ \
[-editnames={column|parameter|array},{\em wildcard-string},{\em edit-string}] \\ \
[-nowarnings] \\ \
[-majorOrder=row|column]}
\end{flushleft} 
\item {\bf switches:} 
    \begin{itemize} 
    \item {\tt -pipe=[input][,output]} --- Standard SDDS pipe options for reading/writing files from stdin/stdout.
    \item {\tt -ifis} --- Specifies names of parameters, arrays, and columns that must exist in {\em input1} if the program is to run as asked.
    \item {\tt -ifnot} --- Specifies names of parameters, arrays, and columns that may not exist in {\em input1} if the program is to run as asked.
    \item {\tt -transfer} --- Specifies names of parameters or arrays to transfer from {\em input2}.
    \item {\tt -take} --- Specifies names of columns to take from {\em input2}.
    \item {\tt -leave} --- Specifies names of columns not to take from {\em input2}. Overrides -take if both name a given column. -leave=* results in no columns being taken.
    \item {\tt -fillIn} --- Specifies filling in NULL and 0 values in rows for which no match is found.  By default, such rows are omitted.
    \item {\tt -match} --- Specifies names of columns to match between {\em input1} and {\em input2} for selection and placement of data taken from {\em input2}.
    \item {\tt -equate} --- Specifies names of columns to equate between {\em input1} and {\em input2} for selection and placement of data taken from {\em input2}.
    \item {\tt -reuse} --- Specifies that rows of {\em input2} may be reused, i.e., matched with more than one row of {\em input1}.  Also, -reuse=page specifies that only the first page of {\em input2} is used.
    \item {\tt -rename} --- Specifies new names for entities in the output data set. The entities must still be referred to by their old names in the other commandline options.
    \item {\tt -editnames} --- Specifies creation of new names for entities of the specified type with names matching the specified wildcard string. Editing is performed using commands reminiscent of emacs keystrokes. if -editnames=<entity>{column|parameter|array},wildcard,ei/\%ld/ is specified, the entity names will be changed to <name>N, N is the position of input files in the command line.
    \item {\tt -nowarnings} --- Specifies that warning messages should be suppressed.
    \item {\tt -majorOrder=row|column} --- Specifies the binary SDDS layout.
\end{itemize} 

\item {\bf author:} C. Saunders, M. Borland, R. Soliday, H. Shang, ANL/APS. 
\end{itemize} 

\begin{sddsprog}{sddsnaff}
  \item \textbf{description:}
    \verb|sddsnaff| is an implementation of Laskar's Numerical Analysis of Fundamental
    Frequencies (NAFF) algorithm. This algorithm determines the frequency components
    of a signal more accurately than Fast Fourier Transforms (FFT). FFTs are used as
    part of the analysis, so if an FFT is sufficient for an application, \progref{sddsfft}
    should be used as it will be much faster.
    The algorithm starts by removing the average value of the signal and applying a
    Hanning window. Next, the signal is FFT'd and the frequency at which the maximum
    FFT amplitude occurs is found. This is taken as the starting frequency for a
    numerical optimization of the ``overlap'' between the signal and $e^{i\omega t}$,
    which allows determining $\omega$ to resolution greater than the frequency spacing
    of the FFT. Once $\omega$ is determined, the overlap is subtracted from the
    original signal and the process is repeated, if desired.
  \item \textbf{examples:}
    Find the first fundamental frequency for each of the BPM signals in \verb|par.bpm|.
    \begin{verbatim}
sddsnaff par.bpm par.naff -column=Time,'P?P?x' -terminateSearch=frequencies=1
    \end{verbatim}
  \item \textbf{synopsis:}
    \begin{verbatim}
sddsnaff [inputfile] [outputfile]
  [-pipe=[input][,output]]
  [-columns=indep-variable[,depen-quantity[,...]]]
  [-pair=<column1>,<column2>]
  [-exclude=depen-quantity[,...]]
  [-terminateSearch={changeLimit=fraction[,maxFrequencies=number] | frequencies=number}]
  [-iterateFrequency=[cycleLimit=number][,accuracyLimit=fraction]]
  [-truncate] [-noWarnings]
    \end{verbatim}
  \item \textbf{files:}
    \emph{inputFile} contains the data to be NAFF'd. One column must be chosen as the
    independent variable. If \emph{inputFile} contains multiple pages, each is treated
    separately and delivered to a separate page of \emph{outputFile}.
    \emph{outputFile} contains two columns for each selected column in \emph{inputFile}.
    These columns have names like \emph{origColumn}\verb|Frequency| and
    \emph{origColumn}\verb|Amplitude|, giving the frequency and amplitude for
    \emph{origColumn}.
  \item \textbf{switches:}
    \begin{itemize}
      \item \verb|-pipe[=input][,output]| --- The standard SDDS Toolkit pipe option.
      \item \verb|-columns=indepVariable[,depenQuantityList]| --- Specifies the name of the
        independent variable column. Optionally, if no \verb|-pair| options are given, this
        specifies a comma-separated list (optionally with wildcards) of dependent quantities to be
        NAFF'd as a function of the independent variable. By default, all numerical columns except
        the independent column are NAFF'd.
      \item \verb|-pair=<column1>,<column2>| --- Specifies the names of conjugate pairs to give
        double the frequency range. \emph{column1} is used to obtain the basic frequency and
        \emph{column2} is used to obtain the phase at the frequency of the first column. The
        relative phase expands the resulting frequency from 0--Fn to 0--2\*Fn. Multiple
        \verb|-pair| options may be provided. The independent column is provided by \verb|-columns|,
        while dependent columns may be provided by either \verb|-columns| or \verb|-pair|.
      \item \verb|-exclude=depenQuantity,...| --- Specifies optionally wildcarded names of columns to
        exclude from analysis.
      \item \verb!-terminateSearch={changeLimit=fraction[,maxFrequencies=number] | frequencies=number}! ---
        Specifies when to stop searching for frequency components. If \verb|changeLimit| is given,
        the program stops when the RMS change in the signal is less than the specified \emph{fraction}
        of the original RMS value. The maximum number of frequencies returned in this mode is set with
        \verb|maxFrequencies| (default is 4). If \verb|frequencies| is given, the program finds the
        specified number of frequencies, if possible. By default, one frequency is found for each
        signal.
      \item \verb|-iterateFrequency=[cycleLimit=number][,accuracyLimit=fraction]| --- Controls the
        optimization procedure that searches for the best frequency. By default, the procedure
        executes 100 passes and attempts to determine the frequency to a precision of 0.00001 of the
        Nyquist frequency. \verb|cycleLimit| changes the number of passes, while \verb|accuracyLimit|
        specifies the desired precision.
      \item \verb|-truncate| --- Specifies that the data should be truncated so that the number of
        points is the largest product of primes from 2 to 19 not greater than the original number of
        points. In some cases, this results in significantly greater speed by making the FFTs faster.
      \item \verb|-noWarnings| --- Suppresses warning messages.
    \end{itemize}
  \item \textbf{see also:}
    \begin{itemize}
      \item \hyperref[exampleData]{Data for Examples}
      \item \progref{sddsfft}
    \end{itemize}
  \item \textbf{author:} M. Borland, ANL/APS.
\end{sddsprog}


\begin{sddsprog}{sddsnormalize}
  \item \textbf{description:} \verb|sddsnormalize| performs various normalizations of column data.
  \item \textbf{examples:}
    \begin{verbatim}
    sddsnormalize data.sdds norm.sdds -columns=mode=spread,column1,column2
    sddsnormalize data.sdds -columns=suffix=Norm,mode=largest,column1 -pipe=out
    \end{verbatim}
  \item \textbf{synopsis:}
    \begin{verbatim}
    sddsnormalize [inputFile] [outputFile] [-pipe[=input][,output]]
      -columns=[mode=mode][,suffix=string][,exclude=wildcardString],columnName[,columnName...]
    \end{verbatim}
  \item \textbf{switches:}
    \begin{itemize}
      \item {\tt -pipe[=input][,output]} --- The standard SDDS Toolkit pipe option.
      \item {\tt -columns=[mode={\em mode}][,suffix={\em string}][,exclude={\em wildcardString}],columnName[,columnName...]} ---
        Any number of these options may be given. Each specifies columns to normalize and in what {\em mode}. {\em mode} may be
        one of {\tt minimum}, {\tt maximum}, {\tt largest}, {\tt signedLargest}, or {\tt spread}, referring to the factor used for normalization.
        {\tt largest} (the default) is the maximum absolute value; {\tt signedLargest} is the same value but with the sign restored;
        {\tt spread} is the maximum minus the minimum. Each {\em columnName} qualifier gives a possibly wildcarded string specifying
        columns to normalize. {\tt exclude} may be used to exclude columns from normalization that are matched by a {\em columnName}.
        The {\tt suffix} qualifier optionally specifies a suffix to be appended to each column name, to create a new column for the output file; if not given, then the original data are replaced with the normalized data.
    \end{itemize}
  \item \textbf{files:} \emph{inputFile} is an SDDS file containing data to be processed. The \emph{outputFile} argument is optional. If it is not given, and if an output pipe is not selected, then the input file will be replaced.
  \item \textbf{see also:} \progref{sddsprocess}
  \item \textbf{author:} M. Borland, ANL/APS.
\end{sddsprog}

%\begin{latexonly}
\newpage
%\end{latexonly}
\subsection{sddsoutlier}
\label{sddsoutlier}

\begin{itemize}
\item {\bf description:}
{\tt sddsoutlier} does outlier elimination of rows from SDDS tabular
data.  An ``outlier'' is a data point that is statistically unlikely
or else invalid.
\item {\bf example:}
Eliminate ``bad'' beam-position-monitor readouts from PAR x BPM data, where
a bad readout is one that is more than three standard deviations from the
mean:
\begin{flushleft}{\tt
sddsoutlier par.bpm par.bpm1 -columns=P?P?x -stDevLimit=3
}\end{flushleft}
Fit a line to readout P1P1x vs P1P2x, then eliminate points too far from the line.
\begin{flushleft}{\tt
sddspfit par.bpm -pipe=out -columns=P1P2x,P1P1x \\
| sddsoutlier -pipe=in par.2bpms -column=P1P1xResidual -stDevLimit=2
}\end{flushleft}
Same, but refit and redo outlier elimination based on the improved fit:
\begin{flushleft}{\tt
sddspfit par.bpm -pipe=out -columns=P1P2x,P1P1x \\
| sddsoutlier -pipe par.2bpms -column=P1P1xResidual -stDevLimit=2 \\
| sddspfit -pipe -columns=P1P2x,P1P1x \\
| sddsoutlier -pipe=in par.2bpms -column=P1P1xResidual -stDevLimit=2 
}\end{flushleft}

\item {\bf synopsis:} 
\begin{flushleft}{\tt
sddsoutlier [-pipe=[input][,output]] [{\em inputFile}] [{\em outputFile}]
[-columns={\em listOfNames}] [-excludeColumns={\em listOfNames}]
[-stDevLimit={\em value}] [-absLimit={\em value}] [-absDeviationLimit={\em value}]
[-minimumLimit={\em value}] [-maximumLImit={\em value}] 
[-chanceLimit={\em value}] [-invert] [-verbose] [-noWarnings] 
[\{-markOnly | -replaceOnly=\{lastValue | nextValue | interpolatedValue | value={\em number}\}\}]
}\end{flushleft}
\item {\bf files:}
{\em inputFile} contains column data that is to be winnowed using outlier elimination.
If {\em inputFile} contains multiple pages, the are treated separately.  {\em
outputFile} contains all of the array and parameter data, but only those rows of the
tabular data that pass the outlier elimination.  {\em Warning}: if {\em outputFile} is
not given and {\tt -pipe=output} is not specified, then {\em inputFile} will be
overwritten.
\item {\bf switches:}
    \begin{itemize}
    \item {\tt -pipe[=input][,output]} --- The standard SDDS Toolkit pipe option.
    \item {\tt -columns={\em listOfNames}} --- Specifies a comma-separated list of
        optionally wildcard containing column names.  Outlier analysis and elimination
        will be applied to the data in each of the specified columns independently.
        No row that is eliminated by outlier analysis of any of these columns will      
        appear in the output.  If this option is not given, all columns are 
        included in the analysis.
    \item {\tt -excludeColumns={\em listOfNames}} --- Specifies a comma-separated list of
        optionally wildcard containing column names that are to be excluded from 
        outlier analysis.
    \item {\tt -stDevLimit={\em value}} --- Specifies the number of standard deviations
        by which a data point from a column may deviate from the average for the column
        before being considered an outlier.
    \item {\tt -absLimit={\em value}} --- Specifies the maximum absolute value that
        a data point from a column may have before being considered an outlier.
    \item {\tt -absDeviationLimit={\em value}} --- Specifies the maximum absolute value
        by which a data point from a column may deviate from the average for the column
        before being considered an outlier.
    \item {\tt -minimumLimit={\em value}}, {\tt -minimumLimit={\em value}} --- 
        Specify minimum or maximum values that data points may have without being considered
        outliers.
    \item {\tt -chanceLimit={\em value}} --- Specifies placing a lower limit on the probability
        of seeing a data point as a means of removing outliers.  Gaussian statistics are
        used to determine the probability that each point would be seen in sampling a gaussian
        distribution a given number of times (equal to the number of points in each page).
        If this probability is less than {\em value}, then the point is considered an outlier.
        Using a larger {\em value} results in elimination of more points.
    \item {\tt -invert} --- Specifies that only outlier points should be kept.
    \item {\tt -markOnly} --- Specifies that instead of deleting outlier points, they
        should be only marked as outliers.  This is done by creating a new column
        ({\tt IsOutlier}) in the output file that contains a 1 (0) if the row has (no)
        outliers.  If {\tt IsOutlier} is in the input file, rows with a value of 1 are
        treated as outliers and essentially ignored in processing.  Hence, successive
        invocations of {\tt sddsoutlier} in a data-processing pipeline make use of 
        results from previous invocations even if {\tt -markOnly} is given.  Note: if {\tt -markOnly}
        is not given, then the presence of {\tt IsOutlier} in the input file has no effect.
    \item {-tt -replaceOnly=\{lastValue | nextValue | interpolatedValue | value={\em number}\}} ---
        Specifies replacing outliers rather than removing them.  {\tt lastValue} ({\tt nextValue})
        specifies replacing with the previous (next) value in the column.  
        {\tt interpolatedValue} specifies interpolating a new value from the last and next value
        (with row number as the independent quantity).  {\tt value={\em number}} specifies 
        replacing outliers with {\em number}. 
    \item {\tt -verbose} --- Specifies that informational printouts should be provided.
    \item {\tt -noWarnings} --- Specifies that warnings should be suppressed.
    \end{itemize}
\item {\bf see also:}
    \begin{itemize}
    \item \hyperref[exampleData]{Data for Examples}
    \item \progref{sddspfit}
    \item \progref{sddsgfit}
    \item \progref{sddsexpfit}
    \item \progref{sddscorrelate}
    \end{itemize}
\item {\bf author:} M. Borland, ANL/APS.
\end{itemize}




\begin{sddsprog}{sddspeakfind}
  \item \textbf{description:}
  \verb|sddspeakfind| finds the locations and values of peaks in a single column of an SDDS file.  It incorporates
  various features to help reject spurious peaks.  The column is considered a function of the row index for the
  purpose of finding peaks.  Hence, the data should be sorted if necessary using \verb|sddssort| prior to using this
  program.  I.e., if the data contains columns x and y, and one wants x values of peaks in y, then one should
  ensure that the rows are sorted into increasing or decreasing x order.

  It may also be helpful to smooth the data using \verb|sddssmooth| in order to eliminate spurious peaks due to
  noisy data.
  \item \textbf{examples:}
    Find peaks in a Fourier transform:
    \begin{verbatim}
sddspeakfind data.fft data.peaks -column=FFTamplitude
    \end{verbatim}
    Sort and smooth the data first:
    \begin{verbatim}
sddssort data.fft -column=f,increasing -pipe=out |
  sddssmooth -pipe -columns=FFTamplitude |
  sddspeakfind -pipe=in data.peaks -column=FFTamplitude
    \end{verbatim}
  \item \textbf{synopsis:}
    \begin{verbatim}
sddspeakfind [-pipe=[input][,output]] [inputFile] [outputFile]
  -column=columnName [-fivePoints] [-threshold=value]
  [-exclusionZone=fractionalInterval] [-changeThreshold=fractionalChange]
    \end{verbatim}
  \item \textbf{files:}
  \emph{inputFile} contains the data to be searched for peaks.  \emph{outputFile} contains all of the array and
  parameter data from \emph{inputFile}, plus data from all rows that contain a peak in the named column.  No new
  data elements are created.  If \emph{inputFile} contains multiple pages, each is treated separately and is
  delivered to a separate page of \emph{outputFile}.
  \item \textbf{switches:}
    \begin{itemize}
      \item \verb|-pipe[=input][,output]| --- The standard SDDS Toolkit pipe option.
      \item {\tt -column={\em columnName}} --- Specifies the name of the column to search for peaks.
      \item \verb|-fivePoints| --- Specifies peak analysis using five adjacent data points, rather than
        the default three.  For three-point mode, a peak is any point which is larger than both of
        its two nearest neighbors.  For five-point mode, the candidate point's nearest neighbors must in turn
        be higher than their nearest neighbors on the side away from the candidate point.
      \item {\tt -threshold={\em value}} --- Specifies a minimum value that a peak value must exceed in order
        to be included in the output.  By default, no threshold is applied.
      \item {\tt -exclusionZone={\em fractionalInterval}} --- Specifies elimination of smaller peaks within a given interval
        around a larger peak.  {\em fractionalInterval} is the width of the interval in units of the length of the data table.
      \item {\tt -changeThreshold={\em fractionalChange}} --- Specifies elimination of peaks for which the fractional
        change between the peak value and the nearest neighbor points is less than the given amount.  If
        \verb|-fivePoints| is given, the nearest neighbors in question are those 2 rows above and below the
        peak.
    \end{itemize}
  \item \textbf{see also:}
    \begin{itemize}
      \item \progref{sddsfft}
      \item \progref{sddssmooth}
    \end{itemize}
  \item \textbf{author:} M. Borland, ANL/APS.
\end{sddsprog}


\begin{sddsprog}{sddspfit}
  \item \textbf{description:} \verb|sddspfit| does ordinary and Chebyshev polynomial fits to column data, including error analysis. It will do fits with a specified number of terms, with specific terms only, and with specific symmetry only. It will also eliminate spurious terms.
  \item \textbf{examples:}
  \begin{verbatim}
sddspfit data.sdds fit.sdds -columns=x,y -terms=3
sddspfit par.bpm par.bpm1 -pipe=out -columns=P1P2x,P1P1x
  \end{verbatim}
  \item \textbf{synopsis:}
  \begin{verbatim}
sddspfit [-pipe=[input][,output]] [inputFile] [outputFile]
  [-evaluate=filename[,begin=value][,end=value][,number=integer]]
  -columns=xName,yName[,xSigma=name][,ySigma=name]
  {-terms=number [-symmetry={none | odd | even}] | -orders=number[,number...]}
  [-reviseOrders[=threshold=chiValue][,verbose][,complete=<chiThreshold>][,goodEnough=<chiValue>]]
  [-chebyshev[=convert]]
  [-xOffset=value] [-xFactor=value]
  [-sigmas={absolute=value | fractional=value}]
  [-modifySigmas] [-generateSigmas[={keepLargest | keepSmallest}]]
  [-sparse=interval] [-range=lower,upper]
  [-normalize[=termNumber]] [-verbose]
  [-fitLabelFormat=sprintfString]
  \end{verbatim}
  \item \textbf{files:} \emph{inputFile} is an SDDS file containing columns of data to be fit. If it contains multiple pages, they are processed separately. \emph{outputFile} is an SDDS file containing one page for each page of \emph{inputFile}. It contains columns of the independent and dependent variable data, plus columns for error bars (``sigmas'') as appropriate. The values of the fit and of the residuals are in columns named \emph{yName}\verb|Fit| and \emph{yName}\verb|Residual|. \emph{outputFile} also contains the following one-dimensional arrays:
    \begin{itemize}
      \item \verb|Order|: a long integer array of the polynomial orders used in the fit.
      \item \verb|Coefficient|: a double-precision array of fit coefficients.
      \item \verb|CoefficientSigma|: a double-precision array of fit coefficient errors. Present only if errors are present for data.
      \item \verb|CoefficientUnits|: a string array of fit coefficient units.
    \end{itemize}
    \emph{outputFile also contains the following parameters:}
    \begin{itemize}
      \item \verb|Basis|: a string identifying the type of polynomials used.
      \item \verb|ReducedChiSquared|: the reduced chi-squared of the fit:
      $$ \chi^2_\nu = \frac{\chi^2}{\nu} = \frac{1}{N-T}\sum_{i=0}^{N-1} \left(\frac{y_i - y(x_i)}{\sigma_i}\right)^2 $$,
      where $\\nu = N-T$ is the number of degrees of freedom for a fit of N points with T terms.
      \item \verb|rmsResidual|
      \item \emph{xName}\verb|Offset|, \emph{xName}\verb|Factor|
      \item \verb|FitIsValid|: a character having values \verb|y| and \verb|n| if the page contains a valid fit or not.
      \item \verb|Terms|: the number of terms in the fit.
      \item \verb|sddspfitLabel|: a string containing an equation showing the fit, suitable for use with \verb|sddsplot|.
      \item \verb|Intercept|, \verb|Slope|, \verb|Curvature|: the three lowest order coefficients for ordinary polynomial fits. These are present only if orders 0, 1, and 2 respectively are requested in fitting. If error analysis is valid, then the errors for these quantities appear as \emph{quantityName}\verb|Sigma|.
    \end{itemize}
  \item \textbf{switches:}
    \begin{itemize}
      \item \verb|-pipe[=input][,output]| --- The standard SDDS Toolkit pipe option.
      \item \verb|-evaluate=filename[,begin=value][,end=value][,number=integer]| --- Specifies creation of an SDDS file called \emph{filename} containing points from evaluation of the fit. The fit is normally evaluated over the range of the input data; this may be changed using the \verb|begin| and \verb|end| qualifiers. Normally, the number of points at which the fit is evaluated is the number of points in the input data; this may be changed using the \verb|number| qualifier.
      \item \verb|-columns=xName,yName[,xSigma=name][,ySigma=name]| --- Specifies the names of the columns to use for the independent and dependent data, respectively. \verb|xSigma| and \verb|ySigma| can be used to specify the errors for the independent and dependent data, respectively.
      \item By default, an ordinary polynomial fit is done using a constant and linear term. Control of what fit terms are used is provided by the following switches:
        \begin{itemize}
          \item \verb|-terms=number| --- Specifies the number of terms to be used in fitting. 2 terms is linear fit, 3 is quadratic, etc.
          \item \verb!-symmetry={none | odd | even}! --- When used with \verb|-terms|, allows specifying the symmetry of the N terms used. \verb|none| is the default. \verb|odd| implies using linear, cubic, etc., while \verb|even| implies using constant, quadratic, etc.
          \item \verb|-orders=number[,number...]| --- Specifies the polynomial orders to be used in fitting. The default is equivalent to \emph{-orders=0,1}.
          \item \verb|-reviseOrders[=threshold=chiValue][,verbose][,complete=<chiThreshold>][,goodEnough=<chiValue>]| --- Specifies adaptive fitting to eliminate spurious terms. When invoked, this switch causes \verb|sddspfit| to repeatedly fit the first page of data with different numbers of terms in an attempt to find a minimal number of terms that gives an acceptable fit. This is done in up to three stages:
            \begin{enumerate}
              \item The process starts by making a fit with all terms. Then, each term is eliminated individually and a new fit is made. If the new fit has a smaller reduced chi-squared by an amount of at least \emph{chiValue}, then the term is permanently eliminated and the process is repeated for each remaining term. By default, the criterion for an improvement is a change of 0.1 in the reduced chi-squared. This step eliminates terms that result in a bad fit due to numerical problems. If the \verb|goodEnough=chiValue| qualifier is given, then the first fit that has reduced chi-squared less than \emph{chiValue} is used.
              \item Next, the individual terms are tested for how well they improve reduced chi-squared. Any term that does not improve the reduced chi-squared by at least \emph{chiValue} is eliminated. This stage eliminates terms that do not sufficiently improve the fit to merit inclusion. Again, if the \verb|goodEnough=chiValue| qualifier is given, then the first fit that has reduced chi-squared less than \emph{chiValue} is used.
              \item Finally, if \verb|complete=chiThreshold| is given, then next stage involves repeating the above procedure with the remaining terms, but instead of eliminating one term at a time, the program tests each possible combination of terms. This can be very time consuming, especially if the \verb|goodEnough=chiValue| qualifier is not given.
            \end{enumerate}
          \item \verb|[-chebyshev[=convert]]| --- Asks that Chebyshev T polynomials be used in fitting. If \verb|convert| is given, the output contains the coefficients for the equivalent ordinary polynomials.
        \end{itemize}
      \item \verb|-xOffset=value|, \verb|-xFactor=value| --- Specify offsetting and scaling of the independent data prior to fitting. The transformation is ${\\rm x \\rightarrow (x - Offset)/Factor}$. This feature can be used to make a fit about a point other than $x=0$, or to scale the data to make high-order fits more accurate.
      \item \verb|sddspfit| will compute error bars (``sigmas'') for fit coefficients if it has knowledge of the sigmas for the data points. These can be supplied using the \verb|-columns| switch, or generated internally in several ways:
        \begin{itemize}
          \item \verb!-sigmas={absolute=value | fractional=value}! --- Specifies that independent-variable errors be generated using a specified value for all points, or a specified fraction for all points.
          \item \verb|-modifySigmas| --- Specifies that independent-variable sigmas be modified to include the effect of uncertainty in the dependent variable values. If this option is not given, any x sigmas specified with \verb|-columns| are ignored.
          \item \verb!-generateSigmas[={keepLargest | keepSmallest}]! --- Specifies that independent-variable errors be generated from the variance of an initial equal-weights fit. If errors are already given (via \emph{-column}), one may request that for every point \verb|sddspfit| retain the larger or smaller of the sigma in the data and the one given by the variance.
        \end{itemize}
      \item \verb|-sparse=interval| --- Specifies sparsing of the input data prior to fitting. This can greatly speed computations when the number of data points is large.
      \item \verb|-range=lower,upper| --- Specifies the range of independent variable over which to do fitting.
      \item \verb|-normalize[=termNumber]| --- Specifies that coefficients be normalized so that the coefficient for the indicated order is unity. By default, the 0-order term (i.e., the constant term) is normalized to unity.
      \item \verb|-verbose| --- Specifies that the results of the fit be printed to the standard error output.
      \item \verb|-fitLabelFormat=sprintfString| --- Specifies the format to use for printing numbers in the fit label. The default is ``\%g''.
    \end{itemize}
  \item \textbf{see also:}
    \begin{itemize}
      \item \hyperref[exampleData]{Data for Examples}
      \item \progref{sddsexpfit}
      \item \progref{sddsgfit}
      \item \progref{sddsplot}
      \item \progref{sddsoutlier}
    \end{itemize}
  \item \textbf{author:} M. Borland, ANL/APS.
\end{sddsprog}

%
%\begin{latexonly}
\newpage
%\end{latexonly}

\subsection{sddsplot}
\label{sddsplot}

\begin{itemize}
\item {\bf description:}
{\tt sddsplot} is a general purpose device-independent graphics program for displaying parameter and column data
from SDDS files.  The program is equally capable of quick-and-dirty plots and publication quality graphics.  It
allows organization of large amounts of data from multiple files into useful plots with minimal effort.  It
provides line, point, symbol, impulse, error-bar, and arrow plotting, with automatic variation of color, linetype,
etc.  It can do data winnowing using the data to be graphed or other data in the file.  Parameters from a file can
be designated for use as plot labels, legends, or for placement on the plot in specified locations.  Data pages may
be tagged and sorted by multiple criteria.

{\tt sddsplot} supports various flavors of Postscript, various windows options, and numerous graphics terminals.  For
X-windows, a GUI interface is generated that supports zoom/pan, cursor readout, movie mode, and much more.

\item {\bf examples:} 
Plot the horizontal beta function for the APS design:
\begin{flushleft}{\tt 
sddsplot -columnNames=s,betax APS0.twi
}
\end{flushleft}
Plot the Twiss functions for the APS design, using different line types for
each quantity:
\begin{flushleft}{\tt 
sddsplot -columnNames=s,'(beta?,etax)' APS0.twi -graph=line,vary 
}
\end{flushleft}
Plot the Twiss functions for APS lattices, one plotting page per lattice (i.e., per
data page), with different linetypes and a legend:
\begin{flushleft}{\tt 
sddsplot -columnNames=s,'(beta?,etax)' APS.twi -graphic=line,vary -legend
        -split=page -separate=page
}\end{flushleft}
Plot the Twiss functions for APS lattices, one plotting page per function, with each 
data page shown with a different line type:
\begin{flushleft}{\tt 
sddsplot -columnNames=s,'(beta?,etax)' APS.twi -graphic=line,vary
        -split=page -groupby=nameIndex -separate=nameIndex
}
\end{flushleft}
Plot the Twiss functions for the APS design, using a common scale for the beta functions
and another for eta:
\begin{flushleft}{\tt
sddsplot -graphic=line,vary APS0.twi -columnNames=s,beta? -yScalesGroup=id=beta 
-columnNames=s,etax -yScalesGroup=id=eta
}

\end{flushleft}

\item {\bf sddsplot concepts:}\\

{\tt sddsplot} has a very large number of options and is very flexible.   In most cases, only a very few
of these options are employed.  In order to make best use of {\tt sddsplot}, it helps to be familiar with
certain concepts. 

{\tt sddsplot} supports multiple ``plot pages'' and multiple ``panels'' per page.  In this context, a
``plot page'' is a separate sheet of paper for hardcopy devices, and the equivalent for interactive
devices.  For example, when using the X-windows interface just described, separate plot pages are held in
memory so that the user may go back and forth between them rapidly, or run them as a movie.  A plot page
may contain several nonoverlapping panels, each displaying essentially independent graphics.  Presently,
{\tt sddsplot} divides the plot page into an array of plot panels, each of equal size.  The default is one
plot panel per plot page.

Within each plot panel, {\tt sddsplot} may display data from any number of ``plot requests''.  A plot
request is a specification to {\tt sddsplot} of what data to plot from what files, and how to do it.  A plot
request must contain or indicate a list of names of columns and parameters to display, as well as the names
of one or more files from which to extract the data.  The data from plot requests are organized into plot
panels and plot pages according to certain defaults or explicit instructions.  One frequent choice is to
move to a new panel for each plot request.  However, one may also regroup data to display data from
different plot requests together.

For each request, the names of columns and parameters are grouped to form sets of sets of data
element names.  For example, {\tt -columnNames=s,(betax,betay,etax)} results in formation of three
sets of pairs: {\tt (s, betax)}, {\tt (s, betay)}, and {\tt (s, etax)}.  In a more complicated
example, the sets of dataname sets might include names of error-bar data (e.g., {\tt (x, y, ySigma)})
or vector components (e.g., {\tt (x, y, Ex, Ey)}).  To avoid confusion, a set of datanames like those
just listed will be referred to as a ``name group''.  Each name group for a request is given a
sequential ``name index'', which can be used as shown in the last example above.

Each panel is divided into two regions, a ``plot space'' (or ``pspace'') and a ``label space''.  The
pspace is the region where data is displayed.  Outside the pspace is the label space, where labels
and legends normally appear.

Any point on any plot panel can be referenced by unit coordinates that start at zero in the lower
left corner of the panel and end at unity in the upper right corner.  The extent of the pspace is
given in these coordinates.  By default, the region the pspace occupies in these coordinates is
[0.15, 0.90]x[0.15, 0.90].  The extent of the pspace may be changed explicitly, or it
may be altered implicitly by certain switches (e.g., to make room for legends).  The data or user's
coordinates, referred to as (x, y), are mapped onto this space, as are the ``pspace coordinates'',
(p, q).  The latter are [0, 1]x[0, 1] coordinates.

When {\tt sddsplot} reads data in from files, it collects it into internal data sets. By default,
each of these internal data sets contains the all of the data for one name group from one file.  That
is, an internal data set normally contains all of the data for a name group from an SDDS data set.
The phrase ``internal data set'' is used to maintain the distinction between the SDDS data set and
the representation of data from an SDDS data set within {\tt sddsplot}.  Associated with each
internal data set is the request number, the filename, the file number within the request, the y
dataname, the name group index within the request, the starting page number from the file, and an
optional user-specified tag value (from the commandline or a parameter in the file).  These values
may be used to sort and group the data in order to place on individual panels sets of similar data
from multiple sources.  An instance of this is shown in the last example above.

\item {\bf synopsis:} 
\begin{flushleft}{\tt
sddsplot [{\em X11Switches}] [{\em commonSwitches}]
{\em plotRequestSwitch} {\em fileNames} {\em localSwitches}
[{\em plotRequestSwitch} {\em fileNames} {\em localSwitches} \ldots]
}\end{flushleft}

The {\tt sddsplot} command line is organized into three categories.  First, one may issue any of the standard X11
switches (e.g., -geometry).  Second, one may give a set of switches, indicated by {\em commonSwitches}, that will
apply to all subsequest plot requests.

Third, one gives a series of ``plot requests''.  A plot request starts with one of several switches that give the
names of data elements to be plotted.  It continues with the names of one or more files from which this data is to
be extracted.  In addition, one may include various switches that apply only to current plot request.  These may,
for example, override any common switches that were set prior to the first plot request.  In general, any switch
may be given as a common switch (so that it applies to all plot requests unless overridden) or as a local switch.

In the examples above, only a single plot request is exhibited.  There are no X11 switches and no common switches
set.  The plot request is initiated by the {\tt -columnNames} switch.  The {-graphic} and {-legend} switches are
local switches.

\item {\bf switches:}
\begin{itemize}
\item Initiating a plot request:\\
  \begin{itemize}
  \item {\tt -columnNames={\em xName},{\em yNameList}[,\{{\em y1NameList} | {\em x1Name},{\em y1NameList}\}] }---
        Specifies the names of columns to be plotted.  {\em xName} may be the name of a numeric or string column,
        which is normally plotted against the horizontal or x axis.
        {\em yNameList} gives the comma-separated, optionally wildcarded names of one or more columns of numeric
        data.  Data for each item in {\em yNameList} will be paired with the x data for plotting.

        Some types of plotting require additional data, such as error bars or vector components.  These are
        specified with the {\em x1Name} and {\em y1NameList}.  Each item in {\em y1NameList} is paired with the
        corresponding item in {\em yNameList}; the lists must have the same length.  The interpretation of the
        additional data is specified with the {\tt -graphic=error} or {\tt -arrow} switches. For error bar
        plotting, one may give error bars for both x and y by giving {\em x1Name} and {\em y1NameList}, or for y
        only by giving {\em y1NameList}.  For arrow plotting, giving {\em y1NameList} only is allowable for
        vectors perpendicular to the page. Giving both {\em x1Name} and {\em y1NameList} is required for vectors
        in the plane of the page.

        One may give several {\tt -columnNames} switches in a row in order to specify additional ``datanames'' for
        the request.  This may be convenient if, for example, one wants several different x variables.
  \item {\tt -xExclude={\em xNameList}}---
        specifies the names of x columns to be excluded from the plot. 
        {\em xNameList} gives the comma-separated, optionally wildcarded names of one or more columns.
  \item {\tt -yExclude={\em yNameList}}---
        specifies the names of y columns to be excluded from the plot. 
        {\em yNameList} gives the comma-separated, optionally wildcarded names of one or more columns.
  \item {\tt -toPage={\em pagenumber}}---
        specifies the page number to which sddsplot plots.
  \item {\tt -fromPage={\em pagenumber}}---
        specifies the page number from which sddsplot plots.      
  \item {\tt -parameterNames={\em xName},{\em yNameList}[,\{{\em y1NameList} | {\em x1Name},{\em y1NameList}\}]}---
        Identical to {\tt -columnNames}, except it specifies parameter data to be plotted.  As with {\tt -columnNames},
        several such options may be given in a row in order to add datanames.
  \item {\tt -arrayNames={\em xName},{\em yNameList}[,\{{\em y1NameList} | {\em x1Name},{\em y1NameList}\}]}---
        Identical to {\tt -columnNames}, except it specifies array data to be plotted.  As with {\tt -columnNames},
        several such options may be given in a row in order to add datanames.

  \item {\tt -keep[=\{names | files\}]} ---
        Rarely used.
        Specifies starting a new plot request, but retaining certain information from the previous request.
        If given without qualifiers, the datanames (as specified by {\tt -columnNames} or {\tt -parameterNames})
        and filenames from the previous request are kept; this allows plotting the same data again in a different
        way.  If the {\tt names} qualifier is given, the datanames from the previous request are retained.
        If the {\tt files} qualifier is given, the filenames from the previous request are retained.

  \item {\tt -mplfiles[=noTitle][,noTopline]} ---
        Provided for compatibility with an older type of data file and rarely used.
        Allows plotting of {\tt mpl} data files with {\tt sddsplot}.  The x and y columns of the {\tt mpl} file
        are used.  The qualifiers may be employed to inhibit use of the {\tt mpl} plot title and topline.

  \item {\tt -namescan=\{all | first\}} --- Specifies whether {\tt sddsplot} should scan all input files when searching
        for matches to wildcard datanames, or only the first.  The default is to scan all files, which may be slow
        for many files with large numbers of columns or parameters.
  \end{itemize}
   
\item Controlling output type:
  \begin{itemize}
  \item {\tt -listDevices}---Lists the names of available graphics devices to the standard error output.
  \item {\tt -device={\em deviceName}[,{\em deviceArguments}]}---Specifies the name of the graphics
        device, plus optional device-specific arguments.  The default device is ``motif'', unless the
        {\tt SDDS\_DEVICE} environment variable if defined, in which case the default device is the one
        named.
        Some commonly-used devices that have device-specific arguments are:
        \begin{itemize} 
        \item {\tt motif} --- The device arguments are a single string of space-separated entries of the
        form {\tt -{\em resourceName} {\em value}}.  These are passed directly to the MOTIF ``outboard-driver''
        without any interpretation.For example,{\tt "-dashes {\em 1}"} qualifier sets the line types with
        built-in dash styles; {\tt "-linetype {\em linetypeFileName }"} forces MOTIF "outboard-driver" to use
        the user-defined line types (color,dash,thickness)  in a SDDS file intead of the default line types.
        Other resource names may be found in the help for the driver.
        \item {\tt qt} --- The device arguments are given as a space-separated string passed to the Qt
        driver.  Valid options include {\tt -dashes <0|1>}, {\tt -linetype <file>}, {\tt -movie 1}
        [{\tt -interval <sec>}], {\tt -keep <number>}, {\tt -share <name>}, {\tt -timeoutHours <hours>}, and
        {\tt -spectrum}.
        \item {\tt postscript},{\tt cpostscript} --- Four qualifiers are presently accepted. {\tt dash} sets
        the cpostscript device to use built-in dash styles for the line drawing.
        {\tt linetypetable={\em linetypeFileName}} replaces the default line types with the customized line types
        defined in a SDDS file. {\tt onblack} and {\tt onwhite} set the background of the plot.
        \item {\tt png} --- PNG devices accept {\tt rootname}, {\tt template}, {\tt onwhite}, {\tt onblack}, 
        {\tt dash} and {\tt linetypetable}  device arguments. {\tt rootname={\em string}} specifies a rootname
        for automatic filename generation; the resulting filenames are of the form {\em rootname}.DDD, where DDD 
        is a three-digit integer. {\tt template={\em string}} provides a more general facility; one uses it to
        specify an sprintf-style format string to use in creating filenames. For example, the behavior obtained
        using {\tt rootname={\em name}} may be obtained  using {\tt template={\em name}.\%03ld}.
        \item {\tt mif} --- Three qualifiers are presently accepted.  {\tt linesizeDefault={\em size}} sets the
        default line thickness (normally 0.25).  {\tt dashsizeDefault={\em size}} sets the default dash size
        (normally 1.0).  {\tt lineIncrement={\em value}} sets the line thickness increment between different
        line types.
        \item {\tt gif}, {\tt tgif}, {\tt sgif}, {\tt mgif}, {\tt lgif} --- No longer supported, use {\tt png}, 
        {\tt tpng}, {\tt spng}, {\tt mpng}, {\tt lpng} instead.
        \end{itemize}
  \item {\tt -output={\em filename}}---Specifies the name of a file to which graphics output will be sent.
        Used primarily for hardcopy devices (e.g., Postscript) where the data will be sent to a printer.
        By default, the data for such devices is printed to the standard output.
  \end{itemize}
\item Controlling type of plotting:
  \begin{itemize}

  \item {\tt -graphic={\em element}[,type={\em integer}][,fill][,subtype=\{{\em integer} | type\}]}
        {\tt [,connect[=\{{\em linetype} | type | subtype\}]][,vary[=\{type | subtype\}]][,modulus={\em integer}]}
        {\tt [,scale={\em factor}][,\{eachFile | eachPage | eachRequest | fixForName | ]}
	{\tt fixForFile | fixForRequest\}]} ---
        Specifies the type of graphic element to use for data in the present plot request.  

{\em element} may be one of {\tt line}, {\tt symbol}, {\tt errorBar}, {\tt impulse}, {\tt yimpulse}, {\tt
bar}, {\tt ybar}, {\tt dot}, or {\tt continue}.  These are largely self-explanatory.  {\tt continue}
specifies continuing whatever was done in the previous request.  {\tt impulse} is a line extending from
y=0 to the data value, while {\tt bar} is a line extending from the bottom of the plot region to the data
value.  {\tt yimpulse} and {\tt ybar} are analogous except that the line extends from x=0 or from the
left-hand vertical border of the plot.

The {\tt type} field for the graphic element has different meanings for different elements.  For
lines, impulses, bars, and dots, the type is the color or line style used, depending on the device.
For most devices, values between 0 and 15 inclusive given unique lines.  For symbols and error bars,
the type specifies the style of symbol or error bar to use; the value is between 0 and 8 inclusive
for symbols and between 0 and 1 inclusive for error bars.  For symbols, one may give the {\tt fill}
qualifier to get filled-in symbols.

The {\tt subtype} field is meaningful only for symbols, error bars, and dots.  It specifies the line style
or color to be used in making a symbol or error bar, and the size for a dot.  As for the type field for
line plotting, the value may be between 0 and 15 inclusive.  The {\tt connect} qualifier is also valid for
symbols and error bars only.  It specifies that the symbols and error bars should be connected by lines.
By default, the line type used is 0.

If one desires automatic variation of the line color, symbol type, and
so on, one may obtain this using the \verb|vary| qualifier.  By
default, the type is varied.  The \verb|eachFile|, \verb|eachPage|,
\verb|eachRequest|, \verb|fixForName|, \verb|fixForFile|, and \verb|fixForRequest|
qualifiers may be given to specify how to
assign type or subtype.  For \verb|eachFile|, variation is done
separately for data from different files.  For \verb|eachPage|,
variation is done separately for data from different pages (hence,
items from different pages would have the same line or symbol).  For
\verb|eachRequest|, variation is done separately for each request.
The \verb|fixForName| qualifier in constrast assigns fixed graphic
attributes to items according to the y name.  The \verb|fixForFile|
qualifier assigns fixed graphic attributes for items according to which
data file they are from.  The \verb|fixForRequest| qualifier assigns fixed
graphic attributes according to the request in which the data originates.

  \item {\tt -arrowSettings=[,autoScale][scale={\em factor}][,linetype={\em integer}]}
{\tt [,centered][,singleBarb][,barbLength={\em value}][,barbAngle={\em value}]}
{\tt [,\{cartesianData | polarData | scalarData\}]}
---Specifies parameters for plotting vectors using arrows.

{\tt autoScale} specifies that the scale factor for the length of arrows should be chosen
automatically; if several data pages are being plotted separately, the same scale is used for all of
them.  {\tt scale} may be used instead of {\tt autoScale} to set the factor manually; if both are
given, then the factor given with {\tt scale} multiplies that computed by {\tt autoScale}.

{\tt linetype} specifies the line type to use for the arrows, using the same mechanism as for lines
in the {\tt -graphic} switch.  The default is 0.

{\tt cartesianData}, {\tt polarData}, and {\tt scalarData} specify the type of data being provided.
For the first two, one must have specified both {\em x1Name} and {\em y1NameList} in the plot
request; for {\tt cartesianData}, x1 and y1 are the x and y vector components, while for {\tt
polarData} x1 is the length and y1 is the angle in radians from the positive x direction.

{\tt centered} specifies that arrows should be centered on the corresponding (x, y) point; by
default, the arrow starts at the (x, y) point.  {\tt singleBarb} specifies that arrows should have
only a single barb, rather than the default two barbs; this can be significantly faster for large
amounts of data.  {\tt barbLength} and {\tt barbAngle} specify the length and angle of arrow barbs;
the barb length is a specified as a fraction of the arrow length, which the barb angle is specified
in degrees.

  \item {\tt -linetypeDefault={\em integer}[,thickness={\em value}]}--- Specifies the default line type for borders, legend
        text, labels, axes, and so on.  If not given, 0 is used.

\end{itemize}

\item Controlling the plotting region:
  \begin{itemize} 

   \item {\tt -scales={\em xmin},{\em xmax},{\em ymin},{\em ymax}}---Specifies the region of the plot in user's
coordinates.  If {\em xmin} and {\em xmax} are equal, then autoscaling is used in x, and similarly for y. Note
that data outside the specified region is still plotted, so that proper clipping of lines occurs.
 
 \item {\tt -range=[\{x|y\}Minimum={\em value}][,\{x|y\}Maximum={\em value}][,\{x|y\}Center={\em value}]}
        --- Constrains the extent
        and center of the plot in user's coordinates.  {\tt xminimum} specifies the minimum allowable
        horizontal extent of the plot; if the autoscaled (or user-specified) range is less than this, the
        range is increased symmetrically to this value.  Similarly, {\tt xmaximum} specifies the maximum
        allowable horizontal extent of the plot.  {\tt xcenter} specifies the center of the horizontal
        range without affecting the extent.  The {\tt y} options are the same, but for the vertical
        coordinates.
  \item {\tt -unsuppressZero[=x][,y]}---Specifies that x=0 and/or y=0 should be within the region of
the plot.  If given without qualifiers, both x and y are ``unsuppressed''.

  \item {\tt -sameScale[=x][,y][,global]}---Specifies that separate panels of data shall be displayed
on the same scales.  In other words, any autoscaling is done based on all of the data from a request,
rather than simply the data on a particular plot panel.  If given without these qualifiers, both x
and y are affected.  {\tt global} forces {\tt sddsplot} to impose the desired condition across all
plot requests.

  \item {\tt -zoom=[\{x|y\}Factor={\em value}][,\{x|p\}Center={\em value}][,\{y|q\}Center={\em
value}]} --- Specifies zoom and pan starting from the scales set by autoscaling or by {\tt -scales}.
A factor less than (greater than) unity zooms out (in).  For each dimension, one may specify the
center of the plot using either the

  \item {\tt -aspectRatio={\em value}} --- Specifies the y/x aspect ratio of the plot.  The value
must be nonzero.  If it is positive, then the desired aspect ratio is obtained by altering the
pspace.  If it is negative, the desired aspect ratio (the absolute value of the value given) is
obtained by altering the data coordinate range.

  \item {\tt -pSpace={\em hMin}{,\em hMax}{,\em vMin}{,\em vMax}}---This option is seldom used, but allows
control of the region of the panel that is mapped to data coordinates, said region being the ``plot space''
or ``pspace''.  The first two coordinates give the horizontal extent, while the second two give the 
vertical extent.  The coordinate values are between 0 and 1.  The defaults are [0.15, 0.9]x[0.15, 0.9].
  \end{itemize}
\item Controlling axes, numeric labels, ticks, and grids:
  \begin{itemize}
  \item {\tt -axes[=x][,y][,linetype={\em integer}]}---Specifies that axes will be placed on the plot,
if they are visible.  By default, both x and y axes are created, with the same linetype as the labels,
scales, and plot border.  One may select a given axis by supply the {\tt x} or {\tt y} qualifier.
One may specify the line type to use for the axes using the {\tt linetype} qualifier.

  \item {\tt -tickSettings=[,[\{x|y\}]grid][[\{x|y\}]spacing={\em value}]} {\tt
[,[\{x|y\}]factor={\em value}][,[\{x|y\}]modulus={\em value}]} {\tt [,[\{x|y\}]size={\em
fraction}][,[\{x|y\}]linetype={\em integer}]} {\tt [,[\{x|y\}]logarithmic]} --- Specifies how to make
ticks and numeric labels for the x and y dimensions. All of the qualifiers have an {\em x} and {\em
y} variant, e.g., {\tt xgrid} and {\tt ygrid}.  Some have a variant that includes both x and y (e.g.,
{\tt grid}).  In the case of the grid option, {\tt xgrid} specifies grid lines rather than ticks for
the x dimension, {\tt ygrid} is similar for the y dimension, and {\tt grid} specifies grid lines in
both dimensions.

The {\tt factor} qualifiers specify factors to apply to the data values in producing the labels.  For example, one
might want to muliply small values by a power of ten in order to get labels that are of order units.  The {\tt
spacing} values give the spacing of the ticks and labels with any factor included.  I.e., to keep the same number
of ticks, {\tt factor} and {\tt spacing} values must be increased together.  Usually, giving the {\tt spacing}
qualifiers is unnecessary, since {\tt sddsplot} chooses appropriate values.

The {\tt modulus} qualifiers allow printing the modulus of the label value rather than the value itself; for
example, one might use {\tt xmodulus=24} if x was the time in hours over many days.  The {\tt size} qualifiers
permit specification of the size of the ticks as a fraction of the range in the opposing dimension; the default is
0.02.  The {\em linetype} qualifiers specify the linetype to be used for ticks and grid lines, using integer values
as for the {\tt -graph=line} switch.  The {\tt logarithmic} qualifiers specify log-style ticks and labels; the
implication is that the data being plotted is the base-ten logarithm of something.

  \item {\tt -subTickSettings=[[\{x|y\}]divisions={\em integer}][,[\{x|y\}]grid]} {\tt [,[\{x|y\}]linetype={\em
integer}][,[\{x|y\}]size={\em fraction}][,xNoLogLabel][,yNoLogLabel]}---Specifies whether and how to make subticks or subgrid lines for the
x and y dimensions.  All of the qualifiers have two or more variants, one that applies to x, one that applies
to y, and (in some cases) one that applies to both.  For example, {\tt xgrid} requests grid lines for x, {\tt
ygrid} requests grid lines for y, and {\tt grid} requests grid lines for both x and y.  The {\tt divisions}
qualifiers specify the number of subdivisions of the major tick intervals; the default is none.  The {\tt linetype}
qualifiers specify the line type to use for subticks or subgrid lines.  The {\tt fraction} qualifiers specify
the size of the subticks as a fraction of the plotting region; the default is 0.01. {\em xNoLogLabel} and {\em yNoLogLabel} specify whether plot subtick labels for log scale axis, they are only valid if the axis uses log scale.

  \item {\tt -yScalesGroup=\{ID={\em string} | fileIndex | nameIndex | nameString | page | request | tag | subpage | iNameString\}} --- 
        Specifies multiple vertical scales.  The most common form is {\tt -yscalesGroup=namestring}, which
        uses a separate scale for every separately-named quantity.  Otherwise, one specifies a separate
        scale for items from different files (by file index), with different name index, different page,
        and so on.  The \verb|tag| is a quality of a dataset specified with the \verb|-tag| option.
        \verb|iNameString| is the name string in inverse order (i.e., so that one compares namestrings
        starting at the end rather than the beginning).  These qualifiers are shared with the
        \verb|groupBy| and \verb|separate| options.
        
  \item {\tt -xScalesGroup} --- Identical to \verb|yScalesGroup| but for x axis scales.

  \item {\tt alignZero[=\{xcenter|xfactor|pPin={\em value}\}][,\{ycenter|yfactor|qPin={\em value}\}]} ---
  This option is provides a facility for lining up zeros on plots with multiple axes.  You must give
at least one of the qualifiers.  The {\tt xfactor} and {\tt yfactor} qualifiers request multiplication
of the upper and lower limits for each scale by the smallest factors that will line up the zeros.
The {\tt xcenter} and {\tt ycenter} qualifiers position the zeros at the center of the plot space,
which may result in empty regions on the plot.  The {\tt pPin} and {\tt qPin} allow specifying the
point at which to ``pin'' the zeros, in plot-space coordinates (0 to 1).

  \item {\tt -grid[=x][,y]}---This option is superseeded by the {\tt -tickSettings} option.  It permits specification
that grids (rather than ticks) will be used for major divisions.

  \item {\tt -noScales}---Specifies that no scales (i.e., no ticks, subticks, or numeric labels) will be plotted.

  \item {\tt -noBorder}---Specifies that no border will be made around the plot region.  Implies {\tt -noScales}.

  \end{itemize}
\item Controlling text labels:
  \begin{itemize}

  \item {\tt -xLabel=[\{@{\em parameterName} | {\em string}\} | use=\{name | symbol |
description\}][,units][,offset={\em value}][,scale={\em value}][,edit={\em string}]}---Controls size,
placement, and content of the x dimension label, which appears directly under the scale labels.  The
default text is of the form {\tt {\em symbol} ({\em units})}, where the symbol and units are taken from
the column or parameter definition fields in the SDDS header for the x data.  If the symbol is blank, then
the element name is used.  Alternatively, the text may be taken from a named string parameter, or from a
string that is given explicitly, or the user may specify with the {\tt use} qualifier that the
element name, symbol, or descrpition be used.  The user may also force the appearance of the units on the
label using the {\tt units} qualifier.   The label text may be edited using Toolkit editing commands
(\progref{SDDS editing}).

The {\tt offset} and {\tt scale} qualifiers allow changing the position and size of the label.  The {\tt
offset} is specified as a fraction of the vertical dimension of the plot region.  The {\tt scale} is
simply a multiplicative factor.

Note that if the value of the parameter {\em parameterName} changes from page to page in a file, and if separate pages are
plotted in different panels, then the label for each panel will be different.  If the pages are plotted together, the value
of the parameter from the first page will be used.

  \item {\tt -yLabel}---This switch has identical usage to {\tt -xLabel}.  {\tt -yLabel} controls the y dimension label.  The
default text contains the y data names of all the columns and parameters being displayed.  If the data all have the same
units, the units are displayed as well.  This information is taken from the appropriate entries in the SDDS header.
The {\tt offset} qualifier gives the label offset as a fraction of the horizontal dimension of the plot region.

  \item {\tt -verticalPrint=\{up | down\}}---Specifies the direction of print for the y dimension label.
The default is {\tt up}.

  \item {\tt -title}---This switch has identical usage to {\tt -xLabel}.  The default text is from the {\tt contents}
field of the {\tt description} command in the first file from which data is displayed.  
  \item {\tt -topTitle}---Normally, the title goes below the x dimension label.  This switch directs that it be placed
at the top of the plot, above the ``topline label''.
  \item {\tt -topline}---This switch has identical usage to {\tt -xLabel}.  It is blank by default.
  \item {\tt -filenamesOnTopline}---Directs that the topline text contain the names of the files from which data
is displayed.
  \item {\tt -labelSize={\em fraction}} --- {\bf Obsolete}: 
Specifies a common size for all labels, including numeric labels.  
In the original version of {\tt sddsplot}, the {\em fraction} was the horizontal size of the characters as a 
fraction of the horizontal size of the plot region.  This meaning is no longer precisely true because the
new version doesn't used fixed character sizes.  However, this option may still be used to scale character
sizes up and down.  The previous nominal value for {\em fraction} was 0.03, which is now used as the reference
point for scaling.  Hence, if you specify 0.06, the character sizes would be doubled.  
  \item {\tt -noLabels}---Specifies that no labels (i.e., x and y dimension labels, title, and topline label) will be
made.  

  \item {\tt -string=\{@{\em parameterName} | {\em string}\},\{x|p\}Coordinate={\em value}}
{\tt \{y|q\}Coordinate={\em value}[,scale={\em factor}][,angle={\em degrees}]}
{\tt [,justifyMode={\em mode}][,linetype={\em integer}][,edit={\em string}]} ---
Specifies display of string data on the plot.  The string may either
be extracted from a named string parameter or given explicitly.  If the value of the parameter {\em parameterName} changes
from page to page in a file, and if separate pages are plotted in different panels, then the label for each panel will be
different.  If the pages are plotted together, the value of the parameter from the first page will be used.

The coordinates of the string may be specified either in users coordinates (i.e., x and y), or unit coordinates (i.e., p and
q); the unit coordinates are (0,0) at the lower left of the plot region and (1,1) at the upper right.  {\tt scale} permits
changing the size of the letters by a specified factor. {\tt angle} permits changing the angle of the string; a value of 90
gives upward vertical print.  

Normally, text is ``left bottom'' justified, which means that the coordinates given are those of the left bottom corner of
the first letter of the string.  Justification may be changed with the {\tt justifyMode} qualifier, which accepts a mode string
of the form {\tt \{ l | r | c\}\{t | b | c\}}.  The letters stand for Left, Right, Center, Top, and Bottom, respectively.
The default justification would thus be specified as {\tt justify=lb}.

The text is normally creating using line type 0.  This may be changed with the {\tt linetype} option.  As with the other
labels, the text may be edited using Toolkit editing commands (\progref{SDDS editing}).

  \item {\tt -dateStamp}---Directs that a time and date stamp be placed on the plot.  It appears in the upper left corner
of the plot.

  \end{itemize}
\item Altering or rearranging data prior to plotting:
  \begin{itemize}

  \item {\tt -swap}---Specifies that the x data will be plotted as y and vice-versa.  

  \item {\tt -transpose}---Specifies that the data matrix be transposed prior to plotting.  This
means, for example, that if the plot request specified N columns of y data and if the table contained
M rows, one would get a plot of M quantities as a function of the index of the column.  The implicit
assumption is that the N columns contain comparable quantities.  This would allow one to display, for
example, how the quantities changed from row to row in the data.  Each row of data thus organized is
marked as a separate ``subpage'' (see the {\tt -groupBy} and {\tt -separate} switches), so that one
can for example split rows onto separate panels.

  \item {\tt -factor=[\{x|y\}Multiplier={\em value}]} --- Specifies that the x
and/or y data for the present request will be multiplied by the given values.  Note that it is the
users responsibility to ensure that the units that are displayed are corrected, if required.

  \item {\tt -offset=[\{x|y\}Change={\em value}][,\{x|y\}Parameter={\em
value}][,\{x|y\}Invert]} --- Specifies that the x and/or the y data be
offset by either specified values, qor by values in named numerical
parameters.  Normally, the offset is of the form $x \rightarrow
x+x_o$.  The {\tt invert} qualifiers cause the offset to be subtracted
rather than added.

If {\tt -factor} is given together with {\tt -offset}, then the offset is applied first.

  \item {\tt -mode=\{x | y\}=\{linear | logarithmic | autolog | normalize | offset | coffset | center | meanCenter | fractionalDeviation | specialScales\}[,...]} ---

Invokes one or more standard transformations of data, independently
for x and y values.  The {\tt linear} mode is the default.  {\tt
normalize} mode directs that data be displayed after independent
normalization to the interval [-1, 1]; to do this, the data is divided
by the maximum absolute value in the data.  The {\tt offset}, {\tt
coffset}, {\tt center}, and {\tt meanCenter} qualifiers result in 
shifting of the data: {\tt offset} directs that data be shifted so 
that the first value plotted is zero; {\tt coffset} directs the data 
to use a common offset from the first plot; {\tt center} directs that 
data be shifted the center of the range is zero; {\tt meanCenter} 
directs that the data be shifted so that the average plotted value 
is zero.

{\tt logarithmic} mode implies that the base-ten logarithmic of the
appropriate values is taken prior to plotting.  Normally, this does
not produce log-type scales; use of the {\tt specialScales} keyword
together with the {\tt logarithmic} keyword will obtain this. One can
also use the {\tt -tickSettings} option for this, which is the preferred
method.  {\tt autolog} mode results in choice of linear or log-scale plotting
based on the range of the data.  If the range of the data is more than a factor
of 15, then log mode is used (with log scales).  Otherwise, linear mode is used.

{\tt fractionalDeviation} plots the data after subtracting and then dividing by the mean value.

\item {\tt -stagger=[xIncrement={\em value}][,yIncrement={\em value}][,files][,datanames]} ---
Directs that data displayed on the same panel will be incrementally offset for display.  This is
useful in order to make mountain range plots, or to offset similar data for clarity.  {\tt
xIncrement} and {\tt yIncrement} are used to specify the increments for each dimension; zero is the
default.  Normally, only data from the same column or parameter is staggered, with the stagger amount
increasing with each page in the file.  The {\tt files} qualifier directs incrementing the offset
when plotting proceeds to a new file on the same panel.  The {\tt datanames} qualifier directs
incrementing the offset when plotting proceeds to a new dataname (i.e., column or parameter name)
within the same file on the same panel.

  \item {\tt -enumeratedScales=[interval={\em integer}][,limit={\em integer}][,scale={\em
factor}]} {\tt [,allTicks][,rotate][,editCommand={\em string}]} --- Allows control of the display of
enumerated value strings when the x data is of string type.  {\tt interval={\em N}} specifies
displaying and making a tick for every $N {th}$ enumerated value; the default is 1.  Also, {\tt
limit={\em M}} specifies displaying and making a tick for only $M$ enumerated values at equal
spacing; the default is unlimited.  If one of these options is employed but one desires to see all
the ticks (even those without labels), the {\tt allTicks} qualifier may be given.  {\tt scale}
specifies a factor by which to increase the size of the text.  {\tt rotate} specifies rotation of the
printed text from the normal orientation to the optional orientation; if enumerated data is displayed
along the x dimension, the normal (optional) orientation is vertical (horizontal) printing.  These
are reversed if the enumerated data is displayed along the y dimension.

  \end{itemize}
\item Creating legends and data labels:
  \begin{itemize}

  \item {\tt -legend[=\{ \{x|y\}Symbol | \{x|y\}Description | \{x|y\}Name | filename | 
 specified={\em string} | parameter={\em name}\}] [,editCommand={\em string}]
  [,firstFileOnly][,scale={\em factor}]} \rm
--- Specifies creation of a legend for the datanames in the current request.  By default, the
legend text is the symbol field for the y data; if the symbol is blank, the dataname is used.  {\tt
xSymbol} and {\tt ySymbol} specify use of the x or y data symbols, or the datanames if the requested
symbol is blank.  {\tt xDescription} and {\tt yDescription} specify use of the indicated description
fields.  {\tt xName} and {\tt yName} specify use of the indicated datanames.  {\tt filename}
specifies use of the name of the file from which the data comes.  {\tt specified={\em string}}
specifies use of the given string.  {\tt parameter={\em name}} specifies use of the contents of the
named string parameter.  Any legend text may be editing using SDDS editing commands\progref{SDDS
Editing} via the {\tt editCommand} qualifier.  If {\tt firstFileOnly} is given, only the first file
in the request will have legends generated.  If {\tt scale={\em factor}} is given, the legend text
size is scaled by the given factor.

  \item {\tt -lSpace={\em qmin},{\em qmax},{\em pmin},{\em pmax}}---Specifies the region in which
legends will be placed.  The coordinates are pspace coordinates.  Since the legends are typically
outside the pspace, the coordinates may be greater than unity.  For example, the default values are
[1.02, 1.18]x[0.0, 1.0].  This option is usually used to place the legend inside the pspace, or to
extend the size of the lspace to accomodate long legend text.

  \item {\tt -pointLabel={\em name}[,edit={\em editCommand}][,scale={\em number}]
        [,justifyMode=\{rcl\}\{bct\}]} --- Specifies labeling of individual data points using
        data from column or parameter {\em name}.  The labels may be edited by specifying an
        {\em editCommand} with the {\tt edit} qualifier.  The {\tt scale} qualifier may be
        used to scale the label size.  The {\tt justifyMode} qualifer is used to change the
        location of the label relative to the point; the first letter gives the horizontal
        justification (right, center, or left) and the second gives the vertical
        justification (bottom, center, or top). The justification mode may also be embedded  in
        the string; if the string ends with \$j{\em XY}, then {\em X} and {\em Y} are the
        horizontal and vertical justification, respectively, for that string.
        
  \end{itemize}

\item Creating overlays:

The overlay feature allows displaying data that has different scales
on the same plot.  In most cases, it is superseded by the {\tt
-yScalesGroup} and {\tt -xScalesGroup} options.  The only exception is
when one wants to overlay data without having scales shown for the
data.  (An example is plotting magnet layouts for Twiss parameter
plots using the {\tt magnets} output from {\tt elegant}.)

{\tt -overlay=[\{x|y\}mode={\em mode}][,\{x|y\}factor={\em value}]} {\tt [,\{x|y\}offset={\em
value}]} {\tt [,\{x|y\}center]}---Normally, {\tt sddsplot} displays all data on a single panel on the same
scale.  In some cases, one wants to overlay data that is on a different scale from other data on the
panel.  One way to do this is with the {\tt -overlay} switch, which gives convenient control of how
overlayed data is displayed.  Any data in a plot request for which this switch is given will be
overlayed as specified.

The {\tt xmode} and {\tt ymode} options allow two types of scaling for x and y independently.  A mode
of {\tt normal} means that the indicated data is treated normally.  The default mode is {\tt unit},
which means that the data is scaled so that its full range is equal to the full coordinate range of
the plot in the appropriate (x or y) dimension.

The data is further adjusted according to any additional qualifiers given.  The {\tt center}
qualifiers offset the data so that the data is centered in the plot space; normally, zero in the data
is mapped to zero in the user's coordinates.  The {\tt factor} qualifiers scale the data by the given
factor about the center value.  The {\tt offset} qualifiers offset the data by specified amounts; if
{\tt mode=normal}, the offset is in user's coordinates, otherwise it is in pspace coordinates.

Users needing only the {\tt factor} facility should consider the {\tt -factor} switch, since it is
easier to use.

\item Controlling plot panels:
  \begin{itemize}
  \item {\tt -newPanel}---Specifies that the current plot request will start a new plot panel.
  \item {\tt -endPanel}---Specifies that the current plot request will end the current plot panel.

  \item {\tt -layout={\em hNumber},{\em vNumber}[,limitPerPage={\em integer}]}---Specifies the layout
of panels on each plot page.  The maximum number of panels on any page is the product of {\em
hNumber} and {\em vNumber}, which are the number of panels horizontally and vertically, respectively.
The default is {\em hNumber}=1 and {\em vNumber=1}.  If {\tt limitPerPage} is given, then only the
specified number of panels will appear on any page; for example, {\tt -layout=2,2,limit=3} would
imply three panel spaces per page, with one left blank.

  \end{itemize}

\item Grouping, sorting, and separating data:

  \begin{itemize} \item {\tt -sever[=xGap={\em value}][,yGap={\em value}]}---For line plotting, {\tt
sddsplot} will normally connect points sequentially without regard for gaps in the data.  The {\tt
-sever} switch specifies various means of locating gaps in data and directs lifting the ``pen''
whenever a gap occurs.  If {\tt -sever} is given without qualifiers, the pen is lifted whenever the x
value decreases; this is useful for plotting data where the x value is expected to increase
monotonically for each group of points.

The {\tt xgap} and {\tt ygap} qualifiers invoke a more sophisticated and more generally applicable form of
severing.  For each dimension for which severing is requested, the pen is lifted whenever the absolute
difference of two successive values exceeds a defined limit.  This limit is specified either in absolute
or fractional terms using the {\em value} entry.  If {\em value} is positive, the gap threshold is equal
to {\em value}.  If {\em value} is negative, the gap threshold is {\em -value} times the mean spacing
between successive points; a value of -1.5 has been found to work well for data that is roughly equispaced
with occasional missing points.

  \item {\tt -tagRequest=\{{\em number} | @{\em parameterName}\}}---Specifies that data from the
current requested will be tagged with either the given (generally floating-point) {\em number}, or
with the values from the numeric parameter {\em parameterName}.  Using the {\tt -groupBy} and {\tt
-separate} options permits grouping and sorting of data by tag values.  If a data set has multiple
pages in the file, and if pages are split (see {\tt -split} below), then parameter-tagged data will
have the parameter value from the first page in each group of pages.

  \item {\tt
 -groupBy[=request][,tag][,fileIndex][,nameIndex][,page][,subpage]} {\tt [,fileString][,nameString][,iNameString]} ---
 Specifies how internal data sets will be ordered.  {\tt -sortBy} might have been a more appropriate
 name for this switch.  The qualifiers that appear in the list are shown in the order that
 corresponds to the default sorting.  The file index is the sequential number within the request of
 the file from which the internal data set is taken; the file string is the name of the file.  The
 name index is the sequential index within the request of the dataname group for the internal data
 set, while the name string is the name of the y data.  The page is the sequential number in the file
 of the first SDDS data page from which data appears in the internal data set.  The subpage is a
 sequential number within each internal data set, which allows subdivision of the internal data set.
 The request is the sequential number of the plot request that resulted in generation of the internal
 data set.  The tag is a single user-supplied value or a value read from a parameter that is
 associated with each internal data set; by default, all data sets are tagged with the value 0.  If a
 file is split into several internal data sets, each may have a different tag value if the tag is
 read from a parameter; in this case, the data sets are eached tagged with the value for the first
 included data page.

The order in which the qualifiers to {\tt -groupBy} are given determines the priority of sorting by
the various criteria.  In the default ordering, data sets are sorted by request number, subsorted by
tag (usually a null operation unless data is tagged by the user), subsubsorted by file index,
subsubsubsorted by dataname index, etc.  Each successive qualifier results in moving the indicated
sort criterion to the next highest priority.  Any qualifiers not given are retained in the default
order.

If one wanted to bring together, for example, internal data sets with the same data name, one would
give {\tt -groupby=nameString}.  In this case, the new sorting priority would be {\tt nameString},
{\tt request}, {\tt tag}, etc.

  \item {\tt -separate[=\{{\em numberToGroup} | groupsOf={\em number} | fileIndex | fileString |
nameIndex | nameString | page | subpage | request | tag | iNameString\}]} --- Specifies how to separate internal data
sets onto panels.  If given with no qualifiers, each internal data set is placed on a separate panel.
If given with a single integer argument, or with the {\tt groupsOf} qualifier, then the specified
number of data sets appear on each panel; the data sets are assign to panels in the order determined
by {\tt -groupBy} or the default thereof.

If one of the other qualifiers is given, then panel separation occurs when the indicated criterion
changes as the data sets are accessed in sorted order.  Most commonly, one uses {\tt -groupby={\em
criterion}} {\tt -separate={\em criterion}}.  For example, one might want to group by filename and
separate by filename.
 
  \item {\tt -split=\{pages[,interval={\em integer}] | parameterChange={\em name}[,width={\em
value}][,offset={\em value}] | columnBin={\em name},width={\em name}[,start={\em
value}][,completely]\}}---As discussed in the introductory sections, when {\tt sddsplot} reads data
for one dataname group from a file, it normally concatenates data from successive pages to form a
single internal data set.  This would mean, for example, that all of the data from the file would be
displayed with the same linetype or symbol.  The {\tt -split} switch overrides this behavior,
splitting the data into multiple internal data sets.

The simplest and most commonly-used way of doing this is to split the data page boundaries; this is
done using the {\tt -split=pages} mode.  The optional {\tt interval} specifies spliting after a
specified number of page boundaries.  Splitting data does not imply that the data will appear on
separate plot panels, but allows this and other possibilities.  (To separate page-split data onto
panels, one uses {\tt -separate=pages}, as discussed above.)

One can also page-split based on the value of a parameter, using {\tt -split=parameterChange}.  This
directs that a new internal data set will be started wheneven the named parameter changes.  For
numeric parameters, the {\tt width} and {\tt start} qualifiers may be used.  If {\tt width} is
specified, the change must exceed the given value before a split occurs.  If {\tt start} is
specified, the reference value for changes is set to the given value; otherwise, the first parameter
value is used.  (For example, one might wish to split when a parameter changed by 5 units referenced
from 2.5 units, giving boundaries of 7.5, 12.5, etc.; this would be obtained with {\tt
width=5,start=2.5}.)

The {\tt columnBin} mode is different from the other two modes.  Rather than splitting data into internal
data sets at page boundaries, it groups or bins data into subpages according to the value in a specified
numeric column.  (It is appropriate only for plotting column data.)  {\tt columnBin} mode may be used with
{\tt pages} mode to split and subsplit data into pages and subpages.  For example, one might have a data
file with many pages of time-series data.  One might want to plot each page separately, but within
each page one might want to color-code the points according to some value in the table (e.g., a valid-data
indicator). This would be accomplished using {\tt -split=pages,columnBin={\em name},width={\em value}} {\tt
-separate=pages} {\tt -graph=dot,vary,eachPage}.

  \item {\tt -omniPresent}---Specifies that the data sets from the current request will appear on all
plot panels.

  \item {\tt -replicate=\{number={\em integer} | match=\{names | pages | requests| files\}\}\{,scroll\}} --- Specifies replication of a dataset so that it can be plotted several times. This is similar to -{\tt omniPresent}, but more flexible.  When a dataset is replicated,
 each replicant appears to have come from a different page of the original file.  The number of replications is controlled by
 the first option: a specific number of replications can be requested, or it can be asked to replicate a number of times 
 equal to the maximum number of pages in any file, data names in any request, plot requests, or files in any request.  
 If the {\tt scroll} qualifier is given, then the replicants do not have the same number of data points.  Instead, successive
 copies are more and more complete until the final replicant has the full dataset.

  \end{itemize}
\item Winnowing data:
  \begin{itemize}
  \item {\tt -limit=[\{x|y\}Minimum={\em value}][,\{x|y\}Maximum={\em value}][,autoscaling]}---
Specifies limits to be placed on x and y values prior to plotting.  Points beyond the indicated limits are 
eliminated from the data prior to plotting.  This complements the facility available from {\tt -filter} and
{\tt -match} in that one need not specify the name of the data one is winnowing with.  This permits easier
filtering of data from many columns or parameters.

The {\tt autoscaling} qualifier specifies that {\tt sddsplot} will not remove data outside the
defined limits, but rather that it will ignore it for purposes of autoscaling.  If lines are used to
connect data points, this could result in lines being drawn to the boundary of the plot region, thus
showing the presence of extreme points.

  \item {\tt -sparse={\em interval}[,{\em offset}]}---Specifies that only every {\em interval}${
{th}}$ point will be used.  If {\em offset} is not given, the first point in the internal data set is
the first taken; otherwise, the {\em offset}${ {th}}$ point is the first taken.

  \item {\tt -sample={\em fraction}}---Specifies random sampling of data to retain only the indicated
fraction of the points.  {\em fraction} gives the probability that any point will be used.  Hence,
the data actually used may vary from run to run since the random number generator is seeded with the
system clock.

  \item {\tt -clip={\em head},{\em tail}[,invert]}---Specifies removal of {\em head} points from the
beginning and {\em tail} points from the end of each internal data set.  If {\tt invert} is given,
the points that would have been removed are instead the only ones used.

  \item {\tt -presparse={\em interval}[,{\em offset}]}---Similar to {\tt -sparse}, except that
sparsing is done at the time the data page is read and only once for all requests and
datanames that draw data from the data page.  This is faster, and is usually what is desired.
However, if one wants to plot sparsed and unsparsed data from the same file at the same time,
{\tt -presparse} cannot be used.  If both {\tt presparse} and {\tt sparse} are given, both
are applied.

   \item {\tt -filter=\{column | parameter\},{\em rangeSpec}[,{\em rangeSpec},{\em logicOp}...] }
--- Specifies winnowing each internal data set based on numerical data in parameters or columns.  A
{\em range-spec} is of the form {\tt {\em name},{\em lower-value},{\em upper-value}[,!] }, where
\verb|!| signifies logical negation.  A point passes a {\tt column}-based filter if the value in the
named column is inside (or outside, if negation is given) the specified range, where the endpoints
are considered inside.  {\tt parameter}-based filters are similar, except that the point passes only
if the value of the named parameter for the page from which it comes is acceptable.  One or more
range specifications may be combined to give a accept/reject status by employing the {\em
logic-operations}, \verb|&| (logical and) and \verb&|& (logical or).
     \item {\tt -timeFilter=\{column | parameter\},[before=YYYY/MM/DD@HH:MM:SS] [,after=YYYY/MM/DD@HH:MM:SS][,invert]} 
--- Specifies date range in YYYY/MM/DD@HH:MM:SS format in time parameters or columns. The invert option cause the
filter to be inverted, so that the data that would otherwise be kept is removed and vice-versa. For example,
if one want to keep data between 8:30AM on Januaray 2, 2003 and 9:20PM on February 6,2003, the option woould be
     -timeFilter=column,Time,before=2003/2/6@21:20,after=2003/1/2@8:30
assume that the time data is in the column Time.
   \item {\tt -match=\{column | parameter\},{\em matchTest}[,{\em matchTest},{\em logicOp}]} ---
Specifies winnowing based on data in string parameters or columns.  A {\em matchTest} is of the form
{\tt {\em name}={\em matchingString}[,!]}, where the matching string may include wildcards.
If the first character of {\em matchingString} is '@', then the remainder of the string is taken to
be the name of a parameter or column.  In this case, the match is performed to the data in the named
entity.

The use of several match tests and logic is done just as for {\tt -filter}.  For example, to match
all the rows for which the column {\tt Name} starts with 'A' or 'B', one could use
{\tt -match=column,Name=A*,Name=B*,|}.  (This could also be done with {\tt -match=column,Name=[AB]*}.)
  \end{itemize}
\item {Miscellaneous:}
\begin{itemize}
        \item {\tt -repeat[=checkSeconds={\em number}][,timeOut={\em seconds}]} ---
Specifies repeated plotting of data from the files, with replotting occuring when any
file is modified.  By default, {\tt sddsplot} checks the files every second and times out
after 900s of no change.  This is available on UNIX systems only.  It is best used with
the \verb|motif| device type and the following device argument:
\verb|-device=motif,''-movie true -keep 1''|.
        \item {\tt -drawLine=}
        {\tt \{x0Value={\em value}|p0Value={\em value}|x0parameter={\em name}|p0parameter={\em name}\}}
        {\tt \{x1Value={\em value}|p1Value={\em value}|x1parameter={\em name}|p1parameter={\em name}\}}
        {\tt \{y0Value={\em value}|q0Value={\em value}|y0parameter={\em name}|q0parameter={\em name}\}}
        {\tt \{y1Value={\em value}|q1Value={\em value}|y1parameter={\em name}|q1parameter={\em name}\}} ---
        Specifies drawing of lines on the plot by giving the two endpoints of the line.
        For each endpoint (labeled '0' and '1'), one must specify the x or p coordinate (for horizontal) and
        the y or q coordinate (for vertical).  Each coordinate name be specified explicitly (e.g.,
        \verb|x0Value=1.7|) or via a parameter (e.g., \verb|x0parameter=alpha|).  If a parameter is
        given, the coordinate can change as the parameter value changes in the file.
        
\end{itemize}

\end{itemize}

\item {\bf special characters:}
{\tt sddsplot} supports Greek and mathematical characters in labels and strings through special sequences embedded in
text strings.  A similar mechanism is used to allow character-by-character control over size and positioning.
The special sequences are of the form \verb|$|{\em character}, where {\em character} may be one of the following:
\begin{itemize}

\item {\tt a}, {\tt b}, {\tt n}: provide subscript and superscript control.  {\tt a} puts the character Above the
normal position (superscript), {\tt b} puts the character Below the normal position (subscript), and {\tt n}
returns to Normal.

\item {\tt g}, {\tt r}: provide for switching between Greek and Roman
character sets. \verb|$g| switches into Greek mode, while \verb|$r|
switches back to Roman mode.  The correspondance between Greek
characters and the alphabet is shown in Figure \ref{CharSet}.  For
example, to make a lower-case alpha, one would use \verb|$ga$r|.

\item {\tt s}, {\tt e}: provide for switching between Special and
normal characters.  \verb|$s| switches to special character mode,
which provides mathematical and other symbols.  Figure \ref{CharSet}
shows the correspondance between special characters and keyboard
characters.  For example, to make a ${\rm \pm}$ symbol, one would
employ \verb|$sa$e|, while a right-pointing arrow would be obtained
with \verb|$s5$e|.

\newcommand{\PSFigure}[4]{\begin{figure}[htb]
  \vspace{-0.38in}
  \includegraphics[width=\columnwidth,height=#1]{#4}
  \vspace{-0.57in}
  \caption[#2]{#2}\label{#3}
  \end{figure}}

\PSFigure{15cm}{Special character set}{CharSet}{charSet.eps}

\item {\tt i}, {\tt d}: provide for Increasing and Decreasing the
character size.  The two sequences \verb|$i| and \verb|$d| are
inverses of each other.  \verb|$i| increases the size of subsequent
characters by 50\%, while \verb|$d| decreases the size of subsequent
characters by ${\rm 33 \frac{1}{3}}$\%.  These are seldom used, since
{\tt sddsplot} provides other means of controlling the size of
characters in labels and strings.

\item {\tt u}, {\tt v}: provide for motion of the baseline Up and down by one half character height.  

\item {\tt t}, {\tt f}: provide for making Taller and Fatter characters.  \verb|$t| makes characters twice as
tall while maintaining width, while \verb|$f| makes characters half as tall while maintaining width.  

\item {\tt h}: specifies moving back one half space.
\end{itemize}


\item {\bf environment variables:}
        \begin{itemize}
        \item {\tt SDDS\_DEVICE} --- Gives the name of the device type to use as the default.
        \end{itemize}
\item {\bf see also:}
    \begin{itemize}
    \item \hyperref[exampleData]{Data for Examples}
    \item \progref{SDDS editing}
    \item \progref{SDDS Wildcard Conventions}
    \end{itemize}
\item {\bf author:} M. Borland, H. Shang and R. Soliday ANL/APS.

\item {\bf acknowledgements}: {\tt sddsplot} uses device driver code from the program {\tt GNUPLOT}, 
with modifications and enhancements made at Argonne.  The GNUPLOT code is covered by a separate
copyright, and is used by permission of the authors.  See the {\tt GNUPLOT\_README} file included
with the distribution for restrictions associated with this code.

The GUI X-windows program ({\tt mpl\_motif}) was written by K. Evans of ANL/APS.

The GIF drivers use the {\tt gd 1.2} library by Thomas Boutell.  The latter is copyrighted by the
Quest Protein Database Center, Cold Spring Harbor Labs.

\end{itemize}



%\begin{latexonly} 
\newpage 
%\end{latexonly} 
\subsection{sddspoly} 
\label{sddspoly} 
 
\begin{itemize} 
\item {\bf description:} \hspace*{1mm}\\ 
{\tt sddspoly} evaluates polynomials for N-dimensional input.
\item {\bf examples:} 
\begin{flushleft}
{\tt sddspoly <inputfile> <outputfile> -evaluate=filename=<polyFilename>,output=y,coefficients=coef,input0=x,power0=power }
\end{flushleft} 
\item {\bf synopsis:}  
\begin{flushleft}
{\tt 
sddspoly [{\em inputfile}] [{\em outputfile}] [-pipe=[input][,output]] \\ \
-evaluate=filename={\em polyFilename},output={\em column},coefficients={\em column}, \\ \
input0={\em inputColumn},power0={\em powerColumn} \\ \
{}[,input1={\em inputColumn},power1={\em polyColumn}][,...] \\ \
{}[-evaluate=...]}
\end{flushleft} 
\item {\bf files:} 
The input file contains the {\em inputColumns}. The polynomial file contains the coefficient column as well asl the {\em powerColumns}. The output file contains the {\em inputColumns} and evaluated output columns.
\item {\bf switches:} 
    \begin{itemize} 
    \item {\tt -pipe=[input][,output]} --- Standard SDDS pipe options for reading/writing files from stdin/stdout.
    \item {\tt -evaluate=...} --- Specifies evaluation of polynomial, specified in file {\em polyFilename}.  The results of the evaluation are in the {\em outputfile} under the name given with output={\em column}. The polynomial coefficients are in the column named with coefficients={\em column}.  The input$<$n$>$ qualifiers give the names of columns in {\em inputfile} that are inputs to the polynomial.  The power$<$n$>$ qualifiers give the names of columns in the {\em polyFilename} file that give the powers of the input$<$n$>$ columns to use for each coefficient.
\end{itemize} 

\item {\bf author:} M. Borland, H. Shang, R. Soliday, ANL/APS. 
\end{itemize} 

\begin{sddsprog}{sddsprintout}
  \item \textbf{description:} \verb|sddsprintout| provides formatted text output of data from columns and parameters. It is similar to \progref{sdds2stream}, but provides better control of the appearance of the text. Note that using \verb|sddsprintout| to create tables of ASCII data for other programs is not recommended. Better alternatives are \progref{sdds2stream}, \progref{sdds2spreadsheet}, and \progref{sdds2plaindata}.
  \item \textbf{examples:}
    \begin{verbatim}
sddsprintout APS0.twi -column=ElementName -column='beta?' -parameters='nu?'
    \end{verbatim}
  \item \textbf{synopsis:}
    \begin{verbatim}
sddsprintout [-pipe[=input]] [SDDSinput] [outputFile]
  -columns=nameList[,format=string][,label=string][,editLabel=command][,useDefaultFormat][,endsLine][,blankLines=number]
  -parameters=nameList[,format=string][,label=string][,editLabel=command][,useDefaultFormat][,endsLine][,blankLines=number]
  -spreadsheet[=delimiter=string][,quoteMark=string][,noLabels][,schFile=filename]
  -fromPage=number -toPage=number
  -formatDefaults=SDDStype=formatString[,SDDStype=formatString...]
  -width=integer -pageAdvance -paginate[=lines=number][,noTitle][,noLabels]
  -postPageLines=number -title=string -noTitle -noWarnings
    \end{verbatim}
  \item \textbf{files:} \emph{SDDSinput} is the SDDS file from which data is printed. \emph{outputFile} is a file to which the printout will go; by default, the printout goes to the standard output.
  \item \textbf{switches:}
    \begin{itemize}
      \item \verb|-pipe[=input]| --- The standard SDDS Toolkit pipe option.
      \item \verb|-columns=nameList[,format=string][,label=string][,editLabel=command][,useDefaultFormat][,endsLine][,blankLines=number]| --- Specifies the names of columns to appear in the printout. \emph{nameList} may contain one or more comma-separated strings, each of which may contain wildcards. If more than one string is given, the list must be enclosed in parentheses, e.g., \verb|-columns='(betax,betay)'|.
        The \verb|format| qualifier may be used to specify a \verb|printf|-style format string for the named columns; in this case, all of the columns must have the same data type. The format string should contain a width field, to ensure proper alignment of text, e.g., \verb|%30s| rather than \verb|%s|. The \verb|useDefaultFormat| qualifier directs that \verb|sddsprintout| use its own default format for the data type in question, as opposed to any format that might be specified in the SDDS header.
        The \verb|label| qualifier can be used to specify the column label in the printout (by default, the column name is used); the label may be edited using the \verb|editLabel| qualifier and a standard editing sequence.
        If the \verb|endsLine| qualifier is given, a line break is issued after the last column of the list is printed. The \verb|blankLines| qualifier may be used to specify that one or more blank lines be emitted following such a line break.
        Any number of \verb|-columns| options may be given.
      \item \verb|-parameters...| --- Identical to \verb|-columns|, except that printout of parameters is specified.
      \item \verb|-spreadsheet[=delimiter=string][,quoteMark=string][,noLabels][,schFile=filename]| --- Specifies spreadsheet compatible output, using the given delimiter between columns. In this mode, simplified header is printed and no line width limits are imposed. The default delimiter is a tab. The default quotation mark is \verb|"|. If the \verb|schFile| qualifier is given, a header file for comma-separated-values data is generated. In this case, the delimiter should be a comma.
      \item \verb|-fromPage=number| --- Specifies the first data page of the file that will appear in the printout. By default, the printout starts with data page 1.
      \item \verb|-toPage=number| --- Specifies the last page of the file that will appear in the printout. By default, the printout ends with the last data page in the file.
      \item \verb|-formatDefaults=SDDStype=formatString[,SDDStype=formatString...]| --- Specifies default \verb|printf| format strings for named SDDS data types. The \emph{SDDStype} qualifier may be one of \verb|float|, \verb|double|, \verb|long|, \verb|short|, \verb|string|, or \emph{character}.
      \item \verb|-width=integer| --- Specifies the width of the output line in number of characters. The default is 130.
      \item \verb|-pageAdvance| --- Specifies that the page be advanced at the end of every data page of the SDDS file. This is done by emitting an ASCII page advance character, which will probably work only if the output is sent to a printer.
      \item \verb|-paginate| --- Specifies pagination of the output, using a default 66 line page. The \verb|lines| qualifier may be used to change the page length. By default, the title and column labels are printed for each page. These may be disabled using the \verb|noTitle| and \verb|noLabels| qualifiers.
      \item \verb|-postPageLines| --- Specifies that a number of blank lines shall be emitted at the end of the printout for each page. By default, there are no blank lines between pages.
      \item \verb|-title=string| --- Specifies the title for the printout.
      \item \verb|-noTitle| --- Specifies that no title be printed.
      \item \verb|-noWarnings| --- Suppresses warning messages, such as those concerning data elements requested in the printout that are not in the input file.
    \end{itemize}
  \item \textbf{see also:}
    \begin{itemize}
      \item \hyperref[exampleData]{Data for Examples}
      \item \progref{sdds2plaindata}
      \item \progref{sdds2spreadsheet}
      \item \progref{sdds2stream}
    \end{itemize}
  \item \textbf{author:} M. Borland, ANL/APS.
\end{sddsprog}

%\begin{latexonly}
\newpage
%\end{latexonly}
\subsection{sddsprocess}
\label{sddsprocess}

\begin{itemize}
\item {\bf description:} \hspace*{1mm}\\
\verb|sddsprocess| operates on the data columns and parameters of an existing SDDS data set and creates a new
data set.  The program supports filtering and matching operations on both tabular data and parameter
data, definition of new parameters and columns in terms of existing ones, units conversions, scanning of 
string data to produce numeric data, composition of string data from other data types, 
statistical and waveform analyses, and other operations.
\item {\bf examples:}
Compute the square-roots of the beta-functions, which are the beam-size envelopes:
\begin{flushleft}{\tt
sddsprocess APS.twi -define=column,sqrtBetax,"betax sqrt"  -define=column,sqrtBetay,"betay sqrt" 
}\end{flushleft}
Compute the horizontal beam-size, given by the equation
\[ \sigma_x = \sqrt{ \epsilon_x \beta_x + (\eta_x \sigma_\delta)^2} \]
\begin{flushleft}{\tt
sddsprocess APS.twi -define=parameter,epsx,8.2e-9,units=nm -define=parameter,sigmaDelta,1e-3 
  -define=column,sigmax,"epsx betax *  sigmaDelta etax * sqr + sqrt",units=m
}\end{flushleft}
\item {\bf synopsis:} 
\begin{flushleft}{\tt
sddsprocess [-pipe[=input][,output]] [{\em inputFile}] [{\em outputFile}] {\em options}
}\end{flushleft}
\item {\bf files:}
{\em inputFile} is an SDDS file containing data to be processed.  If no options are given, it is copied to {\em
outputFile} without change.  {\em Warning:} if no output filename is given, and if an output pipe is not selected,
then the input file will be replaced.

\item {\bf switches:}
    \begin{itemize}
    \item Data winnowing: Any number of the following may be used.  They are applied in the order
given.  Note that {\tt -match} and {\tt -test} are the most time intensive; thus, if several types
of winnowing are to be applied, these should be used last if possible.

   \begin{itemize}

   \item {\tt -filter=\{column | parameter\},{\em rangeSpec}[,{\em rangeSpec}[,{\em logicOp}...]] } ---
Specifies winnowing {\em inputFile} based on numerical data in parameters or columns.  A {\em range-spec} is of the
form {\tt {\em name},{\em lower-value},{\em upper-value}[,!] }, where \verb|!| signifies logical negation. A page
passes a given filter by having the named parameter inside (or outside, if negation is given) the specified range,
where the endpoints are considered inside.  A tabular data row passes a given filter in the analogous fashion, except
that the value from the named column is used.  

One or more range specifications may be combined to give a accept/reject status by employing the {\em
logic-operations}, \verb|&| (logical and) and \verb&|& (logical or).  For example, to select rows for
which A is on [0, 1] and B is on [10, 20], one would use
{\tt -filter=column,A,0,1,B,10,20,\&}.
        \item {\tt -timeFilter=\{column | parameter\},[before=YYYY/MM/DD@HH:MM:SS] [,after=YYYY/MM/DD@HH:MM:SS][,invert]} 
--- Specifies date range in YYY/MM/DD@HH:MM:SS format in time parameters or columns. The invert option cause the
filter to be inverted, so that the data that would otherwise be kept is removed and vice-versa. For example,
if one want to keep data between 8:30AM on Januaray 2, 2003 and 9:20PM on February 6,2003, the option woould be
     -timeFilter=column,Time,before=2003/2/6@21:20,after=2003/1/2@8:30
assume that the time data is in the column Time.

        \item {\tt -match=\{column | parameter\},{\em matchTest}[,{\em matchTest},{\em logicOp}]} --- Specifies
winnowing {\em inputFile} based on data in string parameters or columns.  A {\em match-test} is of the form {\tt {\em
name}={\em matchingString}[,!]}, where the matching string may include the wildcards \verb|*| (matches zero of more
characters) and \verb|?| (matches any one character). 

If the first character of {\em matchingString} is '@', then the remainder of the string is taken to be the name of a
parameter or column.  In this case, the match is performed to the data in the named entity.  For column-based matching,
this is done row-by-row.  For parameter-based matching, it is done page-by-page.

In addition, if instead of \verb|=| one uses \verb|=+|, then matching is case-insensitive.  The plus sign is intended
to be mnemonic, as the case-insensitive matching results in additional matches.

The use of several match tests and logic is done just as for {\tt -filter}.  For example, to match
all the rows for which the column {\tt Name} starts with 'A' or 'B', one could use
{\tt -match=column,Name=A*,Name=B*,|}.  (This could also be done with {\tt -match=column,Name=[AB]*}.)

\item {\tt -numberTest=\{column | parameter\},{\em name}[,invert]} --- 
Specifies testing the values of in a
string column (parameter) to see if they can be (or cannot be, if {\tt
invert} is given) converted to numbers.  If not, the corresponding row
(page) is deleted.  

\item {\tt -test=\{column | parameter\},{\em test}[,autostop][,algebraic]} --- Specifies
winnowing of {\em inputFile} based on a test embodied in an {\tt rpn} expression.  The expression, {\em test}, may use
the names of any parameters or columns.  If {\tt autostop} is specified, the processing of the data set (or data page)
terminates when the parameter-based (or column-based) expression is false.

        \item {\tt -clip={\em head},{\em tail}[,invert]} --- Specifies the number of data points to clip from the head
and tail of each page.  If {\tt invert} is given, the clipping retains rather than deletes the indicated points.
         \item {\tt -fclip={\em head},{\em tail}[,invert]} --- Specifies the fraction of data points to clip from the head
and tail of each page.  If {\tt invert} is given, the clipping retains rather than deletes the indicated points.
        \item {\tt -sparse={\em interval}[,{\em offset}]} --- Specifies sparsing of each page with the indicated
interval.  That is, only every {\em interval}$ {th}$ row starting with row {\em offset} is copied to the
output.  The default value of {\em offset} is 0.

        \item {\tt -sample={\em fraction}} --- Specifies random sampling of rows such that approximately the indicated
fraction is kept.  Since a random number generator is used that is seeded with the system clock, this will usually
never be the same twice.

        \end{itemize}
    \item \verb|rpn| calculator initialization:\\

        \begin{itemize} \item {\tt -rpnDefinitionsFiles={\em filename}...} --- Specifies a list of comma-separated
filenames to be read in as \verb|rpn| definitions files.  By default, the file named in the {\tt RPN\_DEFNS}
environment variable is read.

        \item {\tt -rpnExpression={\em expression}[,repeat][,algebraic]} --- Specifies an \verb|rpn| expression to be executed.  If
{\tt repeat} is not specified, then the expression is executed before processing begins.  If {\tt repeat} is specified,
the expression is executed just after each page is read; it may use values of any of the numerical parameters for that
page. This option may be given any number of times.

        \end{itemize}
  \item Scanning from, editing, printing to, and executing string columns and parameters: \\

         \begin{itemize}
 
         \item {\tt -scan=\{column | parameter\},{\em newName},{\em sourceName},{\em sscanfString}}\\ {\tt [,{\em
 definitionEntries}]} --- Specifies creation of a new numeric column (parameter) by scanning an existing string column
 (parameter) using a \verb|sscanf| format string.  The default type of the new data is double; this may be changed by
 including a {\em definitionEntry} of the form {\tt type=}{\em typeName}.  With the exception of the {\tt name} field,
 any valid namelist command field and value may be given as part of the {\em definitionEntries}.

        If {\em sourceName} contains wildcards, then {\em newName} must contain at least one occurrence of the string
``\%s''.  In this case, for each name that matches {\em sourceName}, an additional element is created, with a
name created by substituting the name for ``\%s'' in {\em newName}.
 
         \item {\tt -edit=\{column | parameter\},{\em newName},{\em sourceName},{\em edit-command}} --- Specifies
creation of a new string column (parameter) called {\em newName} by editing an existing string column (parameter) {\em
sourceName} using an emacs-like editing string.  For details on editing commands, see \hyperref[SDDSediting]{SDDS editing}.
         
        If {\em sourceName} contains wildcards, then {\em newName} must contain at least one occurrence of the string
``\%s''.  In this case, for each name that matches {\em sourceName}, an additional element is created, with a
name created by substituting the name for ``\%s'' in {\em newName}.

         \item {\tt -reedit=\{column | parameter\},{\em name},{\em edit-command}} --- Like {\tt -edit},
except that the element {\em name} must already exist.  Each value is replaced by the value obtained from
applying {\em edit-command}.
         
        \item {\tt -print=\{column | parameter\},{\em newName},{\em sprintfString},{\em sourceName}}\\ {\tt [,{\em
sourceName}...][,{\em definitionEntries}]} --- Specifies creation of a new string column (parameter) by formatted
printing of one or more elements from other columns (parameters).  The {\em sprintfString} is a C-style format string
such as might be given to the routine {\tt sprintf}.  With the exception of the {\tt name} field, any valid namelist
command field and value may be given as part of the {\em definitionEntries}.

        \item {\tt -reprint} --- Identical in syntax and function to {\tt -print}, except that if {\em newName} 
already exists, it is overwritten.  No error or warning is issued.

        \item {\tt -format=\{column | parameter\},{\em newName},{\em
sourceName}} \\ {\tt [,stringFormat={\em sprintfString}][,doubleFormat={\em
sprintfString}]} \\ {\tt [,longFormat={\em sprintfString}]} --- Reformats string
data in different ways depending on the type of data the string
contains.  Each string is separated into tokens at space boundaries.
Each token is separately formatted, either as a long integer, a
double-precision floating point number, or a string, depending on what
the token appears to be.  The formatting is done using the specified
format strings; the default format strings are \%ld for longs,
\%21.15e for doubles, and \%s for strings.

         \item {\tt -system=\{column | parameter\},{\em newName},{\em commandName},}\\ {\tt [{\em
definitionEntries}]} --- Specifies creation of a new string column (parameter) by executing an existing string
column (parameter) using a subprocess.  The first line of output from the subprocess is acquired and placed in the
new column (parameter).

        If {\em commandName} contains wildcards, then {\em newName} must contain at least one occurrence of
``\%s''.   In this case, for each name that matches {\em commandName}, an additional element is created, with a
name created by substituting the name for ``\%s'' in {\em newName}.

 
         \end{itemize}
 
    \item Creation and modification of numeric columns and parameters: \\

        \begin{itemize}

        \item {\tt -convertUnits=\{column | parameter\},{\em
name},{\em newUnits},}{\tt {\em oldUnits},{\em factor}}\\ ---
Specifies units conversion for the column or parameter {\em name}
(which may contain wildcards).  The {\em factor} entry the factor by
which the values must be multiplied to convert them to the desired
units.  It is an error if {\em oldUnits} does not match the original
units of the column or parameter.  Eventually, the {\em factor} entry
will be made optional by inclusion of conversion information in the
program.  This option may be given any number of times.

        \item {\tt -define=\{column | parameter\},{\em name},{\em
equation}[,select={\em matchString}][,exclude={\em matchString}]} 
{\tt [,editSelection={\em editCommand}][,{\em definitionEntries}][,algebraic] }\\ --- Specifies
creation of a new column or parameter using an {\tt rpn} expression to
obtain the values.  For parameters, any parameter value may be
obtained by giving the parameter name in the expression.  For columns,
one may additionally get the value of any column by giving its name in
the expression; the expression given for {\tt -define=column} is
essentially specifying a vector operation on columns with parameters
as scalars.  By default, the type of the new data is \verb|double|.
This and other properties of the new column or parameter may be
altered by giving {\em definitionEntries}, which have the form {\em
fieldName}={\em value}; {\em fieldName} is the name of any namelist
command field (except the name field) for a column or parameter, as
appropriate.  This option may be given any number of times.

        Using the {\tt select} qualifier, it is possible to use a
single {\tt -define} option to specify many instances of new column
definitions.  If {\tt select} is given, the input is searched for all
the column names matching {\em matchString}.  These are then
optionally editted using the {\em editCommand} specified with {\tt
editSelection}.  The resulting strings are then substituted one at a
time into {\em name} and {\em equation}, replacing all occurances of
``\%s''.  For example, suppose a file contained a number of
column-pairs of the form {\em Prefix}V1 and {\em Prefix}V2; to take
the difference of each pair, one could use\\ {\tt
-define=column,\%sDiff,\%sV1 \%sV2 -,select=*V1,edit=\%/V1//} \\[12pt]
{\tt sddsprocess} permits read access to individual elements of a
column of data using the {\tt rpn} array feature.  For each column, an
array of name \verb|&|{\em ColumnName} is created; the ampersand is to
remind the user that the variable \verb|&|{\em ColumnName} is the
address of the start of the array.  To get the first element of a
column named \verb|Data|, one would use {\tt 0 \&Data [}.  This will
function only within or following a {\tt -define=column} or {\tt
-redefine=column} operation.  It is an error to attempt to access data
beyond the bounds of an array.

   The number of columns, and the current page and row number are 
pre-loaded into the rpn calculator memory according to the following table.
\begin{center}
\begin{tabular}{|l|l|} \hline
Quantity    & rpn memory   \\ \hline
Page number & {\tt i\_page}       \\ \hline
Page number & {\tt table\_number} \\ \hline
Row number  & {\tt i\_row}        \\ \hline
Number of rows & {\tt n\_rows}    \\ \hline
\end{tabular}
\end{center}
For example, to generate a column of index number to a file, add the option {\tt
-define=col,Index,i\_row,type=long}.

        \item {\tt -redefine} --- This option is identical to {\tt
-define} except that the column or parameter already exists in the
input.  The equation may use the previous values of the entity being
redefined by including the column name in the expression.

        \item {\tt -evaluate=\{column | parameter\},{\em name},{\em
source}[,{\em definitionEntries}]}\\ --- Specifies creation of a new column or parameter {\em name}
containing values from evaluation of the equation stored in a string column or parameter {\em source}.
The source string is an rpn expression in terms of the other column and parameter values.

        \item {\tt -cast=\{column | parameter\},{\em newName},{\em oldName},{\em newType}} --- This option allows
casting data from one numerical data type to another.  It is much faster than trying to do the same operation using
{\tt -define}.  The string {\em newType} may be any of {\tt double}, {\tt float}, {\tt long}, {\tt short}, or
{\tt character}.

        \item {\tt -process={\em mainColumnName},{\em analysisName},{\em resultName}[,default={\em value}]}
\\{\tt [,description={\em string}][,symbol={\em string}][,weightBy={\em columnName}]}
\\{\tt [,functionOf={\em columnName}[,lowerLimit={\em value}][,upperLimit={\em value}]]}
\\{\tt [,head={\em number}][,tail={\em number}][,fhead={\em fraction}][,ftail={\em fraction}]}
\\{\tt [,topLimit={\em value}][,bottomLimit={\em value}]} 
\\{\tt [,position][,offset={\em value}][,factor={\em value}]}
\\{\tt [,match={\em columnName},value={\em match-value}]} --- This option may be given
any number of times.  It specifies creation of a new parameter {\em resultName} by processing
column {\em mainColumnName} using analysis mode {\em analysisName}.  The column must contain
numeric data, in general, except for a few analysis modes that take any type of data (see below). {\em
mainColumnName} may contain wildcards, in which case the processing is applied to all matching
columns containing numeric data.  {\em resultName} may have a single occurence of the string
``\%s'' embedded in it; if so, {\em mainColumnName} is substituted.  If wildcards are given in
{\em mainColumnName}, then ``\%s'' must appear in {\em resultName}; in this case, the name of
each selected column is substituted.  Similarly, if the {\tt description} field is supplied,
it may contain an embedded ``\%s'' for which the column name will be substituted.
If the processing fails for any reason, the value given by the {\tt default} parameter is subsituted;
if no value is specified, the value is equal to the maximum double-precision value on the system.

Recognized values for {\em analysisName} are:
\begin{itemize}
\item {\tt average}, {\tt rms}, {\tt sum}, {\tt standardDeviation}, {\tt mad} --- The arithmetic average, 
the rms average, the arithmetic sum, the standard deviation, and the mean absolute deviation.  All may be possibly
weighted.
\item {\tt median}, {\tt drange}, {\tt qrange} --- The median value, i.e., the value which is both above and below
50\% of the data points; the decile-range, which is the range excluding the smallest and largest 10\% of the values;
the quartile-range, which is the range excluding the smallest and largest 25\% of the values.
\item {\tt percentile}, {\tt prange} --- These compute percentiles and percentile ranges, as defined by the 
  {\tt percentlevel} qualifier.  For {\tt percentile}, the value returned is the value of the column corresponding
  to the given {\tt percentlevel}.  For {\tt prange}, the value return is the span of the values in the column
  encompassing the given central percentage of the data; for example {\tt percentile=50} would give the quartile range.

\item {\tt minimum}, {\tt maximum}, {\tt spread}, {\tt smallest}, {\tt largest} --- 
The minimum value, maximum value, spread in values, smallest value (minimum absolute value), and largest value
(maximum absolute value).  For all except {\tt spread}, the {\tt position} and {\tt functionOf} qualifiers may be
given to obtain the value in another column when {\em mainColumnName} has the extremal value; the {\tt functionOf} qualifer may name a string column.

\item {\tt first}, {\tt last} --- The values in the first and last rows of the page.  Will accept non-numeric
data.
\item {\tt pick} --- The first value within the filter.  Will accept non-numeric data.
\item {\tt count} ---  The number of values in the page.
\item {\tt baselevel}, {\tt toplevel}, {\tt amplitude} --- Waveform analysis parameters from histogramming
the signal amplitude.  {\tt baselevel} is the baseline, {\tt toplevel} is the height, and {\tt amplitude}
is height above baseline.

\item {\tt risetime}, {\tt falltime}, {\tt center} --- The rise and fall times from the 10\%-90\% and
90\%-10\% transitions.  {\tt center} is the midpoint between the first 50\% rising edge and the first following 50\%
falling edge after rising above 90\% amplitude.  Requires specifying a independent variable column with {\tt
functionOf}.

\item {\tt fwhm}, {\tt fwtm}, {\tt fwha}, {\tt fwta} --- Full-widths of the named column as a function
of the independent variable column specified with {\tt functionOf}.  The letters 'h' and 't' specify Half and Tenth
amplitude widths, while 'm' and 'a' specify Maximum value or Amplitude over baseline.

\item {\tt zerocrossing} --- Zero-crossing point of the column named with {\tt functionOf} of the
column {\em mainColumnName}.

\item {\tt sigma} --- The standard deviation over the square-root of the number of points.  This is an
estimate of the uncertainty in the mean value.

\item {\tt slope}, {\tt intercept}, {\tt lfsd} --- The slope and intercept of a linear fit.  The {\tt functionOf}
qualifier must be given to specify the quantity to fit against.  {\tt lfsd} is the Linear-Fit-Standard-Deviation,
which is the standard deviation of the fit residuals.

\item {\tt gmintegral} --- The integral of the quantity with respect to the quantity named with the {\tt functionOf} qualifier.
  The integral is performed using the Gill-Miller method, which works well for non-equispaced values of the independent
  variable.

\item {\tt correlation} --- The Pearson's correlation coefficient of the quantity and the column of which it is 
  (nominally) a function (as declared with the {\tt functionOf} qualifier).

\end{itemize}

Qualifiers for this switch are:
\begin{itemize}
\item {\tt description={\em string}}, {\tt symbol={\em string}} --- Specify the description and symbol
fields for the new column.
\item {\tt weightBy={\em columnName}} --- Specifies the name of a column to weight values from 
column {\em mainColumnName} by before computing statistics.
\item {\tt functionOf={\em columnName}} --- Specifies the name of a column that {\em mainColumnName}
is to be considered a function of for computing widths, zero-crossings, etc.
\item {\tt topLimit={\em value}}, {\tt bottomLimit={\em value}} --- Specifies winnowing of rows
so that only those with {\em mainColumn} values above the {\tt topLimit} or below the {\tt bottomLimit} are 
included in the computations.
\item {\tt lowerLimit={\em value}}, {\tt upperLimit={\em value}} --- If {\tt functionOf} is given,
specifies winnowing of rows so that only rows for which the independent column data is above
the {\tt lowerLimit} and/or below the {\tt upperLimit} are included in computations.
\item {\tt head={\em number}}, {\tt fhead={\em fraction}} --- Specifies taking the head 
of the data prior to processing.  {\tt head} gives the number of points keep, while {\tt fhead}
gives the fraction of the points to keep.  If {\em number} or {\em fraction} is less than 0, thenthe head points are deleted and the other points are kepts. If head and tail are both used, head  is
performed first.
gives the fraction of the points to clip.
\item {\tt tail={\em number}}, {\tt ftail={\em fraction}} --- Specifies taking the tail 
of the data prior to processing.  {\tt tail} gives the number of points keep, while {\tt ftail}
gives the fraction of the points to keep.  If {\em number} or {\em fraction} is less than 0, thenthe tail points are deleted and the other points are kepts. If head and tail are both used, head  is
performed first.
\item {\tt position} --- For {\tt minimum}, {\tt maximum}, {\tt smallest}, and {\tt largest} analysis modes,
specifies that the results should be the position at which the indicated value occurs.  This position is
the corresponding value of in column named with {\tt functionOf}.
\item {\tt offset={\em value}}, {\tt factor={\em value}} --- Specify an offset and factor for modifying
data prior to processing.  By default, the offset is zero and the factor is 1.  The equation is
$x \rightarrow f*(x+o) $.
\item {\tt match={\em controlName}}, {\tt value={\em match-value}} --- Specify the match column and the
match value (may contain wildcard).
\end{itemize}


\end{itemize}


    \item Miscellaneous:
        \begin{itemize}
        \item {\tt -ifis=\{column | parameter | array\},{\em name}[,{\em name}...]}\\ {\tt -ifnot=\{column | parameter |
array\},{\em name}[,{\em name}...]} \\ --- These options allow conditional execution.  If any column that is named under a
\verb|ifis| option is not present, execution aborts.  If any column that is named under a \verb|ifnot| option is
present, execution aborts.
        \item {\tt -description=[text={\em string}][,contents={\em string}]} --- Specifies the
        description fields for the SDDS dataset.  Use of this feature is disparaged as these fields
        are not manipulated by any tools.  Use of string parameters is suggested.
        \item {\tt -summarize} --- Specifies that a summary of the processing be printed to the screen.
        \item {\tt -verbose} --- Specifies that informational printouts be provided during processing.
        \item {\tt -noWarnings} --- Specifies suppression of warning messages.
    \item {\tt -delete=\{columns | parameters | arrays\},{\em matchingString}[,...]},\\
        {\tt -retain=\{columns | parameters | arrays\},{\em matchingString}[,...]}
         --- These options specify wildcard strings to be used to select entities
        (i.e., columns, parameters, or arrays) that will respectively be deleted or retained (i.e., that will not or
        will appear in the output).   
        The selection is performed by determining which input entities have names matching any of the strings.
        If \verb|retain| is given but \verb|delete| is not, only those entities matching one of the
        strings given with \verb|retain| are retained.  If both \verb|delete| and \verb|retain|
        are given, then all entities are retained except those that match a \verb|delete| string without
        matching any of the \verb|retain| strings.
        \end{itemize}
    \end{itemize}

\item {\bf author:} M. Borland, H. Shang, R. Soliday ANL/APS.
\end{itemize}


\begin{sddsprog}{sddspseudoinverse}
  \item {\bf description:} \verb|sddspseudoinverse| views the numerical tabular data of the input file as though it formed a matrix and produces an output file containing the pseudo-inverse of this matrix. At present the pseudo-inversion is done using a singular value decomposition. Other methods may be made available in the future.

  Command line options specify the number of singular values to be used in the inversion process.

  The column names for the output file are generated either from the data in a selected string column in the input file, from the value of the command line option \verb|-root|, or from an internal default.

  The column names of the input file are collected and made into a string column in the output file.

  \item {\bf examples:}
    \begin{verbatim}
    sddspseudoinverse LTP.R12 LTP.InvR12
    \end{verbatim}
  \item {\bf synopsis:}
    \begin{verbatim}
    sddspseudoinverse [<input>] [<output>] [-pipe[=input][,output]]
      [{-minimumSingularValueRatio=<value> | -largestSingularValues=<number>}]
      [-smallestSingularValues=<number>]
      [-deleteVectors=<list of vectors separated by comma>]
      [-economy] [-printPackage]
      [-oldColumnNames=<string>] [{-root=<string> [-digits=<integer>] |
      -newColumnNames=<column>}] [-sFile=<file>[,matrix]] [-uMatrix=<file>] [-vMatrix=<file>]
      [-weights=<file>,name=<columnname>,value=<columnname>]
      [-reconstruct=<file>] [-symbol=<string>] [-ascii] [-verbose] [-noWarnings]
      [-multiplyMatrix=<file>[,invert]]
    \end{verbatim}
  \item {\bf files:}
    The input file contains the data for the matrix to be inverted. The output file contains the data for the inverted matrix. If only one file is specified, then the input file is overwritten by the output.

    Multiple data pages of the input file will be processed and written to the output file if all the data pages of the input file have the same number of rows. The processing will stop at the first data page which does not have the same number of rows as that of the first page.

    If applicable, the string column selected to generate column names for the output file is assumed to be the same in all input data sets. The string columns of only the first data set are read.

  \item {\bf switches:}
    \begin{itemize}
      \item \verb|-pipe[=input][,output]| --- The standard SDDS Toolkit pipe option.
      \item \verb|-minimumSingularValueRatio=value| --- Used to remove small singular values from the calculation. The smallest singular value retained for the inverse calculation is determined by multiplying this ratio value with the largest singular value of the input matrix.
      \item \verb|-largestSingularValues=number| --- Used to remove small singular values from the calculation. The largest \verb|number| singular values are kept.
      \item \verb|-deleteVectors=n1,n2,n3,...| --- Sets the inverse singular values of modes n1, n2, n3, etc. to zero. The order in which the singular value removal options are processed is \verb|minimumSingularValueRatio|, \verb|largestSingularValues| and then \verb|deleteVectors|.
      \item \verb|-economy| --- If given, only the first \verb|min(m,n)| columns for the \verb|U| matrix are calculated or returned, where \verb|m| is the number of rows and \verb|n| is the number of columns. This can potentially reduce computation time with no loss of useful information. The \verb|economy| option is highly recommended for most practical applications since it uses less memory and runs faster. If \verb|economy| is not given, a full \verb|m| by \verb|m| \verb|U| matrix will be internally computed whether or not \verb|-uMatrix| is provided.
      \item \verb|-oldColumnNames=string| --- A string column of name {\tt string} is created in the output file containing the column names of the input files as string data. If this option is not present, then the default name of \verb|OldColumnNames| is used for the string column.
      \item \verb|-multiplyMatrix=file[,invert]| --- If \verb|invert| is not provided, then the output matrix is the inverse of the input matrix multiplied by this matrix. Otherwise, the output matrix is the product of the multiply matrix and the inverse of the input matrix.
      \item \verb|-root=string| --- A string used to generate column names for the output file data. The first data column is named \verb|string000|, the second \verb|string001|, etc.
      \item \verb|-digits=integer| --- Minimum number of digits used in the number appended to \verb|root| of the output file column names. The default value is 3.
      \item \verb|-sFile=file| --- Writes the singular values vector to file.
      \item \verb|-newColumnNames=string| --- Specifies a string column of the input file which will be used to define column names of the output file.
      \item \verb|-uMatrix=file| --- Writes the $u$ column-orthogonal matrix to a file. The SVD decomposition follows the convention $A = u S v^T$. The ``transformed'' $x$ are $v^T x$, and the ``transformed'' $y$ are $u^T y$.
      \item \verb|-vMatrix=file| --- Writes the $v$ column-orthogonal matrix to a file.
      \item \verb|-removeDCVectors| --- Removes the eigenvectors which have an overall DC component.
      \item \verb|-weights=file,name=columnName,value=columnName| --- Specifies a file which contains weights for each of the rows of the matrix, thus giving different weights for solving the linear equations of the pseudoinverse problem. The equation that is solved is $wAx = wy$ where $w$ is the weight vector turned into a diagonal matrix and $A$ is the input matrix. The matrix solution returned is $(wA)^I w$ where $()^I$ means taking the pseudoinverse. The $u$ matrix now has a different interpretation: the ``transformed'' $x$ are $v^T x$, as before, but the ``transformed'' $y$ are $u^T w y$.
      \item \verb|-symbol=string| --- The string for the symbol field of data column definitions.
      \item \verb|-reconstruct=file| --- Specifies a file which will reconstruct the original matrix with only the singular values retained in the inversion.
      \item \verb|-printPackage| --- Prints out the linear algebra package that was compiled.
      \item \verb|-ascii| --- Produces an output in ASCII mode. Default is binary.
      \item \verb|-verbose| --- Prints out incidental information to stderr.
      \item \verb|-noWarnings| --- Suppresses warning messages.
    \end{itemize}
  \item {\bf see also:}
    \begin{itemize}
      \item \progref{sddsmatrixop}
    \end{itemize}
  \item {\bf author:} L. Emery, ANL.
\end{sddsprog}

\begin{sddsprog}{sddsquery}
  \item \textbf{description:}
    \verb|sddsquery| prints a summary of the SDDS header for a data set. It also prints bare lists of names of defined entities,
    suitable to use with shell scripts that need to detect the existence of entities in the data set. Finally, it will create an
    SDDS file containing information about what is in the header.
  \item \textbf{examples:}
    \begin{verbatim}
    sddsquery APS.twi
    sddsquery APS.twi -columnList
    set names = `sddsquery APS.twi -columnList -delimiter=" "`
    sddsquery APS.twi -columnList -sddsOutput=APS.twi.names
    \end{verbatim}
  \item \textbf{synopsis:}
    \begin{verbatim}
    sddsquery SDDSfilename [SDDSfilename...]
      [-sddsOutput[=filename]]
      {-arrayList | -columnList | -parameterList | -version}
      [-delimiter=delimitingString] [-appendUnits[=bare]] [-readAll]
    \end{verbatim}
  \item \textbf{files:} The input filenames may name arbitrary SDDS files.

    If \verb|-sddsOutput| is given, the output normally contains one page for each data class (i.e., array, parameter, and
    column). The following elements are defined:
    \begin{itemize}
      \item Columns (all string type):
        \begin{itemize}
          \item \verb|Name| --- The name of the element.
          \item \verb|Units| --- The units of the data.
          \item \verb|Symbol| --- The symbol for the element.
          \item \verb|Format| --- The format string for the element (e.g., ``\%f'').
          \item \verb|Type| --- The SDDS data type name (e.g., double, float, etc.).
          \item \verb|Description| --- The description for the element.
          \item \verb|Group| --- The group name (for array elements only).
        \end{itemize}
      \item Parameters:
        \begin{itemize}
          \item \verb|Class| --- The SDDS class for the present page.
          \item \verb|Filename| --- The filename being described by the present page.
        \end{itemize}
    \end{itemize}
  \item \textbf{switches:} Normal operation of \verb|sddsquery| results in a printout summarizing the header of each file. If one of
    the options is given, however, this printout will not appear. Instead, the selected list of names appears for each file.
    \begin{itemize}
      \item \verb|sddsOutput[=filename]| --- Requests that output be delivered in SDDS protocol. If no \verb|filename| is given,
        the output is delivered to the standard output.
      \item \verb|arrayList| --- In non-SDDS output mode, requests that a list of array names be printed to the standard output, one
        name per line. In SDDS output mode, requests that only array information be provided.
      \item \verb|columnList| --- In non-SDDS output mode, requests that a list of column names be printed to the standard output,
        one name per line. In SDDS output mode, requests that only column information be provided.
      \item \verb|parameterList| --- In non-SDDS output mode, requests that a list of parameter names be printed to the standard
        output, one name per line. In SDDS output mode, requests that only parameter information be provided.
      \item \verb|-version| --- Requests that the SDDS version number of the file be printed to the standard output. Valid in
        non-SDDS output mode only.
      \item \verb|-delimiter=delimitingString| --- Requests that listed items be separated by the given string. By default, the
        delimiter is a newline. Valid in non-SDDS output mode only.
      \item \verb|-appendUnits[=bare]| --- Requests that the units of each item be printed directly following the item name. Valid in
        non-SDDS output mode only. If the \verb|bare| qualifier is not given, then the units are enclosed in parentheses.
      \item \verb|-readAll| --- Forces \verb|sddsquery| to read the entire file. On some operating systems this is necessary when
        querying compressed files to prevent ``Broken Pipe'' errors. For large files, use of this option will make \verb|sddsquery|
        slower.
    \end{itemize}
  \item \textbf{see also:} \hyperref[exampleData]{Data for Examples}
  \item \textbf{author:} M. Borland, ANL/APS.
\end{sddsprog}


\begin{sddsprog}{sddsregroup}
  \item \textbf{description:}
  \verb|sddsregroup| swaps the row indexing and page indexing of data
  in an SDDS file. That is, the ${\rm i^{th}}$ row of all data pages in the input file are collected
  and made into the ${\rm i^{th}}$ data page of the output file.

  \item \textbf{examples:}
  The file \verb|bpm.sdds| contains the beam position monitor (BPM) readback as a function of time for a series
  of consecutive BPMs in a beamline. The defined columns are \verb|Time| and \verb|x|. The parameter
  is \verb|bpmIndex|. The file is regrouped to produce data sets of \verb|x| vs \verb|bpmIndex|
  for each time value. The output is suitable to plot as a movie with \verb|sddsplot|.
  \begin{verbatim}
  sddsregroup bpm.sdds bpm.movie -newParameters=Time -newColumns=bpmIndex
  \end{verbatim}

  \item \textbf{synopsis:}
  \begin{verbatim}
  sddsregroup [-pipe=[input][,output]] inputFile outputFile
    [-newParameters=oldColumnName,...]
    [-newColumns=oldParameterName,...] [-warning] [-verbose]
  \end{verbatim}

  \item \textbf{switches:}
  \begin{itemize}
    \item \verb|-pipe[=input][,output]| --- The standard SDDS Toolkit pipe option.
    \item \verb|-newParameters=oldColumnName,...| --- Specifies columns of the input file that become parameters in the output file. By default no new parameters are created.
    \item \verb|-newColumns=oldParameterName,...| --- Specifies parameters of the input file that become columns in the output file. These columns are duplicated in all pages. By default parameter values are discarded.
    \item \verb|-warning| --- Report non-fatal warnings.
    \item \verb|-verbose| --- Print processing information.
  \end{itemize}

  \item \textbf{files:}
  \begin{itemize}
    \item \verb|inputFile| --- SDDS file containing the data sets to be regrouped.
    \item \verb|outputFile| --- SDDS file containing the regrouped data. If only one file is specified, the input file is overwritten.
  \end{itemize}

  \item \textbf{see also:} \progref{sddsplot}

  \item \textbf{author:} L. Emery, ANL/APS.
\end{sddsprog}


%\begin{latexonly} 
\newpage 
%\end{latexonly} 
\subsection{sddsremoveoffsets} 
\label{sddsremoveoffsets} 
 
\begin{itemize} 
\item {\bf description:} \hspace*{1mm}\\ 
{\tt sddsremoveoffsets} removes the offset from bpm waveform data with ``commutation'' on.
\item {\bf examples:} 
\begin{flushleft}
{\tt sddsremoveoffsets <input-file> <output-file> -column=BPMValues }
\end{flushleft} 
\item {\bf synopsis:}  
\begin{flushleft}
{\tt 
sddsremoveoffsets [{\em input-file}] [{\em output-file}] [-pipe=[input][,output]] \\ \
-column={\em name} \\ \
{}[-removeCommutationOffsetOnly] \\ \
{}[-fhead={\em value}] \\ \
{}[-majorOrder=row|column] \\ \
{}[-verbose]}
\end{flushleft} 
\item {\bf files:} 
The input file contains bpm waveform data. The output file contains a bpm waveform data with a zero average. 
\item {\bf switches:} 
    \begin{itemize} 
    \item {\tt -pipe=[input][,output]} --- Standard SDDS pipe options for reading/writing files from stdin/stdout.
    \item {\tt -column={\em name}} --- The data column to be adjusted.
    \item {\tt -removeCommutationOffsetOnly} --- Removes only the commutation offset, leaving the average offset.
    \item {\tt -fhead={\em value}} --- Fraction of rows at the head which will be used to determine the offsets.
    \item {\tt -majorOrder=row|column} --- Specifies the binary SDDS layout.
    \item {\tt -verbose} --- Verbose output to stdout.
\end{itemize} 

\item {\bf author:} L. Emery, J. Luo, M. Borland, R. Soliday, ANL/APS. 
\end{itemize} 
 

%\begin{latexonly}
\newpage
%\end{latexonly}
\subsection{sddsrowstats}
\label{sddsrowstats}

\begin{itemize}
\item {\bf description:}
{\tt sddsrowstats} analyzes data across columns on a row-by-row basis to
find minima, maxima, averages, standard-deviations, etc.
The output is a copy of the input with additional columns that contain
the desired statistics.
\item {\bf examples:}
Find the mean x and y orbits from PAR beam-position-monitor data collected
with one set of x and y values (in 32 columns) per row.
\begin{flushleft}{\tt
sddsrowstats par.bpm par.bpm1 -mean=xMean,P?P?x -mean=yMean,P?P?y
}\end{flushleft}
\item {\bf synopsis:}
\begin{flushleft}{\tt
sddsrowstats [-pipe=[input][,output]] [{\em input}] [{\em output}] 
[-nowarnings]
[-mean={\em newName},[,{\em limitOps}],{\em columnNameList}]
[-rms={\em newName},[,{\em limitOps}],{\em columnNameList}]
[-median={\em newName}[,{\em limitOps}],{\em columnNameList}]
[-minimum={\em newName}[,{\em limitOps}],{\em columnNameList}]
[-maximum={\em newName}[,{\em limitOps}],{\em columnNameList}]
[-standardDeviation={\em newName}[,{\em limitOps}],{\em columnNameList}]
[-sigma={\em newName}[,{\em limitOps}],{\em columnNameList}]
[-mad={\em newName}[,{\em limitOps}],{\em columnNameList}]
[-sum={\em newName}[,{\em limitOps}][,power=<integer>],{\em columnNameList}] 
[-drange={\em newName}[,{\em limitOps}],{\em columnNameList}]
[-qrange={\em newName}[,{\em limitOps}],{\em columnNameList}]
[-smallest={\em newName}[,{\em limitOps}],{\em columnNameList}]
[-largest={\em newName}[,{\em limitOps}],{\em columnNameList}]
[-count={\em newName}[,{\em limitOps}],{\em columnNameList}]
}\end{flushleft}
where {\em columnList} is a comma-separated list of one or more optionally wildcarded names and
{\em limitOps} is of the form {\tt [topLimit={\em value},][bottomLimit={\em value}]}.
\item {\bf switches:}
    \begin{itemize}
    \item {\tt -pipe=[input][,output]} --- The standard SDDS Toolkit pipe option.
    \item {\tt -mean}, {\tt -rms},
        {\tt median}, {\tt minimum}, {\tt maximum}, {\tt standardDeviation}, {\tt sigma}, 
        {\tt mad}, {\tt drange}, {\tt qrange}, {\tt smallest}, {\tt largest}, {\tt count} ---
        Compute indicated statistic across the columns specified in {\em columnList}.
        If {\em limitOps} are given, then values above the {\tt topLimit} or below the
        {\tt bottomLimit} are excluded from computations.
        {\tt sigma} is the standard deviation of the mean.
        {\tt mad} is the mean-absolute-deviation.  {\tt smallest} ({\tt largest}) is
        the minimum (maximum) absolute value.  {\tt drange} and {\tt qrange} are the
        decile and quartile ranges, respectively.
    \item {\tt -sum={\em newName}[,power={\em integer}],{\em columnNameList} } --- 
        Specifies creation of a new column
        {\em newName} containing the row-by-row sums of the columns specified
        in {\em columnList}.  The values are summed after being raised to the given power, which is 1 by
        default.
    \end{itemize}
\item {\bf see also:}
    \begin{itemize}
    \item \hyperref[exampleData]{Data for Examples}
    \item \progref{sddschanges}
    \item \progref{sddsenvelope}
    \item \progref{sddsprocess}
    \end{itemize}
\item {\bf author:} M. Borland, ANL/APS.
\end{itemize}


\begin{sddsprog}{sddsrunstats}
  \item \textbf{description:} \verb|sddsrunstats| computes running or blocked statistics on SDDS tabular data.

  \item \textbf{examples:}
  Smooth PAR x beam-position-monitor data by using a sliding window 32 points long:
  \begin{verbatim}
  sddsrunstats par.bpm par.bpm.rs -mean=Time,P?P?x
  \end{verbatim}
  Same, but use nonoverlapping window for averages:
  \begin{verbatim}
  sddsrunstats par.bpm par.bpm.rs -mean=Time,P?P?x  -noOverlap
  \end{verbatim}

  \item \textbf{synopsis:}
  \begin{verbatim}
  sddsrunstats [-pipe[=input][,output]] [input] [output] \\
    [-points=integer | -window=column=column,width=value] [-noOverlap] \\
    [-partialOk] \\
    [-mean=[limitOps],columnNameList] \\
    [-minimum=[limitOps],columnNameList] \\
    [-maximum=[limitOps],columnNameList] \\
    [-standardDeviation=[limitOps],columnNameList] \\
    [-sigma=[limitOps],columnNameList] \\
    [-sum=[limitOps][,power=integer],columnNameList] \\
    [-sample=[limitOps],columnNameList]
  \end{verbatim}
  where \emph{columnNameList} is a comma-separated list of one or more optionally wildcarded names and \emph{limitOps} is of the form \verb|[topLimit=value,][bottomLimit=value]|.

  \item \textbf{switches:}
    \begin{itemize}
    \item {\tt -pipe=[input][,output]} --- The standard SDDS Toolkit pipe option.
    \item {\tt -points=\emph{integer}} --- The number of points in the statistics window for each output row. If non-overlapping statistics are used, the last output row will be computed from fewer than the specified number of points if the input file number of rows is not a multiple of the specified number of points.
    \item {\tt -window=column=\emph{column},width=\emph{value}} --- Specifies a column to use for determining statistics row boundaries. For example, one might want statistics for 60 second blocks of data when the data is not uniformly sampled in time. In this case, the column would be \verb|Time| and the width 60.
    \item {\tt -partialOk} --- Specifies that \verb|sddsrunstats| should do computations even if the number of available rows is less than specified. By default, such data is simply ignored.
    \item {\tt -noOverlap} --- Specifies non-overlapping statistics. The default is to compute running statistics with a sliding window.
    \item {\tt -mean=[\emph{limitOps}],\emph{columnNameList}}\\
      {\tt -minimum=[\emph{limitOps}],\emph{columnNameList}}\\
      {\tt -maximum=[\emph{limitOps}],\emph{columnNameList}}\\
      {\tt -standardDeviation=[\emph{limitOps}],\emph{columnNameList}}\\
      {\tt -sigma=[\emph{limitOps}],\emph{columnNameList}} --- Specifies computation of the indicated statistic for the columns matching \emph{columnNameList} (see above). The standard deviation is N-1 weighted. Sigma is the standard deviation of the sample mean. \emph{limitOps} (see above for syntax) allows filtering the points in each window to exclude values above the \verb|topLimit| or below the \verb|bottomLimit|.
    \item {\tt -sum=[\emph{limitOps}][,power=\emph{integer}],\emph{columnNameList}} --- Specifies computation of a general sum of powers of values. For example, to get the sum of squares you'd use \verb|power=2|. \emph{columnNameList} and \emph{limitOps} are as for the last item.
    \item {\tt -sample=[\emph{limitOps}],\emph{columnNameList}} --- Results in extraction of a single set of values per group, namely, the first value in the group that passes the \emph{limitOps} criteria.
    \end{itemize}

  \item \textbf{see also:}
    \begin{itemize}
    \item \hyperref[exampleData]{Data for Examples}
    \item \progref{sddssmooth}
    \end{itemize}

  \item \textbf{author:} M. Borland, ANL/APS.
\end{sddsprog}


%\begin{latexonly}
\newpage
%\end{latexonly}
\subsection{sddssampledist}
\label{sddssampledist}

\begin{sddsprog}{sddssampledist}
  \item \textbf{description:} \verb|sddssampledist| provides for pseudo-random sampling of probability distributions. It also provides nonrandom sampling using Halton sequences.
  \item \textbf{examples:} Draw random samples from a normal (Gaussian) distribution, G(z), shifted to have a sigma of 10 and centroid of 5.
    \begin{verbatim}
    sddssampledist gaussian.sdds samples.sdds -samples=100 -columns=indep=z,df=G,output=zSample,factor=10,offset=5
    \end{verbatim}
  \item \textbf{synopsis:}
    \begin{verbatim}
    sddssampledist [input] [output] [-pipe=[in][,out]]
      -columns=independentVariable=name,{cdf=CDFName | df=DFName}
        [,output=name][,units=string][,factor=value]
        [,offset=value][,datafile=filename]
        [,haltonRadix=primeNumber[,randomize[,group=groupID]]]
      [-columns=...] [-samples=integer] [-seed=integer]
    \end{verbatim}
  \item \textbf{files:}
    \emph{input} is the default input file for distribution functions (DFs) and cumulative distribution functions (CDFs). \emph{input} is not required if all \verb|-columns| options give the \verb|datafile| qualifier.
    \emph{output} contains the samples. By default the sampled data names match the independent variable names from the \verb|-columns| options. Use the \verb|output| qualifier to change these names.
  \item \textbf{switches:}
    \begin{itemize}
      \item \verb|-pipe[=input][,output]| --- The standard SDDS Toolkit pipe option.
      \item \verb!-columns=independentVariable=name,{cdf=CDFName | df=DFName}[,output=name][,units=string][,factor=value][,offset=value][,datafile=filename][,haltonRadix=primeNumber[,randomize[,group=groupID]]]!--- Specifies the CDF or DF from which to draw samples (\verb|cdf| or \verb|df| qualifier) and the independent variable. This option may be given multiple times. \verb|output| sets the column name for the samples. \verb|units| specifies the units. \verb|factor| and \verb|offset| apply a transformation $x \rightarrow x*f+o$. \verb|datafile| gives an alternate file containing the distribution function data, otherwise the main input file is used. \verb|haltonRadix| selects the radix for generating a non-random Halton sequence, which provides smoother sampling than a pseudo-random sequence. The radix should be a small prime number. Use \verb|randomize| to remove correlations when using the same radix for multiple sequences. Use \verb|group| to assign options to a group for correlated randomization.
      \item \verb|-samples=integer| --- Specifies the number of samples to generate.
      \item \verb|-seed=integer| --- Specifies the seed for the random number generation. Should be a large, odd integer. If not given, the system clock is used.
    \end{itemize}
  \item \textbf{author:} M. Borland, ANL/APS.
\end{sddsprog}


%\begin{latexonly}
\newpage
%\end{latexonly}
\subsection{sddsselect}
\label{sddsselect}

\begin{itemize}
\item {\bf description:}
{\tt sddsselect} excludes or includes rows from one file based on the presence of matching data in another
file.  It is similar to {\tt sddsxref}, but unlike that program does not import data from the second file.
\item {\bf examples:} 
Use a list of quadrupole names to get just the Twiss parameters are the quadrupoles:
\begin{flushleft}{\tt
sddsselect APS.twi quadNames.sdds APSquad.twi -match=ElementName -reuse
}\end{flushleft}
where {\tt ElementName} is a column in both {\tt APS.twi} and {\tt quadNames.sdds} giving the
name of a magnet.
Use the same file to get the Twiss parameters everywhere but at the quadrupoles:
\begin{flushleft}{\tt
sddsselect APS.twi quadNames.sdds APSnquad.twi -match=ElementName -reuse -invert
}\end{flushleft}
\item {\bf synopsis:} 
\begin{flushleft}{\tt
sddsselect [-pipe[=input][,output]] [{\em input1}] {\em input2} [{\em output}] 
\{-match={\em columnName1}[={\em columnName2}] |
 -equate={\em columnName1}[={\em columnName2}] \}
[-invert] [-reuse[=page][,rows]] [-noWarnings]
}
\end{flushleft}
\item {\bf files:}
{\em input1} is an SDDS file from which rows of data will be selected for inclusion in {\em output}.  
If {\em input1} contains multiple pages, they are processed separately. {\em input2} is an SDDS
file containing rows of data to use in selecting data from {\em input1}.  {\em Warning:} if {\em output} is not given and
{\tt -pipe=output} is not specified, then {\em input1} will be replaced.
\item {\bf switches:}
    \begin{itemize}
    \item {\tt -pipe[=input][,output]} --- The standard SDDS Toolkit pipe option.
    \item {\tt -match={\em columnName1}[={\em columnName2}] } --- Specifies the names of string columns from {\em input1}
        and {\em input2} to compare.  If {\em columnName2} is not given, it taken to be the same as {\em columnName1}.
        Data in {\em columnName} is taken from {\em input1} and {\em columnName2} from {\em input2}.  For each row in a page
        of {\em input1}, a match for the string in {\em columnName1} is sought in any row of {\em columnName2}.  If a match
        is found, the row is accepted.
    \item {\tt -equate={\em columnName1}[={\em columnName2}] } --- Identical to {\tt -match}, except the columns contain
        numerical data.
    \item {\tt -invert} --- Specifies that only rows that have no match or equal should be selected for output.
    \item {\tt -reuse[=rows][,page]} --- By default, if {\em input1}  contains multiple pages, each is selected against
        the corresponding page of {\em input2}.  In addition, each row of {\em input2} is matched or equated to only
        one row of {\em input1}.  If {\tt -reuse=page} is given, then each page of {\em input1}
        is selected against the first page of {\em input2}.   If {\tt -reuse=rows} is given, each row of {\em input2}
        can select any number of rows of {\em input1}.
    \item {\tt -noWarnings} --- Specifies that no warning messages (about, e.g., file length mismatches
        or file overwrites) should be issued.
    \end{itemize}
\item {\bf sddsmselect} --- {\tt sddsmselect} is a variant of {\tt sddsselect} that permits multiple {\tt -match}
 and {\tt -equate} options for more sophisticated cross-referencing.  In other respects, the program is
 used just like {\tt sddsmselect}.  {\tt sddsselect} is much faster, however, for single-criterion matching or
 equating.
\item {\bf see also:}
    \begin{itemize}
    \item \hyperref[exampleData]{Data for Examples}
    \item \progref{sddsxref}
    \end{itemize}
\item {\bf author:} M. Borland, H. Shang and R. Soliday ANL/APS.
\end{itemize}






%\begin{latexonly} 
\newpage 
%\end{latexonly} 
\subsection{sddsseparate} 
\label{sddsseparate} 
 
\begin{itemize} 
\item {\bf description:} \hspace*{1mm}\\ 
{\tt sddsseparate} reorganizes the column data so that data from different columns ends up on different pages.
\item {\bf examples:} 
\begin{flushleft}
{\tt sddsseparate <inputfile> <outputfile> ``-group=Values,(Data1,Data2,Data3)'' }
\end{flushleft} 
\item {\bf synopsis:}  
\begin{flushleft}
{\tt 
sddsseparate [{\em inputfile}] [{\em outputfile}] [-pipe=[input][,output]] \\ \
-group={\em newName},({\em listOfOldNames}) \\ \
{}[-copy={\em listOfNames}]}
\end{flushleft} 
\item {\bf files:} 
The input file is a multi-column SDDS file. The output file is a single column, mutli-page SDDS file. 
\item {\bf switches:} 
    \begin{itemize} 
    \item {\tt -pipe=[input][,output]} --- Standard SDDS pipe options for reading/writing files from stdin/stdout.
    \item {\tt -group={\em newName},({\em listOfOldNames}[,...])} --- Multiple columns are reorganized into a single column that spans multiple pages.
    \item {\tt -copy={\em listOfNames}} --- A list of columns that are duplicated on each page.
\end{itemize} 

\item {\bf author:} M. Borland, R. Soliday, ANL/APS. 
\end{itemize} 
 

\begin{sddsprog}{sddssequence}
  \item \textbf{description:} \verb|sddssequence| generates an SDDS file with a single page and several columns of data of arithmetic sequences. An example application is generating values for an independent variable that may be used by \progref{sddsprocess} to produce a mathematical function.
  \item \textbf{examples:}
    \begin{verbatim}
    sddssequence example.sdds -define=Index,type=long -sequence=begin=1,number=100,delta=1
    \end{verbatim}
  \item \textbf{synopsis:}
    \begin{verbatim}
    sddssequence [-pipe=[output]] [<outputfile>]
                 -define=<columnName>[,<definitionEntries>] [-repeat=<number>]
                 -sequence=begin=<value>[,number=<integer>][,end=<value>][,delta=<value>][,interval=<integer>]
                 [-sequence=begin=<value>[,number=<integer>][,end=<value>][,delta=<value>][,interval=<integer>] ...]
    \end{verbatim}
  \item \textbf{switches:}
    \begin{itemize}
      \item \verb|-pipe=[output]| --- The standard SDDS Toolkit pipe option.
      \item \verb|-define=<columnName>[,<definitionEntries>]| --- Defines a new column. One or more \verb|-sequence| options should follow. Definition entries have the form \verb|fieldName=value| where \verb|fieldName| is the name of any namelist command field (except the name field) for a column. The default data type is double. To get a different type, use \verb|type=<typeName>|. Multiple \verb|-define| options can be used to create multiple columns, each with its own set of \verb|-sequence| options.
      \item \verb|-sequence=begin=<value>[,number=<integer>][,end=<value>][,delta=<value>][,interval=<integer>]| --- Defines the arithmetic sequence for the data column. More than one \verb|-sequence| option can be given for a column definition, allowing arithmetic sequences of different character in one column. The \verb|begin| value must be given in the first \verb|-sequence| option. If subsequent \verb|-sequence| options follow immediately, a default value equal to the previous \verb|end| value plus the previous \verb|delta| value is used. For the rest of the suboptions, the user must supply \verb|(end,delta)|, \verb|(end,number)|, or \verb|(delta,number)|. If \verb|number| isn't supplied, then the set of \verb|begin|, \verb|end|, \verb|delta| must imply a positive number of rows. The \verb|interval| field specifies the number of rows for which the value is frozen within the sequence.
      \item \verb|-repeat=<number>| --- Repeats the sequence identically for the given number of times.
    \end{itemize}
  \item \textbf{files:} \verb|outputfile| is the name of the SDDS file containing data generated.
  \item \textbf{see also:}
    \begin{itemize}
      \item \progref{sddsprocess}
      \item \progref{sddsprintout}
    \end{itemize}
  \item \textbf{author:} M. Borland, ANL/APS.
\end{sddsprog}

%\begin{latexonly} 
\newpage 
%\end{latexonly} 
\subsection{sddsshift} 
\label{sddsshift} 
 
\begin{itemize} 
\item {\bf description:} \hspace*{1mm}\\ 
{\tt sddsshift} shifts the given data columns by rows.
\item {\bf examples:} 
\begin{flushleft}
{\tt sddsshift <inputfile> <outputfile> -columns=Values -shift=5 -zero }
\end{flushleft} 
\item {\bf synopsis:}  
\begin{flushleft}
{\tt 
sddsshift [{\em inputfile}] [{\em outputfile}] [-pipe=[input][,output]] \\ \
-columns={\em inputcol}[,...] \\ \
{}[-zero] \\ \
{}[-shift={\em points} | -match={\em matchcol}] \\ \
{}[-majorOrder=row|column]}
\end{flushleft} 
\item {\bf files:} 
The output file contains all the columns from the input file as well as new columns named {\tt Shifted{\em inputcol}} and new parameters named {\tt {\em inputcol}Shift}. Exposed end-points are set to zero if the zero option is provided, otherwise they are set to the value from the first or last row as appropriate.
\item {\bf switches:} 
    \begin{itemize} 
    \item {\tt -pipe=[input][,output]} --- Standard SDDS pipe options for reading/writing files from stdin/stdout.
    \item {\tt -columns={\em inputcol}[,...]} --- The names of the columns to be shifted.
    \item {\tt -zero} --- Set exposed end-points to zero.
    \item {\tt -shift={\em points}} --- Number of rows to shift columns. Positive and negative numbers are both allowed.
    \item {\tt -match={\em matchcol}} --- The columns are shifted to minimize the least squares error relative to {\em matchcol}.
    \item {\tt -majorOrder=row|column} --- Specifies the binary SDDS layout.
\end{itemize} 

\item {\bf author:} C. Saunders, M. Borland, R. Soliday, H. Shang, ANL/APS. 
\end{itemize} 
 

%\begin{latexonly}
\newpage
%\end{latexonly}
\subsection{sddsshiftcor}
\label{sddsshiftcor}

\begin{itemize}
\item {\bf description:} 
{\tt sddsshiftcor} computes correlation coefficients and correlation
significance between column data as a function of shifting of the data columns
relative to each other.  The correlation coefficient between
columns i and j is defined as
\[ {\rm C_{ij} = \frac{\langle x_i x_j \rangle}{\sqrt{\langle x_i^2\rangle\langle x_j^2 \rangle}}} \]
If ${\rm C_{ij}=1}$, then the variables are perfectly correlated, whereas if ${\rm C_{ij}=-1}$, they
are perfectly anticorrelated.
In some cases, signals are correlated but with a time-lag.  Hence, computing \[ {\rm C_{ij}} \]
as a function of the shifting of one of the signals may reveal relationships that are not
apparent in a simple correlation, such as might be done with {\tt sddscorrelate}.
\item {\bf synopsis:}
\begin{flushleft}{\tt
sddsshiftcor [-pipe=[input][,output]] [{\em inputFile}] [{\em outputFile}] 
-with={\em columnName} 
[-scan[=start={\em startShift}][,end={\em endShift}][,delta={\em deltaShift}]]
[-columns={\em columnNames}] [-excludeColumns={\em columnNames}] 
[-rankOrder] [-stDevOutlier[=limit={\em factor}][,passes={\em integer}]]
[-verbose]
}\end{flushleft}
\item {\bf files:}
        {\em inputFile} is an SDDS file containing two or more columns of data.  {\em outputFile}
        contains one column ({\tt ShiftedBy}) for the amount shifted, plus one column for
        each analyzed column in {\em inputFile}.  The latter each contains the correlation
        coefficient with the shifted signal for the given shift value.
\item {\bf switches:}
    \begin{itemize}
    \item {\tt -pipe=[input][,output] } --- The standard SDDS Toolkit pipe option.
    \item {\tt -with={\em columnName}} --- Specifies the column to be shifted, which is correlated
        with the other columns.
    \item {\tt -scan[=start={\em startShift}][,end={\em endShift}][,delta={\em deltaShift}]} --- 
        Specifies the amount to shift and the step size.  The values are all integers.  By
        default {\em startShift}=-10, {\em endShift}=10, and {\em deltaShift}=1
    \item {\tt -columns={\em columnNames}} --- Specifies the names of columns to be included in the analysis.
        A comma-separated list of optionally wildcard-containing names may be given.
    \item {\tt -excludeColumns={\em columnNames}} --- Specifies the names of columns to be excluded from the
        analysis.  A comma-separated list of optionally wildcard-containing names may be given.
    \item {\tt -rankOrder} --- Specifies computing rank-order correlations rather than standard correlations.
        This is considered more robust that standard correlations.
    \item {\tt -stDevOutlier[=limit={\em factor}][,passes={\em integer}]} --- Specifies standard-deviation-based
        outlier elimination on each pair of columns prior to computation of the correlation coefficient.
        Any pair of values is ignored if one or both values are outliers relative to the column from which they come.
        The {\tt limit} qualifier specifies the allowed deviation from the mean in standard deviations; the
        default is 1.  The {\tt passes} qualifier specifies how many times the outlier elimination (including
        recomputation of the mean and standard deviation) is performed; the default is 1.
    \end{itemize}
\item {\bf see also:}
    \begin{itemize}
    \item \progref{sddscorrelate}
    \end{itemize}
\item {\bf author:} M. Borland, ANL/APS.
\end{itemize}



\begin{sddsprog}{sddssinefit}
  \item \textbf{description:} \verb|sddssinefit| fits data to the form
    $y = \verb|constant| + \verb|factor| \sin(2\pi \verb|freq| x + \verb|phase|)$.
  \item \textbf{examples:}
    \begin{verbatim}
    sddssinefit data.sdds fit.sdds -columns=x,y
    \end{verbatim}
  \item \textbf{synopsis:}
    \begin{verbatim}
    sddssinefit [inputfile] [outputfile] [-pipe=[input][,output]]
      [-fulloutput]
      [-columns=x-name,y-name]
      [-tolerance=value]
      [-limits=[evaluations=number][,passes=number]]
      [-verbosity=integer]
      [-guess=[constant=constant][,factor=factor][,frequency=freq][,phase=phase]]
      [-majorOrder=row|column]
    \end{verbatim}
  \item \textbf{switches:}
    \begin{itemize}
      \item \verb|-fulloutput| --- Includes \verb|y-name| and \verb|y-name|Residual in the output file.
      \item \verb|-columns=x-name,y-name| --- Specifies the independent and dependent columns.
      \item \verb|-tolerance=value| --- Desired tolerance of simplex minimization. Negative values are fractional,
        positive numbers are absolute. Default: 1e-6.
      \item \verb|-limits=[evaluations=number][,passes=number]| --- Limits for simplex minimization.
        Default: evaluations=100, passes=5.
      \item \verb|-verbosity=integer| --- Sets the level of informational messages. Valid values are 0-4.
      \item \verb|-guess=[constant=constant][,factor=factor][,frequency=freq][,phase=phase]| --- Initial guesses for
        simplex minimization.
      \item \verb!-majorOrder=row|column! --- Specifies the binary SDDS layout.
    \end{itemize}
  \item \textbf{files:} The input file must contain columns \verb|x-name| and \verb|y-name|. The output file contains the
    independent column and fitted values in a column named by appending \verb|Fit| to \emph{y-name}. If \verb|-fulloutput| is
    given, a residual column named by appending \verb|Residual| to \emph{y-name} is also included.
  \item \textbf{see also:} \progref{sddsexpfit}, \progref{sddsgenericfit}
  \item \textbf{author:} C. Saunders, M. Borland, R. Soliday, L. Emery, H. Shang, ANL/APS.
\end{sddsprog}


%
%\begin{latexonly}
\newpage
%\end{latexonly}

%
% Substitute the program name for <programName>
%
\subsection{sddsslopes}
\label{sddsslopes}

\begin{itemize}
\item {\bf description:}
%
% Insert text of description (typicall a paragraph) here.
%
\verb+sddsslopes+ makes straight line fits of column data
of the input file with respect to a selected column used as independent variable.
The output file contains a one-row data set of slopes and intercepts
for each data set of the input file.
Errors on the slope and intercept may be
calculated as an option.

\item {\bf examples:} 
%
% Insert text of examples in this section.  Examples should be simple and
% should be preceeded by a brief description.  Wrap the commands for each
% example in the following construct:
% 
%
The file corrector.sdds below contains beam position monitors (bpms) readbacks as a 
function of corrector
setting. The defined columns are {\tt CorrectorSetpoint} and the series
{\tt bpm1}, {\tt bpm2}, etc.
The bpm response to the corrector setpoints are calculated with the use of \verb+sddsslopes+:
\begin{flushleft}{\tt
sddsslopes corrector.sdds corrector.slopes -independentVariable=CorrectorSetpoint
   -columns='bpm*'
}\end{flushleft}
where all columns that match with the wildcard expression \verb+bpm*+ is selected
for fitting.
\item {\bf synopsis:} 
%
% Insert usage message here:
%
\begin{flushleft}{\tt
sddsslopes [-pipe=[input][,output]] {\em inputFile} {\em outputFile}
      -independentVariable={\em parameterName} [-range={\em lower},{\em upper}]
      [-columns={\em listOfNames}] [-excludeColumns={\em listOfNames}] 
      [-sigma[=generate]] [-residual={\em file}] [-ascii] [-verbose]
}\end{flushleft}
\item {\bf files:}
% Describe the files that are used and produced
The input file contains the tabular data for fitting. Multiple
data sets are processed one at a time. 
For optional error processing, additional columns of sigma values
associated with the data to be fitted must be present. These sigma column 
must be named {\tt {\em name}Sigma} or {\tt Sigma{\em name}},
the former one being searched first.

The output file contains a one-row data set for each data set in the 
input file. The columns defined have names
such as {\tt {\em name}Slope}, and {\tt {\em name}Intercept} where {\tt {\em name}} is the name of
the fitted data.  If only one file is specified, then the input file is 
overwritten by the output.
A string column called \verb+IndenpendentVariable+ is defined containing the name of the indepedent variable.

\item {\bf switches:}
%
% Describe the switches that are available
%
    \begin{itemize}
    \item {\tt  -pipe[=input][,output]} --- The standard SDDS Toolkit pipe option.
    \item {\tt  -independentVariable={\em parametername} }
        --- name of independent variable (default is the first valid column).
    \item {\tt  -range={\em lower},{\em upper}} --- The range of the independent
        variable where the fit is calculated. By default, all data points are used.
    \item {\tt  -columns={\em listOfNames}}   
        ---  columns to be individually paired with independentVariable 
        for straight line fitting.
    \item {\tt  -excludeColumns={\em listOfNames}}  ---  columns to exclude from fitting.
    \item {\tt  -sigma[=generate]}  
        ---   calculates errors by interpreting column names 
        {\tt {\em name}Sigma} or {\tt Sigma{\em name}} as
        sigma of column {\tt {\em name}}. If these columns don't exist
        then the program generates a common sigma from the residual of a first fit,
        and refits with these sigmas. If option {\tt -sigma=generate} is given,
        then sigmas are generated from the residual of a first fit for all columns,
        irrespective of the presence of columns {\tt {\em name}Sigma} or {\tt Sigma{\em name}}.
    \item {\tt -residual={\em file}} --- Specifies an output file into which the
        residual of the fits are written. The column names in the residual file
         are the same as they appear in the input file.
    \item {\tt  -ascii }    ---  make output file in ascii mode (binary is the default).
    \item {\tt  -verbose }  ---  prints some output to stderr.

    \end{itemize}
%\item {\bf see also:}
%    \begin{itemize}
%
% Insert references to other programs by duplicating this line and 
% replacing <prog> with the program to be referenced:
%
%    \item \progref{<prog>}
%    \end{itemize}
%
% Insert your name and affiliation after the '}'
%
\item {\bf author: L. Emery } ANL
\end{itemize}


\begin{sddsprog}{sddssmooth}
  \item \textbf{description:} {\tt sddssmooth} smooths columns of data using multipass nearest-neighbor averaging and/or despiking. Any number of columns may be smoothed. The smoothed data may be put in place of the original data, or included as a new column.

  Nearest-neighbor averaging involves repeatedly replacing each point by the average of its N nearest-neighbors; this is the type of smoothing that is done if nothing is specified. Despiking consists of replacing the most extreme of N nearest neighbors with the average of the same points; the most extreme point is the one with the largest mean absolute difference from the other points.
  \item \textbf{examples:}
    \begin{verbatim}
    sddssmooth data.fft data.peaks -column=FFTamplitude
    \end{verbatim}
  \item \textbf{synopsis:}
    \begin{verbatim}
    sddssmooth [-pipe=[input][,output]] [inputfile] [outputfile]
      -columns=name[,name...]
      [-points=oddInteger] [-passes=integer]
      [-SavitzkyGolay=left,right,order[,derivativeOrder]]
      [-despike[=neighbors=integer][,passes=integer]]
      [-newColumns] [-differenceColumns]
      [-medianFilter=windowSize=integer]
    \end{verbatim}
  \item \textbf{files:}
    \emph{inputFile} contains the data to be smoothed. \emph{outputFile} contains all of the array and parameter data from \emph{inputFile}, plus at least one column for every column in \emph{inputFile}. Columns that are not smoothed will appear unchanged in \emph{outputFile}. If \emph{inputFile} contains multiple pages, each is treated separately and is delivered to a separate page of \emph{outputFile}.
  \item \textbf{switches:}
    \begin{itemize}
      \item \verb|-pipe[=input][,output]| --- The standard SDDS Toolkit pipe option.
      \item {\tt -columns={\em columnName}[,{\em columnName...}]} --- Specifies the names of the columns to smooth. The names may include wildcards.
      \item {\tt -points={\em oddInteger}} --- Specifies the number of points to average to create a smoothed value for each point. The default is three, which implies replacing each point by the average of itself and its two nearest neighbors.
      \item {\tt -passes={\em integer}} --- Specifies the number of nearest-neighbor-averaging smoothing passes to make over each column of data. The default is 1. If 0, no such smoothing is done. In the limit of an infinite number of passes, every point will tend toward the average value of the original data. If {\tt -despike} is also given, then despiking occurs first.
      \item {\tt -SavitzkyGolay={\em left},{\em right},{\em order}[,{\em derivativeOrder}]} --- Specifies smoothing by use of a Savitzky-Golay filter, which involves fitting a polynomial of order {\em order} through {\em left}+{\em right}+1 points. Optionally, takes the {\em derivativeOrder}-th derivative of the data. If this option is given, the nearest-neighbor-averaging smoothing is not done. If {\tt -despike} is also given, then despiking occurs first.
      \item {\tt -despike[=neighbors={\em integer}][,passes={\em integer}]} --- Specifies smoothing by despiking, as described above. By default, 4 nearest-neighbors are used and 1 pass is done. If this option is not given, no despiking is done.
      \item {\tt -newColumns} --- Specifies that the smoothed data will be placed in new columns, rather than replacing the data in each column with the smoothed result. The new columns are given names of the form {\tt {\em columnName}Smoothed}, where {\em columnName} is the original name of a column.
      \item {\tt -differenceColumns} --- Specifies that additional columns be created in the output file, containing the difference between the original data and the smoothed data. The new columns are given names of the form {\em columnName}{\tt Unsmooth}, where {\em columnName} is the original name of the column.
      \item {\tt -medianFilter=windowSize={\em integer}} --- Specifies median smoothing and the window size (W, an odd integer, default is 3). It smooths the original data by taking the median of a data point among the nearest left (W-1)/2 points, the data point, and the nearest right (W-1)/2 points.
    \end{itemize}
  \item \textbf{see also:}
    \begin{itemize}
      \item \progref{sddsdigfilter}
    \end{itemize}
  \item \textbf{author:} M. Borland, ANL/APS.
\end{sddsprog}

\begin{sddsprog}{sddssnap2grid}
  \item \textbf{description:}
    \verb|sddssnap2grid| reads data pages from an SDDS file and writes a new SDDS file.
    The output data contains all of the input data, except that one or more columns may be
    modified to ``snap'' the values to a uniform grid.
  \item \textbf{examples:}
  \begin{verbatim}
sddssnap2grid fieldMap.sdds fieldMap1.sdds -column=x -column=y -column=z
sddssnap2grid fieldMap.sdds fieldMap1.sdds -column=x -column=y -column=z,deltaGuess=5e-4
  \end{verbatim}
  \item \textbf{synopsis:}
  \begin{verbatim}
sddssnap2grid [<options>] inputFile outputFile
  \end{verbatim}
  \item \textbf{switches:}
    \begin{itemize}
      \item \verb|-pipe[=input][,output]| --- The standard SDDS Toolkit pipe option.
      \item \verb!-column=name[{maximumBins=value | binFactor=value} | deltaGuess=value]! ---
        Specifies a column to be snapped to a grid. The column must contain numerical data.
        The algorithm uses histograms to group the data points into subsets; this grouping is
        considered valid if there are no adjacent bins in the histogram that have non-zero values.
        By default, the number of bins is 10 times the number of data points, which seems reliable
        if data is actually close to a grid. If the algorithm fails, the user can provide additional
        parameters to attempt to obtain a good result. The first thing to try is providing a guess of
        the grid spacing using \verb|deltaGuess|; in this case the initial number of bins is based on
        10 times the provided spacing. Next, one can try providing a value for the \verb|binFactor|
        parameter that is higher or lower than 10. Finally, the \verb|maximumBins| parameter can be set
        directly.
      \item \verb|-verbose| --- Prints grid parameters to standard output.
    \end{itemize}
  \item \textbf{files:}
    \emph{inputFile} is the name of an SDDS data set to be snapped. \emph{outputFile} is the result.
  \item \textbf{author:} M. Borland, ANL/APS.
\end{sddsprog}


%\begin{latexonly}
\newpage
%\end{latexonly}
\subsection{sddssort}
\label{sddssort}

\begin{sddsprog}{sddssort}
  \item \textbf{description:} \verb|sddssort| sorts the tabular data section of a data set by the values in named columns.
    Any number of columns may be involved in the sort, and sorting order may be individually specified.
  \item \textbf{examples:}
    \begin{verbatim}
    sddssort APS.twi APS.twi.sorted -column=ElementName
    sddssort APS.twi APS.twi.sorted -column=ElementName -unique
    \end{verbatim}
  \item \textbf{synopsis:}
    \begin{verbatim}
    sddssort [-pipe=[input][,output]] [SDDSinput] [SDDSoutput]
      -column=name[, {increasing | decreasing}] [-column...]
      [-parameter=name[, {increasing | decreasing}] ...]
      [-numericHigh] [-nonDominateSort]
      [-unique[=count]] [-noWarnings]
    \end{verbatim}
  \item \textbf{files:} \emph{SDDSinput} is an SDDS file to be sorted. If it contains multiple data pages, they are treated
    separately. \emph{Warning:} if \emph{SDDSoutput} is not given and \verb|-pipe=output| is not specified, then
    \emph{SDDSinput} will be replaced.
  \item \textbf{switches:}
    \begin{itemize}
      \item \verb|-pipe=[input][,output]| --- The standard SDDS pipe option.
      \item \verb!-column=name[, {increasing | decreasing}]! --- Requests that the column \emph{name} be used to order the rows of each tabular data section. Each subsequent \verb|column| request specifies a subsort of the ordering produced by the previous requests. The \verb|increasing| and \verb|decreasing| keywords may be given to specify the desired ordering of the (sub)sort, with increasing order being the default.
      \item \verb!-parameter=name[, {increasing | decreasing}]! --- Similar to column requests, but sort the data pages by parameters.
      \item \verb|-unique[=count]| --- Specifies that for any rows that are identical in the sort column values, only the first should be included in the output file. If the \verb|count| qualifier is given, then a count of the number of identical rows is supplied in a column called \verb|IdenticalCount|.
      \item \verb|-nonDominateSort| --- Perform non-dominated-sort when multiply sorting columns supplied. Non-dominated-sort only works for numeric columns.
      \item \verb|-numericHigh| --- Works for string sorting which rank the numeric characters higher than other characters in a string comparison. It also ranks numeric character sets with fewer characters below numeric character sets with more characters.
      \item \verb|-noWarnings| --- Suppresses warning messages.
    \end{itemize}
  \item \textbf{see also:} \progref{sddssortcolumn}
  \item \textbf{author:} M. Borland, ANL/APS.
\end{sddsprog}


\begin{sddsprog}{sddssortcolumn}
  \item \textbf{description:}
    \verb|sddssortcolumn| rearranges the columns of the input file in a specified order.
  \item \textbf{examples:}
    \begin{verbatim}
sddssortcolumn data.sdds sorted.sdds -sortList=col3,col1,col2
sddssortcolumn data.sdds -pipe=out -sortWith=order.sdds,column=Name
    \end{verbatim}
  \item \textbf{synopsis:}
    \begin{verbatim}
sddssortcolumn [SDDSinput] [SDDSoutput] [-pipe=[input][,output]]
  [-sortList=<list of columns in order>] [-decreasing]
  [-bpmOrder] [-sortWith=<filename>,column=<string>]
    \end{verbatim}
  \item \textbf{switches:}
    \begin{itemize}
      \item \verb|-sortList=<column1>,<column2>,...| --- specify the order of column names in a list.
      \item \verb|-pipe=[input][,output]| --- the standard SDDS Toolkit pipe option.
      \item \verb|-sortWith=<filename>,column=<string>| --- sort columns by the order in column \verb|<string>| of \verb|<filename>|; overrides other sorting.
      \item \verb|-bpmOrder| --- sort columns by the assumed BPM position in the storage ring.
      \item \verb|-decreasing| --- sort columns in decreasing order; default is increasing.
    \end{itemize}
  \item \textbf{files:}
    \emph{inputFile} is an SDDS file whose columns will be rearranged. The \emph{outputFile} argument is optional. If it is not given and an output pipe is not selected, then the input file will be replaced.
  \item \textbf{see also:}
    \begin{itemize}
      \item \progref{sddssort}
    \end{itemize}
  \item \textbf{author:} H. Shang, ANL/APS.
\end{sddsprog}

%\begin{latexonly}
\newpage
%\end{latexonly}
\subsection{sddssplinefit}
\label{sddssplinefit}

\begin{itemize}
\item {\bf description:} 
{\tt sddssplinefit} fits splines to column data using the gsl library. The fits are of the form $y = \Sigma_i \{ A[i] * B(x-xOffset, i)\}$, where B(x,i) is the ith basis spline function evaluated at x.  {\tt sddssplinefit} internally computes the A[i], writes the $y$ in the output file and estimates of the errors in the values. One can specify the order of the spline, and the number of breakpoints (or alternatively the number of coefficients).  The options are very similar to those for {\tt sddsmpfit}.
\item {\bf synopsis:} 
\begin{flushleft}{\tt
usage: sddsmsplinefit [-pipe=[input][,output]] [<inputfile>] [<outputfile>]
  -independent=<xName> -dependent=<yname1-wildcard>[,<yname2-wildcard>...]
  [-sigmaIndependent=<xSigma>] [-sigmaDependent=<ySigmaFormatString>]
  [-order=<number>] [-coefficients=<number>] [-breakpoints=<number>]
  [-xOffset=value] [-xFactor=value]
  [-sigmas=<value>,{absolute | fractional}] 
  [-modifySigmas] [-generateSigmas[={keepLargest, keepSmallest}]]
  [-sparse=<interval>] [-range=<lower>,<upper>[,fitOnly]]
  [-normalize[=<termNumber>]] [-verbose]
  [-evaluate=<filename>[,begin=<value>][,end=<value>][,number=<integer>]]
  [-fitLabelFormat=<sprintf-string>] [-infoFile=<filename>]
  [-copyParameters]
}\end{flushleft}
\item {\bf files:}
{\em inputFile} is an SDDS file containing columns of data to be fit. 
If it contains multiple pages, they are processed separately. 
{\em outputFile} is an SDDS file containing one page for each page of {\em inputFile}. 
It contains columns of the independent and dependent variable data, plus columns for error bars (``sigmas'') as appropriate. 
The values of the fit and of the residuals are in a columns named {\em yName}{\tt Fit} and {\em yName}{\tt Residual}.
In addition, various parameters having names beginning with {\em yName} are created that give reduced chi-squared, slope, intercept, and so on.

\item {\bf switches:}
    \begin{itemize}
    \item {\tt -pipe[=input][,output]} --- The standard SDDS Toolkit pipe option.
    \item {\tt -evaluate={\em filename}[,begin={\em value}][,end={\em value}][,number={\em integer}]} ---
        Specifies creation of an SDDS file called {\em filename} containing points from evaluation of the
        fit.  The fit is normally evaluated over the range of the input data; this may be changed using
        the {\tt begin} and {\tt end} qualifiers, though the spline routines do not allow exceeding the
        range of the fit. Normally, the number of points at which the fit is evaluated is the number of 
        points in the input data; this may be changed using the {\tt number} qualifier.
    \item {\tt infoFile={\em filename}} --- Specifies creation of an SDDS file containing results of
        the fits in columns. Under construction.
    \item {\tt -order={\em number}} --- Specifies the order of the spline, which is one more than the order
        of the local polynomials. For example, order 2 would fit straight 
    \item {\tt [-breakpoints={\em number}]} --- The number of splines pieces the data will be split into for fitting.
    \item {\tt [-coefficients={\em number}]} --- The number of coefficiencts used in hte spline fitting, which 
        essentially controls the number of splines, since the number of coefficients = number of splines - 2
        plus the order.
    \item {\tt -xOffset={\em value}}, {\tt -xFactor={\em value}} --- Specify offseting and scaling of the independent
        data prior to fitting.  The transformation is ${\rm x \rightarrow (x - xOffset)/Factor}$.  This feature can
        be used to make a fit about a point other than x=0, or to scale the data to make high-order fits more
        accurate. The benefits for spline fitting is unknown, as it is left-over from polynomial fitting. 
    \item {\tt -sparse={\em interval}} --- Specifies sparsing of the input data prior to fitting.  This can greatly
        speed computations when the number of data points is large.
    \item {\tt -range={\em lower},{\em upper}} --- Specifies the range of independent variable over which to do fitting.
    \item {\tt -verbose} --- Specifies that the results of the fit be printed to the standard error output.
    \end{itemize}
\item {\bf see also:}
    \begin{itemize}
    \item \hyperref[exampleData]{Data for Examples}
    \item \progref{sddspfit}
    \item \progref{sddsmpfit}
    \item \progref{sddsoutlier}
    \end{itemize}
\item {\bf author:} M. Borland, ANL/APS.
\end{itemize}


%\begin{latexonly}
\newpage
%\end{latexonly}
\subsection{sddssplit}
\label{sddssplit}

\begin{itemize}
\item {\bf description:} \hspace*{1mm}\\
{\tt sddssplit} breaks up an SDDS file into one or more separate files, each containing only a single
page of data.  This may be useful in those instances where a tool or program only processes the first
page of a file.
\item {\bf examples:} \\
Split a Twiss parameter file into separate files:
\begin{flushleft}{\tt
sddssplit APS.twi
}\end{flushleft}
\item {\bf synopsis:} \\
\begin{flushleft}{\tt
sddssplit \{-pipe[=input] | {\em inputFile}\}  [\{-binary | -ascii\}] 
[-digits={\em number}] [-rootname={\em string}] [-extension={\em string}]
[-nameParameter={\em paramName}]
[-firstPage={\em number}] [-lastPage={\em number}] [-interval={\em number}] 
}\end{flushleft}
\item {\bf files:}
{\em inputFile} is an SDDS file to be split.  By default, the output files are created by appending the page number
to a ``rootname'' and adding an extension.  That is, the output files have names 
{\em rootname}{\em Page}.{\em extension}.  The default rootname is the name of {\em inputFile}, while
the default extension is ``sdds''.  By default, {\em Page} is printed using ``%03ld'' format.
less the extension.  
\item {\bf switches:}
    \begin{itemize}
    \item {\tt -pipe[=input][,output]} --- The standard SDDS Toolkit pipe option.
    \item {\tt -binary}, {\tt -ascii} --- Specifies binary or ASCII output, with binary being the default.
    \item {\tt -digits={\em number}} --- Specifies the number of digits to be used in creating filenames.
        Leading zeros are included.
    \item {\tt -rootname={\em string}} --- Specifies the rootname to be used in creating filenames.
    \item {\tt -extension={\em string}} --- Specifies the extension to be used in creating filenames.
    \item {\tt -nameParameter={\em paramName}} --- Specifies that instead of composing names for the output
        files, {\tt sddssplit} take the names from the string parameter {\em paramName} in the input file.
        This provides a limited capability to retrieve the original files from a file made with {\tt sddscombine}.
        Note that if the named parameter takes the same value on two pages, the file created for the first
        of the pages will be overwritten.
    \item {\tt -firstPage={\em number}} --- Specifies the first page of data to use.
    \item {\tt -lastPage={\em number}} --- Specifies the last page of data to use.
    \item {\tt -interval={\em number}} --- Specifies the interval between pages that are used.
    \end{itemize}
\item {\bf see also:}
    \begin{itemize}
    \item \progref{sddsbreak}
    \item \progref{sddscombine}
    \end{itemize}
\item {\bf author:} M. Borland, ANL/APS.
\end{itemize}


%
%\begin{latexonly}
\newpage
%\end{latexonly}
\subsection{sddsspotanalysis}
\label{sddsspotanalysis}

\begin{itemize}
\item {\bf description:} Used to locate and give details about spots in multi-column SDDS image files.
\item {\bf example:} 
\begin{flushleft}{\tt
sddsspotanalysis image.input image.output 
}\end{flushleft}
\item {\bf synopsis:}
\begin{flushleft}{\tt
sddsspotanalysis [{\em Inputfile}] [{\em Outputfile}] [-pipe[=in][,out]] 
[-ROI=[\{xy\}\{01\}value={\em value}][,\{xy\}{01}parameter={\em name}]]
[-spotROIsize=[\{xy\}value={\em value}][,\{xy\}parameter={\em name}]]
[-centerOn=\{\{xy\}centroid|\{xy\}peak\}]
[-imageColumns={\em listOfNames}] 
[-singleSpot]
[-levels=[intensity={\em integer}][,saturation={\em integer}]]
[-blankOut=[\{xy\}\{01\}value={\em value}][,\{xy\}\{01\}parameter={\em name}]]
[-sizeLines=[\{xy\}value={\em value}][,\{xy\}parameter={\em name}]]
[-background=[halfwidth={\em value}][,symmetric][,antihalo][,antiloner]]
[-despike=[=neighbors={\em integer}][,passes={\em integer}][,averageOf={\em integer}][,threshold={\em value}]]
[-spotImage={\em filename}]
}\end{flushleft}
\item {\bf files: }

{\em Inputfile} is the multi-column SDDS input image file.

{\em Outputfile} contains spot property information.

\item {\bf switches:}
    \begin{itemize}
    \item {\tt -pipe[=in][,out]} --- The standard SDDS Toolkit pipe option.
    \item {\tt -ROI=[\{xy\}\{01\}value={\em value}][,\{xy\}{01}parameter={\em name}]} --- Used to give the region of interest in pixel units. All data outside this region is ignored.
    \item {\tt -spotROIsize=[\{xy\}value={\em value}][,\{xy\}parameter={\em name}]} --- Used to give the full size in pixel units of the region of interest (ROI) around the spot. This ROI is used for computing spot properties.
    \item {\tt -centerOn=\{\{xy\}centroid|\{xy\}peak\}} --- Choose whether to center the spot analysis region on the peak value or the centroid value for x and y.
    \item {\tt -imageColumns={\em listOfNames}} --- Give a list of names of columns containing image data. Wildcards may be used.
    \item {\tt -singleSpot} --- Used to eliminate multiple spots.  All pixels not connected to the most intense pixel by a path through nonzero pixels are set to zero.
    \item {\tt -levels=[intensity={\em integer}][,saturation={\em integer}]} --- Use intensity to give the number of intensity levels in the data; 256 is the default, with values from 0 to 255. Use saturation to give the level at which saturation is considered to occur; 254 is the default.
    \item {\tt -sizeLines=[\{xy\}value={\em value}][,\{xy\}parameter={\em name}]} --- Specify the number of lines to use for analysis of the beam size.  The default is 3.
    \item {\tt -background=[halfwidth={\em value}][,symmetric][,antihalo][,antiloner]} --- Use halfwidth to specify the number of intensity levels on either side of the most populous intensity to include for computation of the background. The default is 3.  Setting this to zero means that the background level is equal to the intensity of the most populous level. If symmetric is given, then pixels within this width of 0 after background subtraction are set to zero if the intensity of all but one adjacent pixel is less than this level; this symmetrizes the background removal and avoids leaving positive noise.  If antihalo is given, then each line of the spot ROI is scanned from the outer edge toward the center.  Pixels within the halfwidth of 0 after background subtraction are set to zero, until the first pixel is reached which fails this criterion, after which the next line is processed. If antiloner is given, then after background subtraction, the program removes any pixel that is surrounded by all zero pixels.
    \item {\tt -despike=[=neighbors={\em integer}][,passes={\em integer}][,averageOf={\em integer}][,threshold={\em value}]} --- Enter despiking parameters for smoothing the data for purposes of finding the spot center.  If this isn't used then the program may pick a noise pixel as the spot center. Default equivalent to -despike=neighbors=4,passes=2,averageOf=4.
    \item {\tt -spotImage={\em filename}} --- Provide the name of a file to which images of the spots will be written. The file can be plotted with sddscontour.
    \end{itemize}
\item {\bf see also:}
    \begin{itemize}
    \item \progref{sddscongen}
    \item \progref{sddscontour}
    \item \progref{sddsimageconvert}
    \item \progref{sddsimageprofiles}
    \end{itemize}
\item {\bf author:} R. Soliday, ANL/APS.
\end{itemize}


\begin{sddsprog}{sddstdrpeeling}
  \item \textbf{description:} \verb|sddstdrpeeling| processes time-domain reflectometry (TDR) data using a recursive algorithm to determine the impedance of a nonuniform transmission line.
  \item \textbf{examples:}
    \begin{verbatim}
sddstdrpeeling <inputfile> <outputfile> -columns=Time,Values -inputVoltage=120 -z0=50
    \end{verbatim}
  \item \textbf{synopsis:}
    \begin{verbatim}
sddstdrpeeling [input] [output] [-pipe=[input][,output]]
  -col=time-col,data-column
  [-inputVoltage=value|@<parameter>]
  [-z0=value]
    \end{verbatim}
  \item \textbf{files:} The input file contains both the time and data columns. The output file contains the additional column \verb|PeeledImpedance|.
  \item \textbf{switches:}
    \begin{itemize}
      \item \verb|-pipe=[input][,output]| --- Standard SDDS pipe options for reading/writing files from stdin/stdout.
      \item \verb|-col=time-col,data-column| --- The names of the time and data columns.
      \item \verb|-inputVoltage=value| or \verb|@<parameter>| --- The input voltage in volts of TDR (default 0.2 V).
      \item \verb|-z0=value| --- The line impedance (default 50 ohms).
    \end{itemize}
  \item \textbf{author:} H. Shang, R. Soliday, ANL/APS.
\end{sddsprog}


%\begin{latexonly} 
\newpage 
%\end{latexonly} 
 
\subsection{sddstimeconvert} 
\label{sddstimeconvert} 
 
\begin{itemize} 
\item {\bf description:} 
\verb|sddstimeconvert| does conversions between calendar time in 
terms of (for example) day, month, and year, and ``time-since-epoch''. 
The latter is the time in seconds since a system-defined reference 
time (e.g., 0:00 on January 1, 1970). 
 
\item {\bf examples:}  
Convert column data broken down as day, month, year, and hour to seconds-since-epoch:
\begin{flushleft}{\tt  
sddstimeconvert input.sdds output.sdds 
 -epoch=column,Time,year=TheYear,month=TheMonth,day=DayOfMonth,hour=HourOfDay
} 
\end{flushleft} 
where \verb|TheYear|, \verb|TheMonth|, \verb|DayOfMonth|, and \verb|HourOfDay|
are the names of the columns in the input file and \verb|Time| is the 
column to be created containing the time-since-epoch.

\item {\bf synopsis:}  
 
\begin{flushleft}{\tt 
sddstimeconvert [{\em inputFile}] [{\em outputFile}] [-pipe[=input][,output]] 
[-breakdown=\{column | parameter\},{\em timeName}[,year={\em newName}]
\hspace*{5mm}[,julianDay={\em newName}][,month={\em newName}][,day={\em newName}][,hour={\em newName}][,text={\em newName}]]
[-epoch=\{column | parameter\},{\em newName},year={\em name} \\ \hspace*{5mm}
[,julianDay={\em name} | month={\em name},day={\em name}],hour={\em name}]
}\end{flushleft} 
 
\item {\bf switches:} 
    \begin{itemize} 
    \item {\tt -pipe[=input][,output]} --- The standard SDDS Toolkit pipe option. 
    \item {\tt -breakdown=\{column | parameter\},{\em timeName}[,year={\em newName}]}
        {\tt [,julianDay={\em newName}][,month={\em newName}][,day={\em newName}][,hour={\em newName}]}
        {\tt [,text={\em newName}]} ---  
        Specifies conversion of the column or parameter data named
        {\em timeName} to year, Julian day, month, day, hour, and/or a text string.  
        {\em timeName} contains the time expressed as
        seconds-since-epoch.  Any number of these options may be given. 
    \item {\tt -epoch=\{column | parameter\},{\em newName},year={\em name},}\\ {\tt [julianDay={\em name} | month={\em name},day={\em name}],hour={\em name}} ---  
        Specifies conversion of column or parameter data given as year, Julian day or month/day, and hour
to seconds-since-epoch, with the result being placed in {\em newName}.  
    \end{itemize} 

\item {\bf notes:}
The hour data as used or created by \verb|sddstimeconvert| contains the floating-point time-of-day in hours.
That is, the minutes and seconds are folded into the hour value.

Year values must be the full four-digit year; e.g., year 99 is {\em not} 1999, but rather 99 AD.

\item {\bf author:} M. Borland, ANL/APS. 
\end{itemize} 

% $Log: not supported by cvs2svn $
% Revision 1.5  1996/04/29  22:58:48  emery
% Minor spelling changes to make capitalization consistent.
%
% Revision 1.4  1996/04/28  22:58:12  emery
% Corrected the font of the option or file specification.
%
%
% Template for making SDDS Toolkit manual entries.
%
%\begin{latexonly}
\newpage
%\end{latexonly}

%
% Substitute the program name for <programName>
%
\subsection{sddstranspose}
\label{sddstranspose}

\begin{itemize}
\item {\bf description:}
%
% Insert text of description (typicall a paragraph) here.
%
\verb+sddstranspose+ views the numerical tabular data of the input file
as though it formed a matrix, and produces an output
file with data corresponding to the transpose of the input
file matrix. In other words, the columns of tabular data of the input
file become rows in the output file. String column data
are not transposed but are stored as string parameters in the output file.
Operating on the output file with a second \verb+sddstranpose+ command essentially recovers
the original input file.

The column names for the output file are generated either from the data in
a selected string column in the input file,
from the value of the command line option -root,
or from an internal default.

The column names of the input file are collected and made into
a string column in the output file.

\item {\bf examples:} 
%
% Insert text of examples in this section.  Examples should be simple and
% should be preceeded by a brief description.  Wrap the commands for each
% example in the following construct:
% 
%
The data in file LTP.R12 (matrix of $R_{12}$'s in a beamline called LTP, say)
is transposed to give file LTP.R12.trans:
\begin{flushleft}{\tt
sddstranspose LTP.R12 LTP.R12.trans
}\end{flushleft}
\item {\bf synopsis:} 
%
% Insert usage message here:
%
\begin{flushleft}{\tt
sddstranspose [-pipe=[input][,output]] {\em inputFile} {\em outputFile}
     [-oldColumnNames={\em string}] [\{-root={\em string} [-digits={\em integer}] | 
     -newColumnNames={\em column}\}] 
     [-symbol={\em string}] [-ascii] [-verbose]
}\end{flushleft}
\item {\bf files:}
% Describe the files that are used and produced
The input file contains the data for the matrix to be transposed. The output file
contains the data for the transposed matrix. If only one file is specified,
then the input file is overwritten by the output.


\item {\bf switches:}
%
% Describe the switches that are available
%
    \begin{itemize}
    \item {\tt  -pipe[=input][,output]} --- The standard SDDS Toolkit pipe option.
    \item {\tt  -oldColumnNames={\em string}} --- 
        A string column of name {\tt {\em string}} is created in the output file, containing
        the column names of the input files as string data.
        If this option is not present, then the default name of ``OldColumnNames''
        is used for the string column.
    \item {\tt  -root={\em string}} ---
        A string used to generate columns names for the output file data. 
        The first data column is named ``{\tt {\em string}000}'',
        the second, ``{\tt {\em string}001}'', etc. If the input file has only one
        row, the the root name alone (with no digits following) is used for the 
        column name.
    \item {\tt  -digits={\em integer}} --- minimum number of digits used in the number 
        appended to {\tt {\em root}} of the output file column names. (Default value is 3).
    \item {\tt  -newColumnNames={\em string}} --- Specifies a string column
        of the input file which will be used to define column names
        of the output file.
    \item {\tt  -symbol={\em string}} --- The string for the symbol field of data column definitions.
    \item {\tt  -ascii}  --- Produces an output in ascii mode. Default is binary.
    \item {\tt  -verbose} --- Prints out incidental information to stderr.
    \end{itemize}
%\item {\bf see also:}
%    \begin{itemize}
%
% Insert references to other programs by duplicating this line and 
% replacing {\em prog} with the program to be referenced:
%
%    \item \progref{<prog>}
%    \end{itemize}
%
% Insert your name and affiliation after the '}'
%
\item {\bf author: L. Emery } ANL
\end{itemize}




\begin{sddsprog}{sddsunwrap}
  \item \textbf{description:}
    \verb|sddsunwrap| looks for discontinuities larger than a threshold in a set of data.
    After each discontinuity it adds the appropriate multiple of the modulo to the data set.
  \item \textbf{examples:}
    \begin{verbatim}
    sddsunwrap inputFile outputFile -columns=phase
    \end{verbatim}
  \item \textbf{synopsis:}
    \begin{verbatim}
    sddsunwrap [-pipe=[input][,output]] inputFile outputFile
      [-columns=columnName[,...]] [-threshold=value] [-modulo=value]
    \end{verbatim}
  \item \textbf{switches:}
    \begin{itemize}
      \item \verb|-pipe[=input][,output]| --- The standard SDDS Toolkit pipe option.
      \item \verb|-threshold=\emph{value}| --- Specifies the discontinuity threshold used to identify a wrap in the data, default is PI.
      \item \verb|-modulo=\emph{value}| --- Specifies the value used to unwrap the data, default is 2*PI.
      \item \verb|-columns=\emph{columnName}[,\emph{columnName}...]| --- Specifies the names of the columns to unwrap; wildcards are accepted. If not specified, all numerical columns in the input file are unwrapped. The output column is named as \verb|Unwrap<inputColumn>|.
    \end{itemize}
  \item \textbf{files:}
    The input file contains the column data to be unwrapped. The unwrapped data is saved into \verb|Unwrap<inputColumn>| column in the output file.
  \item \textbf{author:} H. Shang, ANL/APS.
\end{sddsprog}


%
%\begin{latexonly}
\newpage
%\end{latexonly}

%
% Substitute the program name for <programName>
%
\subsection{sddsvslopes}
\label{sddsvslopes}

\begin{itemize}
\item {\bf description:}
%
% Insert text of description (typicall a paragraph) here.
%
\verb|sddsvslopes| makes straight line fits of vectorized data
in the input file with respect to a selected parameter used as independent variable.
The simplest example of vectorized data is a data set with one parameter and two columns,
one string column of rootnames and one numerical column of data.
The fitting is looped over rows across all the 
data sets in the input file (using a selected parameter as the
independent vairable). The output file contains
vectorized slopes and intercepts data for each column specified in the input file.

\item {\bf examples:} 
%
% Insert text of examples in this section.  Examples should be simple and
% should be preceeded by a brief description.  Wrap the commands for each
% example in the following construct:
% 
%
The file corrector.sdds contains vectorized  beam position monitor (bpm)
readbacks as a function of corrector setting. The defined parameter is CorrectorSetpoint.
The defined columns are Rootname and x. Each row of the data set correspond to a different
bpm.
The bpm response to the corrector setpoints are calculated with
\begin{flushleft}{\tt
sddsvslopes corrector.sdds corrector.vslopes -independentVariable=CorrectorSetpoint 
  -columns=x
}\end{flushleft}
\item {\bf synopsis:} 
%
% Insert usage message here:
%
\begin{flushleft}{\tt
sddsvslopes [-pipe=[input][,output]] {\em inputFile} {\em outputFile} 
        -independentVariable={\em parametername}
        [-columns={\em listOfNames}] [-excludeColumns={\em listOfNames}] 
        [-sigma[=generate]] [-verbose]
}\end{flushleft}
\item {\bf files:}
% Describe the files that are used and produced
The input file contains the tabular data for fitting. The column Rootname must be present.

The output file contains one data set of vectorized slopes and intercept data.
The Rootname and Index columns from the input file is transfered to the output file.
In the column Index doesn't exist in the input file, then it is created in the output
file anyway.
The column names are {\em name}\verb|Slope|, and {\em name}\verb|Intercept| 
where {\em name} is the name of the fitted data.
If only one file is specified, then the input file is overwritten by the output.
A string parameter called \verb|IndenpendentVariable| is defined containing the name of the indepedent variable.

\item {\bf switches:}
%
% Describe the switches that are available
%
    \begin{itemize}
%
%   \item {\tt -pipe[=input][,output]} --- The standard SDDS Toolkit pipe option.
%
    \item {\tt  -pipe[=input][,output]} --- The standard SDDS Toolkit pipe option.
    \item {\tt  -independentVariable={\em parametername} }
        --- name of independent variable (default is the first valid column)
    \item {\tt  -columns={\em listOfNames}}   
        ---  columns to be individually paired with independentVariable for straight line fitting
    \item {\tt  -excludeColumns={\em listOfNames}}  ---    columns to exclude from fitting
    \item {\tt  -sigma[=generate]}  
        ---   calculates errors by interpreting column names 
        {\em name}\verb|Sigma| or \verb|Sigma|{\em name} as
        sigma of column {\em name}. If these columns don't exist
        then the program generates a common sigma from the residual of a first fit,
        and refits with these sigmas. If option -sigma=generate is given,
        then sigmas are generated from the residual of a first fit for all columns,
        irrespective of the presence of columns {\em name}\verb|Sigma| or \verb|Sigma|{\em name}.
    \item {\tt  -ascii }    ---         make output file in ascii mode (binary is the default)
    \item {\tt  -verbose }  ---         prints some output to stderr

    \end{itemize}
%\item {\bf see also:}
%    \begin{itemize}
%
% Insert references to other programs by duplicating this line and 
% replacing <prog> with the program to be referenced:
%
%    \item \progref{<prog>}
%    \end{itemize}
%
% Insert your name and affiliation after the '}'
%
\item {\bf author: L. Emery } ANL
\end{itemize}



%\begin{latexonly}
\newpage
%\end{latexonly}
\subsection{sddsxra}
\label{sddsxra}

\begin{itemize}
\item {\bf description:}
\verb|sddsxra| reads an NIST x-ray data base to obtain x-ray interaction constants for a list of photon energies provided by the user. Depending on the mode selected, the output data may include x-ray cross sections, refractive indices, atomic scattering factors, mirror reflectivity, or absorption / transmission coefficients of a film. 
\item {\bf synopsis:} 
\begin{flushleft}{\tt
sddsxra <inputFile> <outputFile> -energy=column=<colname>|begin=<value>,end=<value>,points=<integer> 
-mode=<number> -target=material=<string>|formula=<string>,thickness=<value>,density=<value>,angle=<value>
}\end{flushleft}
\item {\bf examples} List total photo cross sections for silicon nitride from photon energy 10 keV to 50 keV in 400 points: 
\begin{flushleft}{\tt
 sddsxra SiliconNitride.sdds -mode=0 -energy=begin=10000,end=50000,points=400 -target=formula=Si3N4,density=3.44 
}\end{flushleft}
\begin{flushleft}{\tt
 sddsplot -graph=line,vary -legend -mode=y=log,y=special -col=PhotonEnergy,{TotalCS,PhotoCS,CoherentCS,IncoherentCS} SiliconNitride.sdds 
}\end{flushleft}
\item {\bf files:}
        {\em inputFile} is an SDDS file containing photon energy data in a column or a parameter. 
        {\em outputFile} is an SDDS file containing the photon energy data in a column and x-ray interaction constants. 
\item {\bf switches:}
    \begin{itemize}
    \item Source of photon energy: only one of the following may be used. 
         \begin{itemize}
          \item {\tt -energy={column | parameter},{\em columnName}}---
                  Specifies the column or parameter from the inputFile containing photon energy values. 
          \item {\tt -energy=begin={\em start},end={\em end},points={\em points}}---
                  The photon energy array is from start (eV) to end (eV) in npts data points. No input file is needed or used. 
          \item {\tt -energy=specified={e1[,e2,...]}} ---
                  The photon energy array is given by a list e1 (eV), e2 (eV), .... No input file is needed or used. 
         \end{itemize}
    \item Mode of calculation: 
        \begin{itemize}
          \item {\tt -mode=0}  or {\tt -mode=10} ---
            Retrieve interaction constants of a compound (mode 0) or an element (mode 10). The output contains coulmns of photon energy (PhotonEnergy) in eV, photo-absorption cross sections (PhotoCS), coherent scattering cross sections (CoherentCS), incoherent scattering cross sections (IncoherentCS) and total photo cross sections (TotalCS) in units of cm\^{2}/g. 
          \item {\tt -mode=1} ---
            Retrieve index of refraction of target material. The output contains coulmns of photon energy (PhotonEnergy) in eV, and real and imaginary part of the refractive index (RefracIndex\_Re, RefracIndex\_Im). 
          \item {\tt -mode=2}  or {\tt -mode=12} ---
            Calculate x-ray attenuation through a compound (mode 2) or an elemental (mode 12) film target. The output contains coulmns of photon energy (PhotonEnergy) in eV, absorption coefficient (Absorption) and transmission coefficient (Transmission) of the target. 
           \item {\tt -mode=4}  or {\tt -mode=14} ---
            Calculate total electron yield or quantum efficiency of a compound (mode 4) or an elemental (mode 14) film target. The output contains coulmns of photon energy (PhotonEnergy) in eV, total electron yield from the front surface (TotalElectronYieldFront), from the back surface (TotalElectronYieldBack), and from both surfaces (TotalElectronYield). This calculation is based on Henke model and the user needs to provide the material's TEY coefficient (teyEfficiency) except for Au. 
            \item {\tt -mode=6} ---
              Calculate reflectivity of a compound mirror. The output contains coulmns of photon energy (PhotonEnergy) in eV and reflectivity from the front surface (Reflectivity). 
            \item {\tt -mode=11} ---
              Retrieve atomic scattering factor of the target element. The output contains coulmns of photon energy (PhotonEnergy) in eV, and real and imaginary part of the scattering factor (F1, F2). 
             \item {\tt -mode=20} ---
               Retrieve Kissel partial photoelectric cross sections of the target element. The output contains coulmns of photon energy (PhotonEnergy) in eV, photo-absorption cross sections (PhotoCS), coherent scattering cross sections (CoherentCS), incoherent scattering cross sections (IncoherentCS) and total photo cross sections (TotalCS) in cm\^{2}/g.  
          \end{itemize}
     \item Mode of calculation: 
        \begin{itemize}
          \item {\tt -target=material={\em name}[,formulat={\em formula}][,density={\em dd}][,thickness={\em tt}][,theta={\em theta}][,teyEfficiency={\em tey}]} ---
            chemical formula, such as C, target density in g/cm\^{3}, thickness in mm, angle of incidence in degrees, measured from the front surface normal vector of the film, and total electron yield (TEY) efficiency constant in g/cm\^{2}. For materials listed in the build-in material property table, the default density may be provided by the program, but can be overwritten by the value supplied via this option. 
          \item {\tt -target=formula={\em formula}[,density={\em dd}][,thickness={\em tt}][,theta={\em theta}][,teyEfficiency={\em tey}]} ---
            Specifies target composition only with chemical formula formula. The formula will also be used as material name if the material is not listed in the material property table. 
          \item {\tt -target=material={\em name}[,density={\em dd}][,thickness={\em tt}][,theta={\em theta}][,teyEfficiency={\em tey}] } ---
            Specifies target composition with common name name. This is acceptible only for material listed in the build-in material property table. 
        \end{itemize}
    \item {Miscellaneous:}
        \begin{itemize}
          \item {\tt -polarization={\em value} } ---
            Specifies the polarization of the incoming photon for mode 6: 0 = unpolarized (default), +1 = S-polarized, and -1 = P-polarized. 
          \item {\tt --shell=s1[,s2,...,sn] } ---
            Specifies shell name for Kissel partial photoelectric cross sections, where sx = K, L1, L2, L3,... 
        \end{itemize}
    \end{itemize}
\item {\bf see also:}
    \begin{itemize}
    \item A. Brunetti, et al, A library for x-ray matter interaction cross sections for x-ray fluorescence applications, an update, (2011). 
    \item B. L. Henke, et al, The characterization of x-ray photocathode in the 0.1-10-keV photon energy region, J. Appl. Phys. 1509 (52) 1981. 
    \item B. Yang and H. Shang, SDDS-compatible programs for modeling x-ray absorption in film targets, APS/ASD/DIAG Technote DIAG-TN-2013-004, 2013. 
    \end{itemize}
\item {\bf author:}B.Yang, H. Shang, ANL/APS.
\end{itemize}


\begin{sddsprog}{sddsxref}
  \item \textbf{description:}
  \verb|sddsxref| creates a new data set by adding selected rows from one data set to another.
  Rows are selected by matching string or numeric values in a column present in both input data sets.
  Users may specify which columns of the second data set to take or leave and may transfer
  parameter and array data. \verb|sddsmxref| is a variant that permits multiple \verb|-match|
  and \verb|-equate| options.
  \item \textbf{examples:}
    \begin{verbatim}
sddsxref input.sdds xRef.sdds output.sdds -match=Name -take=Value
sddsxref data.sdds xRef.sdds -equate=Index -reuse=rows -fillIn
    \end{verbatim}
  \item \textbf{synopsis:}
    \begin{verbatim}
sddsxref [-pipe[=input][,output]] [input] [xRefFile] [output]
         [-equate=columnName | -match=columnName]
         [-reuse[=rows][,page]] [-fillIn]
         [-take=columnName,...] [-leave=columnName,...]
         [-transfer={parameter | array},name[,name...]]
         [-ifis={column | parameter | array},name[,name...]]
         [-ifnot={column | parameter | array},name[,name...]]
    \end{verbatim}
  \item \textbf{files:}
    \emph{input} is the data set to which data is being added. \emph{xRefFile} is the data set from
    which data is taken. Warning: if \emph{output} is not given and if \verb|-pipe=out| is not
    specified, \emph{input} is overwritten. For pipe input, the first file listed is taken to be
    \emph{xRefFile}. For pipe input and output, the only file listed is \emph{xRefFile}.
  \item \textbf{switches:}
    \begin{itemize}
      \item \verb|-equate=columnName|, \verb|-match=columnName| --- These options specify the name of a column
        that exists in both \emph{input} and \emph{xRefFile}. For \verb|match| the column must contain
        string data, while for \verb|equate| the column must contain numeric data. For each row in
        \emph{input}, \verb|sddsxref| searches \emph{xRefFile} to find the first row for which the
        \verb|match| column is identical or the \verb|equate| column is equal, as appropriate. If neither
        option is given, rows are taken sequentially from \emph{xRefFile} for each row of \emph{input}.
      \item \verb|-reuse[=rows][,page]| --- By default, each row from \emph{xRefFile} is matched to one row in
        \emph{input}. If \verb|-reuse=rows| is given, each row from \emph{xRefFile} may be matched to any
        number of rows in \emph{input}. Also by default, each page of \emph{input} is matched with the
        corresponding page of \emph{xRefFile}. If \verb|-reuse=page| is given, each page of \emph{input}
        is matched anew to the first page of \emph{xRefFile}. The two qualifiers may be given together.
      \item \verb|-fillIn| --- Normally if there is no match in \emph{xRefFile} for a row in \emph{input},
        that row will not appear in \emph{output}. If \verb|-fillIn| is given, the row will appear, but the
        values for the columns that are being transferred from \emph{xRefFile} will be filled with zeros
        and empty strings, as appropriate.
      \item \verb|-take=columnName,...|, \verb|-leave=columnName,...| --- These options specify which columns
        of \emph{xRefFile} to extract from a matching or equal row for addition to a row of \emph{input}.
        Wildcards may be given in the column names. By default, all columns not in \emph{input} are taken.
        If \verb|take| is employed, only the named columns will be taken. In either case, no column
        specified under \verb|leave| will be taken. \verb|-leave=*| causes no columns to be taken.
      \item \verb!-transfer={parameter | array},name[,name...]! --- This option, which may be given
        multiple times, specifies the names of parameters and arrays to be transferred. Wildcards are not
        presently supported in this option.
      \item \verb!-ifis={column | parameter | array},name[,name...]!, \verb!-ifnot={column | parameter | array},name[,name...]! --- These options allow conditional execution. If any
        column named under \verb|ifis| is not present, execution aborts. If any column named under
        \verb|ifnot| is present, execution aborts.
    \end{itemize}
  \item \textbf{see also:}
    \begin{itemize}
      \item \hyperref[exampleData]{Data for Examples}
      \item \progref{sddsselect}
      \item \progref{sddsmxref}
    \end{itemize}
  \item \textbf{author:} M. Borland, L. Emery, H. Shang, R. Soliday ANL/APS.
\end{sddsprog}

\begin{sddsprog}{sddszerofind}
  \item \textbf{description:}
    \verb|sddszerofind| finds the locations of zeroes in a single column of an SDDS file. This is done by finding successive rows for which a sign change occurs in the "dependent column", or any row for which an exact zero is present in this column. For each of the "independent columns", the location of the zero is determined by linear interpolation. Hence, the program is really interpolating multiple columns at locations of zeros in a single column. This single column is in a sense being looked at as a function of each of the interpolated columns.
  \item \textbf{examples:}
    Find zeroes of a Bessel function, $\mathrm{J_0(z)}$:
    \begin{verbatim}
sddszerofind J0.sdds J0.zero -zeroesOf=J0 -columns=z
    \end{verbatim}
    Find zeroes of a Bessel function, $\mathrm{J_0(z)}$, and simultaneously interpolate $\mathrm{J_1(z)}$ at the zero locations:
    \begin{verbatim}
sddszerofind J0.sdds J0.zero -zeroesOf=J0 -columns=z,J1
    \end{verbatim}
    (This isn't the most accurate way to interpolate $\mathrm{J_1(z)}$, of course.)
  \item \textbf{synopsis:}
    \begin{verbatim}
sddszerofind [-pipe=[input][,output]] [inputfile] [outputfile]
  -zeroesOf=columnName [-columns=columnNames]
  [-offset=value] [-slopeOutput]
  [-majorOrder=row|column]
    \end{verbatim}
  \item \textbf{switches:}
    \begin{itemize}
      \item \verb|-pipe[=input][,output]| --- The standard SDDS Toolkit pipe option.
      \item \verb|-zeroesOf=columnName| --- Specifies the name of the dependent quantity for which zeroes will be found.
      \item \verb|-columns=columnNames| --- Specifies the names of the independent quantities for which zero locations will be interpolated. Generally, there is only one of these. \emph{columnNames} is a comma-separated list of optionally wildcarded names.
      \item \verb|-offset=value| --- Specifies a value to add to the values of the \verb|-zeroesOf| column prior to finding the zeroes.
      \item \verb|-slopeOutput| --- Specifies that additional columns will be created containing the slopes of the dependent quantity as a function of each independent quantity. This can be useful, for example, if one wants to pick out only positive-going zero-crossings.
      \item \verb|-majorOrder=row| or \verb|-majorOrder=column| --- Specifies the organization of data in the output file.
    \end{itemize}
  \item \textbf{files:}
    \emph{inputFile} contains the data to be searched for zeroes. \emph{outputFile} contains columns for each of the independent quantities and a column for the dependent quantity. Normally, each dependent quantity is represented by a single column of the same name. If output of slopes is requested, additional columns will be present, having names of the form {\tt columnNameSlope}.

    If \emph{inputFile} contains multiple pages, each is treated separately and is delivered to a separate page of \emph{outputFile}.
  \item \textbf{see also:}
    \begin{itemize}
      \item \progref{sddsinterp}
    \end{itemize}
  \item \textbf{author:} M. Borland, ANL/APS.
\end{sddsprog}

%\begin{latexonly}
\newpage
%\end{latexonly}
\subsection{SDDS Editing}
\label{SDDSediting}

This manual page does not describe a program, but rather a facility that is common to several
programs.  In particular, several SDDS programs use a common syntax for specifying editing of
string data.  The editing commands for these programs are composed of a series of subcommands of
the form
[{\em count}]{\tt commandLetter}[{\em commandSpecificData}]
As indicated, the {\em count} and {\em commandSpecificData} are optional.

The commands are as follows:
\begin{itemize}
\item [] [{\em n}]{\tt f} --- move forward 1 or {\em n} characters.
\item [] [{\em n}]{\tt b} --- move backward 1 or {\em n} characters.
\item [] [{\em n}]{\tt d} --- delete the next character or {\em n} characters.
\item [] [{\em n}]{\tt F} --- move forward 1 or {\em n} words.
\item [] [{\em n}]{\tt B} --- move backward 1 or {\em n} words.
\item [] [{\em n}]{\tt D} --- delete the next word or {\em n} words.
\item [] {\tt a} --- Go to the beginning of the string.
\item [] {\tt e} --- Go to the end of the string.
\item [] [{\em n}]{\tt i}{\em -delim-}{\em text}{\em -delim-} --- Insert {\em text}, delimited
        by the character {\em -delim-} 1 or {\em n} times.  For example, ``i/thisString/'' would insert
        ``thisString'' once.
\item [] [{\em n}]s{\em -delim-}{\em text}{\em -delim-} --- Search for {\em text},  delimited
        by the character {\em -delim-} 1 or {\em n} times.  The position is left at the end
        of the search string.  {\em -delim-} may be any character except a question mark.
\item [] S{\em -delim-}{\em text}{\em -delim-} --- Search for {\em text},  delimited
        by the character {\em -delim-}, leaving the position at the start of the
        search string. {\em -delim-} may be any nonspace character except a question mark.
\item [] [{\em n}]s?{\em -delim-}{\em text}{\em -delim-} --- Search for {\em text},  delimited
         by the character {\em -delim-} 1 or {\em n} times.   Abort all subsequent editing
        if the search fails.  If the search suceeds, leave the position at the end of the
        search string. {\em -delim-} may be any nonspace character except a question mark.
\item [] S?{\em -delim-}{\em text}{\em -delim-} --- Search for {\em text},  delimited
         by the character {\em -delim-}.   Abort all subsequent editing
        if the search fails.  If the search suceeds, leave the position at the start of the
        search string. {\em -delim-} may be any nonspace character except a question mark.
\item [] [{\em n}]k --- Delete forward from the present position 1 or {\em n} characters, placing them in the kill buffer.
\item [] [{\em n}]K --- Delete forward from the present position 1 or {\em n} words, placing them in the kill buffer.
\item [] z{\em char} --- Delete forward from the present position up to the first occurence of the character {\em char},
        placing the deleted text in the kill buffer.
\item [] [{\em n}]Z{\em char} --- Delete 1 or {\em n} times up to and including the
        character {\em char}, placing the deleted text in the kill buffer.
\item [] [{\em n}]y --- Yank the kill buffer into the string 1 or {\em n} times.
\item [] [{\em n}]\%{\em -delim-}{\em text1}{\em -delim-}{\em text2}{\em -delim-} ---
        Replace {\em text1} with {\em text2} 1 or {\em n} times starting at the
        present position.  {\em -delim-} may be any nonspace character.  For example,
        ``10\%/c/C/'' would capitalize the next 10 occurences of the character 'c'.
\item {\bf see also:}
    \begin{itemize}
    \item \progref{sddsprocess}
    \item \progref{sddsplot}
    \item \progref{sddsconvert}
    \end{itemize}
\end{itemize}


%\begin{latexonly}
\newpage
%\end{latexonly}
\subsection{sddsanalyticsignal}
\label{sddsanalyticsignal}

\begin{itemize}
\item {\bf description:}
{\tt sddsanalyticsignal} computes the analytic signal of selected real data columns
using a Hilbert transform.  It outputs the real and imaginary parts as well as
the magnitude and phase (both wrapped and unwrapped).
\item {\bf examples:}
Compute the analytic signal of column {\tt signal} as a function of {\tt time}:
\begin{flushleft}{\tt
sddsanalyticsignal data.sdds result.sdds -columns=time,signal
}\end{flushleft}
\item {\bf synopsis:}
\begin{flushleft}{\tt
sddsanalyticsignal [-pipe=[input][,output]] [\em inputfile] [\em outputfile] \\
  [-columns=\em indep-variable,\em depen-quantity[,...]] \\
  [-unwrapLimit=\em value] \\
  [-majorOrder=row|column]
}\end{flushleft}
\item {\bf switches:}
    \begin{itemize}
    \item \verb|-pipe[=input][,output]| --- The standard SDDS Toolkit pipe option.
    \item {\tt -columns=\em indep-variable,\em depen-quantity}[{\tt ,...}] ---
    Specifies the independent variable and one or more dependent columns to
    analyze.  Wildcards are permitted for the dependent quantity names.
    \item {\tt -unwrapLimit=\em value} --- Sets the relative magnitude threshold
    for phase unwrapping.  Unwrapping is applied only when the signal's
    magnitude exceeds this fraction of its maximum.
    \item {\tt -majorOrder=row|column} --- Specifies the binary SDDS layout of
    the output file.
    \end{itemize}
\item {\bf author:} H. Shang, R. Soliday, ANL/APS.
\end{itemize}

%\begin{latexonly}
\newpage
%\end{latexonly}
\subsection{sddsarray2column}
\label{sddsarray2column}

\begin{itemize}
\item {\bf description:} \hspace*{1mm}\\
{\tt sddsarray2column} converts one or more arrays in an SDDS file into columns. The
number of elements in the converted arrays must match the number of rows in
existing columns or the number of elements in other converted arrays.

\item {\bf examples:}
  Convert array \verb|A| to column \verb|A\_out|:
  \begin{flushleft}{\tt
  sddsarray2column in.sdds out.sdds -convert=A,A\_out
  }\end{flushleft}
Convert the first plane of a three--dimensional array \verb|A|:
  \begin{flushleft}{\tt
  sddsarray2column in.sdds out.sdds -convert=A,A\_out,d0=0
  }\end{flushleft}
Select a rectangular region from array \verb|A|:
  \begin{flushleft}{\tt
  sddsarray2column in.sdds out.sdds "-convert=A,A\_out,d2=(1,3)"
  }\end{flushleft}

\item {\bf synopsis:}
\begin{flushleft}{\tt
  sddsarray2column [{\em source}] [{\em target}] [-pipe=[input][,output]]\\
  {}[-nowarnings] -convert={\em arrayName}[,{\em columnName}][,d<dimension>=<indexList>]...
  }\end{flushleft}

\item {\bf files:}
\begin{itemize}
  \item {\em source} --- SDDS file containing arrays to be converted.
  \item {\em target} --- SDDS file that will contain the converted columns.
\end{itemize}

\item {\bf switches:}
    \begin{itemize}
    \item {\tt -pipe=[input][,output]} --- Standard SDDS Toolkit pipe option.
    \item {\tt -nowarnings} --- Suppress warning messages about overwriting files.
    \item {\tt -convert={\em arrayName}[,{\em columnName}][,d<dimension>=<indexList>]} ---
          Specifies an array to convert to a column. The optional {\em columnName}
          gives the name of the new column. Qualifiers \verb|d0|, \verb|d1| and
          \verb|d2| select indices from dimensions 0, 1 and 2 respectively. Each
          qualifier may contain a single index or a parenthesized list of
          indices.
    \end{itemize}
\item {\bf author:} R. Soliday, M. Borland, ANL/APS.
\end{itemize}

%\begin{latexonly}
\newpage
%\end{latexonly}
\subsection{rpn Calculator Module}
\label{rpn}
\label{rpn Calculator Module}

\begin{itemize}
\item {\bf description:} \hspace*{1mm}\\

Many of the SDDS toolkit programs employ a common Reverse Polish Notation (RPN) calculator module for equation evaluation.
This module is based on the {\tt rpn} programmable calculator program.  It is also available in a commandline
version called {\tt rpnl} for use in shell scripts.  This manual page discusses the programs {\tt rpn} and 
{\tt rpnl},  and indicates how the {\tt rpn} expression evaluator is used in SDDS tools.
\item {\bf examples:} \\
Do some floating-point math using shell variables:
(Note that the asterisk (for multiplication) is escaped in order to protect it from interpretation by the shell.)
\begin{flushleft}{\tt
set pi = 3.141592\\
set radius = 0.15\\
set area = `rpnl \$pi \$radius 2 pow \verb|\|*`\\
}\end{flushleft}
Use {\tt rpn} to do the same calculation:
\begin{flushleft}{\tt
% rpn\\
rpn> 3.141592 sto pi\\
rpn> 0.15 sto radius\\
rpn> radius 2 pow pi *\\
	      0.070685820000000\\
rpn> quit\\
% \\
}\end{flushleft}
\item {\bf synopsis:}
\begin{flushleft}{\tt
rpn [{\em filenames}]\\
rpnl {\em rpnExpression}
}\end{flushleft}
\item {\bf Overview of {\tt rpn} and {\tt rpnl}}:

{\tt rpn} is a program that places the user in a Reverse Polish Notation calculator shell.  Commands to {\tt rpn} consist
of generally of expressions in terms of built-in functions, user-defined variables, and user-defined functions.  Built-in
functions include mathematical operations, logic operations, string operations, and file operations.  User-defined
functions and variables may be defined ``on the fly'' or via files containing {\tt rpn} commands. 

The command {\tt rpn} {\em filename} invokes the {\tt rpn} shell with {\em filename} as a initial command file.
Typically, this file would contain instructions for a computation.  Prior to execution of any files named the
commandline, {\tt rpn} first executes the instructions in the file named by the environment variable {\tt RPN\_DEFNS}, if
it is defined.  This file can be used to store commonly-used variable and function definitions in order to customize the
{\tt rpn} shell.  This same file is read by {\tt rpnl} and all of the SDDS toolkit programs that use the {\tt rpn}
calculator module.  An example of such a file is included with the code.

As with any RPN system, {\tt rpn} uses stacks. Separate stacks are maintained for numerical, logical, string data, and
command files.

{\tt rpnl} is essentially equivalent to executing {\tt rpn}, typing a single command, then exiting.  However, {\tt rpnl}
has the advantage that it evaluates the command and prints the result to the screen without any need for user input.
Thus, it can be used to provide floating point arithmetic in shell scripts.  Because of the wide variety of operations
supported by the {\tt rpn} module and the availability of user-defined functions, this is a very powerful feature even for
command shells that include floating point arithmetic.

Built-in commands may be divided into four broad categories: mathematical operations, logical operations, string
operations, and file operations.  (There are also a few specialized commands such as creating and listing user-defined
functions and variables; these will be addressed in the next section).  Any of these commands may be characterized
by the number of items it uses from and places on the various stacks.

\begin{itemize}
\item Mathematical operations:
\begin{itemize}
\item Using {\tt rpn} variables:\\
The {\tt sto} (store) function allows both the creation of {\tt rpn} variables and modification of their
contents.  {\tt rpn} variables hold double-precision values.  The variable name may be any string starting
with an alphabetic character and containing no whitespace.  The name may not be one used for a built-in
or user-defined function.   There is no limit to the number of variables that may be defined.

For example, {\tt 1 sto one} would create a variable called {\tt one} and store the value 1 in it.  To recall the
value, one simply uses the variable name.  E.g., one could enter {\tt 3.1415925 sto pi} and later enter
{\tt pi } to retrieve the value of $\pi$.

\item Basic arithmetic: {\tt + - * / sum}\\

With the exception of {\tt sum}, these operations all take two values
from the numeric stack and push one result onto the numeric stack.
For example, {\tt 5 2 -} would push 5 onto the stack, push 2 onto the
stack, then push the result (3) onto the stack.  

{\tt sum} is used to sum the top {\tt n} items on the stack, exclusive
of the top of the stack, which gives the number of items to sum.  For
example, {\tt 2 4 6 8 4 sum} would put the value {\tt 20} on the stack.

\item Basic scientific functions: {\tt sin cos acos asin atan atan2 sqrt sqr pow exp ln}\\

With the exception of {\tt atan2} and {\tt pow}, these operations all take one item from the numeric stack and push one
result onto that stack.

{\tt sin} and {\tt cos} are the sine and cosine functions, while {\tt asin}, {\tt acos}, and {\tt atan} are inverse
trigonometic functins.  {\tt atan2} is the two-argument inverse tangent: {\tt x y atan2} pushes the value ${\rm atan(y/x)}$
with the result being in the interval ${\rm [-\pi, \pi]}$.

{\tt sqrt} returns the positive square-root of nonnegative values.  {\tt sqr} returns the square of a value.  {\tt pow}
returns a general power of a number: {\tt x y pow} pushes ${\rm x t}$ onto the stack.  Note that if {\tt y} is
nonintegral, then {\tt x} must be nonnegative.

{\tt exp} and {\tt ln} are the base-e exponential and logarithm functions.

\item Special functions: {\tt Jn Yn cei1 cei2 erf erfc gamP gamQ lngam}\\
{\tt Jn} and {\tt Yn} are the Bessel functions of integer order of the first and second kind\cite{Abramowitz}.  Both take
two items from the stack and place one result on the stack.  For example, {\tt x i Jn} would push ${\rm J_i(x)}$ onto the
stack.  Note that ${\rm Y_n(x)}$ is singular at x=0.

{\tt cei1} and {\tt cei2} are the 1st and 2nd complete elliptic integrals.  The argument is the modulus k, as seen in
the following equations (the functions K and E are those used by Abramowitz\cite{Abramowitz}).
\[{\rm cei1(k) = K(k^2) = \int_0^{\pi/2} \frac{d\theta}{\sqrt{1 - k^2 sin^2 \theta}}  } \] \\
\[{\rm cei2(k) = E(k^2) = \int_0^{\pi/2} \sqrt{1 - k^2 sin^2 \theta} d\theta }\]

{\tt erf} and {\tt erfc} are the error function and complementary error function.  By definition, ${\rm erf(x) + erfc(x)}$
is unity.  However, for large x, {\tt x erf 1 -} will return 0 while {\tt x erfc} will return a small, nonzero value.
The error function is defined as\cite{Abramowitz}:
\[ {\rm erf(x) = \frac{2}{\sqrt{\pi}} \int_0 x e^{-t^2} dt} \]
Note that ${\rm erf(x/\sqrt{2}) }$ is the area under the normal Gaussian curve between ${\rm -x}$ and ${\rm x}$.

{\tt gamP} and {\tt gamQ} are, respectively, the incomplete gamma function and its complement
\cite{Abramowitz}:
\[{\rm gamP(a, x) = 1-gamQ(a, x) = \frac{1}{\Gamma(a)} \int_0^x e^{-t} t^{a-1} dt \hspace*{10mm}  a>0} \]
These functions take two arguments; the 'a' argument is place on the stack first.

{\tt lngam} is the natural log of the gamma function.  For integer arguments, {\tt x lngam} is 
${\rm ln( (x-1)!) }$.  The gamma function is defined as\cite{Abramowitz}:
\[ {\rm \Gamma (x) = \int_0^\infty t^{x-1} e^{-t} dt } \]

\item Numeric stack operations: {\tt cle n= pop rdn rup stlv swap view ==}\\
{\tt cle} clears the entire stack, while {\tt pop} simply removes the top element.  {\tt ==} duplicates the top item on
the stack, while {\tt x n=} duplicates the top x items of the stack (excluding the top itself).  {\tt swap} swaps the top
two items on the stack. {\tt rdn} (rotate down) and {\tt rup} (rotate down) are stack rotation commands, and are the
inverse of one another.  {\tt stlv} pushes the stack level (i.e., the number of items on the stack) onto the stack.
Finally, {\tt view} prints the entire stack from top to bottom.

\item Random number generators: {\tt rnd grnd}\\
{\tt rnd} returns a random number from a uniform distribution on ${\rm [0, 1]}$.  {\tt grnd} returns
a random number from a normal Gaussian distribution.

\item Array operations: {\tt mal [ ]}\\
{\tt mal} is the Memory ALlocation command; it pops a single value from the numeric stack, and returns a
``pointer'' to memory sufficient to store the number of double-precision values specified by that value.  This
pointer is really just an integer, which can be stored in a variable like any other number.  It is used to place
values in and retrieve values from the allocated memory.  

\verb|]| is the memory store operator. A sequence of the form {\tt {\em value} {\em index} {\em addr} ]} results in
{\em value} being stored in the {\em index} position of address {\em addr}.  {\em value}, {\em index}, and {\em
addr} are consumed in this operation.  Indices start from 0.

Similarly, {\tt {\em index} {\em addr} [} value pushes the value in the {\em index} position of address {\em addr}
onto the stack.  {\em index} and {\em addr} are consumed in this operation.

\item Miscellaneous: {\tt tsci int}\\
{\tt tsci} allows one to toggle display between scientific and variable-format notation.  In the former, all
numbers are displayed in scientific notation, whereas in the later, only sufficiently large or small numbers are
so displayed.  (See also the {\tt format} command below.)

{\tt int} returns the integer part of the top value on the stack by truncating the noninteger part.  

\end{itemize}
\item {\bf Logical operations}: \verb.! && < == > ? $ vlog ||.\\
\begin{itemize}
\item Conditional execution: {\tt ?}
The question-mark operator functions to allow program branching.  It is meant to remind the user of the C operator for
conditional evaluation of expressions.  A conditional statement has the form\\
{\tt ? {\em executeIfTrue} : {\em executeIfFalse} \$}\\
The colon and dollar sign function as delimiters for the conditionally-executed instructions.  The {\tt ?} operator pops the
first value from the logic stack.  It branches to the first set of instructions if this value is ``true'', and to the
second if it is ``false''.

\item Comparisons: {\tt < == > }\\
These operations compare two values from the numeric stack and push a value onto the logic stack indicating the result.
Note that the values from the numeric stack are left intact.  That is, these operations push the numeric values back onto the
stack after the comparison.
\item Logic operators: \verb. && || !.\\
These operators consume values from the logic stack and push new results onto that stack.
\verb|&&| returns the logical and of the top two values, while {\tt ||} returns the logical or.
{\tt !} is the logical negation operator.
\item Miscellaneous: {\tt vlog}\\
This operator allows viewing the logic stack.  It lists the values on the stack starting at the top.

\item Examples:\\
Suppose that a quantity is tested for its sign.  If the sign is negative, then have the conditional return a -1, if the sign is positive then return a +1.

Suppose we are running in the rpn shell and that the quantity 4 is initially pushed onto the stack.  The command ``\verb.0 < ? -1 : 1 $.\ '' that accomplishes the sign test will be executed as follows.
\begin{flushleft}
\begin{verbatim}
command     stack    logical stack
0           0        stack empty
            4

<           0        false  <-- new value
            4

? 1 : -1 $  1        stack empty
            0  
            4
\end{verbatim}
\end{flushleft}
In order to keep the stack small, the command should be written ``\verb.0 < pop pop ? -1 : 1 $.\ '', where the pop commands would eliminate the 0 and 4 from the stack before the conditional is executed.

If the command is executed with rpnl command in a C-shell, then the \$
character has to be followed by a blank space to prevent the shell
from interperting the \$ as part of variable:\\

C-shell\verb.>.\ rpnl \verb."4 0 < pop pop ? -1 : 1 $ ".\\

If the command is executed in a C-shell sddsprocess command to create
a new column, then we write:\\
\begin{flushleft}
\begin{verbatim}
sddsprocess <infile> <outfile> \
   -def=col,NewColumn,"OldColumn 0 < pop pop ? -1 : 1 $ "

\end{verbatim}
\end{flushleft}
which is similar to the rpnl command above.

If the sddsprocess command is run in a tcl/tk shell, the \$ character can be escaped with a backslash as well as with a blank space:\\
\begin{flushleft}
\begin{verbatim}
sddsprocess <infile> <outfile> \
   "-def=col,NewColumn,OldColumn 0 < pop pop ? -1 : 1 \$"

\end{verbatim}
\end{flushleft}
Note that the double quotes enclose the whole command argument, not just the sub-argument.

\end{itemize}
\item {\bf String operations}: {\tt "" =str cshs format getformat pops scan sprf vstr xstr}\\

\begin{itemize}
\item Stack operations: {\tt "" =str pops vstr}\\
To place a string on the string stack, one simply encloses it in double quotation marks.
{\tt =str} duplicates the top of the string stack.  {\tt pops} pops the top item off
of the string stack.  {\tt vstr} prints (views) the string stack, starting at the top.
\item Format operations: {\tt format getformat}\\
{\tt format} consumes the top item of the string stack, and causes it to be used as
the default printf-style format string for printing numbers.  {\tt getformat} 
pushes onto the string stack the default printf-style format string for printing numbers.
\item Print/scan operations: {\tt scan sprf}\\
{\tt scan} consumes the top item of the string stack and scans it for a number; it
pushes the number scanned onto the string stack, pushes the remainder of the string
onto the string stack, and pushes true/false onto the logic stack to indicate
success/failure.  {\tt sprf} consumes the top of the string stack to get a sprintf
format string, which it uses to print the top of the numeric stack; the resulting
string is pushed onto the string stack.  The numeric stack is left unchanged.
\item string comparison opertions:
{\tt streq} compares if two strings are the same.
{\tt strgt} compares if left string is greater than right string.
{\tt strlt} compares if left string is less than right string.
{\tt strmatch} compares if left string matches right string pattern.
\item Other operations: {\tt cshs xstr}\\
{\tt cshs} executes the top string of the stack in a C-shell subprocess; note that
if the command requires terminal input, {\tt rpn} will hang.  {\tt xstr} executes
the top string of the stack as an rpn command.
{\tt strlen} returns the lenght of a string.
\end{itemize}

\item {\bf File operations}: {\tt @ clos fprf gets open puts}\\
\begin{itemize}
\item Command file input: {\tt @}\\
The {\tt @} operator consumes the top item of the string stack,
pushing it onto the command file stack.  The command file is executed
following completion of processing of the current input line.  Command
file execution may be nested, since the files are on a stack.  The
name of the command file may have options appended to it in the format
{\tt {\em filename},{\em option}}.  Presently, the only option
recognized is 's', for silent execution.  If not present, the command file
is echoed to the screen as it is executed.  

Example: {\tt "commands.rpn,s" @} would silently execute the {\tt rpn}
commands in the file {\tt commands.rpn}.

\item Opening and closing files: {\tt clos open}\\
{\tt open} consumes the top of the string stack, and opens a file with
the name given in that element.  The string is of the format {\tt {\em
filename},{\em option}}, where {\em option} is either 'w' or 'r' for
write or read.  {\tt open} pushes a file number onto the numeric
stack.  This should be stored in a variable for use with other file IO
commands.  The file numbers 0 and 1 are predefined, respectively, as
the standard input and standard output.

{\tt clos} consumes the top of the numeric stack, and uses it as the
number of a file to close.  

\item Input/output commands: {\tt fprf gets puts}\\
These commands are like the C routines with similar names.
{\tt fprf} is like fprintf; it consumes the top of the string stack
to get a fprintf format string for printing a number.  It consumes
the top of the numeric stack to get the file number, and uses the
next item on the numeric stack as the number to print.  This number
is left on the stack.

{\tt gets} consumes the top of the numeric stack to get a file
number from which to read.  It reads a line of input from the
given file, and pushes it onto the string stack.  The trailing
newline is removed.  If successful, {\tt gets} pushes true onto
the logic stack, otherwise it pushes false.

{\tt puts} consumes the top of the string stack to get a string to
output, and the top of the numeric stack to get a file number.  Unlike
the C routine of the same name, a newline is {\em not} generated.
Both {\tt puts} and {\tt fprf} accept C-style escape sequences for
including newlines and other such characters.

\end{itemize}

\item {\bf author:} M. Borland, ANL/APS.
\end{itemize}

%\begin{latexonly}
\newpage
%\end{latexonly}
\subsection{SDDS Wildcard Conventions}
\label{SDDS Wildcard Conventions}

This manual page does not describe a program, but rather a facility that is common to several
programs.  In particular, several SDDS programs use a common convention for wildcards in
element names.

The characters \verb|*|, \verb|?|, \verb|[|, \verb|]|, and \verb| | are used for wildcard operations.

\item [] \verb|*| matches any zero or more characters.  A sequence like \verb|*a| matches zero or more
characters up to the first occurence of \verb|a|.

\item [] \verb|?| matches any one character.
\item [] \verb|[|{\em rangeSpec}\verb|]| matches any one character in {\em rangeSpec}.  {\em rangeSpec} is
 composed on any number of explicit characters, plus character ranges specified as {\em firstChar}\verb|-|{\em lastChar},
 which matches any character between  {\em firstChar} and {\em lastChar} inclusive in the ASCII character set.
 For example, \verb|[a-z]| would match a lower case alphabetic character, while \verb|[a-z][A-Z][0-9]| would
 match any alphanumeric character.

\item [] \verb|[ |{\em rangeSpec}\verb|]| matches any one character not in {\em rangeSpec}.

\item {\bf see also:}
    \begin{itemize}
    \item \progref{sddschanges}
    \item \progref{sddsconvert}
    \item \progref{sddscorrelate}
    \item \progref{sddsenvelope}
    \item \progref{sddsfft}
    \item \progref{sddsoutlier}
    \item \progref{sddsplot}
    \item \progref{sddsprintout}
    \item \progref{sddsprocess}
    \item \progref{sddssmooth}
    \item \progref{sddsxref}
    \item \progref{sddszerofind}
    \end{itemize}
\end{itemize}



\newpage
\section{Manual Pages for APS-Specific Programs}
\label{APSManualPages}

%\begin{latexonly}
\newpage
%\end{latexonly}
\subsection{awe2sdds}
\label{awe2sdds}

\begin{itemize}
\item {\bf description:} 
Converts a file in \verb|awe| self-describing format to SDDS.  This is of interest to
only a few users at APS, as \verb|awe| format has been superseeded by SDDS and is rarely used.
\item {\bf example:} 
To convert {\tt awe} format Twiss parameter data from an old version of \verb|elegant|:
\begin{flushleft}{\tt 
awe2sdds APS.awe APS.sdds -labelColumnName=ElementName
}\end{flushleft}
\item {\bf synopsis:}
\begin{flushleft}{\tt 
awe2sdds {\em inputFile} {\em outputFile} [-labelColumnName={\em string}] [-asciiOutput]
}\end{flushleft}
\item {\bf files:} 
{\em inputFile} is an {\tt awe}-format file, the SDDS equivalent of which is written to {\em outputFile}.
The ``auxiliary values'' of the {\tt awe} file are converted into SDDS parameters.  The {\tt awe} tables
are converted into SDDS tabular data, all columns being double precision except the ``row label'', which
becomes a string column.
\item {\bf switches:}
    \begin{itemize}
    \item {\tt -labelColumnName={\em string}} --- Requests that the {\tt awe} row label be given the name {\em string}.
        By default, the row label is placed in a column named ``row-label''.
    \item \verb|-asciiOutput| --- Requests that output be in ASCII.  By default, the output is binary.
    \end{itemize}
\item {\bf author:} M. Borland, ANL/APS.
\end{itemize}


\newpage
\subsection{col2sdds}
\label{col2sdds}

\begin{itemize}
\item {\bf description:}
Converts a file in \verb|column| self-describing format to SDDS.  This is of interest to
APS users only, some of whom still have programs that generate \verb|column|-format files.
\item {\bf synopsis:} 
\begin{flushleft}{\tt
col2sdds {\em inputFile} {\em outputFile} [-fixMplNames]
}\end{flushleft}
\item {\bf files:}
{\em inputFile} is a {\tt column}-format file, the SDDS equivalent of which is written to {\em outputFile}.
The ``auxiliary values'' of the {\tt columns} file are converted into SDDS parameters.  The {\tt column} table
is converted into SDDS tabular data, all columns begin double precision except the ``row label'', which
becomes a string column.
\item {\bf switches:}
    \begin{itemize}
    \item \verb|-fixMplNames| --- Requests that any column or parameter names in the input file that contain
        \verb|mpl| character set escape sequences be ``fixed''.  This results in simpler names.  The escape sequences
        are always retained in definition of the symbol for each column or parameter, and hence will appear on
        graphs as expected.
    \end{itemize}
\item {\bf author:} M. Borland, ANL/APS.
\end{itemize}


\begin{sddsprog}{sdds2mpl}
\item \textbf{description:}
  \verb|sdds2mpl| extracts data columns or parameters from an SDDS data set and creates \verb|mpl| data files. The
  program allows creation of \verb|mpl| labels from SDDS parameters. This tool is primarily of interest to APS
  users, some of whom still use the older {\tt mpl} Toolkit. It may be of interest to others who are interested in
  a simple format for use with programs that don't need the full power of SDDS protocol. Such applications can use
  {\tt sdds2mpl} and {\tt mpl2sdds} to mediate between themselves and SDDS-compliant programs.

\item \textbf{examples:}
  \begin{verbatim}
  sdds2mpl APS.twi -rootname=APS -output=column,z,betax -output=column,z,betay
  \end{verbatim}

\item \textbf{synopsis:}
  \begin{verbatim}
  usage: sdds2mpl [SDDSfile] [-pipe[=input]] [-rootName=string] [-separatePages]
  -output={column | parameter},xName,yName[{syName | sxName,syName}]
  [-announceOpenings] [-labelParameters=name[=format]][...]
  \end{verbatim}

\item \textbf{switches:}
  \begin{itemize}
  \item \verb|-pipe[=input]| --- The standard SDDS Toolkit pipe option.
  \item \verb|-announceOpenings| --- Requests that an informational message be printed whenever a new output file is opened.
  \item {\tt -rootName={\em string}} --- Gives the rootname for constructing output filenames.
  \item \verb|-separatePages| --- Requests that tabular-data column output from separate pages in the SDDS data set go to separate files.
  \item {\tt -labelParameters={\em name}[={\em format}][...]} --- Gives the names and optional \verb|printf| format specifications for parameters that will be printed on the title line of the \verb|mpl| files.
  \item {\tt -output\{column | parameter\},{\em xName},{\em yName}[,\{{\em syName} | {\em sxName},{\em syName}\}]} --- Requests that the named columns or parameters be put into a \verb|mpl| file or set of files. If \verb|-separate| is not given or if the data is for parameters, the name of the file is {\tt rootname\_{\em xName}\_{\em yName}.out}. For column output, if \verb|-separate| is given, the names of the files are {\tt rootname\_{\em N}\_{\em xName}\_{\em yName}.out}, where {\em N} is the page number. This option may be given any number of times.
  \end{itemize}

\item \textbf{files:}
  {\em SDDSfile} is the name of an SDDS file from which {\tt mpl}-format files will be made. Each {\tt mpl} file contains two to four columns of data.

\item \textbf{see also:}
  \begin{itemize}
  \item \hyperref[exampleData]{Data for Examples}
  \item \progref{mpl2sdds}
  \end{itemize}

\item \textbf{author:} M. Borland, ANL/APS.
\end{sddsprog}


%
%\begin{latexonly}
\newpage
%\end{latexonly}
\subsection{mpl2sdds}
\label{mpl2sdds}

\begin{itemize}
\item {\bf description:} Adds \verb|mpl| data files to an SDDS data set.  \verb|mpl| is a simple data format used by the
\verb|mpl| Toolkit, which is now largely superseded by SDDS and will not be supported in the future. 
\item {\bf example:} 
\begin{flushleft}{\tt
mpl2sdds APS\_s\_betax.out APS\_s\_betay.out -output=APSbetas.sdds
}\end{flushleft}
\item {\bf synopsis:}
\begin{flushleft}{\tt
mpl2sdds {\em mplFile} [{\em mplFile}...] -output={\em SDDSFile} [-erase]
}\end{flushleft}
\item {\bf files:}
Any number of {\em mplFile} arguments may be given.  These name files in {\tt mpl} format, which
has between two and four columns of data.  {\tt sdds2mpl} attempts to add all of the columns from each
\verb|mpl| data file to the data set.  However, a column that has the same name as an existing column
will not be used.  By default, the data in the {\tt mpl} files is added to {\em SDDSFile}, if it
exists already.
\item {\bf switches:}
    \begin{itemize}
    \item {\tt -output={\em SDDSfile}} --- Specifies that data be added to file {\em SDDSfile}. If
        the file does not exist, it is created. 
    \item \verb|-erase| --- Specifies that if {\em SDDSFile} exists already, it should be
        erased prior to adding any data to the data set.  By default, the data in {\em SDDSFilename}
        is retained.
    \end{itemize}
\item {\bf see also:}
    \begin{itemize}
    \item \progref{sdds2mpl}
    \end{itemize}
\item {\bf author:} M. Borland, ANL/APS.
\end{itemize}



\newpage
\section{Manual Pages for Synchrotron Radiation Programs}
\label{SyncManualPages}

%\begin{latexonly}
\newpage
%\end{latexonly}
\subsection{sddssyncflux}
\label{sddssyncflux}

\begin{itemize}
\item {\bf description:} \verb|sddssyncflux| calculates synchroton radiation photon flux of bend, wigger and undulator magnet. The calculation for undulator has not been implemented yet.

\item {\bf examples:} 
%
% Insert text of examples in this section.  Examples should be simple and
% should be preceeded by a brief description.  Wrap the commands for each
% example in the following construct:
% 
%
{\tt }
\begin{flushleft}{\tt
\bf sddssyncflux bend.test -source=bend -mode=energy,linear,start=1,end=3,step=0.1
\bf sddssyncflux wiggler.test -source=wiggler,period=5,numberOfPeriods=7,field=1 -mode=energy,linear,start=100,end=200,step=2 
}\end{flushleft}

\item {\bf synopsis:} 
%
% Insert usage message here:
%
\begin{flushleft}{\tt
sddssyncflux <outputFile> -verbose [-pipe]
   [-fileValues=<filename>[,energy=<columnName or wavelength=columnName>] 
   [-mode=energy(wavelength),linear(logarithmic),start=<value>,end=<value>,step(factor)=<value>] 
   [-source=bendMagnet[,field=xx[,radius=yy][,criticalEnergy=ZZ]] 
     [-source=wiggler(undulator),period=xx[,field=yy][,K=zz],numberOfPerions=<n>]
   [-eBeamEnergy=<value> [-eBeamCurrent=<value>] [-eBeamGamma=<value>]   
}\end{flushleft}

\item {\bf files:}
{\em outputFile} the results are written to an SDDS file.
\item {\bf switches:}
    \begin{itemize}
    \item {\tt -pipe} --- output result to the pipe.
    \item {\tt -fileValues=<filename>[,energy=<columnName or wavelength=columnName>] } --- get the energy or waveformlengt of filename instead of by -mode option.
    \item {\tt -mode=energy,linear,start=<value>,end=<value>,step=<value>} --- Generate photon energy column in eV linearly, from start to end in steps.
    \item {\tt -mode=energy,logarithmic,start=<value>,end=<value>,factor=<value>} --- Generate photon energy column logarithmically, from start to end by multiplying
                   factor from point to point.
    \item {\tt -mode=wavelength,linear,start=<value>,end=<value>,step=<value>} --- Generate photon wavelength column in nm linearly, from start to end in steps.
    \item {\tt -mode=wavelength,logarithmic,start=<value>,end=<value>,factor=<value} ---
        Generate photon wavelength column in nm logarithmically, from start 
        to end by multiplying factor from point to point.
    \item {\tt -source=bendMagnet[,field=xx][,radius=yy][,critialEnergy=zz]} --- bend 
        magnet source, magnetic filed=xx Tesla (default=0.6 Teslas). 
        bend radius= yy meter (no default value), criticalEnergy=zz eV 
        (no default value). only one of field, radius and K is needed to be provided. 
    \item {\tt -source=wiggler,period=xx[,field=yy][,K=zz],numberOfPeriods=n} ---
        Wiggler source, period=xx cm (default=5 cm).  
        Peak magnetic field=yy Tesla (default=1 Teslas). 
        Undulator parameter, K=zz (no default values) 
        only two of period, field and K is needed to be provided.  
        numberOfPeriods needs to be provided.  
    \item {\tt -source=undulator,period=xx[,field=yy][,K=zz],numberOfPerios=n} ---
        undulator source, period=xx cm (default=5 cm). 
        Peak magnetic field=yy Tesla (default=1 Teslas).
        Undulator parameter, K=zz (no default values) 
        only two of period, field and K is needed to be provided.
        numberOfPeriods needs to be provided. 
{\bf Note that only one source is accepted at one time. }
    \item {\tt -eBeamEnergy} --- Electron beam energy in Gev, default 7Gev.
    \item {\tt -eBeamGamma} --- Electron beam gamma.
    \item {\tt -eBeamCurrent} --- Electron beam current in A.
    \item {\tt -g1ValueFile} --- give the file which contains the values of y and yGy,
      where yGy=y*intergration of (bessel funtion) K5/3 from y to infinity.
    \end{itemize}
\item {\bf author:}H. Shang ANL/APS.
\end{itemize}




\tableofcontents
\end{document}
