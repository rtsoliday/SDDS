\begin{sddsprog}{sddsmultihist}
  \item \textbf{description:} \verb|sddsmultihist| does one-dimensional histograms of multiple columns of data from an SDDS file.
    All columns are histogrammed on the same interval and with the same number of bins. It is similar to \verb|sddshist|,
    except that the latter program only histograms a single column at a time. Unlike \verb|sddshist|, \verb|sddsmultihist|
    does not presently do statistical analyses or filtering.
  \item \textbf{examples:}
    \begin{verbatim}
sddsmultihist par.bpm par.bpmhis -columns=P?P?x -bins=20 -abscissa=xReadout
    \end{verbatim}
  \item \textbf{synopsis:}
    \begin{verbatim}
sddsmultihist [-pipe=[input][,output]] [inputFile] [outputFile]
  -columns=columnName[,columnName...] -abscissa=newName
  [-separate]
  [-exclude=columnName[,columnName...]]
  [{-bins=integer | -sizeOfBins=value | -autobins=target=number[,minimum=integer][,maximum=integer]}]
  [-lowerLimit=value] [-upperLimit=value]
  [-sides]
    \end{verbatim}
  \item \textbf{files:} \emph{inputFile} is the name of an SDDS file containing data to be histogrammed. If \emph{inputFile}
    contains multiple data pages, each is treated separately. The histograms are placed in \emph{outputFile}, which has one
    column of histogram frequencies for each histogrammed input column, plus a column giving the abscissa values for the frequency
    distributions. The former columns have names of the form \emph{columnName}\verb|Frequency|, containing the number of points
    in each bin. The latter column has a name given by the user.
  \item \textbf{switches:}
    \begin{itemize}
      \item \verb|-pipe[=input][,output]| --- The standard SDDS Toolkit pipe option.
      \item \verb|-columns=columnName[,columnName...]| --- Specifies the names of the data columns to be histogrammed. The
        \emph{columnName} items may contain wildcards.
      \item \verb|-separate| --- Specifies that a separate abscissa shall be created for each histogrammed column. If
        \verb|-abscissa| is not given, then the abscissa names are the names of the columns being histogrammed.
      \item \verb|-abscissa=newName[,newName...]| --- Specifies the name or names of the abscissa columns for the histogram
        output. If \verb|-separate| is not given, then only one name is permitted. The units are taken from the units of the
        columns being histogrammed.
      \item \verb|-exclude=columnName[,columnName...]| --- Specifies the names of data columns to exclude from histogramming. The
        \emph{columnName} items may contain wildcards.
      \item \verb|-bins=number| --- Specifies the number of bins to use. The default is 20.
      \item \verb|-sizeOfBins=value| --- Specifies the size of bins to use. The number of bins is computed from the range of the
        data.
      \item \verb|-autoBins=target=number[,minimum=integer][,maximum=integer]| --- Specifies that the number of bins should be
        chosen to attempt to give a target number of samples per bin on average. If \verb|minimum| is given, then no fewer than
        the specified number of bins will be used (default: 5). If \verb|maximum| is given, then no more than the specified number
        of bins will be used (default: number of samples).
      \item \verb|-lowerLimit=value| --- Specifies the lower limit of the histogram. By default, the lower limit is the minimum
        value in the data.
      \item \verb|-upperLimit=value| --- Specifies the upper limit of the histogram. By default, the upper limit is the maximum
        value in the data.
      \item \verb|-sides| --- Specifies that zero-height bins should be attached to the lower and upper ends of the histogram.
        Many prefer the way this looks on a graph.
    \end{itemize}
  \item \textbf{see also:}
    \begin{itemize}
      \item \hyperref[exampleData]{Data for Examples}
      \item \progref{sddshist}
      \item \progref{sddshist2d}
    \end{itemize}
  \item \textbf{author:} M. Borland, ANL/APS.
\end{sddsprog}

