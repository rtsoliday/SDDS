\begin{sddsprog}{raw2sdds}
  \item \textbf{description:}
  \verb|raw2sdds| converts a binary data stream into the SDDS format.
  \item \textbf{examples:}
\begin{verbatim}
raw2sdds inputfile outputfile -definition=Screen1,type=character -size=484,512
\end{verbatim}
  \item \textbf{synopsis:}
\begin{verbatim}
raw2sdds inputfile outputfile -definition=name,definition-entries
  [-size=horiz-pixels,vert-pixels] [-majorOrder=row|column]
\end{verbatim}
  \item \textbf{files:}
  The input file contains the binary data for horizontal and vertical pixels of a screen.
  Only single byte data, per pixel, is currently allowed.
  \item \textbf{switches:}
    \begin{itemize}
      \item \verb|-definition=name,definition-entries| --- The name of the output column.
      \item \verb|-size=horiz-pixels,vert-pixels| --- Specifies dimensions of the screen.
      \item \verb!-majorOrder=row|column! --- Specifies the binary SDDS layout.
    \end{itemize}
  \item \textbf{see also:} \progref{image2sdds}.
  \item \textbf{author:} C. Saunders, R. Soliday, ANL/APS.
\end{sddsprog}
 
