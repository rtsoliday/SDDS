%\begin{latexonly} 
\newpage 
%\end{latexonly} 
\subsection{raw2sdds} 
\label{raw2sdds} 
 
\begin{itemize} 
\item {\bf description:} \hspace*{1mm}\\ 
{\tt raw2sdds} converts a binary data stream into the SDDS format.
\item {\bf examples:} 
\begin{flushleft}
{\tt raw2sdds <inputfile> <outputfile> -definition=Screen1,type=character -size=484,512}
\end{flushleft} 
\item {\bf synopsis:}  
\begin{flushleft}
{\tt 
raw2sdds {\em inputfile} {\em outputfile} 
 -definition=<name>,<definition-entries> 
 [-size=<horiz-pixels>,<vert-pixels>] 
 [-majorOrder=row|column]}
\end{flushleft} 
\item {\bf files:} 
The input file contains the binary data for horizontal and vertical pixels of a screen. Only single byte data, per pixel, is currently allowed.
\item {\bf switches:} 
    \begin{itemize} 
    \item {\tt -definition={\em name},{\em definition-entries}} --- The name of the output column. 
    \item {\tt -size={\em horiz-pixels},{\em vert-pixels}} --- Specifies dimensions of the screen.
    \item {\tt -majorOrder=row|column} --- Specifies the binary SDDS layout.
\end{itemize} 

\item {\bf author:} C. Saunders, R. Soliday, ANL/APS. 
\end{itemize} 
 
