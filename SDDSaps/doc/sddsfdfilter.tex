\begin{sddsprog}{sddsfdfilter}
  \item \textbf{description:}
    \verb|sddsfdfilter| performs frequency-domain filtering of data columns using
    Fourier transforms.  It supports thresholding, highpass, lowpass, notch,
    bandpass, and user-defined filters.  Multiple filters may be cascaded or
    applied in parallel, and the program can create new or difference columns in
    the output.
  \item \textbf{examples:}
    \begin{verbatim}
    sddsfdfilter data.sdds filtered.sdds -columns=t,value -lowpass=start=1,end=10
    sddsfdfilter data.sdds filtered.sdds -columns=t,value \
      -notch=center=60,flatWidth=2,fullWidth=5
    \end{verbatim}
  \item \textbf{synopsis:}
    \begin{verbatim}
    sddsfdfilter [-pipe[=input][,output]] [inputfile] [outputfile]
                 [-columns=indep-variable[,depen-quantity[,depen-quantity...]]]
                 [-exclude=depen-quantity[,depen-quantity]]
                 [-clipFrequencies=[high=number][,low=number]]
                 [-threshold=level=value[,fractional][,start=freq][,end=freq]]
                 [-highpass=start=freq,end=freq]
                 [-lowpass=start=freq,end=freq]
                 [-notch=center=center,flatWidth=width1,fullWidth=width2]
                 [-bandpass=center=center,flatWidth=width1,fullWidth=width2]
                 [-filterFile=filename=filename,frequency=columnName,
                             {real=columnName,imaginary=columnName |
                              magnitude=columnName}]
                 [-cascade] [-newColumns] [-differenceColumns]
    \end{verbatim}
  \item \textbf{files:}
    The optional \verb|inputfile| is an existing SDDS file to filter.  If omitted,
    the program reads from standard input when \verb|-pipe=input| is given.  The
    \verb|outputfile| receives the filtered data, or standard output is used when
    \verb|-pipe=output| is specified.
  \item \textbf{switches:}
    \begin{itemize}
    \item {\tt -pipe[=input][,output]} --- The standard SDDS Toolkit pipe option.
    \item {\tt -columns={\em indepVariable}[,{\em depenQuantity}][,{\em depenQuantity}...]} ---
      Gives the name of the independent variable (typically time).  If no
      {\em depenQuantity} qualifiers are given, all numerical columns are filtered;
      otherwise, only the named columns are processed.
    \item {\tt -exclude={\em depenQuantity}[,{\em depenQuantity}]} --- Specifies columns to
      exclude from filtering, modifying selections from {\tt -columns}.
    \item {\tt -clipFrequencies[=low={\em frequency}][,high={\em frequency}]} --- Clips
      frequencies below the {\tt low} value and/or above the {\tt high} value; clipped
      components are set to zero.
    \item {\tt -threshold=level={\em value}[,fractional][,start={\em freq}][,end={\em freq}]} ---
      Suppress components below a threshold.  With {\tt fractional}, the level is a
      fraction of the largest component.  {\tt start} and {\tt end} restrict the
      frequency range.
    \item {\tt -highpass=start={\em freq},end={\em freq}} --- Apply a highpass filter.
    \item {\tt -lowpass=start={\em freq},end={\em freq}} --- Apply a lowpass filter.
    \item {\tt -notch=center={\em center},flatWidth={\em width1},fullWidth={\em width2}} ---
      Apply a notch filter centered on a frequency.
    \item {\tt -bandpass=center={\em center},flatWidth={\em width1},fullWidth={\em width2}} ---
      Apply a bandpass filter centered on a frequency.
    \item {\tt -filterFile=filename={\em filename},frequency={\em columnName},}
      {\tt \{real={\em columnName},imaginary={\em columnName} | magnitude={\em columnName}\}} ---
      Specifies a filter via an SDDS file of attenuation values.  The column named
      with the {\tt frequency} qualifier gives the frequency values.  Components
      outside the provided range are unaffected.
    \item {\tt -cascade} --- Cascade multiple filter definitions.
    \item {\tt -newColumns} --- Create new columns containing filtered results.
    \item {\tt -differenceColumns} --- Create columns containing differences between
      original and filtered data.
    \end{itemize}
  \item \textbf{see also:}
    \begin{itemize}
    \item \progref{sddsdigfilter}
    \item \progref{sddssmooth}
    \end{itemize}
  \item \textbf{author:} M. Borland, ANL/APS.
\end{sddsprog}

