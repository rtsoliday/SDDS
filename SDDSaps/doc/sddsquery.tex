\begin{sddsprog}{sddsquery}
  \item \textbf{description:}
    \verb|sddsquery| prints a summary of the SDDS header for a data set. It also prints bare lists of names of defined entities,
    suitable to use with shell scripts that need to detect the existence of entities in the data set. Finally, it will create an
    SDDS file containing information about what is in the header.
  \item \textbf{examples:}
    \begin{verbatim}
    sddsquery APS.twi
    sddsquery APS.twi -columnList
    set names = `sddsquery APS.twi -columnList -delimiter=" "`
    sddsquery APS.twi -columnList -sddsOutput=APS.twi.names
    \end{verbatim}
  \item \textbf{synopsis:}
    \begin{verbatim}
    sddsquery SDDSfilename [SDDSfilename...]
      [-sddsOutput[=filename]]
      {-arrayList | -columnList | -parameterList | -version}
      [-delimiter=delimitingString] [-appendUnits[=bare]] [-readAll]
    \end{verbatim}
  \item \textbf{files:} The input filenames may name arbitrary SDDS files.

    If \verb|-sddsOutput| is given, the output normally contains one page for each data class (i.e., array, parameter, and
    column). The following elements are defined:
    \begin{itemize}
      \item Columns (all string type):
        \begin{itemize}
          \item \verb|Name| --- The name of the element.
          \item \verb|Units| --- The units of the data.
          \item \verb|Symbol| --- The symbol for the element.
          \item \verb|Format| --- The format string for the element (e.g., ``\%f'').
          \item \verb|Type| --- The SDDS data type name (e.g., double, float, etc.).
          \item \verb|Description| --- The description for the element.
          \item \verb|Group| --- The group name (for array elements only).
        \end{itemize}
      \item Parameters:
        \begin{itemize}
          \item \verb|Class| --- The SDDS class for the present page.
          \item \verb|Filename| --- The filename being described by the present page.
        \end{itemize}
    \end{itemize}
  \item \textbf{switches:} Normal operation of \verb|sddsquery| results in a printout summarizing the header of each file. If one of
    the options is given, however, this printout will not appear. Instead, the selected list of names appears for each file.
    \begin{itemize}
      \item \verb|sddsOutput[=filename]| --- Requests that output be delivered in SDDS protocol. If no \verb|filename| is given,
        the output is delivered to the standard output.
      \item \verb|arrayList| --- In non-SDDS output mode, requests that a list of array names be printed to the standard output, one
        name per line. In SDDS output mode, requests that only array information be provided.
      \item \verb|columnList| --- In non-SDDS output mode, requests that a list of column names be printed to the standard output,
        one name per line. In SDDS output mode, requests that only column information be provided.
      \item \verb|parameterList| --- In non-SDDS output mode, requests that a list of parameter names be printed to the standard
        output, one name per line. In SDDS output mode, requests that only parameter information be provided.
      \item \verb|-version| --- Requests that the SDDS version number of the file be printed to the standard output. Valid in
        non-SDDS output mode only.
      \item \verb|-delimiter=delimitingString| --- Requests that listed items be separated by the given string. By default, the
        delimiter is a newline. Valid in non-SDDS output mode only.
      \item \verb|-appendUnits[=bare]| --- Requests that the units of each item be printed directly following the item name. Valid in
        non-SDDS output mode only. If the \verb|bare| qualifier is not given, then the units are enclosed in parentheses.
      \item \verb|-readAll| --- Forces \verb|sddsquery| to read the entire file. On some operating systems this is necessary when
        querying compressed files to prevent ``Broken Pipe'' errors. For large files, use of this option will make \verb|sddsquery|
        slower.
    \end{itemize}
  \item \textbf{see also:} \hyperref[exampleData]{Data for Examples}
  \item \textbf{author:} M. Borland, ANL/APS.
\end{sddsprog}

