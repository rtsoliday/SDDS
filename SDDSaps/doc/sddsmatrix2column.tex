%\begin{latexonly} 
\newpage 
%\end{latexonly} 
\subsection{sddsmatrix2column} 
\label{sddsmatrix2column} 
 
\begin{itemize} 
\item {\bf description:} \hspace*{1mm}\\ 
{\tt sddsmatrix2column} transfers matrix into a column.

\item {\bf synopsis:}  
\begin{flushleft}
{\tt 
sddsmatrix2column [{\em inputfile}] [{\em outputfile}] [-pipe=[input],[output]] \\ \
[-rowNameColumn={\em string}] \\ \
[-dataColumnName={\em string}] \\ \
[-rootnameColumnName={\em string}] \\ \
[-majorOrder=row|column]}
\end{flushleft} 

\item {\bf files:}
    \begin{itemize} 
    \item {\em inputfile} Contains a string column (not required) and multiple numerical columns. If string column not provided or rowNameColumn not provided, Row$<$row\_index$>$ will be used as row names in the output file.
    \item {\em outputfile} The output file contain 2 columns: string column and data column. String column would be combination of input string column (or Row$<$row\_index$>$) and input data column names.
    \end{itemize} 

\item {\bf switches:} 
    \begin{itemize} 
    \item {\tt -pipe=[input][,output]} --- Standard SDDS pipe options for reading/writing files from stdin/stdout.
    \item {\tt -rowNameColumn} --- The column name of row names (each row has a name) in the input file. If not provided, Row$<$row\_index$>$ will be used instead. 
    \item {\tt -dataColumnName} --- The column name of the data in the output file, if not provided, ``Rootname'' will be used
    \item {\tt -rootnameColumnName} --- The column name of the String column in output file.
    \item {\tt -majorOrder=row|column} --- Specifies the binary SDDS layout.
    \end{itemize} 

\item {\bf author:} H. Shang, R. Soliday, ANL/APS. 
\end{itemize} 
