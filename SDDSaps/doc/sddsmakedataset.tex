% $Log: not supported by cvs2svn $
% Revision 1.1  2003/06/20 19:34:59  shang
% first version
%
% first version, Hairong Shang
%
%
% Template for making SDDS Toolkit manual entries.
%
%\begin{latexonly}
\newpage
%\end{latexonly}

%
% Substitute the program name for <programName>
%
\subsection{sddsmakedataset}
\label{sddsmakedataset}

\begin{itemize}
\item {\bf description:}
%
% Insert text of description (typicall a paragraph) here.
%
\verb+sddsmakedataset + writes the input data into a file or 
pipe in SDDS format. It can be used to make add SDDS file consisting of
a small amount of data from the script. It is more convenient than {``sdds save''}.

\item {\bf examples:} 
%
% Insert text of examples in this section.  Examples should be simple and
% should be preceeded by a brief description.  Wrap the commands for each
% example in the following construct:
% 
%
{\tt }
\begin{flushleft}{\tt
\bf sddsmakedataset mydata.sdds -parameter=pi,type=double -data=3.1415926 -parameter=UserName,type=string -data=somebody 
\bf  -column=index,type=short -data=1,2,3,4,5,6,7,8,9,10 -column=primeNumbers,type=long -data=1,2,3,5,7,11,13,17,19,23
\bf -column=lettersOfAlphabet,type=character -data=a,b,c,d,e,f,g,h,i,j -ascii
}\end{flushleft}
{\tt}An ascii file mydata.sdds is created by this command. The printout of mydata.sdds is as following: (used sddsprintout to get the prinout).
\begin{flushleft}
\begin{verbatim}
Printout for SDDS file mydata.sdds

pi =           3.141593e+00  UserName =         somebody

 index  primeNumbers  lettersOfAlphabet 
----------------------------------------
     1             1                  a 
     2             2                  b 
     3             3                  c 
     4             5                  d 
     5             7                  e 
     6            11                  f 
     7            13                  g 
     8            17                  h 
     9            19                  i 
    10            23                  j 
\end{verbatim}
\end{flushleft}

\item {\bf synopsis:} 
%
% Insert usage message here:
%
\begin{flushleft}{\tt
sddsmakedataset [<oututFile> | -pipe=out] 
[-defaultType={double|float|long|short|string|character}] [-parameter=<name>[,type=<string>][,units=<string>][,symbol=<string>][,description=<string>]]
[-data=<value>] -parameter=.... -data=...
[-column=<name>[,type=<string>][,units=<string>][,symbol=<string>][,description=<string>]]
[-data=<listOfCommaSeparatedValue>] -column=... -data=... 
[-noWarnings] [-description=<string>] [-contents=<string>] [-mode=<string>]        
}\end{flushleft}

\item {\bf switches:}
%
% Describe the switches that are available
%
    \begin{itemize}
    \item {\tt  outputFile} --- SDDS output file for writing the data to.
    \item {\tt  -pipe=out}  --- output the data in SDDS format to the pipe instead of to a file.
    \item {\tt  -defaultType} --- specify the default data type for paparemeters and columns 
                                  if not specified in the parameter or column definition.
    \item {\tt  -parameter}   --- specify the parameter name, data type, units, 
                                  symbol and description.
    \item {\tt  -column} --- specify the column name, data type, units, symbol and/or description.
    \item {\tt  -noWarnings} --- do not print out warning messages.
    \item {\tt  -ascii} --- output file in ascii mode, the default is binary. 
    \item {\tt  -description} --- description of output file.
    \item {\tt  -contents} --- contents of the description.
    \end{itemize}

%\item {\bf see also:}
%    \begin{itemize}
%
% Insert references to other programs by duplicating this line and 
% replacing {\em prog} with the program to be referenced:
%
%    \item \progref{<prog>}
%    \end{itemize}
%
% Insert your name and affiliation after the '}'
%
\item {\bf author: H. Shang } ANL
\end{itemize}



