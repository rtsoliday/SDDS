\begin{sddsprog}{sddspfit}
  \item \textbf{description:} \verb|sddspfit| does ordinary and Chebyshev polynomial fits to column data, including error analysis. It will do fits with a specified number of terms, with specific terms only, and with specific symmetry only. It will also eliminate spurious terms.
  \item \textbf{examples:}
  \begin{verbatim}
sddspfit data.sdds fit.sdds -columns=x,y -terms=3
sddspfit par.bpm par.bpm1 -pipe=out -columns=P1P2x,P1P1x
  \end{verbatim}
  \item \textbf{synopsis:}
  \begin{verbatim}
sddspfit [-pipe=[input][,output]] [inputFile] [outputFile]
  [-evaluate=filename[,begin=value][,end=value][,number=integer]]
  -columns=xName,yName[,xSigma=name][,ySigma=name]
  {-terms=number [-symmetry={none | odd | even}] | -orders=number[,number...]}
  [-reviseOrders[=threshold=chiValue][,verbose][,complete=<chiThreshold>][,goodEnough=<chiValue>]]
  [-chebyshev[=convert]]
  [-xOffset=value] [-xFactor=value]
  [-sigmas={absolute=value | fractional=value}]
  [-modifySigmas] [-generateSigmas[={keepLargest | keepSmallest}]]
  [-sparse=interval] [-range=lower,upper]
  [-normalize[=termNumber]] [-verbose]
  [-fitLabelFormat=sprintfString]
  \end{verbatim}
  \item \textbf{files:} \emph{inputFile} is an SDDS file containing columns of data to be fit. If it contains multiple pages, they are processed separately. \emph{outputFile} is an SDDS file containing one page for each page of \emph{inputFile}. It contains columns of the independent and dependent variable data, plus columns for error bars (``sigmas'') as appropriate. The values of the fit and of the residuals are in columns named \emph{yName}\verb|Fit| and \emph{yName}\verb|Residual|. \emph{outputFile} also contains the following one-dimensional arrays:
    \begin{itemize}
      \item \verb|Order|: a long integer array of the polynomial orders used in the fit.
      \item \verb|Coefficient|: a double-precision array of fit coefficients.
      \item \verb|CoefficientSigma|: a double-precision array of fit coefficient errors. Present only if errors are present for data.
      \item \verb|CoefficientUnits|: a string array of fit coefficient units.
    \end{itemize}
    \emph{outputFile also contains the following parameters:}
    \begin{itemize}
      \item \verb|Basis|: a string identifying the type of polynomials used.
      \item \verb|ReducedChiSquared|: the reduced chi-squared of the fit:
      $$ \chi^2_\nu = \frac{\chi^2}{\nu} = \frac{1}{N-T}\sum_{i=0}^{N-1} \left(\frac{y_i - y(x_i)}{\sigma_i}\right)^2 $$,
      where $\\nu = N-T$ is the number of degrees of freedom for a fit of N points with T terms.
      \item \verb|rmsResidual|
      \item \emph{xName}\verb|Offset|, \emph{xName}\verb|Factor|
      \item \verb|FitIsValid|: a character having values \verb|y| and \verb|n| if the page contains a valid fit or not.
      \item \verb|Terms|: the number of terms in the fit.
      \item \verb|sddspfitLabel|: a string containing an equation showing the fit, suitable for use with \verb|sddsplot|.
      \item \verb|Intercept|, \verb|Slope|, \verb|Curvature|: the three lowest order coefficients for ordinary polynomial fits. These are present only if orders 0, 1, and 2 respectively are requested in fitting. If error analysis is valid, then the errors for these quantities appear as \emph{quantityName}\verb|Sigma|.
    \end{itemize}
  \item \textbf{switches:}
    \begin{itemize}
      \item \verb|-pipe[=input][,output]| --- The standard SDDS Toolkit pipe option.
      \item \verb|-evaluate=filename[,begin=value][,end=value][,number=integer]| --- Specifies creation of an SDDS file called \emph{filename} containing points from evaluation of the fit. The fit is normally evaluated over the range of the input data; this may be changed using the \verb|begin| and \verb|end| qualifiers. Normally, the number of points at which the fit is evaluated is the number of points in the input data; this may be changed using the \verb|number| qualifier.
      \item \verb|-columns=xName,yName[,xSigma=name][,ySigma=name]| --- Specifies the names of the columns to use for the independent and dependent data, respectively. \verb|xSigma| and \verb|ySigma| can be used to specify the errors for the independent and dependent data, respectively.
      \item By default, an ordinary polynomial fit is done using a constant and linear term. Control of what fit terms are used is provided by the following switches:
        \begin{itemize}
          \item \verb|-terms=number| --- Specifies the number of terms to be used in fitting. 2 terms is linear fit, 3 is quadratic, etc.
          \item \verb!-symmetry={none | odd | even}! --- When used with \verb|-terms|, allows specifying the symmetry of the N terms used. \verb|none| is the default. \verb|odd| implies using linear, cubic, etc., while \verb|even| implies using constant, quadratic, etc.
          \item \verb|-orders=number[,number...]| --- Specifies the polynomial orders to be used in fitting. The default is equivalent to \emph{-orders=0,1}.
          \item \verb|-reviseOrders[=threshold=chiValue][,verbose][,complete=<chiThreshold>][,goodEnough=<chiValue>]| --- Specifies adaptive fitting to eliminate spurious terms. When invoked, this switch causes \verb|sddspfit| to repeatedly fit the first page of data with different numbers of terms in an attempt to find a minimal number of terms that gives an acceptable fit. This is done in up to three stages:
            \begin{enumerate}
              \item The process starts by making a fit with all terms. Then, each term is eliminated individually and a new fit is made. If the new fit has a smaller reduced chi-squared by an amount of at least \emph{chiValue}, then the term is permanently eliminated and the process is repeated for each remaining term. By default, the criterion for an improvement is a change of 0.1 in the reduced chi-squared. This step eliminates terms that result in a bad fit due to numerical problems. If the \verb|goodEnough=chiValue| qualifier is given, then the first fit that has reduced chi-squared less than \emph{chiValue} is used.
              \item Next, the individual terms are tested for how well they improve reduced chi-squared. Any term that does not improve the reduced chi-squared by at least \emph{chiValue} is eliminated. This stage eliminates terms that do not sufficiently improve the fit to merit inclusion. Again, if the \verb|goodEnough=chiValue| qualifier is given, then the first fit that has reduced chi-squared less than \emph{chiValue} is used.
              \item Finally, if \verb|complete=chiThreshold| is given, then next stage involves repeating the above procedure with the remaining terms, but instead of eliminating one term at a time, the program tests each possible combination of terms. This can be very time consuming, especially if the \verb|goodEnough=chiValue| qualifier is not given.
            \end{enumerate}
          \item \verb|[-chebyshev[=convert]]| --- Asks that Chebyshev T polynomials be used in fitting. If \verb|convert| is given, the output contains the coefficients for the equivalent ordinary polynomials.
        \end{itemize}
      \item \verb|-xOffset=value|, \verb|-xFactor=value| --- Specify offsetting and scaling of the independent data prior to fitting. The transformation is ${\\rm x \\rightarrow (x - Offset)/Factor}$. This feature can be used to make a fit about a point other than $x=0$, or to scale the data to make high-order fits more accurate.
      \item \verb|sddspfit| will compute error bars (``sigmas'') for fit coefficients if it has knowledge of the sigmas for the data points. These can be supplied using the \verb|-columns| switch, or generated internally in several ways:
        \begin{itemize}
          \item \verb!-sigmas={absolute=value | fractional=value}! --- Specifies that independent-variable errors be generated using a specified value for all points, or a specified fraction for all points.
          \item \verb|-modifySigmas| --- Specifies that independent-variable sigmas be modified to include the effect of uncertainty in the dependent variable values. If this option is not given, any x sigmas specified with \verb|-columns| are ignored.
          \item \verb!-generateSigmas[={keepLargest | keepSmallest}]! --- Specifies that independent-variable errors be generated from the variance of an initial equal-weights fit. If errors are already given (via \emph{-column}), one may request that for every point \verb|sddspfit| retain the larger or smaller of the sigma in the data and the one given by the variance.
        \end{itemize}
      \item \verb|-sparse=interval| --- Specifies sparsing of the input data prior to fitting. This can greatly speed computations when the number of data points is large.
      \item \verb|-range=lower,upper| --- Specifies the range of independent variable over which to do fitting.
      \item \verb|-normalize[=termNumber]| --- Specifies that coefficients be normalized so that the coefficient for the indicated order is unity. By default, the 0-order term (i.e., the constant term) is normalized to unity.
      \item \verb|-verbose| --- Specifies that the results of the fit be printed to the standard error output.
      \item \verb|-fitLabelFormat=sprintfString| --- Specifies the format to use for printing numbers in the fit label. The default is ``\%g''.
    \end{itemize}
  \item \textbf{see also:}
    \begin{itemize}
      \item \hyperref[exampleData]{Data for Examples}
      \item \progref{sddsexpfit}
      \item \progref{sddsgfit}
      \item \progref{sddsplot}
      \item \progref{sddsoutlier}
    \end{itemize}
  \item \textbf{author:} M. Borland, ANL/APS.
\end{sddsprog}
