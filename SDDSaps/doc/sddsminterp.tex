\begin{sddsprog}{sddsminterp}
  \item \textbf{description:} \verb|sddsminterp| performs multiplicative renormalized model interpolation of a data set using another data set as a model function.
  \item \textbf{examples:}
    \begin{verbatim}
sddsminterp data.sdds output.sdds -columns=x,y \
  -model=model.sdds,abscissa=x,ordinate=y,interp=1 \
  -order=2
    \end{verbatim}
  \item \textbf{synopsis:}
    \begin{verbatim}
sddsminterp [input-file] [output-file] [-pipe=[input][,output]]
  -columns=independent-quantity,name
  [-interpOrder=order]
  -model=modelFile,abscissa=column,ordinate=column[,interp=order]
  [-fileValues=valuesFile[,abscissa=column]]
  [-majorOrder=row|column]
  [-verbose]
  [-ascii]
    \end{verbatim}
  \item \textbf{switches:}
    \begin{itemize}
    \item \verb|-pipe=[input][,output]| --- Standard SDDS pipe options for reading/writing files from stdin/stdout.
    \item \verb|-columns| --- Specifies the data in the input file to be interpolated.
    \item \verb|-interpOrder| --- Interpolation order of the multiplicative factor.
    \item \verb|-model| --- Data representing the model function.
    \item \verb|-fileValues| --- Specifies abscissa at which interpolated values are calculated. If not present, then the abscissa values of the model file are used.
    \item {\tt -majorOrder=row|column} --- Specifies the binary SDDS layout.
    \item \verb|-verbose| --- Print verbose messages to stdout.
    \item \verb|-ascii| --- Save output in ASCII SDDS format.
    \end{itemize}
  \item \textbf{see also:}
    \begin{itemize}
    \item \progref{sddsinterp}
    \end{itemize}
  \item \textbf{author:} L. Emery, C. Saunders, M. Borland, R. Soliday, H. Shang, ANL/APS.
\end{sddsprog}
