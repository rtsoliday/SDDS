%\begin{latexonly} 
\newpage 
%\end{latexonly} 
\subsection{sddsminterp} 
\label{sddsminterp} 
 
\begin{itemize} 
\item {\bf description:} \hspace*{1mm}\\ 
{\tt sddsminterp} performs multiplicative renormalized model interpolation of a data set using another data set as a model function.
\item {\bf synopsis:}  
\begin{flushleft}
{\tt 
sddsminterp [{\em input-file}] [{\em output-file}] [-pipe=[input],[output]] \\ \
-columns={\em independent-quantity},{\em name} \\ \
{}[-interpOrder={\em order}] \\ \
-model={\em modelFile},abscissa={\em column},ordinate={\em column}[,interp={\em order}] \\ \
-fileValues={\em valuesFile}[,abscissa={\em column}] \\ \
{}[-majorOrder=row|column] \\ \
-verbose \\ \
-ascii}
\end{flushleft} 
\item {\bf switches:} 
    \begin{itemize} 
    \item {\tt -pipe=[input][,output]} --- Standard SDDS pipe options for reading/writing files from stdin/stdout.
    \item {\tt -columns} --- Specifies the data in the input file to be interpolated.
    \item {\tt -interpOrder} --- Interpolation order of the multiplicative factor.
    \item {\tt -model} --- Data representing the model function.
    \item {\tt -fileValues} --- Specifies abscissa at which interpolated values are calculated. If not present, then the abscissa values of the model file are used.
    \item {\tt -majorOrder=row|column} --- Specifies the binary SDDS layout.
    \item {\tt -verbose} --- Print verbose messages to stdout.
    \item {\tt -ascii} --- Save output in ASCII SDDS format.
\end{itemize} 

\item {\bf author:} L. Emery, C. Saunders, M. Borland, R. Soliday, H. Shang, ANL/APS. 
\end{itemize} 
