%\begin{latexonly} 
\newpage 
%\end{latexonly} 
 
\subsection{sddscollect} 
\label{sddscollect} 
 
\begin{itemize} 
\item {\bf description:} 
\verb|sddscollect| reorganizes tabular data from the input file 
to bring data from several groups of similarly named columns together into  
a single column per group.  In doing so, it creates one page of output 
for every row on the input file.  It also creates parameters to 
hold data from columns that are not included in any group. 
 
\item {\bf examples:}  
Take a data set with several columns of PAR BPM x and y 
readings (one set of readings per row) and create a new file with one column for x 
readings and one column for y readings, with each  
page containing one set of readings.
\begin{flushleft}{\tt 
sddscollect par.bpm par.orbits -collect=suffix=x -collect=suffix=y 
}\end{flushleft} 
The output file has three columns, called \verb|x|, \verb|y|, and 
\verb|Rootname|.  The latter column contains the original column 
names less the ``x'' (or ``y'') suffix.   
 
Do statistics on PAR BPM x and y readbacks, then collect the statistics  
into columns, one column for each type of statistic: 
\begin{flushleft}{\tt  
sddsprocess par.bpm -pipe=out -process=P?P?[xy],spread,\%sSpread  
-process=P?P?[xy],ave,\%sMean -process=P?P?[xy],stand,\%sStDev  
| sddscollapse  
| sddscollect -pipe=in parbpm.stat \ 
  -collect=suffix=xSpread -collect=suffix=xMean -collect=suffix=xStDev 
  -collect=suffix=ySpread -collect=suffix=yMean -collect=suffix=yStDev 
} 
\end{flushleft} 
The output file has columns named {\tt xSpread}, {\tt ySpread},  
{\tt xMean}, {\tt yMean}, {\tt xStDev}, 
and {\tt yStDev}, plus an additional column named {\tt Rootname}. 
The latter column contains the remnants of each original column 
name after the suffix is removed.  Note that in the example, 
the remnant names are the same for all the collections specified. 
If this were not true, \verb|sddscollect| would abort and give 
an error message. 
 
\item {\bf synopsis:}  
 
\begin{flushleft}{\tt 
sddscollect [{\em input}] [{\em output}] [-pipe[=input][,output]] \\
-collect=\{suffix | prefix | match\}={\em match}[,column={\em newName}][,editCommand=<string>] 
}\end{flushleft} 
 
\item {\bf switches:} 
    \begin{itemize} 
    \item {\tt -pipe[=input][,output]} --- The standard SDDS Toolkit pipe option. 
    \item {\tt -collect=\{suffix | prefix | match\}={\em match}[,column={\em newName}][,editCommand=<string>]} 
                --- Specifies 
    a string, {\em match}, to look for at the end of (suffix mode),  
    beginning of (prefix mode), or anywhere in (match mode) column names in the input file. 
    Data from all matching columns is transferred into a single column 
    in the output file.  For prefix and suffix modes, the default name of this column is the 
    suffix or prefix string, while for match mode it is created by applying the edit command
    to the matching column names. This may be changed with the  {\tt column} qualifier. 
 
    This option may be given any number of times. However, all  
    collections must produce the same number of matches.  Further 
    the set of name remainders (i.e., the original 
    column name less the prefix or suffix, or following editing) must  
    be the same for each collection. 
    \end{itemize} 
\item {\bf see also:} 
    \begin{itemize} 
    \item \progref{sddsregroup} 
    \item \progref{sddstranspose} 
    \item \progref{sddsEditing}
    \end{itemize} 
\item {\bf author:} M. Borland, ANL/APS. 
\end{itemize} 
