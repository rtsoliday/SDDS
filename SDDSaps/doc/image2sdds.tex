\begin{sddsprog}{image2sdds}
  \item \textbf{description:} \verb|image2sdds| converts raw binary image files to SDDS format. It can
    produce single column or multi-column image files and supports both
    binary and ASCII output. Options are provided to transpose the image,
    write data as a 2D array, and generate headers compatible with
    \verb|sddscontour|.
  \item \textbf{examples:}
    \begin{verbatim}
    image2sdds camera.raw image.sdds -2d -xdim 640 -ydim 480 -xmax 10 -ymax 10
    image2sdds camera.raw image.sdds -multicolumnmode -transpose -ascii
    \end{verbatim}
  \item \textbf{synopsis:}
    \begin{verbatim}
    image2sdds <IMAGE infile> <SDDS outfile> [-2d] [-ascii] [-contour]
        [-multicolumnmode] [-transpose] [-xdim value] [-ydim value]
        [-xmin value] [-xmax value] [-ymin value]
        [-ymax value] [-debug] [-help]
    \end{verbatim}
  \item \textbf{switches:}
    \begin{itemize}
    \item {\tt -2d} --- Write the image as a single SDDS 2D array. May not be used
      with \verb|-multicolumnmode|.
    \item {\tt -ascii} --- Produce ASCII SDDS output rather than binary.
    \item {\tt -contour} --- Add parameters expected by \progref{sddscontour}.
    \item {\tt -multicolumnmode} --- Write image data in multiple columns. Incompatible
      with \verb|-2d|.
    \item {\tt -transpose} --- Transpose the image about its diagonal prior to output.
    \item {\tt -xdim {\em value}} --- Number of pixels in the horizontal dimension
      (default {\tt 482}).
    \item {\tt -ydim {\em value}} --- Number of pixels in the vertical dimension
      (default {\tt 512}).
    \item {\tt -xmin {\em value}} --- Minimum x value for the image data (default 0).
    \item {\tt -xmax {\em value}} --- Maximum x value for the image data. If not given
      the spacing is determined from {\tt -xdim}.
    \item {\tt -ymin {\em value}} --- Minimum y value for the image data (default 0).
    \item {\tt -ymax {\em value}} --- Maximum y value for the image data. If not given
      the spacing is determined from {\tt -ydim}.
    \item {\tt -debug {\em level}} --- Enable debugging output at the specified level.
    \item {\tt -help} or {\tt -?} --- Print command usage information.
    \end{itemize}
  \item \textbf{see also:}
    \begin{itemize}
    \item \progref{sddscontour}
    \item \progref{sddsimageconvert}
    \item \progref{sddsimageprofiles}
    \end{itemize}
  \item \textbf{author:} J. Stein, J. Anderson, R. Soliday, ANL/APS.
\end{sddsprog}

