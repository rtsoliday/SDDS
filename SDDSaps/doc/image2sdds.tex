%
%\begin{latexonly}
\newpage
%\end{latexonly}
\subsection{image2sdds}
\label{image2sdds}

\begin{itemize}
\item {\bf description:} \verb|image2sdds| converts raw binary image files to SDDS format. It can
  produce single column or multi-column image files and supports both
  binary and ASCII output.  Options are provided to transpose the image,
  write data as a 2D array, and generate headers compatible with
  \verb|sddscontour|.
\item {\bf example:}
\begin{flushleft}{\tt
image2sdds camera.raw image.sdds -2d -xdim 640 -ydim 480 -xmax 10 -ymax 10\\
image2sdds camera.raw image.sdds -multicolumnmode -transpose -ascii
}\end{flushleft}
\item {\bf synopsis:}
\begin{flushleft}{\tt
image2sdds <IMAGE infile> <SDDS outfile> [-2d] [-ascii] [-contour]\\
    {}[-multicolumnmode] [-transpose] [-xdim {\em value}] [-ydim {\em value}]\\
    {}[-xmin {\em value}] [-xmax {\em value}] [-ymin {\em value}]\\
    {}[-ymax {\em value}] [-debug] [-help]
}\end{flushleft}
\item {\bf switches:}
  \begin{itemize}
  \item {\tt -2d} --- Write the image as a single SDDS 2D array. May not be used
    with \verb|-multicolumnmode|.
  \item {\tt -ascii} --- Produce ASCII SDDS output rather than binary.
  \item {\tt -contour} --- Add parameters expected by \progref{sddscontour}.
  \item {\tt -multicolumnmode} --- Write image data in multiple columns. Incompatible
    with \verb|-2d|.
  \item {\tt -transpose} --- Transpose the image about its diagonal prior to output.
  \item {\tt -xdim {\em value}} --- Number of pixels in the horizontal dimension
    (default {\tt 482}).
  \item {\tt -ydim {\em value}} --- Number of pixels in the vertical dimension
    (default {\tt 512}).
  \item {\tt -xmin {\em value}} --- Minimum x value for the image data (default 0).
  \item {\tt -xmax {\em value}} --- Maximum x value for the image data. If not given
    the spacing is determined from {\tt -xdim}.
  \item {\tt -ymin {\em value}} --- Minimum y value for the image data (default 0).
  \item {\tt -ymax {\em value}} --- Maximum y value for the image data. If not given
    the spacing is determined from {\tt -ydim}.
  \item {\tt -debug {\em level}} --- Enable debugging output at the specified level.
  \item {\tt -help} or {\tt -?} --- Print command usage information.
  \end{itemize}
\item {\bf see also:}
  \begin{itemize}
  \item \progref{sddscontour}
  \item \progref{sddsimageconvert}
  \item \progref{sddsimageprofiles}
  \end{itemize}
\item {\bf author:} J. Stein, J. Anderson, R. Soliday, ANL/APS.
\end{itemize}
