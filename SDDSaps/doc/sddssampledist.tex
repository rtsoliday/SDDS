%\begin{latexonly}
\newpage
%\end{latexonly}
\subsection{sddssampledist}
\label{sddssampledist}

\begin{sddsprog}{sddssampledist}
  \item \textbf{description:} \verb|sddssampledist| provides for pseudo-random sampling of probability distributions. It also provides nonrandom sampling using Halton sequences.
  \item \textbf{examples:} Draw random samples from a normal (Gaussian) distribution, G(z), shifted to have a sigma of 10 and centroid of 5.
    \begin{verbatim}
    sddssampledist gaussian.sdds samples.sdds -samples=100 -columns=indep=z,df=G,output=zSample,factor=10,offset=5
    \end{verbatim}
  \item \textbf{synopsis:}
    \begin{verbatim}
    sddssampledist [input] [output] [-pipe=[in][,out]]
      -columns=independentVariable=name,{cdf=CDFName | df=DFName}
        [,output=name][,units=string][,factor=value]
        [,offset=value][,datafile=filename]
        [,haltonRadix=primeNumber[,randomize[,group=groupID]]]
      [-columns=...] [-samples=integer] [-seed=integer]
    \end{verbatim}
  \item \textbf{files:}
    \emph{input} is the default input file for distribution functions (DFs) and cumulative distribution functions (CDFs). \emph{input} is not required if all \verb|-columns| options give the \verb|datafile| qualifier.
    \emph{output} contains the samples. By default the sampled data names match the independent variable names from the \verb|-columns| options. Use the \verb|output| qualifier to change these names.
  \item \textbf{switches:}
    \begin{itemize}
      \item \verb|-pipe[=input][,output]| --- The standard SDDS Toolkit pipe option.
      \item \verb!-columns=independentVariable=name,{cdf=CDFName | df=DFName}[,output=name][,units=string][,factor=value][,offset=value][,datafile=filename][,haltonRadix=primeNumber[,randomize[,group=groupID]]]!--- Specifies the CDF or DF from which to draw samples (\verb|cdf| or \verb|df| qualifier) and the independent variable. This option may be given multiple times. \verb|output| sets the column name for the samples. \verb|units| specifies the units. \verb|factor| and \verb|offset| apply a transformation $x \rightarrow x*f+o$. \verb|datafile| gives an alternate file containing the distribution function data, otherwise the main input file is used. \verb|haltonRadix| selects the radix for generating a non-random Halton sequence, which provides smoother sampling than a pseudo-random sequence. The radix should be a small prime number. Use \verb|randomize| to remove correlations when using the same radix for multiple sequences. Use \verb|group| to assign options to a group for correlated randomization.
      \item \verb|-samples=integer| --- Specifies the number of samples to generate.
      \item \verb|-seed=integer| --- Specifies the seed for the random number generation. Should be a large, odd integer. If not given, the system clock is used.
    \end{itemize}
  \item \textbf{author:} M. Borland, ANL/APS.
\end{sddsprog}

